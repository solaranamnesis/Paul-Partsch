\documentclass[a4paper, 11pt, oneside, polutonikogreek, german]{article}
\usepackage{gfsbaskerville}
% Load encoding definitions (after font package)
\usepackage[LGR,T1]{fontenc}
\usepackage{wasysym}
\usepackage{textalpha}
\usepackage{longtable}
\usepackage{listings}
\lstset{basicstyle=\ttfamily}
% Babel package:
\usepackage{babel}

% With XeTeX$\$LuaTeX, load fontspec after babel to use Unicode
% fonts for Latin script and LGR for Greek:
\ifdefined\luatexversion \usepackage{fontspec}\fi
\ifdefined\XeTeXrevision \usepackage{fontspec}\fi

% "Lipsiakos" italic font `cbleipzig`:
\newcommand*{\lishape}{\fontencoding{LGR}\fontfamily{cmr}%
		       \fontshape{li}\selectfont}
\DeclareTextFontCommand{\textli}{\lishape}

\usepackage{booktabs}
\setlength{\emergencystretch}{15pt}
\usepackage{fancyhdr}
\usepackage{microtype}
\begin{document}
\begin{titlepage} % Suppresses headers and footers on the title page
	\centering % Centre everything on the title page
	%\scshape % Use small caps for all text on the title page

	%------------------------------------------------
	%	Title
	%------------------------------------------------
	
	\rule{\textwidth}{1.6pt}\vspace*{-\baselineskip}\vspace*{2pt} % Thick horizontal rule
	\rule{\textwidth}{0.4pt} % Thin horizontal rule
	
	\vspace{1\baselineskip} % Whitespace above the title
	
	{\scshape\LARGE Die Meteoriten oder\\[1.25pt] vom Himmel gefallenen \\[1.25pt] Steine und Eisenmassen\\[1.25pt] im k. k. Hof-Mineralien-Kabinette\\[4pt] zu Wien.}
	
	\vspace{1\baselineskip} % Whitespace above the title

	\rule{\textwidth}{0.4pt}\vspace*{-\baselineskip}\vspace{3.2pt} % Thin horizontal rule
	\rule{\textwidth}{1.6pt} % Thick horizontal rule
	
	\vspace{1\baselineskip} % Whitespace after the title block
	
	%------------------------------------------------
	%	Subtitle
	%------------------------------------------------
	
	{\scshape Beschrieben,\\ und durch wissenschaftliche und geschichtliche \\ Zusätze erläutert \\ von \\ Paul Partsch,} % Subtitle or further description
	
	\vspace*{1\baselineskip} % Whitespace under the subtitle
	
    {\scshape\scriptsize Kustos an dem genannten Kabinette. \\ Mit einer Abbildung.} % Subtitle or further description
    
	%------------------------------------------------
	%	Editor(s)
	%------------------------------------------------
    \vspace*{\fill}

	\vspace{1\baselineskip}

	{\small\scshape Wien 1843.}
	
	{\small\scshape{Verlag von Kaulfuss Witwe, Prandel \& Komp.}}
	
	\vspace{0.5\baselineskip} % Whitespace after the title block

    \scshape Internet Archive Online Edition  % Publication year
	
	{\scshape\small Namensnennung Nicht-kommerziell Weitergabe unter gleichen Bedingungen 4.0 International} % Publisher
\end{titlepage}
\setlength{\parskip}{1mm plus1mm minus1mm}
\clearpage
\tableofcontents
\clearpage
\vspace*{\fill}
\begin{quote}
   Es lässt sich als ausgemacht ansehen, dass sie nicht von der Erde, sondern von einem anderen Weltkörper herstammen, und folglich die Beschaffenheit der außerhalb der Erde vorkommenden wägbaren Stoffe verkünden. In dieser Beziehung haben die Meteorsteine ein außerordentliches Interesse. - Berzelius
\end{quote}
\vspace*{\fill}
\clearpage
\section*{Vorwort.}
\paragraph{}
In dem k. k. Hof-Mineralien-Kabinette zu Wien befinden sich acht Sammlungen in Glasschränken zur Schau gestellt, die jede Woche zweimal, Mittwoche und Sonnabend, von Jedermann besehen und benützt werden können. Nachdem der Herausgeber vorliegender Schrift eine kurze allgemeine Übersicht dieser Sammlungen des k. k. Mineralien-Kabinettes kürzlich in Druck gelegt hat, beginnt er das darin gegebene Versprechen, von jeder derselben, je nach Bedürfnis und Zweckmäßigkeit, entweder spezielle Verzeichnisse oder doch ausgedehntere Übersichten nachfolgen zu lassen, dadurch in Ausführung zu bringen, dass er zuerst das vorliegende beschreibende Verzeichnis erscheinen lässt. Die Meteoriten-Sammlung des k. k. Mineralien-Kabinettes ist zwar von allen daselbst befindlichen der Anzahl der Stücke nach die kleinste, aber doch die reichste und vollständigste in der Anzahl von Lokalitäten und Exemplaren unter allen bestehenden Sammlungen ihrer Art, und überhaupt eine der merkwürdigsten Zusammenstellungen von unorganischen Körpern.\footnote{Die Meteoriten-Sammlung des k. k. Mineralien-Kabinettes enthielt im Monate Februar 1843, mit Ausschluss aller Pseudometeoriten, die wir später anführen werden, 94 verschiedene Lokalitäten von Meteoriten, und zwar 69 von Meteorsteinen, und 25 von Meteoreisen in 258 Stücken oder eigentlich Nummern, da zuweilen mehrere kleine Stücke unter Einem Nummer vereinigt sind. (Im Jahre 1806 zählte sie 7, im Jahre 1819. 36, im Jahre 1836. 58 Lokalitäten.) - Von Meteoriten, mit Ausschluss aller Pseudometeoriten, besaßen zwischen den Jahren 1840 und 1842: das k. Mineralien-Kabinett der Universität zu Berlin, mit welcher die Sammlung Chladnis vereinigt ist, 78 Lokalitäten; Baron Reichenbach in Wien 68 (wovon jedoch 19 nur in kleinen Splittern); die Galerie der Mineralogie im k. Museum der Naturgeschichte zu Paris 42; Gubernialrat Neumann in Prag 40 (meistens in ganz kleinen Fragmenten); die Mineralien-Sammlung im britischen Museum zu London und die Mineralien-Sammlung der Universität zu Göttingen jede 35; Professor John in Berlin 28 (ebenfalls meist in ganz kleinen Stückchen); die Mineralien-Sammlung der Akademie der Wissenschaften zu St. Petersburg und Baron Berzelius in Stockholm 18; Straßenbaudirektor Braumüller zu Brünn 17; die Mineralien-Sammlung des Herrn Turner in England, ehemals Eigentum des H. Heuland in London 15; die Mineralien-Sammlung des Marquis de Drée zu Paris 14. Diess sind die an Meteoriten reichsten Sammlungen; andere öffentliche und Privat-Sammlungen besitzen selten mehr als 12 Lokalitäten; so die Ecole des Mines zu Paris und die Mineralien-Sammlung des Grafen Beroldingen in Wien 12; die Privat-Mineralien-Sammlung des Königs von Dänemark zu Kopenhagen 11; das herzogliche Naturalien-Kabinett zu Gotha 10; die Mineralien-Sammlung der Universität zu Uppsala, die Mineralien-Sammlung der Akademie der Wissenschaften zu München, Professor Pfaff zu Kiel und die Universitäts-Sammlung in Parma 9; das Joanneum zu Grätz und die Sammlungen der Bergakademie zu Freiberg 8; die Mineralien-Sammlung der Universität zu Wilna, jetzt in Kiew 7; die Mineralien-Sammlung der vaterländischen Museen zu Prag und Pesth, dann die der Prager Universität jede 6; u. s. w. Die Meteoriten-Sammlung, die von Herrn Heinrich Heuland in London zusammengebracht später Eigentum des Herrn Heath zu Madras in Ostindien wurde, zählte 43 Lokalitäten von echten Meteoriten. Nach Europa zurückgebracht, wurden sie im Jahre 1837 von Herrn Carl Pötschke in Wien angekauft und daselbst vereinzelt.} Wer würde denn nicht mit ungewöhnlichem Interesse eine so große Anzahl jener rätselhaften Ankömmlinge von Außen hier vereiniget betrachten ? diese aus dem großen Weltraume oberhalb unserer Atmosphäre stammenden Massen (entweder fest gewordene kosmische Materie, oder Stücke eines zersprungenen Planeten), daher vom Himmel gefallene Steine und Eisenmassen genannt, Aerolithen oder Luftsteine von denjenigen, die ihre Entstehung in unserer Atmosphäre suchen, Mondsteine von denen, die sie durch Vulkane oder elektrische Entladungen aus diesem Erdtrabanten ausschleudern lassen, gewöhnlich aber Meteorsteine und Meteoreisen, oder mit einem gemeinschaftlichen Nahmen Meteoriten genannt, weil sie am Himmel als Meteore oder Feuerkugeln erscheinen, aus welchen, unter heftigem Schall und Geprassel, jene Massen, meist Steine, seltener wunderbare Eisenblöcke, noch heiß und nach Schwefel riechend, auf die Erde niederstürzen. Das schon seit den ältesten Zeiten beobachtete Niederfallen dieser Massen auf unseren Planeten hat von jeher den größten Eindruck auf das menschliche Gemüt gemacht, daher mehrere Völker des Altertums, Phönizier, Griechen, Römer u. a. m., die sie heilige Steine oder Bätylien nannten, ihnen, zumal als Symbol der Mutter der Götter, abergläubische Verehrung bezeigten, und dieselben, wie uns alte Geschichtsschreiber und antike Münzen lehren, in Tempeln aufbewahrten und in Triumphzügen herumführten.\footnote{Münter: Über die vom Himmel gefallenen Steine, Bätylien genannt, Kopenhagen und Leipzig 1805. 8, (auch in Gilberts Annalen der Physik, B. 21, S. 51-84, unter dem Titel: Vergleichung der Bätylien der Alten mit den Steinen, welche in neueren Zeiten vom Himmel gefallen sind.) - Von Dalberg: Über Meteor-Cultus der Alten, vorzüglich in Bezug auf Steine, die vom Himmel gefallen. Heidelberg 1811. 8.} Obwohl das Ereignis des Niederfallens durch mehrere Dezennien des vorigen Jahrhunderts bezweifelt, ja hartnäckig geleugnet, die daran Glaubenden verspoltet und verlacht wurden, so hat dieser Gegenstand seit dem berühmten Steinregen von L’Aigle in der Normandie am 26. April 1803, den das französische National-Institut durch sein Mitglied, den bekannten Physiker, Herrn Biot, untersuchen ließ, in neuerer Zeit doch so viel allgemeine Aufmerksamkeit erregt, und so verschiedene Untersuchungen und Beleuchtungen von Seite der Gelehrten zur Folge gehabt, dass jeder Gebildete, namentlich seit dem Erscheinen der verdienstvollen Schriften von Izarn\footnote{Des pierres tombées du ciel ou Lithologie atmosphérique. Paris 1803. 8.} und Bigot de Morogues,\footnote{Mémoire historique et physique sur les chutes des pierres tombées sur la surface de la terre a diverses époques. Orléans 1812. 8.} vorzüglich aber durch die klassischen Arbeiten von Howard,\footnote{Experiments and Observations on certain stony and metalline Substances, wich at different Times are said to have fallen on the Earth, also on various Kinds of native Iron, in den Philos. Transact. of the Roy. Soc. of London for 1802. Part 1. S. 168; deutsch in Gilberts Annalen der Physik, B. 13, S. 291, unter dem Titel: Versuche und Bemerkungen über Stein- und Metallmassen, die zu verschiedenen Zeiten auf die Erde gefallen sein sollen, und über die gediegenen Eisenmassen.} Chladni\footnote{Über Feuer-Meteore und über die mit denselben herabgefallenen Massen. Wien 1819, im Verlage bei J. G. Heubner. 8. Nebst vielen Aufsätzen in Gilberts Annalen.} und Karl von Schreibers\footnote{Nachrichten von dem Steinregen zu Stannern in Mähren, in Gilberts Annalen der Physik. B. 29. 1808. S. 225. - Beiträge zur Geschichte und Kenntnis meteorischer Stein- und Metallmassen und der Erscheinungen, welche deren Niederfallen zu begleiten pflegen, Wien 1820, im Verlage von J. G. Heubner, Folio. Mit Abbildungen. - Über den Meteorstein-Niederfall auf der Herrschaft Wessely in Mähren, in Baumgartners Zeitschrift für Physik und verwandte Wissenschaften. B. 1. 1832. S. 193-256.} wenigstens mit den Tatsachen des Phänomens, wenn auch nicht über die Herkunft dieser merkwürdigen Massen, die uns nie völlig klar werden, und immer Gegenstand mehr oder weniger gewagter Theorien bleiben wird, im Reinen ist. Die erwähnten wissenschaftlichen Untersuchungen haben jedoch in der naturhistorischen Betrachtung der Meteorsteine und Meteoreisenmassen, zu denen die Arbeiten des Herrn von Schreibers, des verdienstvollen Gründers unserer Meteoriten-Sammlung, und die technischen Untersuchungen einiger Meteoreisenmassen durch Herrn von Widmannstätten den Grund legten, ungeachtet der schönen Beiträge, welche die Herren Gustav Rose\footnote{Poggendorffs Annalen der Physik und Chemie. B. 4. S. 173.} und Cordier,\footnote{Annales de Chemie et de Physique. T. 34. pag. 132.} vorzüglich aber Berzelius\footnote{Poggendorffs Annalen. Bd. 33. S. 1 und 113 auch Jahresbericht über die Fortschritte der physischen Wissenschaften 15. Jahrgang S. 227.} dazu in neuerer Zeit lieferten, noch große Lücken in der genauen naturhistorischen Kenntnis dieser rätselhaften Körper gelassen. Die Ursache mag darin liegen, dass nur wenige bedeutende Sammlungen von Meteoriten bestehen, und in diesen wenigen diese kostbaren Produkte nicht in jenem Zustande vorhanden sind, der zu einer genauen Untersuchung und Kenntnis dieser, gleich den Gebirgsarten gemengten Massen unumgänglich notwendig ist; nämlich in einem durch künstliche Zubereitung entstehenden Zustand, der ihr Inneres aufschließt, und ihre wahre Beschaffenheit erst kennen lehrt. Wir meinen die Anfertigung von gut polierten Schnittflächen bei Meteorsteinen; von fein polierten Schnittflächen, die sonst keine andere Veränderung zu erleiden brauchen, dann von polierten Flächen, die weiter entweder durch Hitze-Einwirkung blau, violett oder rot anlaufen gemacht, oder durch Anwendung von metallischen Säuren (Salz- oder Salpetersäure) mehr oder weniger stark geätzt worden sind, bei Meteoreisenmassen. Da dieses mit vieler Mühe und großem Zeitaufwande, mit nicht unbedeutenden Kosten und nicht geringer Verminderung des Volums und Gewichts der so wertvollen Meteoriten in der Sammlung des k. k. Mineralien-Kabinettes ausgeführt worden ist (die Ätzung der Eisenmassen meist von Herrn von Widmannstätten, dem Entdecker der nach ihm benannten merkwürdigen Figuren), so bietet sie ganz allein unter allen bestehenden Meteoriten-Sammlungen Gelegenheit dar, die Eigenschaften, den Charakter und die Verwandtschaften der Meteoriten vollständig ins Klare zu bringen. Dieser Umstand hat uns bestimmt, dieselben nach den einzelnen Lokalitäten mit kurzen Beschreibungen oder Diagnosen zu versehen, durch die Darstellung ihrer Anordnung und ihrer Reihenfolge und eine angehängte Verwandtschaftstabelle die Ähnlichkeiten und Verschiedenheiten, die sie darbieten (wovon die ersteren im Allgemeinen geringer, die anderen viel grösser sind, als sich mancher Mineraloge vorstellt), zu zeigen, ohne dabei jedoch in eine mikroskopische Untersuchung der Meteorsteine einzugehen, die besseren Augen vorbehalten bleibt und wozu einer der ausgezeichnetsten hiesigen Gelehrten, selbst im Besitze einer der bedeutendsten Meteoriten-Sammlungen und, was bei derlei Untersuchungen fast unumgänglich notwendig ist, zugleich Chemiker, bereits zahlreiche Materialien gesammelt hat, deren baldige Bekanntmachung zu wünschen ist. Wir haben somit, soweit es der Hauptzweck dieses beschreibenden Verzeichnisses gestattete (das übrigens mit Ausschluss der Tabellen, Anmerkungen, Zusätze u. s. w. größtenteils ein Abdruck des von uns verfassten amtlichen Kabinetts-Kataloges ist) bei der Herausgabe desselben gestrebt, zugleich einen wissenschaftlichen Beitrag zur Kenntnis der Meteoriten zu geben, In dieser Absicht haben wir auch am Schlusse eine Tabelle über die spezifischen Gewichte sämtlicher im k. k. Mineralien-Kabinette aufbewahrter Meteoriten beigefügt. Die Wiegungen hat der Kustos-Adjunkt an diesem Kabinette Herr Karl Rumler mit aller Sorgfalt bei einer Temperatur von 140 R. ausgeführt, und es wurden dieser Tabelle auch alle anderen in verschiedenen Werken und Abhandlungen zerstreuten Angaben der spezifischen Gewichte von Meteoriten und auch einige noch nicht veröffentlichte beigefügt. Die historischen Beigaben und erläuternden wissenschaftlichen Anmerkungen werden Wissenschaftsfreunden in diesen Blättern vielleicht ebenfalls nicht unwillkommen sein. Noch manches Material (worunter schön ausgeführte Zeichnungen von sämtlichen durch Ätzen bei den verschiedenen Meteoreisenmassen zum Vorschein kommenden Figuren), liegt zur Bekanntmachung bereit, und wird, falls die Annalen des Wiener Museums der Naturgeschichte wieder aufleben sollten, dem Publikum vorgelegt werden. Möge dasjenige, was wir hier bieten, ein freundliches Andenken denjenigen sein, die Gelegenheit haben, die Meteoriten-Sammlung des k. k. Mineralien-Kabinettes zu sehen und Anderen, namentlich Eigentümern oder Vorstehern von Mineralien-Sammlungen, Besitzern von einzelnen Meteoriten u. s. w. Veranlassung werden, der Sammlung des k. k. Mineralien-Kabinettes im Interesse der Wissenschaft Bereicherungen an Meteoriten zukommen zu lassen. Für eine bereits so reiche Sammlung ist jede neue Lokalität ein hochanzuschlagender Gewinn, und daher dem Geber (nebst der Gegengabe von anderen Meteoriten oder Mineralien, wenn es gewünscht wird), der vollste Dank gesichert.

Wien, den 23. Februar 1843.
\clearpage
\section{Übersicht der Meteoriten im k. k. Mineralien-Kabinette nach der Reihenfolge ihrer Aufstellung.}
\begin{center}
\small
(Die Nummern dienen zur Erleichterung des Aufsuchens im vorliegenden Kataloge.)
\end{center}
\subsection{Meteorsteine.}
\begin{enumerate}
    \small
    \item Alais (St. Etienne de Lolm und Valence).
    \item Simonod
    \item Kapland (Bokkeveld).
    \item Chassigny (Langres).
    \item Juvenas.
    \item Stannern.
    \item Konstantinopel.
    \item Jonzac.
    \item Bialistock.
    \item Lontalax.
    \item Nobleborough (Nobleboro, Maine).
    \item Mässing (Eggenfelden).
    \item Parma (Casignano).
    \item Siena.
    \item Ensisheim.
    \item L'Aigle.
    \item Liponas.
    \item Chantonnay.
    \item Renazzo (Ferrara).
    \item Richmond (Virginien).
    \item Weston (Connecticut).
    \item La Baffe (Épinal).
    \item Benares (Krakhut).
    \item Gouv. Poltawa.
    \item Krasno-Ugol.
    \item Erxleben.
    \item Gouv. Simbirsk.
    \item Mauerkirchen.
    \item Nashville (Tennessee).
    \item Lucé.
    \item Lissa.
    \item Owahu (Hanaruru).
    \item Charkow (Ukraine).
    \item Zaborzika.
    \item Bachmut.
    \item Politz (Köstriz).
    \item Kuleschofka.
    \item Slobodka.
    \item Milena.
    \item Forsyth (Georgien).
    \item Yorkshire (Wold-Cottage).
    \item Glasgow (High Possil).
    \item Berlanguillas (Burgos).
    \item Apt (Saurette).
    \item Vouillé (Poitiers).
    \item Château-Renard (Triguères).
    \item Salés (Villefranche).
    \item Agen.
    \item Nanjemoy (Maryland).
    \item Asco.
    \item Toulouse.
    \item Blansko.
    \item Wessely.
    \item Limerick (Adair).
    \item Grüneberg (Heinrichau).
    \item Tipperary (Mooresfort).
    \item Gouv. Kursk.
    \item Lixna (Dünaburg).
    \item Tabor (Plan).
    \item Charsonville (Orléans).
    \item Doroninsk.
    \item Seres (Makedonien).
    \item Sigena (Sena).
    \item Barbotan (Roquefort, Créon Juillac).
    \item Eichstädt (Wittens).
    \item Groß-Divina (Budetin).
    \item Zebrak (Horzowitz).
    \item Timochin (Smolensk).
    \item Macao (Rio Assu).
\end{enumerate}
\subsection{Meteoreisen.}
\begin{enumerate}
    \small
    \item Atacama.
    \item Krasnojarsk (Sibirien, Pallas).
    \item Brahin.
    \item Sachsen (Steinbach oder Grimma ? mit dem Eisen, angeblich aus Norwegen).
    \item Bitburg.
    \item Toluca (Xiquipilco).
    \item Elbogen.
    \item Agram (Hraschina).
    \item Lenarto.
    \item Red-River (Louisiana oder Texas).
    \item Durango.
    \item Guilford.
    \item Caille (Grasse).
    \item Ashville (Buncombe).
    \item Tennessee.
    \item Bohumilitz.
    \item Bahia (Bemdegò).
    \item Zacatecas.
    \item Rasgatà.
    \item Tucuman (Otumpa).
    \item Senegal.
    \item Kap der guten Hoffnung.
    \item Clairborne (Alabama).
\end{enumerate}
\subsection{Anhang.}
\begin{enumerate}
    \small
    \item Oaxaca.
    \item Grönland (Baffingsbay).
\end{enumerate}
\clearpage
\section{Übersicht der Meteoriten im k. k. Mineralien-Kabinette, nach den Fall- oder Fundorten.}
\begin{center}
\small
(Die Nummern beziehen sich auf die Reihenfolge in der Übersicht Nr. 1., und dienen zur Erleichterung des Aufsuchens im vorliegenden Kataloge.)
\end{center}
\subsection{Meteorsteine.}
\subsubsection{Europa}
\begin{center}
Frankreich.
\end{center}
\begin{itemize}
    \small
    \item[48.] Agen, Dépt. Lot et Garonne.
    \item[1.] Alais, Dépt. du Gard.
    \item[44.] Apt, Dépt. de Vaucluse.
    \item[50.] Asco, Insel Korsika.
    \item[64.] Barbotan (und Roquefort) ehemals Gascogne, Dépt. du Gers (und Dépt. des Landes).
    \item[18.] Chantonnay, Dépt. de la Vendée.
    \item[60.] Charsonville, Dépt. du Loiret.
    \item[4.] Chassigny, Dépt. de la haute Marne.
    \item[46.] Château-Renard, Dépt. du Loiret.
    \item[15.] Ensisheim, ehemals Elsass, jetzt Dépt. du Haut-Rhin.
    \item[8.] Jonzac, Dépt. de la Charente inferieure.
    \item[5.] Juvenas, Dépt. de l'Ardeche.
    \item[22.] La Baffe, Dépt. des Vosges.
    \item[16.] L'Aigle, ehemals Normandie, Dépt. de l'Orne.
    \item[17.] Liponas, Dépt. de l'Ain.
    \item[30.] Lucé, Dépt. de la Sarthe.
    \item[47.] Salés, Dépt. du Rhone.
    \item[2.] Simonod, Dépt. de l'Ain.
    \item[51] Toulouse, Dépt. de la Haute-Garonne.
    \item[45] Vouillé, Dépt. de la Vienne.
\end{itemize}
\begin{center}
England.
\end{center}
\begin{itemize}
    \small
    \item[41.] Wold-Cottage, Yorkshire.
\end{itemize}
\begin{center}
Schottland.
\end{center}
\begin{itemize}
    \small
    \item[42.] High-Possil, Glasgow.
\end{itemize}
\begin{center}
Irland.
\end{center}
\begin{itemize}
    \small
    \item[56.] Mooresfort, Grafschaft Tipperary.
    \item[54.] Adair, Grafschaft Limerick.
\end{itemize}
\begin{center}
Spanien.
\end{center}
\begin{itemize}
    \small
    \item[43.] Berlanguillas, Alt-Kastilien.
    \item[63.] Sigena, Aragonien.
\end{itemize}
\begin{center}
Italien.
\end{center}
\begin{itemize}
    \small
    \item[13.] Casignano, Herzogtum Parma.
    \item[19.] Renazzo, Provinz Ferrara, Kirchenstaat.
    \item[14.] Siena, Toskana.
\end{itemize}
\begin{center}
Deutschland.
\end{center}
\begin{itemize}
    \small
    \item[59.] Tabor, ehemals Bechiner, jetzt Taborer Kreis, Böhmen.
    \item[67.] Zebrak, Berauner Kreis, Böhmen.
    \item[31.] Lissa, Bunzlauer Kreis, Böhmen.
    \item[6.] Stannern, Iglauer Kreis, Mähren.
    \item[52.] Blansko, Brünner Kreis, Mähren.
    \item[53.] Wessely, Hradischer Kreis, Mähren.
    \item[28.] Mauerkirchen, ehemals Bayern, jetzt Inn-Kreis, Ober-Österreich.
    \item[12.] Mässing, Unter-Donau-Kreis, Niederbaiern.
    \item[65.] Eichstädt, Regenkreis, Franken, Baiern.
    \item[36.] Politz bei Gera, Fürstentum Reuß.
    \item[26.] Erxleben, Regierungsbezirk Magdeburg, preußische Provinz Sachsen.
    \item[55.] Grüneberg, Regierungsbezirk Liegnitz, Provinz Schlesien.
\end{itemize}
\begin{center}
Ungarn.
\end{center}
\begin{itemize}
    \small
    \item[66.] Groß-Divina, Trentschiner-Komitat.
\end{itemize}
\begin{center}
Kroatien.
\end{center}
\begin{itemize}
    \small
    \item[39.] Milena, Warasdiner-Komitat.
\end{itemize}
\begin{center}
Russland.
\end{center}
\begin{itemize}
    \small
    \item[35.] Bachmut, Gouv. Ekaterinoslaw.
    \item[9.] Bialistock, gleichnamige Provinz.
    \item[33.] Charkow, gleichnamiges Gouvernement.
    \item[25.] Krasno-Ugol, Gouv. Räsan.
    \item[37.] Kuleschofka, Gouv. Poltawa.
    \item[57.] Kursk (Gouv.)
    \item[58.] Lixna, Dünaburger Kreis, Gouv. Witepsk.
    \item[10.] Lontalax, Finnland.
    \item[24.] Poltawa (Gouv.)
    \item[27.] Simbirsk (Gouv.)
    \item[38.] Slobodka, Gouv. Smolensk.
    \item[68.] Timochin, Gouv. Smolensk.
    \item[34.] Zaborzika, Gouv. Wolhynien.
\end{itemize}
\begin{center}
Türkei.
\end{center}
\begin{itemize}
    \small
    \item[7.] Konstantinopel.
    \item[62.] Seres, Makedonien.
\end{itemize}
\subsubsection{Asien.}
\begin{itemize}
    \small
    \item[61.] Doroninsk, Gouv. Irkutsk, Sibirien.
    \item[23.] Benares, Bengalen, Ostindien.
\end{itemize}
\subsubsection{Afrika.}
\begin{itemize}
    \small
    \item[3.] Kapland (Bokkeveld bei Tulpagh).
\end{itemize}
\subsubsection{Amerika.}
\begin{itemize}
    \small
    \item[11.] Nobleborough, Maine, Vereinigte Staaten von Nord-Amerika.
    \item[21.] Weston, Connecticut, Vereinigte Staaten von Nord-Amerika.
    \item[49.] Nanjemoy, Maryland, Vereinigte Staaten von Nord-Amerika.
    \item[20.] Richmond, Virginien, Vereinigte Staaten von Nord-Amerika.
    \item[29.] Nashville, Tennessee, Vereinigte Staaten von Nord-Amerika.
    \item[40.] Forsyth, Georgien, Vereinigte Staaten von Nord-Amerika.
    \item[69.] Macao, Provinz Rio grande do Norte, Brasilien.
\end{itemize}
\subsubsection{Australien.}
\begin{itemize}
    \small
    \item[32.] Owahu, eine der Sandwich-Inseln.
\end{itemize}
\subsection{Meteoreisen.}
\subsubsection{Europa}
\begin{center}
Frankreich.
\end{center}
\begin{itemize}
    \small
    \item[82.] Caille, Dépt. du Var.
\end{itemize}
\begin{center}
Deutschland.
\end{center}
\begin{itemize}
    \small
    \item[76.] Elbogen, Elbogner Kreis, Böhmen.
    \item[85.] Bohumilitz, Prachiner Kreis, Böhmen.
    \item[73.] Sachsen (Steinbach bei Eibenstock im Erzgebirgischen Kreise oder Grimma ? im Leipziger Kreise).
    \item[74.] Bitburg, Regierungsbezirk Trier, Rheinpreußen.
\end{itemize}
\begin{center}
Ungarn.
\end{center}
\begin{itemize}
    \small
    \item[78.] Lenarto, Saroscher Komitat.
\end{itemize}
\begin{center}
Kroatien.
\end{center}
\begin{itemize}
    \small
    \item[77.] Agram, Agramer Komitat.
\end{itemize}
\begin{center}
Russland.
\end{center}
\begin{itemize}
    \small
    \item[72.] Brahin, Gouv. Minsk, ehemals Litauen.
\end{itemize}
\subsubsection{Asien.}
\begin{center}
Sibirien.
\end{center}
\begin{itemize}
    \small
    \item[71.] Krasnojarsk, Gouv. Jeniseisk.
\end{itemize}
\subsubsection{Afrika.}
\begin{itemize}
    \small
    \item[90.] Senegambien (am oberen Teil des Senegalstromes).
    \item[91.] Kap der guten Hoffnung (zwischen dem Sonntags- und Boschesmannsflüsse).
\end{itemize}
\subsubsection{Amerika.}
\begin{itemize}
    \small
    \item[94.] Grönland (Baffingsbay)
\end{itemize}
\begin{center}
Vereinigte Staaten von Nord-Amerika.
\end{center}
\begin{itemize}
    \small
    \item[84.] Tennessee. (Cocke-County in Staate Tennessee).
    \item[83.] Ashville, Nord-Carolina.
    \item[81.] Guilford, Nord-Carolina.
    \item[92.] Clairborne, Staat Alabama.
    \item[79.] Louisiana oder Texas ? (am Red-River oder roten Flüsse). 
\end{itemize}
\begin{center}
Vereinigte Mexikanische Bundesstaaten.
\end{center}
\begin{itemize}
    \small
    \item[80.] Durango, im gleichnamigen Staate.
    \item[87.] Zacatecas, im gleichnamigen Staate.
    \item[75.] Toluca, (Xiquipilio, im Staate Mexiko).
    \item[93.] Oaxaca, (in der Misteca, im Staate Oaxaca).
\end{itemize}
\begin{center}
Columbien. (Neu-Granada.)
\end{center}
\begin{itemize}
    \small
    \item[88.] Rasgatà, nordöstlich von Santa Fe de Bogotá.
\end{itemize}
\begin{center}
Bolivia. (ehemals Peru.)
\end{center}
\begin{itemize}
    \small
    \item[70.] Atacama. (Wüste Atacama, an der Grenze von Chili).
\end{itemize}
\begin{center}
Brasilien.
\end{center}
\begin{itemize}
    \small
    \item[86.] Bahia, am Bache Bemdegò bei Monte Santo, Capitanie Bahia.
\end{itemize}
\begin{center}
Vereinigte Staaten am Rio de la Plata.
\end{center}
\begin{itemize}
    \small
    \item[89.] Tucuman. (Otumpa, im Staate Tucuman.)
\end{itemize}
\clearpage
\section{Übersicht der Meteoriten im k. k. Mineralien-Kabinette, nach der Zeitfolge ihres Niederfallens.}
\begin{center}
\small
(Die Nummern beziehen sich auf die Reihenfolge in der Übersicht Nr. 1, und dienen zur Erleichterung des Aufsuchens im vorliegenden Kataloge.)
\end{center}
\begin{center}
    \footnotesize
    \begin{longtable}{|p{6mm}|p{9mm}|p{60mm}|p{27mm}|}
    \hline
        Nr. & Jahr & Monat und Tag &   \\ \hline
        ~ & ~ & ~ & \textbf{1. Meteorsteine.} \\ \hline
        15 & 1492 & 7. November & Ensisheim. \\ \hline
        59 & 1753 & 3. Juli & Tabor. \\ \hline
        17 & 1753 & September & Liponas. \\ \hline
        30 & 1768 & 13. September & Lucé. \\ \hline
        28 & 1768 & 20. November & Mauerkirchen. \\ \hline
        63 & 1773 & 17. November & Sigena. \\ \hline
        65 & 1785 & 19. Februar & Eichstädt. \\ \hline
        33 & 1787 & 1. Oktober & Charkow. \\ \hline
        64 & 1790 & 24. Juli & Barbotan. \\ \hline
        14 & 1794 & 16. Juni & Siena. \\ \hline
        41 & 1795 & 13. Dezember & Yorkshire. \\ \hline
        47 & 1798 & 8. oder 12. Marz & Salés. \\ \hline
        23 & 1798 & 13. Dezember & Benares. \\ \hline
        16 & 1803 & 6. April & L’Aigle \\ \hline
        44 & 1803 & 8. Oktober & Apt. \\ \hline
        12 & 1803 & 13. Dezember & Massing \\ \hline
        42 & 1804 & 5. April & Glasgow \\ \hline
        61 & 1805 & 25. Marz & Doroninsk \\ \hline
        7 & 1805 & Juni & Konstantinopel \\ \hline
        50 & 1805 & November & Asco. \\ \hline
        1 & 1806 & 15. Marz & Alais. \\ \hline
        68 & 1807 & 13. Marz & Timochin. \\ \hline
        21 & 1807 & 14. Dezember & Weston. \\ \hline
        13 & 1808 & 19. April & Parma. \\ \hline
        6 & 1808 & 22. Mai & Stannern. \\ \hline
        31 & 1808 & 3. September & Lissa. \\ \hline
        56 & 1810 & August & Tipperary. \\ \hline
        60 & 1810 & 23. November & Charsonville. \\ \hline
        37 & 1811 & zwischen d. 12. u. 13. Marz um Mitternacht & Kuleschofka. \\ \hline
        43 & 1811 & 8. Juli & Berlanguillas. \\ \hline
        51 & 1812 & 12. April & Toulouse. \\ \hline
        26 & 1812 & 15. April & Erxleben. \\ \hline
        18 & 1812 & 5. August & Chantonnay. \\ \hline
        54 & 1813 & 10. September & Limerick. \\ \hline
        10 & 1813 & 13. Dezember & Lontalax. \\ \hline
        35 & 1814 & 3. Februar & Bachmut. \\ \hline
        48 & 1814 & 5. September & Agen. \\ \hline
        4 & 1815 & 3. Oktober & Chassigny. \\ \hline
        34 & 1818 & 30. Marz & Zaborzika. \\ \hline
        62 & 1818 & Juni & Seres. \\ \hline
        38 & 1818 & 10. August & Slobodka. \\ \hline
        8 & 1819 & 13. Juni & Jonzac. \\ \hline
        36 & 1819 & 13. Oktober & Poliz. \\ \hline
        58 & 1820 & 12. Juli & Lixna. \\ \hline
        5 & 1821 & 15. Juni & Juvenas. \\ \hline
        22 & 1822 & 13. September & La Baffe. \\ \hline
        11 & 1823 & 7. August & Nobleborough. \\ \hline
        19 & 1824 & 15. Januar & Renazzo. \\ \hline
        67 & 1824 & 14. Oktober & Zebrak. \\ \hline
        49 & 1825 & 10. Februar & Nanjemoy. \\ \hline
        32 & 1825 & 14. September & Owahu. \\ \hline
        29 & 1827 & 9. Mai & Nashville. \\ \hline
        9 & 1827 & 5. oder 6. Oktober & Bialistock. \\ \hline
        20 & 1828 & 4. Juni & Richmond. \\ \hline
        40 & 1829 & 8. Mai & Forsyth. \\ \hline
        25 & 1829 & 9. September & Krasno-Ugol. \\ \hline
        45 & 1831 & 18. Juli (nach anderen Angaben 13. Mai) & Vouillé. \\ \hline
        53 & 1831 & 9. September & Wessely. \\ \hline
        52 & 1833 & 25. November & Blansko. \\ \hline
        2 & 1835 & 13. November & Simonod. \\ \hline
        69 & 1836 & 11. November (nach anderen Angaben 11. Dezember & Macao. \\ \hline
        66 & 1837 & 24. Juli & Groß-Divina. \\ \hline
        3 & 1838 & 13. Oktober & Kapland. \\ \hline
        55 & 1841 & 22. Marz & Grüneberg. \\ \hline
        46 & 1841 & 12. Juni & Château-Renard. \\ \hline
        39 & 1842 & 26. April & Milena. \\ \hline
        24 & ~ & Die Fallzeit unbekannt. & Gouv. Poltawa. \\ \hline
        57 & ~ & Die Fallzeit unbekannt. & Gouv. Kursk. \\ \hline
        27 & ~ & Die Fallzeit unbekannt. & Gouv. Simbirsk. \\ \hline
        ~ & ~ & ~ & \textbf{2. Meteoreisen.} \\ \hline
        77 & 1751 & 26. Mai & Agram. \\ \hline
        70 bis 76 & ~ & Die Fallzeit unbekannt. & Alle andern Eisenmassen. \\ \hline
        76 & ~ & Die Fallzeit unbekannt. & Alle andern Eisenmassen. \\ \hline
        78 bis 94 & ~ & Die Fallzeit unbekannt. & Alle andern Eisenmassen. \\ \hline
        94 & ~ & Die Fallzeit unbekannt. & Alle andern Eisenmassen. \\ \hline
    \end{longtable}
\end{center}
\clearpage
\section{Wegweiser.}
\paragraph{}
Die Meteoriten-Sammlung des k. k. Mineralien-Kabinettes ist in einem langen pultförmigen Glasschrank, mit nach zwei Seiten abfallenden Glaswänden, in der Mitte des vierten Saales aufgestellt. Auf der waagerechten Ebene des Glasschrankes erheben sich, nach der Länge desselben ziehend, jedoch beiderseits noch Raum lassend, drei breite niedere Stufen, wodurch im Ganzen fünf Abteilungen entstehen. Die obere oder zweite, beiden Seiten des Pult-Schrankes gemeinschaftliche Stufe, (mit Abteilung Nr. 1 bezeichnet) enthält die größten Stücke, deren Volum eine systematische Einreihung unter die anderen nicht erlaubte, nämlich die zwei berühmten großen Eisenmassen von Elbogen und Agram, große Stücke der Eisenmassen von Atacama, Lenarto, Bohumilitz, Bahia und Krasnojarsk, einen großen ganzen Meteorstein von Tabor, einen solchen von Wessely, und einen von Lissa, drei große ganze Steine von Stannern, ein großes Fragment des Steines von Chantonnay und zwei große ganze Steine von L’Aigle (letztere zwei auf der Rückseite des Schrankes). Die Reihenfolge der nach ihren Verwandtschaften zusammengestellten Meteoriten kleineren Formates beginnt in der vorderen, gegen den dritten Saal des Mineralien-Kabinettes gekehrten Hälfte des Schrankes; hier sind auf der untersten, mit Nr. 2 bezeichneten Abteilung, auf der Ebene des Schrankes, unterhalb der ersten Stufe die Meteorsteine, welche kein gediegenes Eisen enthalten (Nr. 1 bis 12 der Tabelle Nr. 1.) aufgestellt; von da wendet sich die Reihe auf die Rückseite des Glasschrankes, der auf der ersten Stufe (Abteilung Nr. 3) und in der Abteilung unterhalb derselben (Abteilung Nr.4) auf einem ausgedehnten Raume die anderen, weit zahlreicheren Meteorsteine, welche gediegenes Eisen einschließen (von Nr. 13 bis 69 der Tabelle Nr. 1.) enthält. Die Reihe springt von der Abteilung Nr. 4 nun wieder auf die Vorderseite des Glasschrankes, wo die erste Stufe, mit Abteilung Nr.5 bezeichnet, die kleineren Stücke von Meteoreisen trägt; Anfangs die ästigen mit Olivin (von Nr. 70 bis 73), darauf die derben oder formlosen (von Nr. 74 bis 94), womit die Sammlung endet. — Alle Stücke liegen auf ovalen, weiß lackierten, mit goldenen Leisten gezierten Untersätzen von verschiedener Größe und Höhe, auf welchen eine Etiquette den Namen der Lokalität, das Falljahr, und wenn (wie bei allen Eisenmassen, mit alleiniger Ausnahme der Agramer) die Fallzeit nicht bekannt ist, die Zeit ihrer Auffindung oder Bekanntwerdung angibt. Die bei jeder Lokalität mit Nr. 1 beginnenden Nummern auf den Untersätzen beziehen sich auf die Beschreibung der Lokalität, sowohl in dem Kabinetts- als dem vorliegenden gedruckten Kataloge.
\clearpage
\section{Meteorsteine.}
\begin{center}
Nr. 1 bis 69.
\end{center}
\subsection{Alais.}
\begin{center}
\small
St. Etienne de Lolm und Valence, Dépt. du Gard, Frankreich.

15. Mai 1806, 5 Uhr Abends.
\end{center}
\paragraph{}
Bräunlich schwarze, teils bröckliche und zerreibliche, teils (durch Zerreibung entstandene) pulverige Substanz, hie und da mit weißen Salz-Effloreszierungen (nach Berzelius: Bittersalz mit Nickelvitriol), in welcher selbst mittelst der Lupe weder kugelige Ausscheidungen, noch gediegenes Eisen und Magnetkies (die jedoch den Analysen zufolge in sehr kleiner Menge vorhanden sind), unterschieden werden können.

1. Größere und kleinere Bröckchen, mit Pulver vermischt und, bis auf zwei, ohne Rindensubstanz; von einem der zwei allda gefallenen, und alsbald zerbröckelten Steine, die zusammen 12 Pfund wogen. — Etwas über $\frac{3}{32}$ Loth oder $25\frac{1}{2}$ Gran — 1816. 35. 44, und 1838. 27. 2.\footnote{Die hier und bei allen anderen Lokalitäten von Meteoriten befindlichen Zahlen bedeuten das Jahr und die Nummer des Acquisitions-Postens, dann die Nummer des Stückes in dem respektiven Acquisitions-Posten der Kabinetts-Kataloge.} — Teils aus der Mineralien-Sammlung des Marquis de Drée in Paris durch den Direktor der vereinigten k. k. Hof-Naturalien-Kabinette, Karl von Schreibers, in Tausch erhalten, teils von Herrn Gubernialrat Neumann in Prag eingetauscht.
\subsection{Simonod.}
\begin{center}
\small
Gemeinde Belmont, Arrondissement Belley, Dép. de l’Ain, Frankreich.

13. November 1835, 9 Uhr Abends.
\end{center}
\paragraph{}
1. Kleine, eckige und scharfkantige Fragmentchen, samtschwarz, schwach glänzend, von Fettglanz, spröde, schwer zerreiblich, vollkommen homogen aussehend; von einem der zwei allda gefallenen etwa eigroßen Steine, die wohl bald in kleine Fragmente zerfallen sind. — $\frac{3}{32}$ Loth und 4 Gran. — 1840. 28. 1. — Von Herrn Marquis de Drée in Paris in Tausch erhalten. Marquis de Drée erhielt die Substanz durch einen Gendarmerie-Beamten des Dép. de l'Ain.

Ob die Fragmentchen von Simonod oder Belley wirklich einer mit Detonation zersprungenen Feuerkugel, die einen wahren, überrindeten Meteorstein gab, angehören, oder Produkt einer Sternschnuppe sind, ist noch zweifelhaft. Die Nacht des Falles war eine der Sternschnuppen-Nächte. Herr Millet d’Aubenton berichtete Herrn Arago, dass er zu der oben angegebenen Zeit ein Feuermeteor beobachtete, welches in der Gemeinde Belmont zersprang, und zwar über Häusern und Strohdächern, die es entzündete. Derselbe will auch zwei eigroße Stücke gefunden haben, die ganz die Beschaffenheit eines Aerolithen besaßen. — Später hat Herr Millet Stücke davon der Pariser Akademie übersendet. Er schrieb dabei, dass sie im Allgemeinen das Ansehen von Obsidian haben (was ganz richtig ist), dass der Magnet kleine Metallkügelchen davon ausziehe, bestehend aus Eisen, Schwefel, Kupfer, Arsenik und vielleicht Silber?! (was wir in unseren Fragmentchen nicht finden konnten). Er glaubte auch Spuren von Nickel und Chrom darin gefunden zu haben, Die eingesendeten Stücke sind von der Pariser Akademie Hrn. Dumas zur Analyse übergeben worden. (Siehe Poggendorffs Annalen B. 36. S. 562 und Bd. 37. S. 460.) — Nach einer Mittheilung, die wir Herrn Marquis de Drée verdanken, fand Herr Damour darin Kieselerde, Eisenoxyd, Kupferoxyd, Schwefel, Kohle und Kalk. — Merkwürdig ist das spezifische Gewicht dieser Fragmente‚ nämlich 1,35. (nach einer Wiegung von Herren Rumler) das geringste von allen bekannten Meteorsteinen.
\subsection{Kapland.}
\begin{center}
\small
Bokkeveld bei Tulpagh, 70 englische Meilen von der Kapstadt, am Vorgebirge der guten Hoffnung in Afrika.

13. Oktober 1838, 9 Uhr Morgens.
\end{center}
\paragraph{}
In die schwarze, matte, durch den Strich Glanz erlangende, weiche und milde Grundmasse sind weißliche und grünliche, undeutliche Körner (die wie Flecken aussehen und wenig Körper zu haben scheinen) eingemengt; gediegen Eisen und Schwefelkies sind nicht sichtbar. — Ein höchst eigentümlicher Meteorstein.

Fragment mit etwas Rinde; von einem großen, einzeln gefallenen Steine von einigen Zentnern an Gewicht, der in viele Trümmer zersprang. — $\frac{3}{8}$ Loth. — 1842. 36. 1. — Von dem kaiserl. russischen Minister in Hamburg, geheimen Rath von Struve, in Tausch erhalten. Dieser bekam das Fragment von Professor Mayer, der es vom Kap mitbrachte.
\subsection[Chassigny.]{Chassigny,}
\begin{center}
\small
unweit Langres, Dép. de la Haute-Marne, Frankreich.

3. Oktober 1815, 8 Uhr Vormittags.
\end{center}
\paragraph{}
Lichte, blass gelblichgrüne, ins Graue ziehende Grundmasse, von kleinen, eckig-körnigen Zusammensetzungsstücken, welche Teilbarkeit besitzen und hie und da glänzende Schüppchen zeigen, die man leicht für fein eingemengten Magnetkies ansehen könnte, der jedoch, ebenso wie das metallische Eisen ganz fehlt; in die Grundmasse sind nur schwarze, sehr feine Pünktchen von Chromeisen, oder Magneteisenstein eingestreut; die Rinde ist dick, matt, glatt und rissig. - Ein durch seine Beschaffenheit ganz isoliert stehender, höchst merkwürdiger Meteorstein.

Zwei Bruchstücke von einem einzeln (?) gefallenen Steine, dessen Bruchstücke zusammen 8 Pfund wogen.

1. Bruchstück mit etwas Rinde, — $3\frac{3}{8}$ Loth. — 1840. 4. 2. — Aus der Heuland’schen, später Heath’schen Meteoriten-Sammlung durch Herrn Pötschke gekauft. Stammt aus der von Herrn Heuland angekauften Mineralien-Sammlung des Marquis de Drée in Paris.

2. Bruchstück mit Rinde und einer anpolierten Fläche. — $2\frac{5}{16}$ Loth. — 1816. 77. 1. — Ein Geschenk des verstorbenen Lucas Sohn, Garde adjoint am naturhistorischen Museum zu Paris.
\subsection{Juvenas (Juvinas).}
\begin{center}
\small
(Libonez), Dép. de l'Ardeche, Languadoc, Frankreich.

15. Juni 1821, zwischen 3 und 4 Uhr Nachmittags.
\end{center}
\paragraph{}
Aschgraue, deutlich aus zwei Gemengteilen, einem weißen, zuweilen gelblichen, und einem schmutzig dunkelgrünen, welche in kristallinischen, eckigen Körnern und Blättchen erscheinen, zusammengesetzte Grundmasse; hie und da mit kleinen Höhlungen, in welchen diese zwei Gemengteile (Labrador ? und Augit) in kleinen, undeutlichen Krystallen erscheinen; an einigen Stellen sind die Gemengteile von etwas gröberem Korne und in runden oder länglichen Partien ausgeschieden, was Jedoch nur auf polierten Flächen ganz deutlich ist. Wenig und höchst fein eingesprengter Magnetkies. Glänzende, aderige Rinde, hie und da mit braunen Tröpfchen.

Ein großes und drei kleine Bruchstücke, von einem großen Steine von 220 Pfund, wovon das Pariser Museum noch ein Stück von 84 Pfund verwahrt. (Es fielen nebstdem noch einige kleinere Steine, deren Gewicht nicht bekannt ist.)

1. Ein großes Bruchstück mit einem kleinen Flecken Rinde — 28 1/2 Loth. — 1822. 55. 1.— Von Herrn Leman in Paris gekauft.

2. Bruchstück mit anpolierter Fläche, ohne Rinde — 495 Loth. — 1822. 56. 1. — Ebenfalls von Herrn Leman gekauft.

3. Bruchstück mit Rinde, woran kleine Tröpfchen sich zeigen. — 2 25/32 Loth. — 1822. 55. 2. — Von Herrn Leman gekauft.

4. Bruchstück mit einer anpolierten Fläche (worauf die erwähnten kugeligen und länglichen, grobkörnigen Ausscheidungen zu sehen sind) und ziemlich viel Rinde. — 2 3/32 Loth. — 1823. 59. 1. — Von Herrn Leman gekauft.
\subsection{Stannern.}
\begin{center}
\small
Iglauer Kreis, Mähren.

22. Mai 1808, gegen 6 Uhr Morgens.
\end{center}
\paragraph{}
Die lockere, etwas poröse Grundmasse ist von zweierlei Beschaffenheit; entweder (und dies ist meist der Fall) deutlich aus zwei Substanzen, einer weißen und einer grünlich schwarzen, bald ziemlich grob-, bald fein- und sehr feinkörnig gemengt; oder, wenn das Gemenge ganz innig ist, ganz einfach erscheinende Grundmasse; letztere überhaupt seltener, und ganze, wenn auch meist kleine Steine konstituierend. Die verschiedenen Grade des grob- oder feinkörnigen, aber doch noch unterscheidbaren Gemengtseins sind meist in einem und demselben Steine vorhanden, und verursachen ein fleckiges Aussehen. Einzelne schwärzliche, meist längliche Körner, zuweilen auch unvollkommen kugelige schwarze Ausscheidungen, von einer anderen Art des Gemengtseins herrührend, geben dem Steine zuweilen ein porphyr- oder breccienartiges Ansehen. Schwarze, die Masse durchziehende Gänge oder Adern sind höchst selten. Schwefelkies ist ziemlich sparsam, meist fein, zuweilen aber auch in einzelnen bohnengroßen Körnern eingemengt; metallisches Eisen fehlt. Die Rinde ist aderig, oft rissig; mehr oder weniger, aber stets glänzend (wenn nicht durch längeres Liegen in der Erde Verwitterung eintrat), zuweilen wie gefirnisst; auch sind verschiedenartige und unvollkommene Überrundungen nicht selten.

Vier und dreißig Stücke, teils ganze Steine, teils größere oder kleinere Fragmeute, in einem Gesamtgewichte von 27 Pfund, 22 5/32, Loth, von den vielen (etwa 200 bis 300) der allda gefallenen Steine.

Die folgende Reihe von ganzen Steinen und Bruchstücken der Meteorsteine von Stannern ist die größte und vollständigste, die je von einem Steinfall zusammengebracht worden ist, und stellt die interessantesten Verhältnisse dieser Steine hinsichtlich ihrer Gestalt, ihrer Überrundung, der Mengung der Grundmasse u. s. w. dar. Sie ist, mit Ausnahme einiger Stücke, das Resultat der Bemühungen der Herren von Schreibers und von Widmanstätten, die unmittelbar nach dem Ereignis als kaiserl. Kommissäre zur Untersuchung desselben nach Stannern abgeordnet wurden, Der von dem ersteren darüber in Gilberts Annalen der Physik, B. 29, vom Jahre 1808, erstattete Bericht ist das Vollständigste, das Je über einen Meteorstein-Niederfall bekannt gemacht worden ist, und hat, nebst Biots Bericht über den Steinregen von L’Aigle am meisten zur Beobachtung und Bekanntwerdung späterhin vorgefallener Niederfälle, auf die nun mehr Aufmerksamkeit gerichtet wurde, beigetragen.

A. Ganze und fast ganze Steine, oder doch in dem Zustande, wie sie auf die Erde kamen.

1. Der größte bekannte von den bei Stannern gefallenen und nicht zertrümmerten Steinen, wahrscheinlich überhaupt der größte aller da gefallenen. — Beschrieben und abgebildet in des Direktors v. Schreibers Beiträgen zur Geschichte und Kenntnis meteorischer Stein- und Metallmassen, Seite 20, Taf. 4. — 11 Pfund 10 3/4 Loth. — 1809. 8. 1. — Von Herrn Professor Mikan in Prag gekauft. Wurde von Herrn Apotheker Heller in Iglau in einem deshalb abgelassenen Teiche aufgefunden.

2. Einer der größten von den bei Stannern gefallenen Steinen; besonders frisch, schön überrindet, auch merkwürdig wegen verschiedenartiger Beschaffenheit der Rinde. — Beschrieben und abgebildet in v. Schreibers Beiträgen, S. 27, Taf. 5. Fig. 5. — 3 Pfund 21 Loth. — 1808. 24. 1. — Wurde während des Aufenthaltes der Untersuchungs-Kommission zu Stannern, im Monate Mai 1808, bei angeordneter Aufsuchung der gefallenen Steine aufgefunden.

3. Einer von den großen Steinen von Stannern; höchst ausgezeichnet und vortrefflich erhalten; merkwürdig wegen seiner keilförmigen Gestalt und der Beschaffenheit der Rinde. — Beschrieben und abgebildet in v. Schreibers Beiträgen, Seite 30, Taf. 6. Fig. 1. — 2 Pfund 12 1/2 Loth. — 1808. 24, 2. — Wie bei 2) erwähnt, während des Aufenthaltes der Kommission aufgefunden.

4. Ebenfalls einer von den größeren Steinen; wegen der strahlenförmigen Überrundung der Grundfläche merkwürdig. — Beschrieben und abgebildet im angeführten Werke, Seite 32, Taf. 6. Fig. 2. — 1 Pfund 11 3/4 Loth: — 1808. 24. 3. — Ebenfalls in Folge der gemachten Aufforderung während der Anwesenheit der Kommission zu Stannern aufgefunden.

5. Noch einer der größeren Steine; sehr lehrreich wegen einer unvollkommen überrindeten Fläche, aus welcher die Grundmasse durchblickt. — Beschrieben und abgebildet im angeführten Werke, Seite 33, Taf. 6. Fig. 3. — 1 Pfund 6 7/8 Loth. — 1808. 24. 4. — Aufgefunden wie Nr. 2-4.

6. Ein mittelgroßer Stein, anscheinend ein Bruchstück, oder die Hälfte eines Steines, aber im Herabfallen zerbrochen; die natürliche Bruchfläche teilweise verändert (etwas braun gefärbt und mit einzelnen kleinen Rindetröpfchen besetzt), also in dem Zustande wie er auf die Erde kam. Ein sehr belehrendes Stück. — Beschrieben und abgebildet im angeführten Werke, S. 36, Taf. 6. Fig. 4. — 1 Pfund 7/8 Loth. — 1808. 24. 5. — Wurde am Tage des Ereignisses aufgefunden und später der Kommission übergeben.

7. Ganzer, mittelgroßer Stein, mit stark glänzender Rinde, an einigen Stellen etwas entblößt. — 1 Pfd. 1/4 Loth. — 1808. 24. 6. — Durch die Kommission überbracht.

8. Ein mittelgroßer ganzer Stein, wenig verletzt, einige Kanten mit hervorragenden, scharfen Linien von Rindensubstanz. - 23 3/16 Loth. — 1809. 4. 2. — Von Herrn von Well gekauft.

9. Mittelgroßer ganzer Stein. An einer Stelle ist ein Stückchen weggeschnitten und die Fläche anpoliert. — 19 7/8 Loth. — 1827. 27. 4048. — Aus der im Jahre 1827 angekauften von der Nüll’schen Mineralien-Sammlung.

10. Mittelgroßer ganzer Stein (oder doch in dem Zustande, wie er herabkam), nur mit einer kleinen frischen Bruchfläche, dann einer größeren Fläche, die während des Herabfallens entstand, braun gefärbt und mit hervorgequollenen Tröpfchen von Rindensubstanz übersäet ist. Lehrreiches, sehr interessantes Stück. — 15 Loth. — 1809. 4. 4. — Von Herrn von Well gekauft.

11. Ein kleiner Stein, fast ganz, nur an dem einen Ende, wahrscheinlich beim Fallen abgebrochen, von zungenförmiger Gestalt; von der Rinde glänzt, wahrscheinlich in Folge von Verwitterung während derselbe in der Erde lag, nur das hervorragende Adergeflechte. — 10 7/8 Loth. — 1809. 7. 1. — Von Herrn Sonsluk gekauft.

12. Ein kleiner Stein, wenig verletzt, unvollkommen prismatisch. — 10 1/2 Loth. — 1809. 4. 1. — Von Herrn von Well gekauft.

13. Ein kleiner, vollkommen ganzer, nicht im geringsten verletzter Stein; verschoben viereckig, — 6 1/2 Loth. — 1827. 27. 4045. — Aus der von der Nüll’schen Mineralien-Sammlung.

14. Ein kleiner, ganzer, fast prismatischer Stein, mit einer im Falle entstandenen, mehr oder weniger, meist jedoch sehr unvollkommen überrindeten Bruchfläche; ausgezeichnet starke Überrundung der Bruchkanten. — 6 1/2 Loth. — 1827. 27. 4046. — Aus der von der Nüll’schen Mineralien-Sammlung.

15. Kleiner Stein, vollkommen ganz (nur eine etwas gekrümmte Ecke ist abgebrochen und schwach angeklebt), die Form dreiseitigpyramidal; die Rinde schwach glänzend. — Beschrieben und abgebildet im angeführten Werke, Seite 23, Tafel 5. Fig. 1. — 5 7/10 Loth. — 1808. 24. 7. — Wie bei Nr. 2. bemerkt aufgefunden, und durch die Kommission überbracht.

16a. Ganzer, sehr merkwürdiger Stein, von einer Seite zugerundet, von der anderen kantig; auch von verschiedener Beschaffenheit der Rinde, welche, wo sie dicker ist, an den Kanten Hervorragenden bildet, die beim Festwerden der Rinde durch den Widerstand der Luft beim Herabfallen, und durch Verschiebungen an der damals zähflüssigen Oberfläche entstanden sein müssen. — 4 13/16 Lth. — 1840. 4. 5. — Von Herrn Pötschke gekauft. 

16b. Kleiner, ganzer Stein, nur eine Ecke etwas abgestoßen, und die Spitze teilweise abgeschlagen; von vierseitig pyramidaler Form mit schiefer Grundfläche; zwei Seiten dick überrindet, stark glänzend, ziemlich glatt, die anderen matter und aderiger. — Beschrieben und abgebildet im angeführten Werke, S. 24, Taf. 5. Fig. 2 a et b. — 4 1/2 Loth. — 1808. 24. 8. — Durch die bei Nr. 2 erwähnte Kommission überbracht.

17. Kleiner, ganzer, an einer Kante der Länge nach entblößter Stein, von ungewöhnlicher Form, wie ein flaches Geschiebe. — 4 7/16 Loth. — 1832. 17. 1. — Von dem k. k. Kämmerer, Grafen Eugen von Czernin, eingetauscht.

18. Kleiner, unregelmäßiger Stein, an einer Kante der Länge nach angebrochen, wodurch eine feinkörnige, fast homogen erscheinende, bläulichgraue Grundmasse, mit ein Paar sehr feinen schwarzen Adern zum Vorschein kam; ziemlich stark glänzende Rinde, mit scharfen Erhöhungen. Der ungewöhnlichen Grundmasse wegen merkwürdig. — 4 3/16 Loth. — 1808. 24. 9. — Durch die bei Nr. 2 erwähnte Kommission überbracht.

19. Sehr kleiner, ganzer, nur an einer Kante etwas angebrochener Stein, einer der kleinsten von diesem Steinfalle. — Beschrieben und abgebildet im angeführten Werke, S. 25, Taf. 5. Fig. 3. — Kaum 5/8 Loth. — 1808. 25. 1. — Durch das Kreisamt zu Iglau eingesendet.

20. Ein sehr kleiner, und, so viel bekannt, der kleinste, der bei Stannern gefallenen. Steine, vollkommen ganz, flach, fast linsenförmig. — Beschrieben und abgebildet im angeführten Werke, S. 27, Taf. 5. Fig. 4. — 7/32 Loth. — 1808. 25. 2. — Durch das Kreisamt zu Iglau eingesendet.

B. Bruchstücke.

21. Größeres Bruchstück mit Rinde, merkwürdig wegen der deutlichen Ausscheidungen von Magnetkies, wovon eine erbsengroß ist. — Beschrieben und teilweise abgebildet in dem angeführten Werke, Seite 69, Taf. 7., untere Reihe, Mittel-Figur. — 13 7/16 Loth. — 1808. 24. 10. — Durch die bei Nr. 2 erwähnte Kommission überbracht.

22. Größeres Bruchstück mit Rinde und einer unvollkommen überrindeten Bruchfläche; die Grundmasse teils grob-, teils feinkörnig, grau. — 11 1/16 Lth. — 1809. 4. 3. — Von Herrn von Well gekauft.

23. Fast rundes Bruchstück, mit ganz frischen Bruchflächen und etwas Rinde; die Gemengteile von dem gewöhnlichen mittelfeinen Korne, und vorzüglich auf einer der Flächen sehr deutlich erkennbar; auch Magnetkies ist deutlich, aber sparsam eingesprengt. — 7 9/16 Loth. — 1827. 27. 4049. — Aus der von der Nüll’schen Mineralien-Sammlung.

24. Längliches Bruchstück, mit etwas Rinde und einer anpolierten Fläche; merkwürdig wegen der Rinde, die teils glänzend, teils durch Verwitterung matt, und mit Tropfen und Perlen von Rindensubstanz besetzt ist. Ein Teil der Rinde ist auch, was höchst selten vorkommt, buntfärbig angelaufen. Die polierte Fläche zeigt Ausscheidungen des schwärzlichen Bestandteiles, daher eine unvollkommen porphyrartige Struktur. — 6 1/2 Lth. — 1808. 24. 11. — Durch die bei Nr. 2 erwähnte Kommission überbracht.

25. Bruchstück, allerseits mit sehr frischen Bruchflächen, ohne Rinde; das Gemenge ist ziemlich feinkörnig und an einigen Stellen von dunklerem Grau; der Magnetkies ist darin nicht zu unterscheiden. — 6 5/16 Lth. — 1809. 24. 12. — Durch die bei Nr. 2 erwähnte Kommission überbracht.

26. Längliches Bruchstück, mit ziemlich viel Rinde. Man sieht beinahe noch die ganze Kontur des ursprünglichen Steines. Das Stück ist deshalb merkwürdig, weil an den oberen Bruchflächen Spuren von neuer Rindenbildung sichtbar sind, und der Magnetkies daselbst bunt angelaufen ist. — 4 5/16 Loth. — 1808. 24. 13. — Durch die bei Nr. 2 erwähnte Kommission überbracht.

27. Viereckiges Bruchstück, mit abgenützter Bruchfläche und mit Rinde; in der Grundmasse sind dunkelgraue, dichte Ausscheidungen vorhanden. — Beschrieben und (nicht gut) abgebildet in dem angeführten Werke. S. 59, Taf. 7. 1 Fig, der oberen Reihe. — 3 11/16 Loth. — 1808. 24. 14. — Durch die bei Nr. 2 erwähnte Kommission überbracht.

28. Kleines Bruchstück mit poröser Rinde, von welcher ein Teil schuppig abgesprungen ist, und eine zweite matte und raue Rindenlage zum Vorscheine brachte, Merkwürdig ist dieses Fragment noch durch die Erscheinung, dass, wahrscheinlich auf einer während des Falles entstandenen Kluft, Rindensubstanz in das Innere des Steines einzudringen begann, und nun, innerhalb des Randes, der Bruchfläche aufsitzt. — Beschrieben und abgebildet im angeführten Werke, S. 38, Taf. 6. Fig. 5. — 3 1/4 Loth. — 1808. 24. 15. — Durch die bei Nr. 2 erwähnte Kommission überbracht.

29. Kleines Bruchstück mit Rinde und einer anpolierten Fläche von marmoriertem Ansehen, welche, wie die ganze Masse des Stückes (eine Seltenheit bei den Steinen von Stannern), ein Paar dünne schwarze Adern durchziehen. — 3 1/8 Loth. — 1808. 24. 16. — Durch die bei Nr. 2 erwähnten Kommission überbracht.

30. Kleines Bruchstück mit Rinde, von welcher die obere, glänzende Lage teilweise abgesprungen ist. Die Grundmasse ist dicht, dunkelgrau, hie und da sind undeutliche, kugeliche Ausscheidungen von derselben Substanz wahrnehmbar. — 2 1/2 Loth. — 1808. 24. 17. — Durch die bei Nr. 2 erwähnte Kommission überbracht.

31. Kleines Bruchstück mit Rinde. Die Grundmasse feinkörnig, von einer dünnen, schwarzen Ader durchzogen. — 1 5/32 Loth. — 1808. 24. 20. — Durch die bei Nr. 2 erwähnte Kommission überbracht.

32. Kleines Bruchstück mit Rinde; die zwei erdigen Gemengteile an ein Paar Stellen mit deutlicher Teilbarkeit. — 1 3/32 Loth. — 1808. 24. 19. — Durch die bei Nr. 2 erwähnte Kommission überbracht.

33. Acht kleine Fragmente zum Studium der Rinde und der Grundmasse. — 1 1/2 Loth. — 1808. 24. 20. — Aus dem durch die Kommission überbrachten Doubletten-Vorrate.
\subsection{Konstantinopel.}
\begin{center}
\small
Auf dem Fleischplatze, im Inneren dieser Stadt.

Juni 1805, an hellem Tage.
\end{center}
\paragraph{}
Graue, durch innige Mengung der zwei erdigen Gemengteile homogen erscheinende Grundmasse, ganz wie bei der zweiten, selteneren Varietät der Steine von Stannern; schwach glänzende Rinde.

1. Fragment mit etwas Rinde, von einer dünnen schwarzen Ader durchzogen; von einem der mehreren allda gefallenen Steine. — 7/16 Loth. — 1832. 28. 1. — Wurde vor mehreren Jahren (zwischen 1818-1820) durch Herrn Leopold Fitzingers Vermittlung von Freiherrn Nell von Nellenburg, jetzt Hofrat der k. k. Hofkammer in Wien, der den Stein durch den verstorbenen Sohn des damaligen k. k. Internuntius in Konstantinopel, Baron von Stürmer, bekam, als Geschenk erhalten.

Wir haben uns in Konstantinopel durch Reisende wiederholt, aber immer erfolglos bemüht, uns von diesem, mitten in einer großen Stadt erfolgten Meteorstein-Fall, der allda nun schon ganz vergessen ist, weitere Musterstücke zu verschaffen.
\subsection{Jonzac.}
\begin{center}
\small
(Barbezieux) Dép. de la basse Charente, Frankreich.

13. Juni 1819, 6 Uhr Morgens.
\end{center}
\paragraph{}
Lichtaschgraue Grundmasse, aus zwei ziemlich gleichförmig gemengten Substanzen, einer weißen und einer schwärzlich grauen, bestehend; die letztere fast vorherrschend und in eckigen Kryställchen oder rundlichen Körnern erscheinend. Sehr wenig und höchst fein eingemengter Magnetkies. Glänzende, aderige Rinde. — Ein der ersten, gewöhnlichen Varietät der Meteorsteine von Stannern ähnlicher Meteorit.

Ein fast ganzer Stein und Ein Bruchstück von den mehreren allda gefallenen Steinen, deren Anzahl und Gesamtgewicht nicht bekannt geworden ist.

1. Ein fast ganzer Stein; eine Ecke abgeschnitten, die Schnittfläche unvollkommen poliert; außerdem auch noch andere kleine Entblößungen des Innern. — 31 11/16 Loth. — 1829. 34. 1. — Aus der Verlassenschaft des Herrn Leman in Paris, durch Professor Desmarest eingetauscht.

2. Fragment mit Rinde und ganz frischem Bruche. — 4 1/4 Loth. — 1840. 4. 3. — Aus der Heuland'schen, später Heath'schen Meteoriten-Sammlung durch Herrn Pötschke gekauft. Stammt aus der de Drée'schen Mineralien-Sammlung.
\subsection{Bialistock.}
\begin{center}
\small
(Belostock), Dorf Knasti-Knasti, im gleichnamigen Gouv., Russland.

5. Oktober alten Styls, 1827, zwischen 9 und 10 Uhr Morgens.
\end{center}
\paragraph{}
Lichtaschgraue, wenig zusammenhängende, nicht schwer zerreibliche Grundmasse, aus einem schneeweißen, einem graulich schwarzen und einem schmutzig spargelgrünen Minerale gemengt; die letzteren, nämlich das schwarze und grüne Mineral, treten auch in größeren eckigen Körnern, und zum Teile auch in rundlichen Partien auf, und verleihen dem Ganzen ein breccien- und konglomeratartiges Aussehen; auch die weiße feldspatartige Substanz sondert sich an einigen Stellen, doch noch immer mit den anderen Substanzen gemengt, deutlicher aus, und verursachet dadurch eine gefleckte Zeichnung. Der Magnetkies ist in geringer Menge vorhanden. Glänzende poröse Rinde. (Nahe verwandt mit den Steinen von Lontalax, Nobleborough und Mässing.)

1. Fragment mit Rinde von einem der mehreren allda gefallenen Steine, wovon der größte 4 Pfund wog. — 3 3/8 Loth. — 1839. 22. 1. — Aus der Mineralien-Sammlung der königl. Universität zu Berlin durch Professor Weiss eingetauscht.
\subsection{Lontalax.}
\begin{center}
\small
Friederichshamm, Switaipola (nach Chladni Sawotaipola), Gouv. Wiburg, Finnland.

13. Dezember 1813.
\end{center}
\paragraph{}
Lichtgraue, körnige, wenig zusammenhängende Grundmasse, angefüllt mit Einmengungen von kleinen, olivengrünen, dann schwärzlichen, eckigen, selten rundlichen Körnern, die vorwaltend sind und dem Ganzen ein porphyr- oder breccienartiges Aussehen geben, endlich weißen feldspatartigen Körnern. Ein Korn von Magnetkies ist deutlich wahrnehmbar, sonst scheint derselbe fein eingesprengt zu sein. Die Rinde glänzend, aderig.

1. Bruchstück von einem der mehreren allda gefallenen, aber bei dem Schmelzen des Eises meist in einen See versunkenen Steine; etwa die Hälfte eines kleinen Steines; mit Rinde und einer geschnittenen Fläche. — 1 Loth, schwach. — 1832. 30. 1. — Von dem verstorbenen Grafen Gregor von Razoumovsky in Tausch erhalten.
\subsection[Nobleborough.]{Nobleborough,}
\begin{center}
\small
oder Nobleboro, im Staate Maine, in den vereinigten Staaten von Nord-Amerika.

7. August 1823, zwischen 4 und 5 Uhr Abends.
\end{center}
\paragraph{}
In jeder Beziehung dem Steine von Lontalax so ähnlich, dass die dort gegebene Beschreibung auch vollkommen auf den Stein von Nobleborough angewendet werden kann; nur scheint letzterer noch weniger Zusammenhang zu besitzen, und daher zerreiblicher zu sein.

1. Drei Bröckchen, wovon das größte mit Rinde, von einem allda gefallenen Steine von 4 bis 6 Pfund, (außer welchem noch andere gefallen sein sollen.) — 3/8 Loth. — 1838. 25. 5. — Aus der ehemals Heuland‘schen Meteoriten-Sammlung durch Herrn Pötschke gekauft. Herr Heuland erhielt diese Fragmente durch Professor Silliman aus Nord-Amerika.
\subsection{Mässing.}
\begin{center}
\small
(St. Nicolas) bei Altötting, Landgericht Eggenfelden in Bayern.

13. Dezember 1803, zwischen 10 und 11 Uhr Vormittags.
\end{center}
\paragraph{}
Graulich weiße, ziemlich lockere Grundmasse, meist aus einem, wie Feldspat aussehenden, schneeweißen Mineral bestehend‚ worin kuglige Ausscheidungen von unreiner, pistaziengrüner Farbe, mit ziemlich vollkommenen schiefwinklichen Teilungsflächen, dann eckige, schwarze, und endlich ganz kleine Körner von olivengrüner Farbe eingemengt sind. Von metallischen Gemengteilen ist Magnetkies allein deutlich zu erkennen. — Ein höchst ausgezeichneter, dem Steine von Lontalax verwandter Meteorstein.

Zwei kleine Fragmente von einem daselbst einzeln gefallenen Steine von 3 1/4 Pfund.

1a. Ein kleines Fragment ohne Rinde. — 3/32 Loth, schwach. — 1832. 29. 3. — Durch Direktor von Schreibers im Jahre 1832 als Geschenk erhalten, welcher dasselbe im Jahre 1811 von Herrn Lavater in Zürch bekam.

1b. Kleines Fragment mit frischem Bruch und ohne Rinde. — 3/32 Loth. — 1843. 22. 1. — Von Herrn Johann von Charpentier, Bergwerks-Direktor zu Bex in der Schweiz, in Tausch erhalten. Herr von Charpentier bekam das Fragment von Chladni.
\subsection{Parma.}
\begin{center}
\small
Casignano, oder eigentlich Pieve die Casignano, bei Borgo St. Domino, im Herzogtum Parma.

19. April 1808, Mittags.
\end{center}
\paragraph{}
Lichtgraue Grundmasse, mit vielen kleinen kugelichen und eckigen Ausscheidungen, welche letztere dem Steine ein breccienartiges Ansehen geben; mit fein eingesprengtem gediegenen Eisen und Magnetkies, welch letzterer vorwaltet und auch in größeren Partien auftritt. Schwach glänzende, fast matte Rinde.

Zwei Bruchstücke von einem der allda in größerer Anzahl gefallenen Steine.

1. Bruchstück mit Rinde und einer anpolierten Fläche. — 3 19/32 Loth. — 1816. 31. 33. c. — Auf Vermittlung des Direktors von Schreibers während seiner Anwesenheit zu Paris im Jahre 1815, durch Tausch aus dem königl. Museum der Naturgeschichte erhalten.

2. Kleines Bruchstück mit Rinde und einer nicht polierten Schnittfläche. — 1 Loth — 1841. 14. 11. — Aus der Heuland'schen Meteoriten-Sammlung durch Herrn Pötschke gekauft. Stammt aus der Mineralien-Sammlung des Marquis de Drée. (Durch Verwechslung mit dem Fallorte Berlanguillas erhalten, passt aber an das vom Pariser Museum erhaltene Stück Nr. 1 an, ist also davon in Paris abgebrochen worden.)
\subsection[Siena.]{Siena,}
\begin{center}
\small
im Großherzogtum Toskana.

16. Juni 1794, nach 7 Uhr Abends.
\end{center}
\paragraph{}
Hellgraue, zuweilen rostbraun gefleckte Grundmasse, mit vielen, teils lichtgrünlichen, teils schwärzlichen, selten kugeligen, meist eckigen Ausscheidungen, die dem Ganzen ein breccien- oder porphyrartiges Ansehen verleihen; mit vielem, größtenteils fein eingesprengten, manchmal aber auch in Körnern eingewachsenen Magnetkies und weniger, fein eingesprengtem metallischen Eisen. Matte, zum Teil rissige und dadurch weiß geaderte Rinde.

Drei vollkommen ganze, aber sehr kleine, dann drei ganze, aber angebrochene oder angeschnittene Steine, und Ein Bruchstück (etwas mehr als die Hälfte eines Steines), zusammen also sieben Stücke von den sehr vielen, jedoch meist kleinen allda gefallenen Steinen.

1. Ein sehr kleiner ganzer Stein. — 7/32 Loth, schwach. — 1832. 29. 4. — Geschenk von Herrn Direktor von Schreibers.

2. Ein ebenfalls sehr kleiner ganzer Stein. — 9/32 Loth, schwach. — 1817. 47. 1. — Durch Vermittlung des Professors, Freiherrn von Jacquin, aus Italien zu Kauf erhalten.

3. Ein kleiner, fast ganzer Stein, mit einer Bruchfläche. — 17/32 Lth. — 1817. 47. 2. — Wie Nr. 2 durch Freiherrn von Jacquin erhalten.

4. Ein kleiner, länglicher, ganzer Stein, mit Rinde von zweifacher Beschaffenheit. — 5/8 Loth, schwach. — 1827. 27. 4051. — Aus der im Jahre 1827 angekauften von der Nüll'schen Mineralien-Sammlung.

5. Ein für diese Lokalität nicht ganz kleiner, fast ganzer Stein, mit einer größeren und einigen kleineren Bruchflächen. — 1 13/16 Loth. — 1817. 47. 3. — Wie Nr. 3 durch Freiherrn von Jacquin erhalten.

6. Ein größeres Bruchstück (etwas mehr als die Hälfte eines Steines), mit einer Bruch- und einer anpolierten Fläche. — Beschrieben und abgebildet in v. Schreibers Beiträgen, S. 14, Taf. 2. und S. 61, Taf 7. — 1 3/4 Lth. — 1809. 20. 1. — Vom Obersten von Tihavsky als Geschenk erhalten.

7. Ein größerer, fast ganzer Stein, mit einer Schnitt- und einer polierten Fläche, auch einer Bruchfläche mit zwei Vertiefungen, worin sich eine schwarze Substanz zeigt, Die Rinde zum Teil mit Eindrücken. — 6 1/32 Lth. — 1822. 20. 1. — Durch Herrn Chierici aus Florenz zu Kauf erhalten.
\subsection[Ensisheim.]{Ensisheim,}
\begin{center}
\small
im ehemaligen Elsass, jetzt Dép. du Haut-Rhin, Frankreich.

7. November 1492, zwischen 11 und 12 Uhr Mittags.
\end{center}
\paragraph{}
Dunkelgraue, rostbraun gefleckte Grundmasse, stellenweise lichter, wodurch ein unvollkommen breccienartiges Aussehen entsteht, das auf polierten Flächen noch deutlicher wahrzunehmen ist. Das nicht häufig und meist fein eingesprengte metallische Eisen, und der vorwaltende, teils fein eingesprengte, teils in kleinen Flecken und Adern auftretende Magnetkies sind, vorzüglich ersteres, auf den Bruchflächen Schwer, dagegen auf polierten Flächen deutlich zu erkennen; sehr ausgezeichnete und zahlreiche, schwarze, glänzende Ablösungsflächen, die den Stein fast Schiefrig, und daher leicht spaltbar machen; auch schwarze glänzende Blättchen‚ die kurze Ablösungsflächen sind. — Ein höchst eigentümlicher, mit keinem anderen verwechselbarer Meteorstein.

Ein großes und vier kleine Bruchstücke, sämtlich ohne Rinde, von einem sehr großen, einzeln gefallenen Steine von 270 Pfund.

1. Ein großes Bruchstück, — 24 1/8 Lth. — 1813. 40. 1. — Durch Vermittlung des kaiserl. Ministers Freiherrn von Hügel, während der Invasion der verbündeten Mächte im Jahre 1813, aus Colmar in Elsass als Geschenk erhalten.

2. Kleineres Bruchstück, — 5 1/32 Lth. — 1841. 6. 71. — Von der königl. sächsischen Mineralien-Niederlage zu Freiberg gekauft.

3. Bruchstück, — 4 3/32 Loth. — 1827. 27. 4053.
— Aus der von der Nüll’schen Mineralien-Sammlung.

4. Längliches Bruchstück, mit zwei anpolierten Flächen. — 2 3/4 Loth. — 1825. 42. 59. — Aus der Mineralien-Sammlung des Grafen Fries gekauft.

5. Kleines Bruchstück , mit einer anpolierten Fläche, — 1 5/8 Loth. — Von 1809. 19. 1. — Geschenk vom verstorbenen Major v. Schwarz.

Der Meteorstein von Ensisheim ist der älteste von allen, die sich bis an unsere Zeit der Zertrümmerung, dem Verstreuen und endlichem Vergessen und Wegwerfen entzogen haben. Er verdankt seine Erhaltung dem Umstande, dass Kaiser Maximilian 1. während seines Falles sich in oder bei Ensisheim befand, und den Stein in den Chor der Kirche zu Ensisheim aufhängen ließ, mit dem Verbote, für Niemanden etwas davon abzuschlagen. In der Revolutionszeit wurde der Stein auf die öffentliche Bibliothek zu Colmar gebracht, und viele Stücke davon abgeschlagen. Er befindet sich jetzt, beträchtlich vermindert, neuerdings in der Kirche zu Ensisheim.
\subsection{L'Aigle.}
\begin{center}
\small
(La Vassolerie, Fontenil, St. Michel, St. Nicolas, Bas-Vernet etc.\footnote{Es werden hier und bei anderen ausgedehnteren Steinfällen mehrere Orte genannt, teils weil die Steine hei allen diesen Orten niederfielen, teils weil sie zuweilen mit verschiedenen Ortsbezeichnungen in Handel kommen, und man sie dann für das Produkt verschiedener Ereignisse halten könnte.}) Normandie. Dépt. de l'Orne, Frankreich.

26. April 1803, 1 Uhr Nachmittags.
\end{center}
\paragraph{}
Teils licht-, teils dunkelgraue, meist rostbraun gefleckte Grundmasse; die lichteren und dunkleren Partien entweder fleckenartig nebeneinander, oder die lichte Grundmasse von einem dunkleren, bald dickeren, bald dünneren aderigen Gewebe durchzogen, dessen Zellen die lichteren Stellen sind. In diese ungleich gefärbte Grundmasse sind breccien- oder porphyrartig lichtere oder dunklere, eckige Körner oder Ausscheidungen eingemengt (zuweilen auch schwarze, bohnengroße Partien, durch das Zusammenfließen des schwarzen Aderngeflechtes entstanden). Das gediegene Eisen ist in ziemlicher Menge, zum Teil grob, der Magnetkies nur äußerst fein eingesprengt. Schwarze Ablösungsflächen sind nicht selten. Die Rinde ist matt, nicht rau. — Ein Meteorstein von eigentümlicher Beschaffenheit.

Dreizehn Stücke von den sehr vielen (2000 bis 3000) der allda gefallenen Steine, darunter vier ganze Steine.

1. Großer, ganzer, ringsum überrundeter Stein. — 2 Pfd. 22 Lth. — 1841. 14. 1. — Aus der Heuland'schen Sammlung durch Herrn Pötschke gekauft. Herr Heuland kaufte den Stein von Herrn Lambotin in Paris.

2. Großer, ganzer, überrundeter Stein, von dem ein dabei befindliches und anpassendes Eck abgebrochen ist (auch die Kanten sind hie und da, wie gewöhnlich, etwas abgestoßen); an ein Paar Seiten mit Eindrücken. — Beschrieben und abgebildet in von Schreibers Beiträgen, S. 12, Taf. 2. — 1 Pfund 30 3/8 Loth — \mars 1. 6. — Wurde durch den verstorbenen k. k. Naturalien-Kabinetts-Direktor Stütz, im Jahre 1803 von einem Franzosen gekauft.

3. Fast ganzer Stein, mit einer anpolierten Fläche. — 22 1/8 Loth. — 1840. 11. 2. — Von Herrn von Scala gekauft, Stammt aus der gräflich Razoumovsky'schen Mineralien-Sammlung.

4. Ein sehr kleiner, aber ganzer Stein, nur an einer Kante, und auch hier zum Teil während des Falles verbrochen und wieder unvollkommen überrindet; hellgraue Grundmasse. — 29/32 Loth. — 1816. 36. 35. — Durch Direktor v. Schreibers während seiner Anwesenheit in Paris im Jahre 1815 vom Mineralienhändler Lambotin erkauft.

5. Ein Fragment (wohl 2/3 des ganzen Steines); mit anpolierter Fläche. — 8 3/4 Loth. — 1827. 27. 4050. — Aus der Mineralien-Sammlung des Herrn von der Nüll.

6. Ein frisches Bruchstück mit etwas Rinde, und den in der Beschreibung erwähnten, schwarzen, bohnengroßen Einmengungen. — 6 23/32 Lth. — 1824. 48. 1. — Durch den Herausgeber zu Kauf erhalten.

7. Bruchstück mit gekrümmter Ablösungsfläche; eine polierte Fläche ist rostbraun gefleckt. — 3 9/32 Loth. — 1808. 4. 1. — Durch Herrn Apotheker Moser in Paris gekauft.

8. a. und b. Zwei Bruchstücke mit Rinde und anpolierten Flächen, welche viele rostbraune Flecken zeigen. (Waren zu einem Versuche einige Zeit in der Erde vergraben). — 1 5/32 Loth und 19/32 Loth. — Aus den Doubletten. — Von Herrn Lambotin in Paris im Jahre 1815 gekauft.

9. a. und b. Zwei Bruchstücke mit rostbraunen, anpolierten Flächen. — 3/4 Loth und 11/16 Loth. — Von 1816. 40. 31. — Durch Direktor v. Schreibers im Jahre 1815 in Paris gekauft.

10. Kleines Bruchstück mit frischem Bruche; die Rinde mit weißen Adern. — 3/4 Loth. — 1816. 40. 31. — Wie Nr. 9 angekauft.

11. Ansehnliches Bruchstück, mit großer, frischer Bruchfläche, welche das in der Beschreibung erwähnte aderige Gewebe, wodurch ein marmoriertes oder breccienartiges Ansehen entsteht, deutlich wahrnehmen lässt; mit Rinde, — 13 3/4 Loth — 1843. 29. 1. — Von Hrn. Francois Marguier in Tausch erhalten.

Die Meteorsteine von L’Aigle sind die verbreitetsten und gemeinsten in Mineralien-Sammlungen. Ein Mineralien-Händler in Paris, Herr Lambotin, kaufte davon so viel auf, als er in L’Aigle und der Umgegend zusammenbringen konnte. Lange war der Preis derselben 8 bis 10 Francs für die Unze. Jetzt ist davon in Paris nichts mehr zu erhalten.
\subsection[Liponas.]{Liponas,}
\begin{center}
\small
(in Chladni, vielleicht durch einen Druckfehler, unrichtig Laponas) bei Pont de Vesle und Bourg en Bresse, Dépt. de l'Ain, Frankreich.

7. September 1753, 2 Uhr Nachmittags.
\end{center}
\paragraph{}
Dunkel asch- oder bläulichgraue Grundmasse mit schwärzlich grauen Partien, welche dieselbe durchziehen und fleckig oder marmoriert aussehen machen; beide mit Rostflecken und ziemlich deutlichen, aber mit der Grundmasse fest verwachsenen, kugeligen Ausscheidung; mit fein und mittelfein eingesprengtem, metallischen Eisen und sehr fein eingesprengtem Magnetkies. Matte Rinde. — Gleicht fast vollkommen den Meteorsteinen von L'Aigle.

Zwei Bruchstücke von einem der zwei allda gefallenen Steine, welche zusammen 31 1/2 Pfund wogen.

1. Fragment mit viel Rinde und ausgezeichneten Eindrücken an der Oberfläche. — 4 15/32 Loth. — 1838. 25. 3. — Aus der Heuland'schen Meteoriten-Sammlung durch Herrn Pötschke gekauft. Das Stück lag früher in der von Herrn Heuland erkauften Mineralien-Sammlung des Herrn Marquis de Drée in Paris.

2. Ein ganz kleines, anpoliertes Bruchstück, ohne Rinde. — 5/16 Loth. — 1832. 29. 1. — Geschenk von Herrn Direktor von Schreibers, welcher dieses kleine Fragment während seines Aufenthaltes zu Paris im Jahre 1815, aus der de Drée'schen Mineralien-Sammlung erhielt.

Nach dem, was Bigot de Morogues in dem Werke: Memoire sur les chutes des pierres, Seite 334, von einem in dem Museum de Drée befindlichen Meteorstein von unbekannter Abkunft erwähnt, ist es wohl nur Vermutung, dass die, obwohl nur in sehr wenig Sammlungen vorhandenen Steine von Liponas wirklich von dieser Lokalität sind, Es heißt da, nachdem die auf unsere Exemplare, die aus des Marquis de Drée Sammlung stammen, vollkommen anwendbare Diagnose gegeben ist: Je presume qu'elle (la pierre d'origine inconnue) peut être l'une de celles tombées à Liponas en 1753, ce qui paroit probable à M. Léman, tant à cause de la manière, dont elle est parvenue à M. de Drée, que par son volume et ses autres caractères.
\subsection{Chantonnay.}
\begin{center}
\small
Zwischen Nantes und La Rochelle, Dépt. de la Vendée, Frankreich.

5. August 1812, Nachts 2 Uhr.
\end{center}
\paragraph{}
Die Grundmasse zeigt stellenweise eine ganz verschiedene Beschaffenheit; sie ist nämlich teils, und zwar bei weitem vorherrschend, schwarz, schwach schimmernd und dicht, wie mancher Basalt; teils dunkelgrau, braun gefleckt, mit schwarzen Streifen oder Linien durchzogen und daher von marmoriertem Ansehen. (Auch die schwarze Grundmasse hat, was aber nur auf polierten Flächen zu bemerken ist, vereinzelte, meist aber undeutliche, lichtere Flecken, und ist mit einem breiten, noch schwärzeren, aderigen Geflechte durchzogen). Ziemlich viel teils fein, teils in hirsekorngroßen Körnern eingesprengtes metallisches Eisen; weit weniger und höchst fein eingesprengter Magnetkies. Undeutliche, matte Rinde. — Ein höchst eigentümlicher Meteorstein; nur die lichteren Stellen gleichen zum Teile den Steinen von Seres und Barbotan.

Ein großes und drei kleinere Fragmente von einem einzeln gefallenen Steine von 69 Pfund.

1. Großes Bruchstück; die schwarze Grundmasse vorherrschend; hie und da Rinde; mit einer anpolierten Fläche. — 4 Pfund 5 1/4 Loth. — 1818. 38. 1. — Auf Vermittlung des Herausgebers während seines Aufenthaltes zu Paris im Jahre 1818 von Professor Brochant zu Kauf erhalten.

2. Bruchstücke mit polierter Fläche, ohne Rinde, von dem Stücke Nr. 1 abgeschnitten. — 7 3/8 Loth. — Von 1818. 38. 1. — Wie Nr. 1 erhalten.

3. Frisches Bruchstück, ganz schwarz, zum Teil verrostet, mit einer undeutlichen Ablösungsfläche. — 12 23/32 Loth. — 1834. 19. 12. — Von Herrn Doktor Bondi in Dresden gekauft.

4. Kleines Bruchstück, grau und schwarz gefleckt; ohne Rinde, — 2 1/32 Loth. — Aus den Doubletten. — Von Herrn G. B. Sowerby Sohn in London erhalten.
\subsection[Renazzo.]{Renazzo,}
\begin{center}
\small
bei Cento, Provinz Ferrara, im Kirchenstaate.

15. Janner 1824, zwischen 8 und 9 Uhr Abends.
\end{center}
\paragraph{}
Matte, schwarze Grundmasse, mit reichlich eingemengten, mit der Grundmasse porphyrartig und fest verwachsenen, lichtgrauen, kugelichen Ausscheidungen; ziemlich viel metallisches Eisen, teils sehr fein, teils gröblich, meist in die Grundmasse, selten in die kugelichen Ausscheidungen eingesprengt und die letzteren oft ringförmig umgebend; der Magnetkies, wenn er vorhanden ist, so fein eingesprengt, dass er nicht unterschieden werden kann, Matte, oder schwach schimmernde Rinde, mit rundlichen, wie schuppig aussehenden Erhöhungen. — Ein höchst merkwürdiger Meteorstein von ganz eigentümlichem Aussehen, fast wie Obsidianporphyr.

Ein Fragment und ein Blättchen von einem der drei allda aufgefundenen Steine.

1. Fragment mit Rinde von zweierlei Beschaffenheit und einer anpolierten Fläche. — 2 7/16 Loth. — 1839. 12. 1. — Von Professor Abbate Ranzani in Bologna in Tausch erhalten.

2. Plättchen mit zwei anpolierten Flächen und mit etwas Rinde (von Nr. 1 abgeschnitten). — 7/32 Loth. — Von 1839. 12. 1. —
\subsection{Richmond.}
\begin{center}
\small
Chesterfield-County, Staat Virginien, Nord-Amerika.

4. Juni 1828, 9 Uhr Morgens.
\end{center}
\paragraph{}
Schwarzgraue, weißlichgrau gesprenkelte und rostbraun gefleckte Grundmasse, worin sich kleine Höhlungen befinden; mit vielen kugeligen Ausscheidungen, zum Teile von schmutziggrüner Farbe; mit viel eingesprengtem, fein zerteiltem Magnetkies (der, wie bei vielen anderen Meteoriten, auf Bruchflächen deutlicher zu sehen ist, als auf polierten Flächen) und mäßig und mittelfein eingesprengtem metallischen Eisen, Der Magnetkies kleidet einige der oben erwähnten Höhlungen aus, und ist darin zuweilen kugelig und bunt angelaufen. In einer Vertiefung eines der Bruchstücke ist ein Eisenkorn sichtbar. Matte, poröse und, wie es scheint, leicht ablösbare Rinde. — Ein merkwürdiger Meteorstein, von ganz eigentümlicher Beschaffenheit.

Drei Bruchstücke von einem einzeln gefallenen Steine von 4 Pfund.

1. Frisches Fragment mit Rinde, — 3 7/8 Loth. — 1840. 19. 4. — Von Herrn Heuland in London gekauft, der das Stück von Herrn Shepard aus Nord-Amerika erhielt.

2. Bruchstück ohne Rinde, — 3 21/32 Loth. — 1834. 31. 21. — Durch Baron Lederer, k. k. General-Konsul in New-York, in Tausch erhalten.

3. Kleines Bruchstück mit einer anpolierten Fläche. — 1/2 Loth. — Von 1830. 11. 14. — Ebenfalls durch Baron Lederer aus Nord-Amerika in Tausch erhalten.
\subsection[Weston.]{Weston,}
\begin{center}
\small
im Staate Connecticut, Nord-Amerika,

14. Dezember 1807, 6 1/2 Uhr Morgens.
\end{center}
\paragraph{}
Die Grundmasse zeigt zwei verschiedene Farbennuancen, eine dunkelaschgraue und eine helle, graulichweiße, die wohl meist in einem und demselben Steine neben einander auftreten, vielleicht aber doch jede für sich auch ganze, wenn gleich kleine Steine konstituieren mögen. Jedenfalls sind von den Fragmenten, die uns zu Gebote stehen, oder die wir zu sehen Gelegenheit hatten, die einen manchmal bloß hell graulichweiß, und dann meist mit braunen Rostflecken besäet, die anderen bloß dunkelaschgrau, so dass man Steine von verschiedenen Steinfällen vor sich zu haben glaubt. In anderen meist größeren Stücken, sieht man jedoch die hellgraue Nuance bald in größeren Partien, bald in Flecken in der dunkelgrauen auftreten und überzeugt sich dadurch leicht von der Identität des Fund- oder Fallortes. Höchst ausgezeichnet sind in den Meteorsteinen von Weston, die in großer Menge und Vollkommenheit, aber nur in geringer Größe auftretenden kugligen Ausscheidungen, die jedoch in den dunkleren Partien weit ausgezeichneter erscheinen. Metallisches Eisen ist in ziemlicher Menge vorhanden aber meist fein eingesprengt; noch feiner der auf Bruchflächen leicht wahrnehmbare Magnetkies. Die Rinde ist sehr rau und uneben, matt oder schimmernd. — Eine sehr charakteristische, leicht erkennbare Varietät von Meteorsteinen.

Fünf Fragmente von ungleicher Beschaffenheit von den sehr vielen und mitunter sehr großen daselbst gefallenen Steinen.

1. Fragment mit frischem Bruch und unvollkommener Rinde; die Substanz des Steines vorherrschend dunkelgrau mit lichtgrauen Flecken. — 3 Loth. — 1840. 4. 4. — Aus der ehemals Heuland'schen, später Heath'schen Meteoriten-Sammlung durch Herrn Pötschke angekauft. Stammt aus der de Drée'schen Mineralien-Sammlung.

2. Fragment mit anpolierter Fläche ohne Rinde; dunkelgraue Grundmasse mit sehr vielen und ausgezeichneten kugligen Ausscheidungen; die Bruchfläche zum Teil rostbraun gefleckt. — 2 9/16 Loth. — 1812. 13. 6. — Von dem verstorbenen Mineralien-Händler Barton eingetauscht.

3. Fragment mit sehr unebener Rinde; die Grundmasse dunkelgrau mit einzelnen lichtgrauen Flecken. — 2 13/32 Loth. — 1838. 8. 1. — Von der Frau Johanna von Henikstein, geborenen von Dieckmann-Secherau eingetauscht. Befand sich früher in der Mineralien-Sammlung des k. k. Hofrates von Gersdorf.

4. Kleines Fragment mit etwas Rinde. Lichtgraue Grundmasse mit Rostflecken; die kugligen Ausscheidungen nicht sehr deutlich; das metallische Eisen und der Magnetkies fein eingesprengt. — 1 11/32 Loth. — 1821. 50. 42. — Durch Baron Lederer, k. k. General-Konsul in New-York, von Dr. Mitchill in Tausch erhalten.

5. Kleines Fragment ohne Rinde; die Grundmasse teils hellgrau mit Rostflecken, teils dunkelgrau mit kugligen Ausscheidungen. — 1 7/32 Loth. — 1812. 13. 7. — Mit Nr. 2 von dem verstorbenen Mineralien-Händler Barton eingetauscht. —
\subsection[La Baffe.]{La Baffe,}
\begin{center}
\small
2 Lieues südlich von Épinal, Dépt. des Vosges, Frankreich.

13. September 1822. 7 Uhr Morgens.
\end{center}
\paragraph{}
Lichtaschgraue oder graulichweiße, rostbraun gefleckte, durch eine große Menge von klein kugligen Ausscheidungen körnig erscheinende Grundmasse; mit vielem teils fein, teils mittelfein eingesprengten metallischen Eisen und sehr fein eingesprengtem Magnetkies; matte oder schwach schimmernde Rinde. — (Ist von den lichteren Abänderungen der Steine von Weston nicht zu unterscheiden.)

1. Fragment mit Rinde und kleiner, (ohne Smirgel) unvollkommen anpolierter Fläche (von einem einzeln gefallenen Steine von unbekanntem Gewichte). — 15/16 Loth. — 1840. 29. 2. — Vom königl. Museum der Naturgeschichte zu Paris auf Vermittlung des Herausgebers, während seines Aufenthaltes zu Paris in Tausch erhalten.
\subsection{Benares.}
\begin{center}
\small
(Krakhut) in Ostindien.

13. Dezember 1798. 8 Uhr Abends.
\end{center}
\paragraph{}
Lichtgraue Grundmasse, ganz angefüllt mit teils kugligen, teils unvollkommen nierförmigen, oder seltener auch eckigen Ausscheidungen von grünlicher Farbe, die mit der Masse nur wenig zusammenhängen, daher aus der Grundmasse hervorragen, oder beim Herausfallen kuglige Eindrücke hinterlassen. Von den metallischen Einmengungen ist der Magnetkies in größerer Menge als das gediegene Eisen, beide jedoch ziemlich sparsam vorhanden. Matte Rinde, durch welche noch die eingemengten Kugeln zu unterscheiden sind.

Drei Bruchstücke von den vielen allda gefallenen Steinen.

1. Großes Fragment mit ausgezeichneten und großen Kugeln; mit Rinde. — 1 Pfund, 1/16 Loth. — 1840. 4. 1. — Aus der Heath'schen Meteoriten-Sammlung durch Hrn. Pötschke gekauft. Herr Heath bekam das Stück in Madras.

2. Bruchstück mit Rinde von doppelter Beschaffenheit und einer anpolierten Fläche. — Beschrieben und abgebildet in v. Schreibers Beiträgen, Seite 62. Taf. 7. — 4 11/16 Loth. — 1807. 44. 1. — Geschenk von dem verstorbenen Lord Greville in London.

3. Bruchstück mit Rinde und einer frischen Bruchfläche; die eingeschlossenen Kügelchen sehr klein — 1 3/16 Loth. — 1838. 40. 1. — Von Herrn Doktor Jakob Baader in Wien eingetauscht. Ist der kleinere Teil eines Fragmentes, das aus der Heuland'schen, später Heath'schen Meteoriten-Sammlung stammt.
\subsection{Gouvernement Poltava.}
\begin{center}
\small
Ohne nähere Angabe des Fundortes erhalten; (nicht zu verwechseln mit Kuleschofka, dass ebenfalls im Gouv. Poltava liegt).

Auch die Fallzeit ist nicht mitgeteilt worden.
\end{center}
\paragraph{}
Dunkelaschgraue Grundmasse, ganz erfüllt mit einer Menge von kugligen, zuweilen auch eckigen Ausscheidungen von schmutzig grünlichgrauer Farbe. Der Magnetkies, zuweilen bunt angelaufen, sondert sich in größeren körnigen Partien aus, ist jedoch meist nur sehr fein eingesprengt. Das metallische Eisen ist in ziemlicher Menge und meist fein eingesprengt. Matte, poröse Rinde. — Einer der aus gezeichnetesten Meteorsteine‚ am nächsten den Steinen von Weston und Krasno-Ugol verwandt.

1. Bruchstück mit Rinde und einer anpolierten Fläche. — 5 1/8 Loth gut. — 1838. 28. 1. — Von der kaiserl. russischen Akademie der Wissenschaften zu Petersburg durch Professor Kupffer in Tausch erhalten.

Über diesen nicht öffentlich bekannt gewordenen Steinfall), der, wie schon oben bemerkt wurde, mit dem von Kuleschofka nicht zu verwechseln ist, fehlen alle historischen Nachrichten.
\subsection{Krasno-Ugol.}
\begin{center}
\small
(Krasnoi-Ugol) Gouv. Räsan, Russland.

9. September 1829.
\end{center}
\paragraph{}
Dunkelgraue Grundmasse, etwas dunkler, als bei dem Steine aus dem Gouv. Poltava, mit welchem der Meteorit von Krasno-Ugol fast vollkommen identisch ist; nur zeigt das kleine Fragment keine größeren Ausscheidungen von Magnetkies; auch ist die Rinde etwas verschieden, weniger porös und mehr glatt.

1. Fragment mit Rinde und einer anpolierten Fläche — 19/32 Loth. — 1839. 28. 1. — Von der Mineralien-Sammlung der königl. Universität zu Berlin durch Herrn Professor Weiss in Tausch erhalten. Wurde von dem dort aufbewahrten Fragment abgeschnitten, welches diese Universität aus der Sammlung der kaiserl. Akademie der Wissenschaften in Petersburg durch Herrn Professor Kupffer erhielt.
\subsection[Erxleben.]{Erxleben,}
\begin{center}
\small
zwischen Magdeburg und Helmstädt, preußische Provinz Sachsen.

15. April 1812, 4 Uhr Nachmittags.
\end{center}
\paragraph{}
Dunkelaschgraue, sehr dichte und auf Bruchflächen ziemlich homogene Grundmasse, mit etwas dunkleren, klein kugeligen Ausscheidungen, die auf Bruchflächen fast gar nicht, deutlich hingegen auf polierten Flächen zu erkennen sind; viel, aber sehr fein und gleichförmig eingesprengtes gediegenes Eisen; viel, höchst fein eingesprengter Magnetkies, der, wie gewöhnlich, auf Bruchflächen leichter wahrzunehmen ist, als auf polierten Flächen. (Das Umgekehrte gilt vom metallischen Eisen.) Dünne, matte Rinde, die zuweilen nur in Flecken und Pünktchen auftritt und wie ausgeschwitzt aussieht. — Ein durch seine Dichtheit, anscheinende Homogenität der Grundmasse, und das feine und gleichförmige Gemenge der letzten mit den zwei metallischen Gemengteilen sehr ausgezeichneter Meteorstein; von allen anderen, mit Ausnahme desjenigen aus dem Gouvernement Simbirsk höchst verschieden.

1. Ein dreieckiges Bruchstück, von einen einzeln gefallenen Steine von 4 1/2 Pfund; mit etwas Rinde und einer anpolierten Fläche. — 3 Loth — 1814. 22. a. 1. — Geschenk von dem verstorbenen Professor Blumenbach in Göttingen.
\clearpage
\end{document}
