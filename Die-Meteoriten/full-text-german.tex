\documentclass[a4paper, 11pt, oneside, polutonikogreek, german]{article}
\usepackage{gfsbaskerville}
% Load encoding definitions (after font package)
\usepackage[LGR,T1]{fontenc}
\usepackage{wasysym}
\usepackage{textalpha}
\usepackage{longtable}
\usepackage{listings}
\lstset{basicstyle=\ttfamily}
% Babel package:
\usepackage{babel}

% With XeTeX$\$LuaTeX, load fontspec after babel to use Unicode
% fonts for Latin script and LGR for Greek:
\ifdefined\luatexversion \usepackage{fontspec}\fi
\ifdefined\XeTeXrevision \usepackage{fontspec}\fi

% "Lipsiakos" italic font `cbleipzig`:
\newcommand*{\lishape}{\fontencoding{LGR}\fontfamily{cmr}%
		       \fontshape{li}\selectfont}
\DeclareTextFontCommand{\textli}{\lishape}

\usepackage{booktabs}
\setlength{\emergencystretch}{15pt}
\usepackage{fancyhdr}
\usepackage{microtype}
\begin{document}
\begin{titlepage} % Suppresses headers and footers on the title page
	\centering % Centre everything on the title page
	%\scshape % Use small caps for all text on the title page

	%------------------------------------------------
	%	Title
	%------------------------------------------------
	
	\rule{\textwidth}{1.6pt}\vspace*{-\baselineskip}\vspace*{2pt} % Thick horizontal rule
	\rule{\textwidth}{0.4pt} % Thin horizontal rule
	
	\vspace{1\baselineskip} % Whitespace above the title
	
	{\scshape\LARGE Die Meteoriten oder\\[1.25pt] vom Himmel gefallenen \\[1.25pt] Steine und Eisenmassen\\[1.25pt] im k. k. Hof-Mineralien-Kabinette\\[4pt] zu Wien.}
	
	\vspace{1\baselineskip} % Whitespace above the title

	\rule{\textwidth}{0.4pt}\vspace*{-\baselineskip}\vspace{3.2pt} % Thin horizontal rule
	\rule{\textwidth}{1.6pt} % Thick horizontal rule
	
	\vspace{1\baselineskip} % Whitespace after the title block
	
	%------------------------------------------------
	%	Subtitle
	%------------------------------------------------
	
	{\scshape Beschrieben,\\ und durch wissenschaftliche und geschichtliche \\ Zusätze erläutert \\ von \\ Paul Partsch,} % Subtitle or further description
	
	\vspace*{1\baselineskip} % Whitespace under the subtitle
	
    {\scshape\scriptsize Kustos an dem genannten Kabinette. \\ Mit einer Abbildung.} % Subtitle or further description
    
	%------------------------------------------------
	%	Editor(s)
	%------------------------------------------------
    \vspace*{\fill}

	\vspace{1\baselineskip}

	{\small\scshape Wien 1843.}
	
	{\small\scshape{Verlag von Kaulfuss Witwe, Prandel \& Komp.}}
	
	\vspace{0.5\baselineskip} % Whitespace after the title block

    \scshape Internet Archive Online Edition  % Publication year
	
	{\scshape\small Namensnennung Nicht-kommerziell Weitergabe unter gleichen Bedingungen 4.0 International} % Publisher
\end{titlepage}
\setlength{\parskip}{1mm plus1mm minus1mm}
\clearpage
\tableofcontents
\clearpage
\vspace*{\fill}
\begin{quote}
   Es lässt sich als ausgemacht ansehen, dass sie nicht von der Erde, sondern von einem anderen Weltkörper herstammen, und folglich die Beschaffenheit der außerhalb der Erde vorkommenden wägbaren Stoffe verkünden. In dieser Beziehung haben die Meteorsteine ein außerordentliches Interesse. - Berzelius
\end{quote}
\vspace*{\fill}
\clearpage
\section*{Vorwort.}
\paragraph{}
In dem k. k. Hof-Mineralien-Kabinette zu Wien befinden sich acht Sammlungen in Glasschränken zur Schau gestellt, die jede Woche zweimal, Mittwoche und Sonnabend, von Jedermann besehen und benützt werden können. Nachdem der Herausgeber vorliegender Schrift eine kurze allgemeine Übersicht dieser Sammlungen des k. k. Mineralien-Kabinettes kürzlich in Druck gelegt hat, beginnt er das darin gegebene Versprechen, von jeder derselben, je nach Bedürfnis und Zweckmäßigkeit, entweder spezielle Verzeichnisse oder doch ausgedehntere Übersichten nachfolgen zu lassen, dadurch in Ausführung zu bringen, dass er zuerst das vorliegende beschreibende Verzeichnis erscheinen lässt. Die Meteoriten-Sammlung des k. k. Mineralien-Kabinettes ist zwar von allen daselbst befindlichen der Anzahl der Stücke nach die kleinste, aber doch die reichste und vollständigste in der Anzahl von Lokalitäten und Exemplaren unter allen bestehenden Sammlungen ihrer Art, und überhaupt eine der merkwürdigsten Zusammenstellungen von unorganischen Körpern.\footnote{Die Meteoriten-Sammlung des k. k. Mineralien-Kabinettes enthielt im Monate Februar 1843, mit Ausschluss aller Pseudometeoriten, die wir später anführen werden, 94 verschiedene Lokalitäten von Meteoriten, und zwar 69 von Meteorsteinen, und 25 von Meteoreisen in 258 Stücken oder eigentlich Nummern, da zuweilen mehrere kleine Stücke unter Einem Nummer vereinigt sind. (Im Jahre 1806 zählte sie 7, im Jahre 1819. 36, im Jahre 1836. 58 Lokalitäten.) - Von Meteoriten, mit Ausschluss aller Pseudometeoriten, besaßen zwischen den Jahren 1840 und 1842: das k. Mineralien-Kabinett der Universität zu Berlin, mit welcher die Sammlung Chladnis vereinigt ist, 78 Lokalitäten; Baron Reichenbach in Wien 68 (wovon jedoch 19 nur in kleinen Splittern); die Galerie der Mineralogie im k. Museum der Naturgeschichte zu Paris 42; Gubernialrat Neumann in Prag 40 (meistens in ganz kleinen Fragmenten); die Mineralien-Sammlung im britischen Museum zu London und die Mineralien-Sammlung der Universität zu Göttingen jede 35; Professor John in Berlin 28 (ebenfalls meist in ganz kleinen Stückchen); die Mineralien-Sammlung der Akademie der Wissenschaften zu St. Petersburg und Baron Berzelius in Stockholm 18; Straßenbaudirektor Braumüller zu Brünn 17; die Mineralien-Sammlung des Herrn Turner in England, ehemals Eigentum des H. Heuland in London 15; die Mineralien-Sammlung des Marquis de Drée zu Paris 14. Diess sind die an Meteoriten reichsten Sammlungen; andere öffentliche und Privat-Sammlungen besitzen selten mehr als 12 Lokalitäten; so die Ecole des Mines zu Paris und die Mineralien-Sammlung des Grafen Beroldingen in Wien 12; die Privat-Mineralien-Sammlung des Königs von Dänemark zu Kopenhagen 11; das herzogliche Naturalien-Kabinett zu Gotha 10; die Mineralien-Sammlung der Universität zu Uppsala, die Mineralien-Sammlung der Akademie der Wissenschaften zu München, Professor Pfaff zu Kiel und die Universitäts-Sammlung in Parma 9; das Joanneum zu Grätz und die Sammlungen der Bergakademie zu Freiberg 8; die Mineralien-Sammlung der Universität zu Wilna, jetzt in Kiew 7; die Mineralien-Sammlung der vaterländischen Museen zu Prag und Pesth, dann die der Prager Universität jede 6; u. s. w. Die Meteoriten-Sammlung, die von Herrn Heinrich Heuland in London zusammengebracht später Eigentum des Herrn Heath zu Madras in Ostindien wurde, zählte 43 Lokalitäten von echten Meteoriten. Nach Europa zurückgebracht, wurden sie im Jahre 1837 von Herrn Carl Pötschke in Wien angekauft und daselbst vereinzelt.} Wer würde denn nicht mit ungewöhnlichem Interesse eine so große Anzahl jener rätselhaften Ankömmlinge von Außen hier vereiniget betrachten ? diese aus dem großen Weltraume oberhalb unserer Atmosphäre stammenden Massen (entweder fest gewordene kosmische Materie, oder Stücke eines zersprungenen Planeten), daher vom Himmel gefallene Steine und Eisenmassen genannt, Aerolithen oder Luftsteine von denjenigen, die ihre Entstehung in unserer Atmosphäre suchen, Mondsteine von denen, die sie durch Vulkane oder elektrische Entladungen aus diesem Erdtrabanten ausschleudern lassen, gewöhnlich aber Meteorsteine und Meteoreisen, oder mit einem gemeinschaftlichen Nahmen Meteoriten genannt, weil sie am Himmel als Meteore oder Feuerkugeln erscheinen, aus welchen, unter heftigem Schall und Geprassel, jene Massen, meist Steine, seltener wunderbare Eisenblöcke, noch heiß und nach Schwefel riechend, auf die Erde niederstürzen. Das schon seit den ältesten Zeiten beobachtete Niederfallen dieser Massen auf unseren Planeten hat von jeher den größten Eindruck auf das menschliche Gemüt gemacht, daher mehrere Völker des Altertums, Phönizier, Griechen, Römer u. a. m., die sie heilige Steine oder Bätylien nannten, ihnen, zumal als Symbol der Mutter der Götter, abergläubische Verehrung bezeigten, und dieselben, wie uns alte Geschichtsschreiber und antike Münzen lehren, in Tempeln aufbewahrten und in Triumphzügen herumführten.\footnote{Münter: Über die vom Himmel gefallenen Steine, Bätylien genannt, Kopenhagen und Leipzig 1805. 8, (auch in Gilberts Annalen der Physik, B. 21, S. 51-84, unter dem Titel: Vergleichung der Bätylien der Alten mit den Steinen, welche in neueren Zeiten vom Himmel gefallen sind.) - Von Dalberg: Über Meteor-Cultus der Alten, vorzüglich in Bezug auf Steine, die vom Himmel gefallen. Heidelberg 1811. 8.} Obwohl das Ereignis des Niederfallens durch mehrere Dezennien des vorigen Jahrhunderts bezweifelt, ja hartnäckig geleugnet, die daran Glaubenden verspoltet und verlacht wurden, so hat dieser Gegenstand seit dem berühmten Steinregen von L’Aigle in der Normandie am 26. April 1803, den das französische National-Institut durch sein Mitglied, den bekannten Physiker, Herrn Biot, untersuchen ließ, in neuerer Zeit doch so viel allgemeine Aufmerksamkeit erregt, und so verschiedene Untersuchungen und Beleuchtungen von Seite der Gelehrten zur Folge gehabt, dass jeder Gebildete, namentlich seit dem Erscheinen der verdienstvollen Schriften von Izarn\footnote{Des pierres tombées du ciel ou Lithologie atmosphérique. Paris 1803. 8.} und Bigot de Morogues,\footnote{Mémoire historique et physique sur les chutes des pierres tombées sur la surface de la terre a diverses époques. Orléans 1812. 8.} vorzüglich aber durch die klassischen Arbeiten von Howard,\footnote{Experiments and Observations on certain stony and metalline Substances, wich at different Times are said to have fallen on the Earth, also on various Kinds of native Iron, in den Philos. Transact. of the Roy. Soc. of London for 1802. Part 1. S. 168; deutsch in Gilberts Annalen der Physik, B. 13, S. 291, unter dem Titel: Versuche und Bemerkungen über Stein- und Metallmassen, die zu verschiedenen Zeiten auf die Erde gefallen sein sollen, und über die gediegenen Eisenmassen.} Chladni\footnote{Über Feuer-Meteore und über die mit denselben herabgefallenen Massen. Wien 1819, im Verlage bei J. G. Heubner. 8. Nebst vielen Aufsätzen in Gilberts Annalen.} und Karl von Schreibers\footnote{Nachrichten von dem Steinregen zu Stannern in Mähren, in Gilberts Annalen der Physik. B. 29. 1808. S. 225. - Beiträge zur Geschichte und Kenntnis meteorischer Stein- und Metallmassen und der Erscheinungen, welche deren Niederfallen zu begleiten pflegen, Wien 1820, im Verlage von J. G. Heubner, Folio. Mit Abbildungen. - Über den Meteorstein-Niederfall auf der Herrschaft Wessely in Mähren, in Baumgartners Zeitschrift für Physik und verwandte Wissenschaften. B. 1. 1832. S. 193-256.} wenigstens mit den Tatsachen des Phänomens, wenn auch nicht über die Herkunft dieser merkwürdigen Massen, die uns nie völlig klar werden, und immer Gegenstand mehr oder weniger gewagter Theorien bleiben wird, im Reinen ist. Die erwähnten wissenschaftlichen Untersuchungen haben jedoch in der naturhistorischen Betrachtung der Meteorsteine und Meteoreisenmassen, zu denen die Arbeiten des Herrn von Schreibers, des verdienstvollen Gründers unserer Meteoriten-Sammlung, und die technischen Untersuchungen einiger Meteoreisenmassen durch Herrn von Widmannstätten den Grund legten, ungeachtet der schönen Beiträge, welche die Herren Gustav Rose\footnote{Poggendorffs Annalen der Physik und Chemie. B. 4. S. 173.} und Cordier,\footnote{Annales de Chemie et de Physique. T. 34. pag. 132.} vorzüglich aber Berzelius\footnote{Poggendorffs Annalen. Bd. 33. S. 1 und 113 auch Jahresbericht über die Fortschritte der physischen Wissenschaften 15. Jahrgang S. 227.} dazu in neuerer Zeit lieferten, noch große Lücken in der genauen naturhistorischen Kenntnis dieser rätselhaften Körper gelassen. Die Ursache mag darin liegen, dass nur wenige bedeutende Sammlungen von Meteoriten bestehen, und in diesen wenigen diese kostbaren Produkte nicht in jenem Zustande vorhanden sind, der zu einer genauen Untersuchung und Kenntnis dieser, gleich den Gebirgsarten gemengten Massen unumgänglich notwendig ist; nämlich in einem durch künstliche Zubereitung entstehenden Zustand, der ihr Inneres aufschließt, und ihre wahre Beschaffenheit erst kennen lehrt. Wir meinen die Anfertigung von gut polierten Schnittflächen bei Meteorsteinen; von fein polierten Schnittflächen, die sonst keine andere Veränderung zu erleiden brauchen, dann von polierten Flächen, die weiter entweder durch Hitze-Einwirkung blau, violett oder rot anlaufen gemacht, oder durch Anwendung von metallischen Säuren (Salz- oder Salpetersäure) mehr oder weniger stark geätzt worden sind, bei Meteoreisenmassen. Da dieses mit vieler Mühe und großem Zeitaufwande, mit nicht unbedeutenden Kosten und nicht geringer Verminderung des Volums und Gewichts der so wertvollen Meteoriten in der Sammlung des k. k. Mineralien-Kabinettes ausgeführt worden ist (die Ätzung der Eisenmassen meist von Herrn von Widmannstätten, dem Entdecker der nach ihm benannten merkwürdigen Figuren), so bietet sie ganz allein unter allen bestehenden Meteoriten-Sammlungen Gelegenheit dar, die Eigenschaften, den Charakter und die Verwandtschaften der Meteoriten vollständig ins Klare zu bringen. Dieser Umstand hat uns bestimmt, dieselben nach den einzelnen Lokalitäten mit kurzen Beschreibungen oder Diagnosen zu versehen, durch die Darstellung ihrer Anordnung und ihrer Reihenfolge und eine angehängte Verwandtschaftstabelle die Ähnlichkeiten und Verschiedenheiten, die sie darbieten (wovon die ersteren im Allgemeinen geringer, die anderen viel grösser sind, als sich mancher Mineraloge vorstellt), zu zeigen, ohne dabei jedoch in eine mikroskopische Untersuchung der Meteorsteine einzugehen, die besseren Augen vorbehalten bleibt und wozu einer der ausgezeichnetsten hiesigen Gelehrten, selbst im Besitze einer der bedeutendsten Meteoriten-Sammlungen und, was bei derlei Untersuchungen fast unumgänglich notwendig ist, zugleich Chemiker, bereits zahlreiche Materialien gesammelt hat, deren baldige Bekanntmachung zu wünschen ist. Wir haben somit, soweit es der Hauptzweck dieses beschreibenden Verzeichnisses gestattete (das übrigens mit Ausschluss der Tabellen, Anmerkungen, Zusätze u. s. w. größtenteils ein Abdruck des von uns verfassten amtlichen Kabinetts-Kataloges ist) bei der Herausgabe desselben gestrebt, zugleich einen wissenschaftlichen Beitrag zur Kenntnis der Meteoriten zu geben, In dieser Absicht haben wir auch am Schlusse eine Tabelle über die spezifischen Gewichte sämtlicher im k. k. Mineralien-Kabinette aufbewahrter Meteoriten beigefügt. Die Wiegungen hat der Kustos-Adjunkt an diesem Kabinette Herr Karl Rumler mit aller Sorgfalt bei einer Temperatur von 140 R. ausgeführt, und es wurden dieser Tabelle auch alle anderen in verschiedenen Werken und Abhandlungen zerstreuten Angaben der spezifischen Gewichte von Meteoriten und auch einige noch nicht veröffentlichte beigefügt. Die historischen Beigaben und erläuternden wissenschaftlichen Anmerkungen werden Wissenschaftsfreunden in diesen Blättern vielleicht ebenfalls nicht unwillkommen sein. Noch manches Material (worunter schön ausgeführte Zeichnungen von sämtlichen durch Ätzen bei den verschiedenen Meteoreisenmassen zum Vorschein kommenden Figuren), liegt zur Bekanntmachung bereit, und wird, falls die Annalen des Wiener Museums der Naturgeschichte wieder aufleben sollten, dem Publikum vorgelegt werden. Möge dasjenige, was wir hier bieten, ein freundliches Andenken denjenigen sein, die Gelegenheit haben, die Meteoriten-Sammlung des k. k. Mineralien-Kabinettes zu sehen und Anderen, namentlich Eigentümern oder Vorstehern von Mineralien-Sammlungen, Besitzern von einzelnen Meteoriten u. s. w. Veranlassung werden, der Sammlung des k. k. Mineralien-Kabinettes im Interesse der Wissenschaft Bereicherungen an Meteoriten zukommen zu lassen. Für eine bereits so reiche Sammlung ist jede neue Lokalität ein hochanzuschlagender Gewinn, und daher dem Geber (nebst der Gegengabe von anderen Meteoriten oder Mineralien, wenn es gewünscht wird), der vollste Dank gesichert.

Wien, den 23. Februar 1843.
\clearpage
\section{Übersicht der Meteoriten im k. k. Mineralien-Kabinette nach der Reihenfolge ihrer Aufstellung.}
\begin{center}
\small
(Die Nummern dienen zur Erleichterung des Aufsuchens im vorliegenden Kataloge.)
\end{center}
\subsection{Meteorsteine.}
\begin{enumerate}
    \small
    \item Alais (St. Etienne de Lolm und Valence).
    \item Simonod
    \item Kapland (Bokkeveld).
    \item Chassigny (Langres).
    \item Juvenas.
    \item Stannern.
    \item Konstantinopel.
    \item Jonzac.
    \item Bialistock.
    \item Lontalax.
    \item Nobleborough (Nobleboro, Maine).
    \item Mässing (Eggenfelden).
    \item Parma (Casignano).
    \item Siena.
    \item Ensisheim.
    \item L'Aigle.
    \item Liponas.
    \item Chantonnay.
    \item Renazzo (Ferrara).
    \item Richmond (Virginien).
    \item Weston (Connecticut).
    \item La Baffe (Épinal).
    \item Benares (Krakhut).
    \item Gouv. Poltawa.
    \item Krasno-Ugol.
    \item Erxleben.
    \item Gouv. Simbirsk.
    \item Mauerkirchen.
    \item Nashville (Tennessee).
    \item Lucé.
    \item Lissa.
    \item Owahu (Hanaruru).
    \item Charkow (Ukraine).
    \item Zaborzika.
    \item Bachmut.
    \item Politz (Köstriz).
    \item Kuleschofka.
    \item Slobodka.
    \item Milena.
    \item Forsyth (Georgien).
    \item Yorkshire (Wold-Cottage).
    \item Glasgow (High Possil).
    \item Berlanguillas (Burgos).
    \item Apt (Saurette).
    \item Vouillé (Poitiers).
    \item Château-Renard (Triguères).
    \item Salés (Villefranche).
    \item Agen.
    \item Nanjemoy (Maryland).
    \item Asco.
    \item Toulouse.
    \item Blansko.
    \item Wessely.
    \item Limerick (Adair).
    \item Grüneberg (Heinrichau).
    \item Tipperary (Mooresfort).
    \item Gouv. Kursk.
    \item Lixna (Dünaburg).
    \item Tabor (Plan).
    \item Charsonville (Orléans).
    \item Doroninsk.
    \item Seres (Makedonien).
    \item Sigena (Sena).
    \item Barbotan (Roquefort, Créon Juillac).
    \item Eichstädt (Wittens).
    \item Groß-Divina (Budetin).
    \item Zebrak (Horzowitz).
    \item Timochin (Smolensk).
    \item Macao (Rio Assu).
\end{enumerate}
\subsection{Meteoreisen.}
\begin{enumerate}
    \small
    \item Atacama.
    \item Krasnojarsk (Sibirien, Pallas).
    \item Brahin.
    \item Sachsen (Steinbach oder Grimma ? mit dem Eisen, angeblich aus Norwegen).
    \item Bitburg.
    \item Toluca (Xiquipilco).
    \item Elbogen.
    \item Agram (Hraschina).
    \item Lenarto.
    \item Red-River (Louisiana oder Texas).
    \item Durango.
    \item Guilford.
    \item Caille (Grasse).
    \item Ashville (Buncombe).
    \item Tennessee.
    \item Bohumilitz.
    \item Bahia (Bemdegò).
    \item Zacatecas.
    \item Rasgatà.
    \item Tucuman (Otumpa).
    \item Senegal.
    \item Kap der guten Hoffnung.
    \item Clairborne (Alabama).
\end{enumerate}
\subsection{Anhang.}
\begin{enumerate}
    \small
    \item Oaxaca.
    \item Grönland (Baffingsbay).
\end{enumerate}
\clearpage
\section{Übersicht der Meteoriten im k. k. Mineralien-Kabinette, nach den Fall- oder Fundorten.}
\begin{center}
\small
(Die Nummern beziehen sich auf die Reihenfolge in der Übersicht Nr. 1., und dienen zur Erleichterung des Aufsuchens im vorliegenden Kataloge.)
\end{center}
\subsection{Meteorsteine.}
\subsubsection{Europa}
\begin{center}
Frankreich.
\end{center}
\begin{itemize}
    \small
    \item[48.] Agen, Dépt. Lot et Garonne.
    \item[1.] Alais, Dépt. du Gard.
    \item[44.] Apt, Dépt. de Vaucluse.
    \item[50.] Asco, Insel Korsika.
    \item[64.] Barbotan (und Roquefort) ehemals Gascogne, Dépt. du Gers (und Dépt. des Landes).
    \item[18.] Chantonnay, Dépt. de la Vendée.
    \item[60.] Charsonville, Dépt. du Loiret.
    \item[4.] Chassigny, Dépt. de la haute Marne.
    \item[46.] Château-Renard, Dépt. du Loiret.
    \item[15.] Ensisheim, ehemals Elsass, jetzt Dépt. du Haut-Rhin.
    \item[8.] Jonzac, Dépt. de la Charente inferieure.
    \item[5.] Juvenas, Dépt. de l'Ardeche.
    \item[22.] La Baffe, Dépt. des Vosges.
    \item[16.] L'Aigle, ehemals Normandie, Dépt. de l'Orne.
    \item[17.] Liponas, Dépt. de l'Ain.
    \item[30.] Lucé, Dépt. de la Sarthe.
    \item[47.] Salés, Dépt. du Rhone.
    \item[2.] Simonod, Dépt. de l'Ain.
    \item[51] Toulouse, Dépt. de la Haute-Garonne.
    \item[45] Vouillé, Dépt. de la Vienne.
\end{itemize}
\begin{center}
England.
\end{center}
\begin{itemize}
    \small
    \item[41.] Wold-Cottage, Yorkshire.
\end{itemize}
\begin{center}
Schottland.
\end{center}
\begin{itemize}
    \small
    \item[42.] High-Possil, Glasgow.
\end{itemize}
\begin{center}
Irland.
\end{center}
\begin{itemize}
    \small
    \item[56.] Mooresfort, Grafschaft Tipperary.
    \item[54.] Adair, Grafschaft Limerick.
\end{itemize}
\begin{center}
Spanien.
\end{center}
\begin{itemize}
    \small
    \item[43.] Berlanguillas, Alt-Kastilien.
    \item[63.] Sigena, Aragonien.
\end{itemize}
\begin{center}
Italien.
\end{center}
\begin{itemize}
    \small
    \item[13.] Casignano, Herzogtum Parma.
    \item[19.] Renazzo, Provinz Ferrara, Kirchenstaat.
    \item[14.] Siena, Toskana.
\end{itemize}
\begin{center}
Deutschland.
\end{center}
\begin{itemize}
    \small
    \item[59.] Tabor, ehemals Bechiner, jetzt Taborer Kreis, Böhmen.
    \item[67.] Zebrak, Berauner Kreis, Böhmen.
    \item[31.] Lissa, Bunzlauer Kreis, Böhmen.
    \item[6.] Stannern, Iglauer Kreis, Mähren.
    \item[52.] Blansko, Brünner Kreis, Mähren.
    \item[53.] Wessely, Hradischer Kreis, Mähren.
    \item[28.] Mauerkirchen, ehemals Bayern, jetzt Inn-Kreis, Ober-Österreich.
    \item[12.] Mässing, Unter-Donau-Kreis, Niederbaiern.
    \item[65.] Eichstädt, Regenkreis, Franken, Baiern.
    \item[36.] Politz bei Gera, Fürstentum Reuß.
    \item[26.] Erxleben, Regierungsbezirk Magdeburg, preußische Provinz Sachsen.
    \item[55.] Grüneberg, Regierungsbezirk Liegnitz, Provinz Schlesien.
\end{itemize}
\begin{center}
Ungarn.
\end{center}
\begin{itemize}
    \small
    \item[66.] Groß-Divina, Trentschiner-Komitat.
\end{itemize}
\begin{center}
Kroatien.
\end{center}
\begin{itemize}
    \small
    \item[39.] Milena, Warasdiner-Komitat.
\end{itemize}
\begin{center}
Russland.
\end{center}
\begin{itemize}
    \small
    \item[35.] Bachmut, Gouv. Ekaterinoslaw.
    \item[9.] Bialistock, gleichnamige Provinz.
    \item[33.] Charkow, gleichnamiges Gouvernement.
    \item[25.] Krasno-Ugol, Gouv. Räsan.
    \item[37.] Kuleschofka, Gouv. Poltawa.
    \item[57.] Kursk (Gouv.)
    \item[58.] Lixna, Dünaburger Kreis, Gouv. Witepsk.
    \item[10.] Lontalax, Finnland.
    \item[24.] Poltawa (Gouv.)
    \item[27.] Simbirsk (Gouv.)
    \item[38.] Slobodka, Gouv. Smolensk.
    \item[68.] Timochin, Gouv. Smolensk.
    \item[34.] Zaborzika, Gouv. Wolhynien.
\end{itemize}
\begin{center}
Türkei.
\end{center}
\begin{itemize}
    \small
    \item[7.] Konstantinopel.
    \item[62.] Seres, Makedonien.
\end{itemize}
\subsubsection{Asien.}
\begin{itemize}
    \small
    \item[61.] Doroninsk, Gouv. Irkutsk, Sibirien.
    \item[23.] Benares, Bengalen, Ostindien.
\end{itemize}
\subsubsection{Afrika.}
\begin{itemize}
    \small
    \item[3.] Kapland (Bokkeveld bei Tulpagh).
\end{itemize}
\subsubsection{Amerika.}
\begin{itemize}
    \small
    \item[11.] Nobleborough, Maine, Vereinigte Staaten von Nord-Amerika.
    \item[21.] Weston, Connecticut, Vereinigte Staaten von Nord-Amerika.
    \item[49.] Nanjemoy, Maryland, Vereinigte Staaten von Nord-Amerika.
    \item[20.] Richmond, Virginien, Vereinigte Staaten von Nord-Amerika.
    \item[29.] Nashville, Tennessee, Vereinigte Staaten von Nord-Amerika.
    \item[40.] Forsyth, Georgien, Vereinigte Staaten von Nord-Amerika.
    \item[69.] Macao, Provinz Rio grande do Norte, Brasilien.
\end{itemize}
\subsubsection{Australien.}
\begin{itemize}
    \small
    \item[32.] Owahu, eine der Sandwich-Inseln.
\end{itemize}
\subsection{Meteoreisen.}
\subsubsection{Europa}
\begin{center}
Frankreich.
\end{center}
\begin{itemize}
    \small
    \item[82.] Caille, Dépt. du Var.
\end{itemize}
\begin{center}
Deutschland.
\end{center}
\begin{itemize}
    \small
    \item[76.] Elbogen, Elbogner Kreis, Böhmen.
    \item[85.] Bohumilitz, Prachiner Kreis, Böhmen.
    \item[73.] Sachsen (Steinbach bei Eibenstock im Erzgebirgischen Kreise oder Grimma ? im Leipziger Kreise).
    \item[74.] Bitburg, Regierungsbezirk Trier, Rheinpreußen.
\end{itemize}
\begin{center}
Ungarn.
\end{center}
\begin{itemize}
    \small
    \item[78.] Lenarto, Saroscher Komitat.
\end{itemize}
\begin{center}
Kroatien.
\end{center}
\begin{itemize}
    \small
    \item[77.] Agram, Agramer Komitat.
\end{itemize}
\begin{center}
Russland.
\end{center}
\begin{itemize}
    \small
    \item[72.] Brahin, Gouv. Minsk, ehemals Litauen.
\end{itemize}
\subsubsection{Asien.}
\begin{center}
Sibirien.
\end{center}
\begin{itemize}
    \small
    \item[71.] Krasnojarsk, Gouv. Jeniseisk.
\end{itemize}
\subsubsection{Afrika.}
\begin{itemize}
    \small
    \item[90.] Senegambien (am oberen Teil des Senegalstromes).
    \item[91.] Kap der guten Hoffnung (zwischen dem Sonntags- und Boschesmannsflüsse).
\end{itemize}
\subsubsection{Amerika.}
\begin{itemize}
    \small
    \item[94.] Grönland (Baffingsbay)
\end{itemize}
\begin{center}
Vereinigte Staaten von Nord-Amerika.
\end{center}
\begin{itemize}
    \small
    \item[84.] Tennessee. (Cocke-County in Staate Tennessee).
    \item[83.] Ashville, Nord-Carolina.
    \item[81.] Guilford, Nord-Carolina.
    \item[92.] Clairborne, Staat Alabama.
    \item[79.] Louisiana oder Texas ? (am Red-River oder roten Flüsse). 
\end{itemize}
\begin{center}
Vereinigte Mexikanische Bundesstaaten.
\end{center}
\begin{itemize}
    \small
    \item[80.] Durango, im gleichnamigen Staate.
    \item[87.] Zacatecas, im gleichnamigen Staate.
    \item[75.] Toluca, (Xiquipilio, im Staate Mexiko).
    \item[93.] Oaxaca, (in der Misteca, im Staate Oaxaca).
\end{itemize}
\begin{center}
Columbien. (Neu-Granada.)
\end{center}
\begin{itemize}
    \small
    \item[88.] Rasgatà, nordöstlich von Santa Fe de Bogotá.
\end{itemize}
\begin{center}
Bolivia. (ehemals Peru.)
\end{center}
\begin{itemize}
    \small
    \item[70.] Atacama. (Wüste Atacama, an der Grenze von Chili).
\end{itemize}
\begin{center}
Brasilien.
\end{center}
\begin{itemize}
    \small
    \item[86.] Bahia, am Bache Bemdegò bei Monte Santo, Capitanie Bahia.
\end{itemize}
\begin{center}
Vereinigte Staaten am Rio de la Plata.
\end{center}
\begin{itemize}
    \small
    \item[89.] Tucuman. (Otumpa, im Staate Tucuman.)
\end{itemize}
\clearpage
\section{Übersicht der Meteoriten im k. k. Mineralien-Kabinette, nach der Zeitfolge ihres Niederfallens.}
\begin{center}
\small
(Die Nummern beziehen sich auf die Reihenfolge in der Übersicht Nr. 1, und dienen zur Erleichterung des Aufsuchens im vorliegenden Kataloge.)
\end{center}
\begin{center}
    \footnotesize
    \begin{longtable}{|p{6mm}|p{9mm}|p{60mm}|p{27mm}|}
    \hline
        Nr. & Jahr & Monat und Tag &   \\ \hline
        ~ & ~ & ~ & \textbf{1. Meteorsteine.} \\ \hline
        15 & 1492 & 7. November & Ensisheim. \\ \hline
        59 & 1753 & 3. Juli & Tabor. \\ \hline
        17 & 1753 & September & Liponas. \\ \hline
        30 & 1768 & 13. September & Lucé. \\ \hline
        28 & 1768 & 20. November & Mauerkirchen. \\ \hline
        63 & 1773 & 17. November & Sigena. \\ \hline
        65 & 1785 & 19. Februar & Eichstädt. \\ \hline
        33 & 1787 & 1. Oktober & Charkow. \\ \hline
        64 & 1790 & 24. Juli & Barbotan. \\ \hline
        14 & 1794 & 16. Juni & Siena. \\ \hline
        41 & 1795 & 13. Dezember & Yorkshire. \\ \hline
        47 & 1798 & 8. oder 12. Marz & Salés. \\ \hline
        23 & 1798 & 13. Dezember & Benares. \\ \hline
        16 & 1803 & 6. April & L’Aigle \\ \hline
        44 & 1803 & 8. Oktober & Apt. \\ \hline
        12 & 1803 & 13. Dezember & Massing \\ \hline
        42 & 1804 & 5. April & Glasgow \\ \hline
        61 & 1805 & 25. Marz & Doroninsk \\ \hline
        7 & 1805 & Juni & Konstantinopel \\ \hline
        50 & 1805 & November & Asco. \\ \hline
        1 & 1806 & 15. Marz & Alais. \\ \hline
        68 & 1807 & 13. Marz & Timochin. \\ \hline
        21 & 1807 & 14. Dezember & Weston. \\ \hline
        13 & 1808 & 19. April & Parma. \\ \hline
        6 & 1808 & 22. Mai & Stannern. \\ \hline
        31 & 1808 & 3. September & Lissa. \\ \hline
        56 & 1810 & August & Tipperary. \\ \hline
        60 & 1810 & 23. November & Charsonville. \\ \hline
        37 & 1811 & zwischen d. 12. u. 13. Marz um Mitternacht & Kuleschofka. \\ \hline
        43 & 1811 & 8. Juli & Berlanguillas. \\ \hline
        51 & 1812 & 12. April & Toulouse. \\ \hline
        26 & 1812 & 15. April & Erxleben. \\ \hline
        18 & 1812 & 5. August & Chantonnay. \\ \hline
        54 & 1813 & 10. September & Limerick. \\ \hline
        10 & 1813 & 13. Dezember & Lontalax. \\ \hline
        35 & 1814 & 3. Februar & Bachmut. \\ \hline
        48 & 1814 & 5. September & Agen. \\ \hline
        4 & 1815 & 3. Oktober & Chassigny. \\ \hline
        34 & 1818 & 30. Marz & Zaborzika. \\ \hline
        62 & 1818 & Juni & Seres. \\ \hline
        38 & 1818 & 10. August & Slobodka. \\ \hline
        8 & 1819 & 13. Juni & Jonzac. \\ \hline
        36 & 1819 & 13. Oktober & Poliz. \\ \hline
        58 & 1820 & 12. Juli & Lixna. \\ \hline
        5 & 1821 & 15. Juni & Juvenas. \\ \hline
        22 & 1822 & 13. September & La Baffe. \\ \hline
        11 & 1823 & 7. August & Nobleborough. \\ \hline
        19 & 1824 & 15. Januar & Renazzo. \\ \hline
        67 & 1824 & 14. Oktober & Zebrak. \\ \hline
        49 & 1825 & 10. Februar & Nanjemoy. \\ \hline
        32 & 1825 & 14. September & Owahu. \\ \hline
        29 & 1827 & 9. Mai & Nashville. \\ \hline
        9 & 1827 & 5. oder 6. Oktober & Bialistock. \\ \hline
        20 & 1828 & 4. Juni & Richmond. \\ \hline
        40 & 1829 & 8. Mai & Forsyth. \\ \hline
        25 & 1829 & 9. September & Krasno-Ugol. \\ \hline
        45 & 1831 & 18. Juli (nach anderen Angaben 13. Mai) & Vouillé. \\ \hline
        53 & 1831 & 9. September & Wessely. \\ \hline
        52 & 1833 & 25. November & Blansko. \\ \hline
        2 & 1835 & 13. November & Simonod. \\ \hline
        69 & 1836 & 11. November (nach anderen Angaben 11. Dezember & Macao. \\ \hline
        66 & 1837 & 24. Juli & Groß-Divina. \\ \hline
        3 & 1838 & 13. Oktober & Kapland. \\ \hline
        55 & 1841 & 22. Marz & Grüneberg. \\ \hline
        46 & 1841 & 12. Juni & Château-Renard. \\ \hline
        39 & 1842 & 26. April & Milena. \\ \hline
        24 & ~ & Die Fallzeit unbekannt. & Gouv. Poltawa. \\ \hline
        57 & ~ & Die Fallzeit unbekannt. & Gouv. Kursk. \\ \hline
        27 & ~ & Die Fallzeit unbekannt. & Gouv. Simbirsk. \\ \hline
        ~ & ~ & ~ & \textbf{2. Meteoreisen.} \\ \hline
        77 & 1751 & 26. Mai & Agram. \\ \hline
        70 bis 76 & ~ & Die Fallzeit unbekannt. & Alle andern Eisenmassen. \\ \hline
        76 & ~ & Die Fallzeit unbekannt. & Alle andern Eisenmassen. \\ \hline
        78 bis 94 & ~ & Die Fallzeit unbekannt. & Alle andern Eisenmassen. \\ \hline
        94 & ~ & Die Fallzeit unbekannt. & Alle andern Eisenmassen. \\ \hline
    \end{longtable}
\end{center}
\clearpage
\section{Wegweiser.}
\paragraph{}
Die Meteoriten-Sammlung des k. k. Mineralien-Kabinettes ist in einem langen pultförmigen Glasschrank, mit nach zwei Seiten abfallenden Glaswänden, in der Mitte des vierten Saales aufgestellt. Auf der waagerechten Ebene des Glasschrankes erheben sich, nach der Länge desselben ziehend, jedoch beiderseits noch Raum lassend, drei breite niedere Stufen, wodurch im Ganzen fünf Abteilungen entstehen. Die obere oder zweite, beiden Seiten des Pult-Schrankes gemeinschaftliche Stufe, (mit Abteilung Nr. 1 bezeichnet) enthält die größten Stücke, deren Volum eine systematische Einreihung unter die anderen nicht erlaubte, nämlich die zwei berühmten großen Eisenmassen von Elbogen und Agram, große Stücke der Eisenmassen von Atacama, Lenarto, Bohumilitz, Bahia und Krasnojarsk, einen großen ganzen Meteorstein von Tabor, einen solchen von Wessely, und einen von Lissa, drei große ganze Steine von Stannern, ein großes Fragment des Steines von Chantonnay und zwei große ganze Steine von L’Aigle (letztere zwei auf der Rückseite des Schrankes). Die Reihenfolge der nach ihren Verwandtschaften zusammengestellten Meteoriten kleineren Formates beginnt in der vorderen, gegen den dritten Saal des Mineralien-Kabinettes gekehrten Hälfte des Schrankes; hier sind auf der untersten, mit Nr. 2 bezeichneten Abteilung, auf der Ebene des Schrankes, unterhalb der ersten Stufe die Meteorsteine, welche kein gediegenes Eisen enthalten (Nr. 1 bis 12 der Tabelle Nr. 1.) aufgestellt; von da wendet sich die Reihe auf die Rückseite des Glasschrankes, der auf der ersten Stufe (Abteilung Nr. 3) und in der Abteilung unterhalb derselben (Abteilung Nr.4) auf einem ausgedehnten Raume die anderen, weit zahlreicheren Meteorsteine, welche gediegenes Eisen einschließen (von Nr. 13 bis 69 der Tabelle Nr. 1.) enthält. Die Reihe springt von der Abteilung Nr. 4 nun wieder auf die Vorderseite des Glasschrankes, wo die erste Stufe, mit Abteilung Nr.5 bezeichnet, die kleineren Stücke von Meteoreisen trägt; Anfangs die ästigen mit Olivin (von Nr. 70 bis 73), darauf die derben oder formlosen (von Nr. 74 bis 94), womit die Sammlung endet. — Alle Stücke liegen auf ovalen, weiß lackierten, mit goldenen Leisten gezierten Untersätzen von verschiedener Größe und Höhe, auf welchen eine Etiquette den Namen der Lokalität, das Falljahr, und wenn (wie bei allen Eisenmassen, mit alleiniger Ausnahme der Agramer) die Fallzeit nicht bekannt ist, die Zeit ihrer Auffindung oder Bekanntwerdung angibt. Die bei jeder Lokalität mit Nr. 1 beginnenden Nummern auf den Untersätzen beziehen sich auf die Beschreibung der Lokalität, sowohl in dem Kabinetts- als dem vorliegenden gedruckten Kataloge.
\clearpage
\section{}
\begin{center}
{\LARGE Meteorsteine.}

Nr. 1 bis 69.
\end{center}
\subsection{Alais.}
\begin{center}
\small
St. Etienne de Lolm und Valence, Dépt. du Gard, Frankreich.

15. Mai 1806, 5 Uhr Abends.
\end{center}
\paragraph{}
Bräunlich schwarze, teils bröckliche und zerreibliche, teils (durch Zerreibung entstandene) pulverige Substanz, hie und da mit weißen Salz-Effloreszierungen (nach Berzelius: Bittersalz mit Nickelvitriol), in welcher selbst mittelst der Lupe weder kugelige Ausscheidungen, noch gediegenes Eisen und Magnetkies (die jedoch den Analysen zufolge in sehr kleiner Menge vorhanden sind), unterschieden werden können.

1. Größere und kleinere Bröckchen, mit Pulver vermischt und, bis auf zwei, ohne Rindensubstanz; von einem der zwei allda gefallenen, und alsbald zerbröckelten Steine, die zusammen 12 Pfund wogen. — Etwas über $\frac{3}{32}$ Loth oder $25\frac{1}{2}$ Gran — 1816. 35. 44, und 1838. 27. 2.\footnote{Die hier und bei allen anderen Lokalitäten von Meteoriten befindlichen Zahlen bedeuten das Jahr und die Nummer des Acquisitions-Postens, dann die Nummer des Stückes in dem respektiven Acquisitions-Posten der Kabinetts-Kataloge.} — Teils aus der Mineralien-Sammlung des Marquis de Drée in Paris durch den Direktor der vereinigten k. k. Hof-Naturalien-Kabinette, Karl von Schreibers, in Tausch erhalten, teils von Herrn Gubernialrat Neumann in Prag eingetauscht.
\subsection{Simonod.}
\begin{center}
\small
Gemeinde Belmont, Arrondissement Belley, Dép. de l’Ain, Frankreich.

13. November 1835, 9 Uhr Abends.
\end{center}
\paragraph{}
1. Kleine, eckige und scharfkantige Fragmentchen, samtschwarz, schwach glänzend, von Fettglanz, spröde, schwer zerreiblich, vollkommen homogen aussehend; von einem der zwei allda gefallenen etwa eigroßen Steine, die wohl bald in kleine Fragmente zerfallen sind. — $\frac{3}{32}$ Loth und 4 Gran. — 1840. 28. 1. — Von Herrn Marquis de Drée in Paris in Tausch erhalten. Marquis de Drée erhielt die Substanz durch einen Gendarmerie-Beamten des Dép. de l'Ain.

Ob die Fragmentchen von Simonod oder Belley wirklich einer mit Detonation zersprungenen Feuerkugel, die einen wahren, überrindeten Meteorstein gab, angehören, oder Produkt einer Sternschnuppe sind, ist noch zweifelhaft. Die Nacht des Falles war eine der Sternschnuppen-Nächte. Herr Millet d’Aubenton berichtete Herrn Arago, dass er zu der oben angegebenen Zeit ein Feuermeteor beobachtete, welches in der Gemeinde Belmont zersprang, und zwar über Häusern und Strohdächern, die es entzündete. Derselbe will auch zwei eigroße Stücke gefunden haben, die ganz die Beschaffenheit eines Aerolithen besaßen. — Später hat Herr Millet Stücke davon der Pariser Akademie übersendet. Er schrieb dabei, dass sie im Allgemeinen das Ansehen von Obsidian haben (was ganz richtig ist), dass der Magnet kleine Metallkügelchen davon ausziehe, bestehend aus Eisen, Schwefel, Kupfer, Arsenik und vielleicht Silber?! (was wir in unseren Fragmentchen nicht finden konnten). Er glaubte auch Spuren von Nickel und Chrom darin gefunden zu haben, Die eingesendeten Stücke sind von der Pariser Akademie Hrn. Dumas zur Analyse übergeben worden. (Siehe Poggendorffs Annalen B. 36. S. 562 und Bd. 37. S. 460.) — Nach einer Mittheilung, die wir Herrn Marquis de Drée verdanken, fand Herr Damour darin Kieselerde, Eisenoxyd, Kupferoxyd, Schwefel, Kohle und Kalk. — Merkwürdig ist das spezifische Gewicht dieser Fragmente‚ nämlich 1,35. (nach einer Wiegung von Herren Rumler) das geringste von allen bekannten Meteorsteinen.
\subsection{Kapland.}
\begin{center}
\small
Bokkeveld bei Tulpagh, 70 englische Meilen von der Kapstadt, am Vorgebirge der guten Hoffnung in Afrika.

13. Oktober 1838, 9 Uhr Morgens.
\end{center}
\paragraph{}
In die schwarze, matte, durch den Strich Glanz erlangende, weiche und milde Grundmasse sind weißliche und grünliche, undeutliche Körner (die wie Flecken aussehen und wenig Körper zu haben scheinen) eingemengt; gediegen Eisen und Schwefelkies sind nicht sichtbar. — Ein höchst eigentümlicher Meteorstein.

Fragment mit etwas Rinde; von einem großen, einzeln gefallenen Steine von einigen Zentnern an Gewicht, der in viele Trümmer zersprang. — $\frac{3}{8}$ Loth. — 1842. 36. 1. — Von dem kaiserl. russischen Minister in Hamburg, geheimen Rath von Struve, in Tausch erhalten. Dieser bekam das Fragment von Professor Mayer, der es vom Kap mitbrachte.
\subsection[Chassigny.]{Chassigny,}
\begin{center}
\small
unweit Langres, Dép. de la Haute-Marne, Frankreich.

3. Oktober 1815, 8 Uhr Vormittags.
\end{center}
\paragraph{}
Lichte, blass gelblichgrüne, ins Graue ziehende Grundmasse, von kleinen, eckig-körnigen Zusammensetzungsstücken, welche Teilbarkeit besitzen und hie und da glänzende Schüppchen zeigen, die man leicht für fein eingemengten Magnetkies ansehen könnte, der jedoch, ebenso wie das metallische Eisen ganz fehlt; in die Grundmasse sind nur schwarze, sehr feine Pünktchen von Chromeisen, oder Magneteisenstein eingestreut; die Rinde ist dick, matt, glatt und rissig. - Ein durch seine Beschaffenheit ganz isoliert stehender, höchst merkwürdiger Meteorstein.

Zwei Bruchstücke von einem einzeln (?) gefallenen Steine, dessen Bruchstücke zusammen 8 Pfund wogen.

1. Bruchstück mit etwas Rinde, — $3\frac{3}{8}$ Loth. — 1840. 4. 2. — Aus der Heuland’schen, später Heath’schen Meteoriten-Sammlung durch Herrn Pötschke gekauft. Stammt aus der von Herrn Heuland angekauften Mineralien-Sammlung des Marquis de Drée in Paris.

2. Bruchstück mit Rinde und einer anpolierten Fläche. — $2\frac{5}{16}$ Loth. — 1816. 77. 1. — Ein Geschenk des verstorbenen Lucas Sohn, Garde adjoint am naturhistorischen Museum zu Paris.
\subsection{Juvenas (Juvinas).}
\begin{center}
\small
(Libonez), Dép. de l'Ardeche, Languadoc, Frankreich.

15. Juni 1821, zwischen 3 und 4 Uhr Nachmittags.
\end{center}
\paragraph{}
Aschgraue, deutlich aus zwei Gemengteilen, einem weißen, zuweilen gelblichen, und einem schmutzig dunkelgrünen, welche in kristallinischen, eckigen Körnern und Blättchen erscheinen, zusammengesetzte Grundmasse; hie und da mit kleinen Höhlungen, in welchen diese zwei Gemengteile (Labrador ? und Augit) in kleinen, undeutlichen Krystallen erscheinen; an einigen Stellen sind die Gemengteile von etwas gröberem Korne und in runden oder länglichen Partien ausgeschieden, was Jedoch nur auf polierten Flächen ganz deutlich ist. Wenig und höchst fein eingesprengter Magnetkies. Glänzende, aderige Rinde, hie und da mit braunen Tröpfchen.

Ein großes und drei kleine Bruchstücke, von einem großen Steine von 220 Pfund, wovon das Pariser Museum noch ein Stück von 84 Pfund verwahrt. (Es fielen nebstdem noch einige kleinere Steine, deren Gewicht nicht bekannt ist.)

1. Ein großes Bruchstück mit einem kleinen Flecken Rinde — 28 1/2 Loth. — 1822. 55. 1.— Von Herrn Leman in Paris gekauft.

2. Bruchstück mit anpolierter Fläche, ohne Rinde — 495 Loth. — 1822. 56. 1. — Ebenfalls von Herrn Leman gekauft.

3. Bruchstück mit Rinde, woran kleine Tröpfchen sich zeigen. — 2 25/32 Loth. — 1822. 55. 2. — Von Herrn Leman gekauft.

4. Bruchstück mit einer anpolierten Fläche (worauf die erwähnten kugeligen und länglichen, grobkörnigen Ausscheidungen zu sehen sind) und ziemlich viel Rinde. — 2 3/32 Loth. — 1823. 59. 1. — Von Herrn Leman gekauft.
\subsection{Stannern.}
\begin{center}
\small
Iglauer Kreis, Mähren.

22. Mai 1808, gegen 6 Uhr Morgens.
\end{center}
\paragraph{}
Die lockere, etwas poröse Grundmasse ist von zweierlei Beschaffenheit; entweder (und dies ist meist der Fall) deutlich aus zwei Substanzen, einer weißen und einer grünlich schwarzen, bald ziemlich grob-, bald fein- und sehr feinkörnig gemengt; oder, wenn das Gemenge ganz innig ist, ganz einfach erscheinende Grundmasse; letztere überhaupt seltener, und ganze, wenn auch meist kleine Steine konstituierend. Die verschiedenen Grade des grob- oder feinkörnigen, aber doch noch unterscheidbaren Gemengtseins sind meist in einem und demselben Steine vorhanden, und verursachen ein fleckiges Aussehen. Einzelne schwärzliche, meist längliche Körner, zuweilen auch unvollkommen kugelige schwarze Ausscheidungen, von einer anderen Art des Gemengtseins herrührend, geben dem Steine zuweilen ein porphyr- oder breccienartiges Ansehen. Schwarze, die Masse durchziehende Gänge oder Adern sind höchst selten. Schwefelkies ist ziemlich sparsam, meist fein, zuweilen aber auch in einzelnen bohnengroßen Körnern eingemengt; metallisches Eisen fehlt. Die Rinde ist aderig, oft rissig; mehr oder weniger, aber stets glänzend (wenn nicht durch längeres Liegen in der Erde Verwitterung eintrat), zuweilen wie gefirnisst; auch sind verschiedenartige und unvollkommene Überrundungen nicht selten.

Vier und dreißig Stücke, teils ganze Steine, teils größere oder kleinere Fragmeute, in einem Gesamtgewichte von 27 Pfund, 22 5/32, Loth, von den vielen (etwa 200 bis 300) der allda gefallenen Steine.

Die folgende Reihe von ganzen Steinen und Bruchstücken der Meteorsteine von Stannern ist die größte und vollständigste, die je von einem Steinfall zusammengebracht worden ist, und stellt die interessantesten Verhältnisse dieser Steine hinsichtlich ihrer Gestalt, ihrer Überrundung, der Mengung der Grundmasse u. s. w. dar. Sie ist, mit Ausnahme einiger Stücke, das Resultat der Bemühungen der Herren von Schreibers und von Widmanstätten, die unmittelbar nach dem Ereignis als kaiserl. Kommissäre zur Untersuchung desselben nach Stannern abgeordnet wurden, Der von dem ersteren darüber in Gilberts Annalen der Physik, B. 29, vom Jahre 1808, erstattete Bericht ist das Vollständigste, das Je über einen Meteorstein-Niederfall bekannt gemacht worden ist, und hat, nebst Biots Bericht über den Steinregen von L’Aigle am meisten zur Beobachtung und Bekanntwerdung späterhin vorgefallener Niederfälle, auf die nun mehr Aufmerksamkeit gerichtet wurde, beigetragen.

A. Ganze und fast ganze Steine, oder doch in dem Zustande, wie sie auf die Erde kamen.

1. Der größte bekannte von den bei Stannern gefallenen und nicht zertrümmerten Steinen, wahrscheinlich überhaupt der größte aller da gefallenen. — Beschrieben und abgebildet in des Direktors v. Schreibers Beiträgen zur Geschichte und Kenntnis meteorischer Stein- und Metallmassen, Seite 20, Taf. 4. — 11 Pfund 10 3/4 Loth. — 1809. 8. 1. — Von Herrn Professor Mikan in Prag gekauft. Wurde von Herrn Apotheker Heller in Iglau in einem deshalb abgelassenen Teiche aufgefunden.

2. Einer der größten von den bei Stannern gefallenen Steinen; besonders frisch, schön überrindet, auch merkwürdig wegen verschiedenartiger Beschaffenheit der Rinde. — Beschrieben und abgebildet in v. Schreibers Beiträgen, S. 27, Taf. 5. Fig. 5. — 3 Pfund 21 Loth. — 1808. 24. 1. — Wurde während des Aufenthaltes der Untersuchungs-Kommission zu Stannern, im Monate Mai 1808, bei angeordneter Aufsuchung der gefallenen Steine aufgefunden.

3. Einer von den großen Steinen von Stannern; höchst ausgezeichnet und vortrefflich erhalten; merkwürdig wegen seiner keilförmigen Gestalt und der Beschaffenheit der Rinde. — Beschrieben und abgebildet in v. Schreibers Beiträgen, Seite 30, Taf. 6. Fig. 1. — 2 Pfund 12 1/2 Loth. — 1808. 24, 2. — Wie bei 2) erwähnt, während des Aufenthaltes der Kommission aufgefunden.

4. Ebenfalls einer von den größeren Steinen; wegen der strahlenförmigen Überrundung der Grundfläche merkwürdig. — Beschrieben und abgebildet im angeführten Werke, Seite 32, Taf. 6. Fig. 2. — 1 Pfund 11 3/4 Loth: — 1808. 24. 3. — Ebenfalls in Folge der gemachten Aufforderung während der Anwesenheit der Kommission zu Stannern aufgefunden.

5. Noch einer der größeren Steine; sehr lehrreich wegen einer unvollkommen überrindeten Fläche, aus welcher die Grundmasse durchblickt. — Beschrieben und abgebildet im angeführten Werke, Seite 33, Taf. 6. Fig. 3. — 1 Pfund 6 7/8 Loth. — 1808. 24. 4. — Aufgefunden wie Nr. 2-4.

6. Ein mittelgroßer Stein, anscheinend ein Bruchstück, oder die Hälfte eines Steines, aber im Herabfallen zerbrochen; die natürliche Bruchfläche teilweise verändert (etwas braun gefärbt und mit einzelnen kleinen Rindetröpfchen besetzt), also in dem Zustande wie er auf die Erde kam. Ein sehr belehrendes Stück. — Beschrieben und abgebildet im angeführten Werke, S. 36, Taf. 6. Fig. 4. — 1 Pfund 7/8 Loth. — 1808. 24. 5. — Wurde am Tage des Ereignisses aufgefunden und später der Kommission übergeben.

7. Ganzer, mittelgroßer Stein, mit stark glänzender Rinde, an einigen Stellen etwas entblößt. — 1 Pfd. 1/4 Loth. — 1808. 24. 6. — Durch die Kommission überbracht.

8. Ein mittelgroßer ganzer Stein, wenig verletzt, einige Kanten mit hervorragenden, scharfen Linien von Rindensubstanz. - 23 3/16 Loth. — 1809. 4. 2. — Von Herrn von Well gekauft.

9. Mittelgroßer ganzer Stein. An einer Stelle ist ein Stückchen weggeschnitten und die Fläche anpoliert. — 19 7/8 Loth. — 1827. 27. 4048. — Aus der im Jahre 1827 angekauften von der Nüll’schen Mineralien-Sammlung.

10. Mittelgroßer ganzer Stein (oder doch in dem Zustande, wie er herabkam), nur mit einer kleinen frischen Bruchfläche, dann einer größeren Fläche, die während des Herabfallens entstand, braun gefärbt und mit hervorgequollenen Tröpfchen von Rindensubstanz übersäet ist. Lehrreiches, sehr interessantes Stück. — 15 Loth. — 1809. 4. 4. — Von Herrn von Well gekauft.

11. Ein kleiner Stein, fast ganz, nur an dem einen Ende, wahrscheinlich beim Fallen abgebrochen, von zungenförmiger Gestalt; von der Rinde glänzt, wahrscheinlich in Folge von Verwitterung während derselbe in der Erde lag, nur das hervorragende Adergeflechte. — 10 7/8 Loth. — 1809. 7. 1. — Von Herrn Sonsluk gekauft.

12. Ein kleiner Stein, wenig verletzt, unvollkommen prismatisch. — 10 1/2 Loth. — 1809. 4. 1. — Von Herrn von Well gekauft.

13. Ein kleiner, vollkommen ganzer, nicht im geringsten verletzter Stein; verschoben viereckig, — 6 1/2 Loth. — 1827. 27. 4045. — Aus der von der Nüll’schen Mineralien-Sammlung.

14. Ein kleiner, ganzer, fast prismatischer Stein, mit einer im Falle entstandenen, mehr oder weniger, meist jedoch sehr unvollkommen überrindeten Bruchfläche; ausgezeichnet starke Überrundung der Bruchkanten. — 6 1/2 Loth. — 1827. 27. 4046. — Aus der von der Nüll’schen Mineralien-Sammlung.

15. Kleiner Stein, vollkommen ganz (nur eine etwas gekrümmte Ecke ist abgebrochen und schwach angeklebt), die Form dreiseitigpyramidal; die Rinde schwach glänzend. — Beschrieben und abgebildet im angeführten Werke, Seite 23, Tafel 5. Fig. 1. — 5 7/10 Loth. — 1808. 24. 7. — Wie bei Nr. 2. bemerkt aufgefunden, und durch die Kommission überbracht.

16a. Ganzer, sehr merkwürdiger Stein, von einer Seite zugerundet, von der anderen kantig; auch von verschiedener Beschaffenheit der Rinde, welche, wo sie dicker ist, an den Kanten Hervorragenden bildet, die beim Festwerden der Rinde durch den Widerstand der Luft beim Herabfallen, und durch Verschiebungen an der damals zähflüssigen Oberfläche entstanden sein müssen. — 4 13/16 Lth. — 1840. 4. 5. — Von Herrn Pötschke gekauft. 

16b. Kleiner, ganzer Stein, nur eine Ecke etwas abgestoßen, und die Spitze teilweise abgeschlagen; von vierseitig pyramidaler Form mit schiefer Grundfläche; zwei Seiten dick überrindet, stark glänzend, ziemlich glatt, die anderen matter und aderiger. — Beschrieben und abgebildet im angeführten Werke, S. 24, Taf. 5. Fig. 2 a et b. — 4 1/2 Loth. — 1808. 24. 8. — Durch die bei Nr. 2 erwähnte Kommission überbracht.

17. Kleiner, ganzer, an einer Kante der Länge nach entblößter Stein, von ungewöhnlicher Form, wie ein flaches Geschiebe. — 4 7/16 Loth. — 1832. 17. 1. — Von dem k. k. Kämmerer, Grafen Eugen von Czernin, eingetauscht.

18. Kleiner, unregelmäßiger Stein, an einer Kante der Länge nach angebrochen, wodurch eine feinkörnige, fast homogen erscheinende, bläulichgraue Grundmasse, mit ein Paar sehr feinen schwarzen Adern zum Vorschein kam; ziemlich stark glänzende Rinde, mit scharfen Erhöhungen. Der ungewöhnlichen Grundmasse wegen merkwürdig. — 4 3/16 Loth. — 1808. 24. 9. — Durch die bei Nr. 2 erwähnte Kommission überbracht.

19. Sehr kleiner, ganzer, nur an einer Kante etwas angebrochener Stein, einer der kleinsten von diesem Steinfalle. — Beschrieben und abgebildet im angeführten Werke, S. 25, Taf. 5. Fig. 3. — Kaum 5/8 Loth. — 1808. 25. 1. — Durch das Kreisamt zu Iglau eingesendet.

20. Ein sehr kleiner, und, so viel bekannt, der kleinste, der bei Stannern gefallenen. Steine, vollkommen ganz, flach, fast linsenförmig. — Beschrieben und abgebildet im angeführten Werke, S. 27, Taf. 5. Fig. 4. — 7/32 Loth. — 1808. 25. 2. — Durch das Kreisamt zu Iglau eingesendet.

B. Bruchstücke.

21. Größeres Bruchstück mit Rinde, merkwürdig wegen der deutlichen Ausscheidungen von Magnetkies, wovon eine erbsengroß ist. — Beschrieben und teilweise abgebildet in dem angeführten Werke, Seite 69, Taf. 7., untere Reihe, Mittel-Figur. — 13 7/16 Loth. — 1808. 24. 10. — Durch die bei Nr. 2 erwähnte Kommission überbracht.

22. Größeres Bruchstück mit Rinde und einer unvollkommen überrindeten Bruchfläche; die Grundmasse teils grob-, teils feinkörnig, grau. — 11 1/16 Lth. — 1809. 4. 3. — Von Herrn von Well gekauft.

23. Fast rundes Bruchstück, mit ganz frischen Bruchflächen und etwas Rinde; die Gemengteile von dem gewöhnlichen mittelfeinen Korne, und vorzüglich auf einer der Flächen sehr deutlich erkennbar; auch Magnetkies ist deutlich, aber sparsam eingesprengt. — 7 9/16 Loth. — 1827. 27. 4049. — Aus der von der Nüll’schen Mineralien-Sammlung.

24. Längliches Bruchstück, mit etwas Rinde und einer anpolierten Fläche; merkwürdig wegen der Rinde, die teils glänzend, teils durch Verwitterung matt, und mit Tropfen und Perlen von Rindensubstanz besetzt ist. Ein Teil der Rinde ist auch, was höchst selten vorkommt, buntfärbig angelaufen. Die polierte Fläche zeigt Ausscheidungen des schwärzlichen Bestandteiles, daher eine unvollkommen porphyrartige Struktur. — 6 1/2 Lth. — 1808. 24. 11. — Durch die bei Nr. 2 erwähnte Kommission überbracht.

25. Bruchstück, allerseits mit sehr frischen Bruchflächen, ohne Rinde; das Gemenge ist ziemlich feinkörnig und an einigen Stellen von dunklerem Grau; der Magnetkies ist darin nicht zu unterscheiden. — 6 5/16 Lth. — 1809. 24. 12. — Durch die bei Nr. 2 erwähnte Kommission überbracht.

26. Längliches Bruchstück, mit ziemlich viel Rinde. Man sieht beinahe noch die ganze Kontur des ursprünglichen Steines. Das Stück ist deshalb merkwürdig, weil an den oberen Bruchflächen Spuren von neuer Rindenbildung sichtbar sind, und der Magnetkies daselbst bunt angelaufen ist. — 4 5/16 Loth. — 1808. 24. 13. — Durch die bei Nr. 2 erwähnte Kommission überbracht.

27. Viereckiges Bruchstück, mit abgenützter Bruchfläche und mit Rinde; in der Grundmasse sind dunkelgraue, dichte Ausscheidungen vorhanden. — Beschrieben und (nicht gut) abgebildet in dem angeführten Werke. S. 59, Taf. 7. 1 Fig, der oberen Reihe. — 3 11/16 Loth. — 1808. 24. 14. — Durch die bei Nr. 2 erwähnte Kommission überbracht.

28. Kleines Bruchstück mit poröser Rinde, von welcher ein Teil schuppig abgesprungen ist, und eine zweite matte und raue Rindenlage zum Vorscheine brachte, Merkwürdig ist dieses Fragment noch durch die Erscheinung, dass, wahrscheinlich auf einer während des Falles entstandenen Kluft, Rindensubstanz in das Innere des Steines einzudringen begann, und nun, innerhalb des Randes, der Bruchfläche aufsitzt. — Beschrieben und abgebildet im angeführten Werke, S. 38, Taf. 6. Fig. 5. — 3 1/4 Loth. — 1808. 24. 15. — Durch die bei Nr. 2 erwähnte Kommission überbracht.

29. Kleines Bruchstück mit Rinde und einer anpolierten Fläche von marmoriertem Ansehen, welche, wie die ganze Masse des Stückes (eine Seltenheit bei den Steinen von Stannern), ein Paar dünne schwarze Adern durchziehen. — 3 1/8 Loth. — 1808. 24. 16. — Durch die bei Nr. 2 erwähnten Kommission überbracht.

30. Kleines Bruchstück mit Rinde, von welcher die obere, glänzende Lage teilweise abgesprungen ist. Die Grundmasse ist dicht, dunkelgrau, hie und da sind undeutliche, kugeliche Ausscheidungen von derselben Substanz wahrnehmbar. — 2 1/2 Loth. — 1808. 24. 17. — Durch die bei Nr. 2 erwähnte Kommission überbracht.

31. Kleines Bruchstück mit Rinde. Die Grundmasse feinkörnig, von einer dünnen, schwarzen Ader durchzogen. — 1 5/32 Loth. — 1808. 24. 20. — Durch die bei Nr. 2 erwähnte Kommission überbracht.

32. Kleines Bruchstück mit Rinde; die zwei erdigen Gemengteile an ein Paar Stellen mit deutlicher Teilbarkeit. — 1 3/32 Loth. — 1808. 24. 19. — Durch die bei Nr. 2 erwähnte Kommission überbracht.

33. Acht kleine Fragmente zum Studium der Rinde und der Grundmasse. — 1 1/2 Loth. — 1808. 24. 20. — Aus dem durch die Kommission überbrachten Doubletten-Vorrate.
\subsection{Konstantinopel.}
\begin{center}
\small
Auf dem Fleischplatze, im Inneren dieser Stadt.

Juni 1805, an hellem Tage.
\end{center}
\paragraph{}
Graue, durch innige Mengung der zwei erdigen Gemengteile homogen erscheinende Grundmasse, ganz wie bei der zweiten, selteneren Varietät der Steine von Stannern; schwach glänzende Rinde.

1. Fragment mit etwas Rinde, von einer dünnen schwarzen Ader durchzogen; von einem der mehreren allda gefallenen Steine. — 7/16 Loth. — 1832. 28. 1. — Wurde vor mehreren Jahren (zwischen 1818-1820) durch Herrn Leopold Fitzingers Vermittlung von Freiherrn Nell von Nellenburg, jetzt Hofrat der k. k. Hofkammer in Wien, der den Stein durch den verstorbenen Sohn des damaligen k. k. Internuntius in Konstantinopel, Baron von Stürmer, bekam, als Geschenk erhalten.

Wir haben uns in Konstantinopel durch Reisende wiederholt, aber immer erfolglos bemüht, uns von diesem, mitten in einer großen Stadt erfolgten Meteorstein-Fall, der allda nun schon ganz vergessen ist, weitere Musterstücke zu verschaffen.
\subsection{Jonzac.}
\begin{center}
\small
(Barbezieux) Dép. de la basse Charente, Frankreich.

13. Juni 1819, 6 Uhr Morgens.
\end{center}
\paragraph{}
Lichtaschgraue Grundmasse, aus zwei ziemlich gleichförmig gemengten Substanzen, einer weißen und einer schwärzlich grauen, bestehend; die letztere fast vorherrschend und in eckigen Kryställchen oder rundlichen Körnern erscheinend. Sehr wenig und höchst fein eingemengter Magnetkies. Glänzende, aderige Rinde. — Ein der ersten, gewöhnlichen Varietät der Meteorsteine von Stannern ähnlicher Meteorit.

Ein fast ganzer Stein und Ein Bruchstück von den mehreren allda gefallenen Steinen, deren Anzahl und Gesamtgewicht nicht bekannt geworden ist.

1. Ein fast ganzer Stein; eine Ecke abgeschnitten, die Schnittfläche unvollkommen poliert; außerdem auch noch andere kleine Entblößungen des Innern. — 31 11/16 Loth. — 1829. 34. 1. — Aus der Verlassenschaft des Herrn Leman in Paris, durch Professor Desmarest eingetauscht.

2. Fragment mit Rinde und ganz frischem Bruche. — 4 1/4 Loth. — 1840. 4. 3. — Aus der Heuland'schen, später Heath'schen Meteoriten-Sammlung durch Herrn Pötschke gekauft. Stammt aus der de Drée'schen Mineralien-Sammlung.
\subsection{Bialistock.}
\begin{center}
\small
(Belostock), Dorf Knasti-Knasti, im gleichnamigen Gouv., Russland.

5. Oktober alten Styls, 1827, zwischen 9 und 10 Uhr Morgens.
\end{center}
\paragraph{}
Lichtaschgraue, wenig zusammenhängende, nicht schwer zerreibliche Grundmasse, aus einem schneeweißen, einem graulich schwarzen und einem schmutzig spargelgrünen Minerale gemengt; die letzteren, nämlich das schwarze und grüne Mineral, treten auch in größeren eckigen Körnern, und zum Teile auch in rundlichen Partien auf, und verleihen dem Ganzen ein breccien- und konglomeratartiges Aussehen; auch die weiße feldspatartige Substanz sondert sich an einigen Stellen, doch noch immer mit den anderen Substanzen gemengt, deutlicher aus, und verursachet dadurch eine gefleckte Zeichnung. Der Magnetkies ist in geringer Menge vorhanden. Glänzende poröse Rinde. (Nahe verwandt mit den Steinen von Lontalax, Nobleborough und Mässing.)

1. Fragment mit Rinde von einem der mehreren allda gefallenen Steine, wovon der größte 4 Pfund wog. — 3 3/8 Loth. — 1839. 22. 1. — Aus der Mineralien-Sammlung der königl. Universität zu Berlin durch Professor Weiss eingetauscht.
\subsection{Lontalax.}
\begin{center}
\small
Friederichshamm, Switaipola (nach Chladni Sawotaipola), Gouv. Wiburg, Finnland.

13. Dezember 1813.
\end{center}
\paragraph{}
Lichtgraue, körnige, wenig zusammenhängende Grundmasse, angefüllt mit Einmengungen von kleinen, olivengrünen, dann schwärzlichen, eckigen, selten rundlichen Körnern, die vorwaltend sind und dem Ganzen ein porphyr- oder breccienartiges Aussehen geben, endlich weißen feldspatartigen Körnern. Ein Korn von Magnetkies ist deutlich wahrnehmbar, sonst scheint derselbe fein eingesprengt zu sein. Die Rinde glänzend, aderig.

1. Bruchstück von einem der mehreren allda gefallenen, aber bei dem Schmelzen des Eises meist in einen See versunkenen Steine; etwa die Hälfte eines kleinen Steines; mit Rinde und einer geschnittenen Fläche. — 1 Loth, schwach. — 1832. 30. 1. — Von dem verstorbenen Grafen Gregor von Razoumovsky in Tausch erhalten.
\subsection[Nobleborough.]{Nobleborough,}
\begin{center}
\small
oder Nobleboro, im Staate Maine, in den vereinigten Staaten von Nord-Amerika.

7. August 1823, zwischen 4 und 5 Uhr Abends.
\end{center}
\paragraph{}
In jeder Beziehung dem Steine von Lontalax so ähnlich, dass die dort gegebene Beschreibung auch vollkommen auf den Stein von Nobleborough angewendet werden kann; nur scheint letzterer noch weniger Zusammenhang zu besitzen, und daher zerreiblicher zu sein.

1. Drei Bröckchen, wovon das größte mit Rinde, von einem allda gefallenen Steine von 4 bis 6 Pfund, (außer welchem noch andere gefallen sein sollen.) — 3/8 Loth. — 1838. 25. 5. — Aus der ehemals Heuland‘schen Meteoriten-Sammlung durch Herrn Pötschke gekauft. Herr Heuland erhielt diese Fragmente durch Professor Silliman aus Nord-Amerika.
\subsection{Mässing.}
\begin{center}
\small
(St. Nicolas) bei Altötting, Landgericht Eggenfelden in Bayern.

13. Dezember 1803, zwischen 10 und 11 Uhr Vormittags.
\end{center}
\paragraph{}
Graulich weiße, ziemlich lockere Grundmasse, meist aus einem, wie Feldspat aussehenden, schneeweißen Mineral bestehend‚ worin kuglige Ausscheidungen von unreiner, pistaziengrüner Farbe, mit ziemlich vollkommenen schiefwinklichen Teilungsflächen, dann eckige, schwarze, und endlich ganz kleine Körner von olivengrüner Farbe eingemengt sind. Von metallischen Gemengteilen ist Magnetkies allein deutlich zu erkennen. — Ein höchst ausgezeichneter, dem Steine von Lontalax verwandter Meteorstein.

Zwei kleine Fragmente von einem daselbst einzeln gefallenen Steine von 3 1/4 Pfund.

1a. Ein kleines Fragment ohne Rinde. — 3/32 Loth, schwach. — 1832. 29. 3. — Durch Direktor von Schreibers im Jahre 1832 als Geschenk erhalten, welcher dasselbe im Jahre 1811 von Herrn Lavater in Zürch bekam.

1b. Kleines Fragment mit frischem Bruch und ohne Rinde. — 3/32 Loth. — 1843. 22. 1. — Von Herrn Johann von Charpentier, Bergwerks-Direktor zu Bex in der Schweiz, in Tausch erhalten. Herr von Charpentier bekam das Fragment von Chladni.
\subsection{Parma.}
\begin{center}
\small
Casignano, oder eigentlich Pieve die Casignano, bei Borgo St. Domino, im Herzogtum Parma.

19. April 1808, Mittags.
\end{center}
\paragraph{}
Lichtgraue Grundmasse, mit vielen kleinen kugelichen und eckigen Ausscheidungen, welche letztere dem Steine ein breccienartiges Ansehen geben; mit fein eingesprengtem gediegenen Eisen und Magnetkies, welch letzterer vorwaltet und auch in größeren Partien auftritt. Schwach glänzende, fast matte Rinde.

Zwei Bruchstücke von einem der allda in größerer Anzahl gefallenen Steine.

1. Bruchstück mit Rinde und einer anpolierten Fläche. — 3 19/32 Loth. — 1816. 31. 33. c. — Auf Vermittlung des Direktors von Schreibers während seiner Anwesenheit zu Paris im Jahre 1815, durch Tausch aus dem königl. Museum der Naturgeschichte erhalten.

2. Kleines Bruchstück mit Rinde und einer nicht polierten Schnittfläche. — 1 Loth — 1841. 14. 11. — Aus der Heuland'schen Meteoriten-Sammlung durch Herrn Pötschke gekauft. Stammt aus der Mineralien-Sammlung des Marquis de Drée. (Durch Verwechslung mit dem Fallorte Berlanguillas erhalten, passt aber an das vom Pariser Museum erhaltene Stück Nr. 1 an, ist also davon in Paris abgebrochen worden.)
\subsection[Siena.]{Siena,}
\begin{center}
\small
im Großherzogtum Toskana.

16. Juni 1794, nach 7 Uhr Abends.
\end{center}
\paragraph{}
Hellgraue, zuweilen rostbraun gefleckte Grundmasse, mit vielen, teils lichtgrünlichen, teils schwärzlichen, selten kugeligen, meist eckigen Ausscheidungen, die dem Ganzen ein breccien- oder porphyrartiges Ansehen verleihen; mit vielem, größtenteils fein eingesprengten, manchmal aber auch in Körnern eingewachsenen Magnetkies und weniger, fein eingesprengtem metallischen Eisen. Matte, zum Teil rissige und dadurch weiß geaderte Rinde.

Drei vollkommen ganze, aber sehr kleine, dann drei ganze, aber angebrochene oder angeschnittene Steine, und Ein Bruchstück (etwas mehr als die Hälfte eines Steines), zusammen also sieben Stücke von den sehr vielen, jedoch meist kleinen allda gefallenen Steinen.

1. Ein sehr kleiner ganzer Stein. — 7/32 Loth, schwach. — 1832. 29. 4. — Geschenk von Herrn Direktor von Schreibers.

2. Ein ebenfalls sehr kleiner ganzer Stein. — 9/32 Loth, schwach. — 1817. 47. 1. — Durch Vermittlung des Professors, Freiherrn von Jacquin, aus Italien zu Kauf erhalten.

3. Ein kleiner, fast ganzer Stein, mit einer Bruchfläche. — 17/32 Lth. — 1817. 47. 2. — Wie Nr. 2 durch Freiherrn von Jacquin erhalten.

4. Ein kleiner, länglicher, ganzer Stein, mit Rinde von zweifacher Beschaffenheit. — 5/8 Loth, schwach. — 1827. 27. 4051. — Aus der im Jahre 1827 angekauften von der Nüll'schen Mineralien-Sammlung.

5. Ein für diese Lokalität nicht ganz kleiner, fast ganzer Stein, mit einer größeren und einigen kleineren Bruchflächen. — 1 13/16 Loth. — 1817. 47. 3. — Wie Nr. 3 durch Freiherrn von Jacquin erhalten.

6. Ein größeres Bruchstück (etwas mehr als die Hälfte eines Steines), mit einer Bruch- und einer anpolierten Fläche. — Beschrieben und abgebildet in v. Schreibers Beiträgen, S. 14, Taf. 2. und S. 61, Taf 7. — 1 3/4 Lth. — 1809. 20. 1. — Vom Obersten von Tihavsky als Geschenk erhalten.

7. Ein größerer, fast ganzer Stein, mit einer Schnitt- und einer polierten Fläche, auch einer Bruchfläche mit zwei Vertiefungen, worin sich eine schwarze Substanz zeigt, Die Rinde zum Teil mit Eindrücken. — 6 1/32 Lth. — 1822. 20. 1. — Durch Herrn Chierici aus Florenz zu Kauf erhalten.
\subsection[Ensisheim.]{Ensisheim,}
\begin{center}
\small
im ehemaligen Elsass, jetzt Dép. du Haut-Rhin, Frankreich.

7. November 1492, zwischen 11 und 12 Uhr Mittags.
\end{center}
\paragraph{}
Dunkelgraue, rostbraun gefleckte Grundmasse, stellenweise lichter, wodurch ein unvollkommen breccienartiges Aussehen entsteht, das auf polierten Flächen noch deutlicher wahrzunehmen ist. Das nicht häufig und meist fein eingesprengte metallische Eisen, und der vorwaltende, teils fein eingesprengte, teils in kleinen Flecken und Adern auftretende Magnetkies sind, vorzüglich ersteres, auf den Bruchflächen Schwer, dagegen auf polierten Flächen deutlich zu erkennen; sehr ausgezeichnete und zahlreiche, schwarze, glänzende Ablösungsflächen, die den Stein fast Schiefrig, und daher leicht spaltbar machen; auch schwarze glänzende Blättchen‚ die kurze Ablösungsflächen sind. — Ein höchst eigentümlicher, mit keinem anderen verwechselbarer Meteorstein.

Ein großes und vier kleine Bruchstücke, sämtlich ohne Rinde, von einem sehr großen, einzeln gefallenen Steine von 270 Pfund.

1. Ein großes Bruchstück, — 24 1/8 Lth. — 1813. 40. 1. — Durch Vermittlung des kaiserl. Ministers Freiherrn von Hügel, während der Invasion der verbündeten Mächte im Jahre 1813, aus Colmar in Elsass als Geschenk erhalten.

2. Kleineres Bruchstück, — 5 1/32 Lth. — 1841. 6. 71. — Von der königl. sächsischen Mineralien-Niederlage zu Freiberg gekauft.

3. Bruchstück, — 4 3/32 Loth. — 1827. 27. 4053.
— Aus der von der Nüll’schen Mineralien-Sammlung.

4. Längliches Bruchstück, mit zwei anpolierten Flächen. — 2 3/4 Loth. — 1825. 42. 59. — Aus der Mineralien-Sammlung des Grafen Fries gekauft.

5. Kleines Bruchstück , mit einer anpolierten Fläche, — 1 5/8 Loth. — Von 1809. 19. 1. — Geschenk vom verstorbenen Major v. Schwarz.

Der Meteorstein von Ensisheim ist der älteste von allen, die sich bis an unsere Zeit der Zertrümmerung, dem Verstreuen und endlichem Vergessen und Wegwerfen entzogen haben. Er verdankt seine Erhaltung dem Umstande, dass Kaiser Maximilian 1. während seines Falles sich in oder bei Ensisheim befand, und den Stein in den Chor der Kirche zu Ensisheim aufhängen ließ, mit dem Verbote, für Niemanden etwas davon abzuschlagen. In der Revolutionszeit wurde der Stein auf die öffentliche Bibliothek zu Colmar gebracht, und viele Stücke davon abgeschlagen. Er befindet sich jetzt, beträchtlich vermindert, neuerdings in der Kirche zu Ensisheim.
\subsection{L'Aigle.}
\begin{center}
\small
(La Vassolerie, Fontenil, St. Michel, St. Nicolas, Bas-Vernet etc.\footnote{Es werden hier und bei anderen ausgedehnteren Steinfällen mehrere Orte genannt, teils weil die Steine hei allen diesen Orten niederfielen, teils weil sie zuweilen mit verschiedenen Ortsbezeichnungen in Handel kommen, und man sie dann für das Produkt verschiedener Ereignisse halten könnte.}) Normandie. Dépt. de l'Orne, Frankreich.

26. April 1803, 1 Uhr Nachmittags.
\end{center}
\paragraph{}
Teils licht-, teils dunkelgraue, meist rostbraun gefleckte Grundmasse; die lichteren und dunkleren Partien entweder fleckenartig nebeneinander, oder die lichte Grundmasse von einem dunkleren, bald dickeren, bald dünneren aderigen Gewebe durchzogen, dessen Zellen die lichteren Stellen sind. In diese ungleich gefärbte Grundmasse sind breccien- oder porphyrartig lichtere oder dunklere, eckige Körner oder Ausscheidungen eingemengt (zuweilen auch schwarze, bohnengroße Partien, durch das Zusammenfließen des schwarzen Aderngeflechtes entstanden). Das gediegene Eisen ist in ziemlicher Menge, zum Teil grob, der Magnetkies nur äußerst fein eingesprengt. Schwarze Ablösungsflächen sind nicht selten. Die Rinde ist matt, nicht rau. — Ein Meteorstein von eigentümlicher Beschaffenheit.

Dreizehn Stücke von den sehr vielen (2000 bis 3000) der allda gefallenen Steine, darunter vier ganze Steine.

1. Großer, ganzer, ringsum überrundeter Stein. — 2 Pfd. 22 Lth. — 1841. 14. 1. — Aus der Heuland'schen Sammlung durch Herrn Pötschke gekauft. Herr Heuland kaufte den Stein von Herrn Lambotin in Paris.

2. Großer, ganzer, überrundeter Stein, von dem ein dabei befindliches und anpassendes Eck abgebrochen ist (auch die Kanten sind hie und da, wie gewöhnlich, etwas abgestoßen); an ein Paar Seiten mit Eindrücken. — Beschrieben und abgebildet in von Schreibers Beiträgen, S. 12, Taf. 2. — 1 Pfund 30 3/8 Loth — \mars 1. 6. — Wurde durch den verstorbenen k. k. Naturalien-Kabinetts-Direktor Stütz, im Jahre 1803 von einem Franzosen gekauft.

3. Fast ganzer Stein, mit einer anpolierten Fläche. — 22 1/8 Loth. — 1840. 11. 2. — Von Herrn von Scala gekauft, Stammt aus der gräflich Razoumovsky'schen Mineralien-Sammlung.

4. Ein sehr kleiner, aber ganzer Stein, nur an einer Kante, und auch hier zum Teil während des Falles verbrochen und wieder unvollkommen überrindet; hellgraue Grundmasse. — 29/32 Loth. — 1816. 36. 35. — Durch Direktor v. Schreibers während seiner Anwesenheit in Paris im Jahre 1815 vom Mineralienhändler Lambotin erkauft.

5. Ein Fragment (wohl 2/3 des ganzen Steines); mit anpolierter Fläche. — 8 3/4 Loth. — 1827. 27. 4050. — Aus der Mineralien-Sammlung des Herrn von der Nüll.

6. Ein frisches Bruchstück mit etwas Rinde, und den in der Beschreibung erwähnten, schwarzen, bohnengroßen Einmengungen. — 6 23/32 Lth. — 1824. 48. 1. — Durch den Herausgeber zu Kauf erhalten.

7. Bruchstück mit gekrümmter Ablösungsfläche; eine polierte Fläche ist rostbraun gefleckt. — 3 9/32 Loth. — 1808. 4. 1. — Durch Herrn Apotheker Moser in Paris gekauft.

8. a. und b. Zwei Bruchstücke mit Rinde und anpolierten Flächen, welche viele rostbraune Flecken zeigen. (Waren zu einem Versuche einige Zeit in der Erde vergraben). — 1 5/32 Loth und 19/32 Loth. — Aus den Doubletten. — Von Herrn Lambotin in Paris im Jahre 1815 gekauft.

9. a. und b. Zwei Bruchstücke mit rostbraunen, anpolierten Flächen. — 3/4 Loth und 11/16 Loth. — Von 1816. 40. 31. — Durch Direktor v. Schreibers im Jahre 1815 in Paris gekauft.

10. Kleines Bruchstück mit frischem Bruche; die Rinde mit weißen Adern. — 3/4 Loth. — 1816. 40. 31. — Wie Nr. 9 angekauft.

11. Ansehnliches Bruchstück, mit großer, frischer Bruchfläche, welche das in der Beschreibung erwähnte aderige Gewebe, wodurch ein marmoriertes oder breccienartiges Ansehen entsteht, deutlich wahrnehmen lässt; mit Rinde, — 13 3/4 Loth — 1843. 29. 1. — Von Hrn. Francois Marguier in Tausch erhalten.

Die Meteorsteine von L’Aigle sind die verbreitetsten und gemeinsten in Mineralien-Sammlungen. Ein Mineralien-Händler in Paris, Herr Lambotin, kaufte davon so viel auf, als er in L’Aigle und der Umgegend zusammenbringen konnte. Lange war der Preis derselben 8 bis 10 Francs für die Unze. Jetzt ist davon in Paris nichts mehr zu erhalten.
\subsection[Liponas.]{Liponas,}
\begin{center}
\small
(in Chladni, vielleicht durch einen Druckfehler, unrichtig Laponas) bei Pont de Vesle und Bourg en Bresse, Dépt. de l'Ain, Frankreich.

7. September 1753, 2 Uhr Nachmittags.
\end{center}
\paragraph{}
Dunkel asch- oder bläulichgraue Grundmasse mit schwärzlich grauen Partien, welche dieselbe durchziehen und fleckig oder marmoriert aussehen machen; beide mit Rostflecken und ziemlich deutlichen, aber mit der Grundmasse fest verwachsenen, kugeligen Ausscheidung; mit fein und mittelfein eingesprengtem, metallischen Eisen und sehr fein eingesprengtem Magnetkies. Matte Rinde. — Gleicht fast vollkommen den Meteorsteinen von L'Aigle.

Zwei Bruchstücke von einem der zwei allda gefallenen Steine, welche zusammen 31 1/2 Pfund wogen.

1. Fragment mit viel Rinde und ausgezeichneten Eindrücken an der Oberfläche. — 4 15/32 Loth. — 1838. 25. 3. — Aus der Heuland'schen Meteoriten-Sammlung durch Herrn Pötschke gekauft. Das Stück lag früher in der von Herrn Heuland erkauften Mineralien-Sammlung des Herrn Marquis de Drée in Paris.

2. Ein ganz kleines, anpoliertes Bruchstück, ohne Rinde. — 5/16 Loth. — 1832. 29. 1. — Geschenk von Herrn Direktor von Schreibers, welcher dieses kleine Fragment während seines Aufenthaltes zu Paris im Jahre 1815, aus der de Drée'schen Mineralien-Sammlung erhielt.

Nach dem, was Bigot de Morogues in dem Werke: Memoire sur les chutes des pierres, Seite 334, von einem in dem Museum de Drée befindlichen Meteorstein von unbekannter Abkunft erwähnt, ist es wohl nur Vermutung, dass die, obwohl nur in sehr wenig Sammlungen vorhandenen Steine von Liponas wirklich von dieser Lokalität sind, Es heißt da, nachdem die auf unsere Exemplare, die aus des Marquis de Drée Sammlung stammen, vollkommen anwendbare Diagnose gegeben ist: Je presume qu'elle (la pierre d'origine inconnue) peut être l'une de celles tombées à Liponas en 1753, ce qui paroit probable à M. Léman, tant à cause de la manière, dont elle est parvenue à M. de Drée, que par son volume et ses autres caractères.
\subsection{Chantonnay.}
\begin{center}
\small
Zwischen Nantes und La Rochelle, Dépt. de la Vendée, Frankreich.

5. August 1812, Nachts 2 Uhr.
\end{center}
\paragraph{}
Die Grundmasse zeigt stellenweise eine ganz verschiedene Beschaffenheit; sie ist nämlich teils, und zwar bei weitem vorherrschend, schwarz, schwach schimmernd und dicht, wie mancher Basalt; teils dunkelgrau, braun gefleckt, mit schwarzen Streifen oder Linien durchzogen und daher von marmoriertem Ansehen. (Auch die schwarze Grundmasse hat, was aber nur auf polierten Flächen zu bemerken ist, vereinzelte, meist aber undeutliche, lichtere Flecken, und ist mit einem breiten, noch schwärzeren, aderigen Geflechte durchzogen). Ziemlich viel teils fein, teils in hirsekorngroßen Körnern eingesprengtes metallisches Eisen; weit weniger und höchst fein eingesprengter Magnetkies. Undeutliche, matte Rinde. — Ein höchst eigentümlicher Meteorstein; nur die lichteren Stellen gleichen zum Teile den Steinen von Seres und Barbotan.

Ein großes und drei kleinere Fragmente von einem einzeln gefallenen Steine von 69 Pfund.

1. Großes Bruchstück; die schwarze Grundmasse vorherrschend; hie und da Rinde; mit einer anpolierten Fläche. — 4 Pfund 5 1/4 Loth. — 1818. 38. 1. — Auf Vermittlung des Herausgebers während seines Aufenthaltes zu Paris im Jahre 1818 von Professor Brochant zu Kauf erhalten.

2. Bruchstücke mit polierter Fläche, ohne Rinde, von dem Stücke Nr. 1 abgeschnitten. — 7 3/8 Loth. — Von 1818. 38. 1. — Wie Nr. 1 erhalten.

3. Frisches Bruchstück, ganz schwarz, zum Teil verrostet, mit einer undeutlichen Ablösungsfläche. — 12 23/32 Loth. — 1834. 19. 12. — Von Herrn Doktor Bondi in Dresden gekauft.

4. Kleines Bruchstück, grau und schwarz gefleckt; ohne Rinde, — 2 1/32 Loth. — Aus den Doubletten. — Von Herrn G. B. Sowerby Sohn in London erhalten.
\subsection[Renazzo.]{Renazzo,}
\begin{center}
\small
bei Cento, Provinz Ferrara, im Kirchenstaate.

15. Janner 1824, zwischen 8 und 9 Uhr Abends.
\end{center}
\paragraph{}
Matte, schwarze Grundmasse, mit reichlich eingemengten, mit der Grundmasse porphyrartig und fest verwachsenen, lichtgrauen, kugelichen Ausscheidungen; ziemlich viel metallisches Eisen, teils sehr fein, teils gröblich, meist in die Grundmasse, selten in die kugelichen Ausscheidungen eingesprengt und die letzteren oft ringförmig umgebend; der Magnetkies, wenn er vorhanden ist, so fein eingesprengt, dass er nicht unterschieden werden kann, Matte, oder schwach schimmernde Rinde, mit rundlichen, wie schuppig aussehenden Erhöhungen. — Ein höchst merkwürdiger Meteorstein von ganz eigentümlichem Aussehen, fast wie Obsidianporphyr.

Ein Fragment und ein Blättchen von einem der drei allda aufgefundenen Steine.

1. Fragment mit Rinde von zweierlei Beschaffenheit und einer anpolierten Fläche. — 2 7/16 Loth. — 1839. 12. 1. — Von Professor Abbate Ranzani in Bologna in Tausch erhalten.

2. Plättchen mit zwei anpolierten Flächen und mit etwas Rinde (von Nr. 1 abgeschnitten). — 7/32 Loth. — Von 1839. 12. 1. —
\subsection{Richmond.}
\begin{center}
\small
Chesterfield-County, Staat Virginien, Nord-Amerika.

4. Juni 1828, 9 Uhr Morgens.
\end{center}
\paragraph{}
Schwarzgraue, weißlichgrau gesprenkelte und rostbraun gefleckte Grundmasse, worin sich kleine Höhlungen befinden; mit vielen kugeligen Ausscheidungen, zum Teile von schmutziggrüner Farbe; mit viel eingesprengtem, fein zerteiltem Magnetkies (der, wie bei vielen anderen Meteoriten, auf Bruchflächen deutlicher zu sehen ist, als auf polierten Flächen) und mäßig und mittelfein eingesprengtem metallischen Eisen, Der Magnetkies kleidet einige der oben erwähnten Höhlungen aus, und ist darin zuweilen kugelig und bunt angelaufen. In einer Vertiefung eines der Bruchstücke ist ein Eisenkorn sichtbar. Matte, poröse und, wie es scheint, leicht ablösbare Rinde. — Ein merkwürdiger Meteorstein, von ganz eigentümlicher Beschaffenheit.

Drei Bruchstücke von einem einzeln gefallenen Steine von 4 Pfund.

1. Frisches Fragment mit Rinde, — 3 7/8 Loth. — 1840. 19. 4. — Von Herrn Heuland in London gekauft, der das Stück von Herrn Shepard aus Nord-Amerika erhielt.

2. Bruchstück ohne Rinde, — 3 21/32 Loth. — 1834. 31. 21. — Durch Baron Lederer, k. k. General-Konsul in New-York, in Tausch erhalten.

3. Kleines Bruchstück mit einer anpolierten Fläche. — 1/2 Loth. — Von 1830. 11. 14. — Ebenfalls durch Baron Lederer aus Nord-Amerika in Tausch erhalten.
\subsection[Weston.]{Weston,}
\begin{center}
\small
im Staate Connecticut, Nord-Amerika,

14. Dezember 1807, 6 1/2 Uhr Morgens.
\end{center}
\paragraph{}
Die Grundmasse zeigt zwei verschiedene Farbennuancen, eine dunkelaschgraue und eine helle, graulichweiße, die wohl meist in einem und demselben Steine neben einander auftreten, vielleicht aber doch jede für sich auch ganze, wenn gleich kleine Steine konstituieren mögen. Jedenfalls sind von den Fragmenten, die uns zu Gebote stehen, oder die wir zu sehen Gelegenheit hatten, die einen manchmal bloß hell graulichweiß, und dann meist mit braunen Rostflecken besäet, die anderen bloß dunkelaschgrau, so dass man Steine von verschiedenen Steinfällen vor sich zu haben glaubt. In anderen meist größeren Stücken, sieht man jedoch die hellgraue Nuance bald in größeren Partien, bald in Flecken in der dunkelgrauen auftreten und überzeugt sich dadurch leicht von der Identität des Fund- oder Fallortes. Höchst ausgezeichnet sind in den Meteorsteinen von Weston, die in großer Menge und Vollkommenheit, aber nur in geringer Größe auftretenden kugligen Ausscheidungen, die jedoch in den dunkleren Partien weit ausgezeichneter erscheinen. Metallisches Eisen ist in ziemlicher Menge vorhanden aber meist fein eingesprengt; noch feiner der auf Bruchflächen leicht wahrnehmbare Magnetkies. Die Rinde ist sehr rau und uneben, matt oder schimmernd. — Eine sehr charakteristische, leicht erkennbare Varietät von Meteorsteinen.

Fünf Fragmente von ungleicher Beschaffenheit von den sehr vielen und mitunter sehr großen daselbst gefallenen Steinen.

1. Fragment mit frischem Bruch und unvollkommener Rinde; die Substanz des Steines vorherrschend dunkelgrau mit lichtgrauen Flecken. — 3 Loth. — 1840. 4. 4. — Aus der ehemals Heuland'schen, später Heath'schen Meteoriten-Sammlung durch Herrn Pötschke angekauft. Stammt aus der de Drée'schen Mineralien-Sammlung.

2. Fragment mit anpolierter Fläche ohne Rinde; dunkelgraue Grundmasse mit sehr vielen und ausgezeichneten kugligen Ausscheidungen; die Bruchfläche zum Teil rostbraun gefleckt. — 2 9/16 Loth. — 1812. 13. 6. — Von dem verstorbenen Mineralien-Händler Barton eingetauscht.

3. Fragment mit sehr unebener Rinde; die Grundmasse dunkelgrau mit einzelnen lichtgrauen Flecken. — 2 13/32 Loth. — 1838. 8. 1. — Von der Frau Johanna von Henikstein, geborenen von Dieckmann-Secherau eingetauscht. Befand sich früher in der Mineralien-Sammlung des k. k. Hofrates von Gersdorf.

4. Kleines Fragment mit etwas Rinde. Lichtgraue Grundmasse mit Rostflecken; die kugligen Ausscheidungen nicht sehr deutlich; das metallische Eisen und der Magnetkies fein eingesprengt. — 1 11/32 Loth. — 1821. 50. 42. — Durch Baron Lederer, k. k. General-Konsul in New-York, von Dr. Mitchill in Tausch erhalten.

5. Kleines Fragment ohne Rinde; die Grundmasse teils hellgrau mit Rostflecken, teils dunkelgrau mit kugligen Ausscheidungen. — 1 7/32 Loth. — 1812. 13. 7. — Mit Nr. 2 von dem verstorbenen Mineralien-Händler Barton eingetauscht. —
\subsection[La Baffe.]{La Baffe,}
\begin{center}
\small
2 Lieues südlich von Épinal, Dépt. des Vosges, Frankreich.

13. September 1822. 7 Uhr Morgens.
\end{center}
\paragraph{}
Lichtaschgraue oder graulichweiße, rostbraun gefleckte, durch eine große Menge von klein kugligen Ausscheidungen körnig erscheinende Grundmasse; mit vielem teils fein, teils mittelfein eingesprengten metallischen Eisen und sehr fein eingesprengtem Magnetkies; matte oder schwach schimmernde Rinde. — (Ist von den lichteren Abänderungen der Steine von Weston nicht zu unterscheiden.)

1. Fragment mit Rinde und kleiner, (ohne Smirgel) unvollkommen anpolierter Fläche (von einem einzeln gefallenen Steine von unbekanntem Gewichte). — 15/16 Loth. — 1840. 29. 2. — Vom königl. Museum der Naturgeschichte zu Paris auf Vermittlung des Herausgebers, während seines Aufenthaltes zu Paris in Tausch erhalten.
\subsection{Benares.}
\begin{center}
\small
(Krakhut) in Ostindien.

13. Dezember 1798. 8 Uhr Abends.
\end{center}
\paragraph{}
Lichtgraue Grundmasse, ganz angefüllt mit teils kugligen, teils unvollkommen nierförmigen, oder seltener auch eckigen Ausscheidungen von grünlicher Farbe, die mit der Masse nur wenig zusammenhängen, daher aus der Grundmasse hervorragen, oder beim Herausfallen kuglige Eindrücke hinterlassen. Von den metallischen Einmengungen ist der Magnetkies in größerer Menge als das gediegene Eisen, beide jedoch ziemlich sparsam vorhanden. Matte Rinde, durch welche noch die eingemengten Kugeln zu unterscheiden sind.

Drei Bruchstücke von den vielen allda gefallenen Steinen.

1. Großes Fragment mit ausgezeichneten und großen Kugeln; mit Rinde. — 1 Pfund, 1/16 Loth. — 1840. 4. 1. — Aus der Heath'schen Meteoriten-Sammlung durch Hrn. Pötschke gekauft. Herr Heath bekam das Stück in Madras.

2. Bruchstück mit Rinde von doppelter Beschaffenheit und einer anpolierten Fläche. — Beschrieben und abgebildet in v. Schreibers Beiträgen, Seite 62. Taf. 7. — 4 11/16 Loth. — 1807. 44. 1. — Geschenk von dem verstorbenen Lord Greville in London.

3. Bruchstück mit Rinde und einer frischen Bruchfläche; die eingeschlossenen Kügelchen sehr klein — 1 3/16 Loth. — 1838. 40. 1. — Von Herrn Doktor Jakob Baader in Wien eingetauscht. Ist der kleinere Teil eines Fragmentes, das aus der Heuland'schen, später Heath'schen Meteoriten-Sammlung stammt.
\subsection{Gouvernement Poltava.}
\begin{center}
\small
Ohne nähere Angabe des Fundortes erhalten; (nicht zu verwechseln mit Kuleschofka, dass ebenfalls im Gouv. Poltava liegt).

Auch die Fallzeit ist nicht mitgeteilt worden.
\end{center}
\paragraph{}
Dunkelaschgraue Grundmasse, ganz erfüllt mit einer Menge von kugligen, zuweilen auch eckigen Ausscheidungen von schmutzig grünlichgrauer Farbe. Der Magnetkies, zuweilen bunt angelaufen, sondert sich in größeren körnigen Partien aus, ist jedoch meist nur sehr fein eingesprengt. Das metallische Eisen ist in ziemlicher Menge und meist fein eingesprengt. Matte, poröse Rinde. — Einer der aus gezeichnetesten Meteorsteine‚ am nächsten den Steinen von Weston und Krasno-Ugol verwandt.

1. Bruchstück mit Rinde und einer anpolierten Fläche. — 5 1/8 Loth gut. — 1838. 28. 1. — Von der kaiserl. russischen Akademie der Wissenschaften zu Petersburg durch Professor Kupffer in Tausch erhalten.

Über diesen nicht öffentlich bekannt gewordenen Steinfall), der, wie schon oben bemerkt wurde, mit dem von Kuleschofka nicht zu verwechseln ist, fehlen alle historischen Nachrichten.
\subsection{Krasno-Ugol.}
\begin{center}
\small
(Krasnoi-Ugol) Gouv. Räsan, Russland.

9. September 1829.
\end{center}
\paragraph{}
Dunkelgraue Grundmasse, etwas dunkler, als bei dem Steine aus dem Gouv. Poltava, mit welchem der Meteorit von Krasno-Ugol fast vollkommen identisch ist; nur zeigt das kleine Fragment keine größeren Ausscheidungen von Magnetkies; auch ist die Rinde etwas verschieden, weniger porös und mehr glatt.

1. Fragment mit Rinde und einer anpolierten Fläche — 19/32 Loth. — 1839. 28. 1. — Von der Mineralien-Sammlung der königl. Universität zu Berlin durch Herrn Professor Weiss in Tausch erhalten. Wurde von dem dort aufbewahrten Fragment abgeschnitten, welches diese Universität aus der Sammlung der kaiserl. Akademie der Wissenschaften in Petersburg durch Herrn Professor Kupffer erhielt.
\subsection[Erxleben.]{Erxleben,}
\begin{center}
\small
zwischen Magdeburg und Helmstädt, preußische Provinz Sachsen.

15. April 1812, 4 Uhr Nachmittags.
\end{center}
\paragraph{}
Dunkelaschgraue, sehr dichte und auf Bruchflächen ziemlich homogene Grundmasse, mit etwas dunkleren, klein kugeligen Ausscheidungen, die auf Bruchflächen fast gar nicht, deutlich hingegen auf polierten Flächen zu erkennen sind; viel, aber sehr fein und gleichförmig eingesprengtes gediegenes Eisen; viel, höchst fein eingesprengter Magnetkies, der, wie gewöhnlich, auf Bruchflächen leichter wahrzunehmen ist, als auf polierten Flächen. (Das Umgekehrte gilt vom metallischen Eisen.) Dünne, matte Rinde, die zuweilen nur in Flecken und Pünktchen auftritt und wie ausgeschwitzt aussieht. — Ein durch seine Dichtheit, anscheinende Homogenität der Grundmasse, und das feine und gleichförmige Gemenge der letzten mit den zwei metallischen Gemengteilen sehr ausgezeichneter Meteorstein; von allen anderen, mit Ausnahme desjenigen aus dem Gouvernement Simbirsk höchst verschieden.

1. Ein dreieckiges Bruchstück, von einen einzeln gefallenen Steine von 4 1/2 Pfund; mit etwas Rinde und einer anpolierten Fläche. — 3 Loth — 1814. 22. a. 1. — Geschenk von dem verstorbenen Professor Blumenbach in Göttingen.
\subsection{Gouvernement Simbirsk.}
\begin{center}
\small
Ohne nähere Angabe des Fallortes und ohne Angabe der Fallzeit erhalten.
\end{center}
\paragraph{}
Dunkelgraue, feste und dichte Grundmasse, aus welcher auf polierten Flächen kleine, dunklere, ins grünlichgraue ziehende Körner unterscheidbar sind; mit mäßig viel, jedoch meist mikroskopisch fein und nur in einzelnen Körnchen etwas gröber eingesprengtem metallischen Eisen, und höchst fein eingesprengtem Magnetkies, Matte, dünne, unterbrochene und fast nur schorfartige Rinde. (So wenigstens an dem kleinen uns zu Gebote stehenden Stücke). — Ein sehr merkwürdiger und eigentümlicher, nur dem Meteorsteine von Erxleben verwandter Meteorit.

1. Fragment mit Rinde und einer kleinen anpolierten Fläche. — 17/32 Loth. — 1838. 28. 4. — Von der kaiserlich russischen Akademie der Wissenschaften zu Petersburg durch Herrn Professor Kupffer (leider ohne irgendeine historische Notiz, um die wir uns später erfolglos bemühten), in Tausch erhalten.
\subsection[Mauerkirchen.]{Mauerkirchen,}
\begin{center}
\small
im Innkreise, Österreich ob der Enns. (Gehörte zur Zeit des Falles zu Baiern.)

20. November 1768, 4 Uhr Nachmittags.
\end{center}
\paragraph{}
Sehr hellgraue, fast weiße, wenig zusammenhängende, nicht schwer zerreibliche Grundmasse, mit ziemlich vielen, auf den Bruchflächen wenig wahrnehmbaren, auf Schnittflächen aber leichter erkennbaren kugeligen Ausscheidungen; fein eingesprengtes metallisches Eisen und viel, sehr fein eingesprengter Magnetkies, der sich zuweilen in größeren Körnern, von Hanfkorn- bis Bohnengröße, aussondert. Matte Rinde. — Ein durch seine helle Farbe und leichte Zerreiblichkeit sehr ausgezeichneter Meteorstein.

Zwei Bruchstücke von einem einzeln gefallenen Steine von 38 Pfund; beide mit Rinde.

1. Ein größeres Bruchstück mit nicht ganz frischen Bruchflächen, und einer kleinen polierten Fläche. — 23 27/32 Loth. — \mars 1. 7. Durch Professor Chladni im Jahre 1805 in Tausch erhalten.

2. Ein kleines, aber ganz frisches Bruchstück. — 9 13/32 Loth. — 1825. 42. 8. — Aus der Mineralien-Sammlung des Grafen Fries gekauft.
\subsection{Nashville.}
\begin{center}
\small
Dorf oder Gegend ? Drake-Creek, 18 engl. Meilen von Nashville, Staat Tennessee, Nord-Amerika.

9. Mai 1827, 4 Uhr Nachmittags.
\end{center}
\paragraph{}
Lichtgrane, durch undeutliche, kugelige Ausscheidungen schwach gefleckte, nicht stark zusammenhängende, und daher schwer polirbare Grundmasse; sehr viel fein eingesprengter Magnetkies, der auch in hanfgroßen Partien auftritt; das metallische Eisen fein zerstreut und in geringer Menge eingesprengt. Matte, ziemlich glatte Rinde.

1. Fragment mit Rinde, von einem faustgroßen Stücke im Museum des Yale-College zu New-Haven in Nord-Amerika abgeschlagen. (Es fielen mehrere Steine, wovon 5 gesammelt wurden, deren einer 11 Pfd. wog.) — 1 27/32 Loth. — 1840. 32. 1. — Durch Vermittlung des Herrn Baron Lederer, österr. General-Konsuls in New-York, von Herrn Professor Silliman in New-Haven in Tausch erhalten.
\subsection[Lucé.]{Lucé,}
\begin{center}
\small
en Maine, jetzt Dép. de la Sarthe, Frankreich.

13. September 1768, 4 1/2 Uhr Nachmittags.
\end{center}
\paragraph{}
Lichtgraue, mit Rostflecken durchsäete Grundmasse, mit undeutlichen kugeligen Ausscheidungen; fein und mittelfein eingesprengtes gediegenes Eisen; sehr fein eingesprengter Magnetkies; raue, matte Rinde.

Fragment von einem einzeln (?) gefallenen Steine von 7 1/2 Pfund.

1. Kleines Fragment mit Rinde und einer anpolierten Fläche. — 17/32 Loth. — 1838. 25. 6. — Aus der ehemals Heuland'schen, dann Heath'schen Meteoriten-Sammlung durch Herrn Pötschke gekauft. Das Stück lag früher in der von Herrn Heuland angekauften Mineralien-Sammlung des Marquis de Drée in Paris.

In der königl, Mineralien-Sammlung zu Berlin befindet sich ein Fragment von Luce, das mit der Chladni'schen Meteoriten-Sammlung dahin kam, und von dem am Wiener kaiserl. Mineralien-Kabinette befindlichen verschieden ist. Das Berliner ist dunkelaschgrau, und ganz den Steinen von Limerick und Tipperary ähnlich. Chladni sagt nicht, von wem er sein Stück erhielt. Das unsere stammt aus der de Drée'schen Sammlung, wo es schon Bigot de Morogues sah, der vom Steine von Luce sagt: La pierre tombée à Lucé, en 1768, est facile à reconnoître à cause de la teinte uniforme tres-claire; elle est assez compacte à grains fins, et ne présente aucun filon. (Siehe den Anhang: Description comparative de quelques pierres tombées du ciel, Seite 335 in Bigots Werke: Mémoire historique et physique sur les chutes des pierres tombées sur la surface de la terre a diverses époques. Orléans 1812.) Léman charakterisiert den Stein von Lucé folgender Massen: Pierre d’un gris cendré pâle, avec une infinité de petits points brillants, d'un jaune pale, enveloppée d'une croûte noire etc. (Siehe den Artikel Pierres météoriques in dem Nouveau Dictionnaire d'histoire naturelle Vol. 26. Paris 1818.)

Anhang. Die nachfolgenden zwei Stücke sind mit dem Meteorstein-Fragment Nr. 1 von Lucé, mit Ausnahme der Beschaffenheit der Rinde, was jedoch nicht viel bedeuten will, vollkommen identisch, wurden aber mit der Angabe anderer Lokalitäten erhalten, mit denen sie nicht übereinstimmen. Wir lassen sie deshalb hier anhangsweise folgen, ohne übrigens die Lokalität Lucé zu verbürgen.

2. Bruchstück mit Rinde und polierter Fläche. — 8 13/32 Loth. — 1841. 14. 5. — Aus der Heuland'schen Sammlung durch Herrn Pötschke gekauft, mit der Etiquette: Limerick. Herr Heuland glaubt den Stein von Herrn Professor Gieseke aus Dublin erhalten zu haben; es dürfte jedoch eine Verwechslung eingetreten sein, welche leicht Statt findet, wenn man die Stücke nicht durch aufgeklebte Etiquetten oder Nummern unterscheidet. — Dieses Fragment wird wohl gleichfalls aus der Sammlung des Marquis de Drée herstammen.

3. Kleines Bruchstück mit etwas Rinde, dicken, schwarzen Adern und anpolierter Fläche. — 19/32 Loth. — 1841. 14. 6. — Aus der Heuland'schen Sammlung durch Herrn Pötschke angekauft. Erhalten mit der Lokalität Toulouse. Stammt aus des Marquis de Drée Mineralien-Sammlung.
\subsection{Lissa.}
\begin{center}
\small
(Strattow, Wustra.) Bunzlauer Kreis, Böhmen.

3. September 1808, 3 1/2 Uhr Nachmittags.
\end{center}
\paragraph{}
Lichtgraue, feinkörnige Grundmasse, in welcher zwar nicht auf Bruch-, doch auf anpolierten Flächen, hellgraue, kugelige oder ovale, mit der Grundmasse innig zusammenhängende Ausscheidungen wahrzunehmen sind; ziemlich viel, aber höchst fein eingesprengtes metallisches Eisen, ungefähr eben so viel, sehr fein eingesprengter, zuweilen aber auch in linsengroßen Partien auftretender Magnetkies. Feine, seltener dicke Adern durchziehen die Grundmasse nach verschiedenen Richtungen. Matte Rinde.

Ein großer, ganzer Stein und zwei Fragmente von den vier oder fünf der allda gefallenen Steine.

1. Großer, ganzer, 7 Zoll langer Stein von unregelmäßiger Form (unvollkommen vierseitig, prismatisch), mit vielen kleinen Eindrücken an der Oberfläche; an zwei Stellen mit Bruchflächen und kleinen, beim Falle entstandenen Beschädigungen an den Kanten. — Beschrieben und abgebildet in v. Schreibers Beiträgen, S. 17, Taf. 3. — 5 Pfund, 17 5/8 Loth. — 1809. 17. 1. — Ist vom Lissaer Wirtschaftsamte an das Bunzlauer Kreisamt, durch dieses an das königl. böhmische Gubernium und weiter an die vereinigte k. k. Hofkanzlei in Wien eingesendet worden, welche den Meteorstein Seiner Majestät dem Kaiser Franz überreichte, der ihn 1809 durch den Herrn Oberstkämmerer, Grafen von Wrbna, dem k. k. Hof-Mineralien-Kabinette übergeben ließ.

2. Bruchstück mit Rinde und einer anpolierten Fläche. — 3 1/32 Lth. — 1808. 26. 1. — Wurde durch das königl. böhm. Gubernium eingesendet.

3. Bruchstück mit Rinde und teilweise frischem Bruche. — 2 17/32 Loth. — 1838. 24. 1. — Von Herrn Gubernialrat Neumann in Prag in Tausch erhalten.
\subsection{Owahu.}
\begin{center}
\small
(Oahu oder Woahoo), eine der Sandwich-Inseln, deren Hauptort Honororu (oder Honololu).
\end{center}
\paragraph{}
Lichtaschgraue, etwas ins Grünliche ziehende, durch eingemengte kugelige Ausscheidungen mehr oder weniger deutlich gefleckte Grundmasse, durchzogen von einer großen Anzahl schwarzer Adern, die sich zum Teil auch verästeln; schwarze, graphitartig glänzende Ablösungen; ziemlich viel, meist fein eingesprengtes gediegenes Eisen; sehr fein eingesprengter Magnetkies; matte, schwarze, zum Teil ins Bräunlichrote umgeänderte Rinde. Eines von den zwei vorhandenen Stücken zeigt auch rostbraune Flecken in der Grundmasse, in Folge der durch Umstände erfolgten, teilweisen Oxydierung des metallischen Eisens, welches die Umgebungen färbte. Dieses Kennzeichen kann daher vorhanden sein, oder auch fehlen, und ist somit überhaupt nicht bezeichnend. (Dieser Meteorstein steht dem von Lissa am nächsten.)

Zwei Fragmente von den zwei allda gefallenen Steinen, wovon jeder ungefähr 15 Pfund wog.

1. Fragment mit Rinde und einer unvollkommen anpolierten Fläche. — 2 Lth. — 1842. 34. 1. — Durch den Kurator des Yale-College zu New-Haven in Nord-Amerika, Herrn B. Silliman, in Tausch erhalten. Der 2 Pfund schwere Stein, von dem dieses Fragment herrührt, wurde von dem Rev. Henry Bingham von den Sandwich-Inseln nach Nord-Amerika gebracht.

2. Fragment mit Rinde und anpolierter Fläche. — 3 1/2 Loth. — 1839. 37. 1. — Von der kaiserl. russischen Universität zu Dorpat durch den Professor und Staatsrat, Moritz von Engelhardt, in Tausch erhalten. Wurde von dem in der Dorpater Universitäts-Sammlung aufbewahrten Stücke, welches der damals auf Owahu anwesende Herr Ernst Hoffmann, jetzt Professor in Kiew, von der Kotzebue'schen Weltumseglung mitbrachte, abgeschnitten (dabei aber leider mit Oel getränkt).
\subsection{Charkow.}
\begin{center}
\small
(Bobrik) Gouv. Charkow, Ukraine, Russland.

1. Oktober 1787, 3 Uhr Nachmittags.
\end{center}
\paragraph{}
Lichtaschgraue Grundmasse, mit eingemengten, undeutlichen Körnern, die etwas ins Grünliche ziehen; in mäßiger Menge und meist fein eingemengtes metallisches Eisen; sehr fein eingesprengter Magnetkies; matte, glatte Rinde. — Das sehr kleine Fragment zeigt keine schwarze Adern oder Ablösungen, die jedoch vorhanden sein können.

1. Kleines Fragment mit Rinde und einer kleinen, mit Quarzpulver polierten Fläche, von einem der mehreren ? allda gefallenen Steine. — 3/32 Loth. — 1839. 22. 4. — Von der Mineralien-Sammlung der königl. Universität zu Berlin durch Herrn Professor Weiss in Tausch erhalten, Stammt aus der Chladni'schen Meteoriten-Sammlung.
\subsection{Zaborczika.}
\begin{center}
\small
(Nach einer brieflichen Angabe von Professor Eichwald Saborytz) am Flüsse Slucz oder Sluisch ? Gouv. Wolhynien, Russland.

30. März alten Styls 1818.
\end{center}
\paragraph{}
Lichtaschgraue, durch undeutlich eingemengte Körner von einer anderen Nuance von Grau nicht ganz homogen aussehende Grundmasse, mit kleinen braunen Rostflecken; ziemlich viel eingesprengter Magnetkies. Über das Verhältnis des eingemengten metallischen Eisens lässt sich aus Mangel einer polierten Fläche an dem kleinen zerklüfteten Stücke nicht urteilen. Rinde ist an dem Fragmente nicht vorhanden.

1. Kleines Fragment ohne Rinde. (Die Zahl der gefallenen Steinen ist unbekannt.) — 5/16 Loth. — 1839. 22. 3. — Von der Mineralien-Sammlung der königl. Universität zu Berlin durch Herrn Professor Weiss in Tausch erhalten, welcher das Fragment vom Professor Storodeki in Wilna erhielt.
\subsection{Bachmut.}
\begin{center}
\small
Gouv. Ekaterinoslaw, Russland.

3. Februar 1814.
\end{center}
\paragraph{}
Lichtaschgraue Grundmasse, durch undeutliche, auf polierten Flächen mehr wahrnehmbare, kugelige Einmengungen schwach gefleckt; nicht viel mittelfein eingesprengtes metallisches Eisen; ziemlich viel, meist sehr fein eingesprengter Magnetkies. Rinde fehlt dem vorhandenen Fragmente.

1. Fragment ohne Rinde und einer unvollkommen anpolierten Fläche von einem einzeln gefallenen Steine von 40 Pfund. — 7/8 Loth. — 1840. 1. 1. — Vom Mineralien-Kabinette der königl. Universität zu Berlin in Tausch erhalten. Das Stück, von welchem dieses Fragment abgeschnitten wurde, stammt aus Klaproths Sammlung.
\subsection{Politz.}
\begin{center}
\small
(Köstritz) bei Gera im Fürstentume Reuß.

13. Oktober 1819, 8 Uhr Morgens.
\end{center}
\paragraph{}
Lichtaschgraue Grundmasse, mit undeutlichen braunen Flecken und schwarzen Punkten; die kugeligen Ausscheidungen mehr oder weniger deutlich; schwarze Adern scheinen (so viel nach den vorhandenen kleinen Stücken geurteilt werden kann) nur seltener aufzutreten; ziemlich viel, jedoch meist fein eingesprengtes metallisches Eisen; weniger und höchst fein eingesprengter Magnetkies; matte, ziemlich dicke Rinde.

Drei Bruchstücke von einem einzeln gefallenen, 7 Pfund schweren Steine.

1. Flaches Bruchstück mit viel Rinde. — 9/16 Lth. — 1840. 23. 2. — Von Doktor Bondi in Dresden gekauft, der es von Herrn Laspe in Gera erhielt.

2. a. Kleines Bruchstück mit Rinde. — 7/32 Loth. — 1839. 22. 5. — Von der Mineralien-Sammlung der königl. Universität zu Berlin durch Herrn Professor Weiss in Tausch erhalten. Stammt aus der Chladni'schen Meteoriten-Sammlung.

2. b. Kleines Bruchstück ohne Rinde; von zwei Seiten anpoliert. — 3/8 Loth. — 1839. 22, 6. — Wie Nr. 2a. erhalten.
\subsection{Kuleschofka.}
\begin{center}
\small
Romenskyscher Kreis, Gouv. Poltawa, Russland.

12. März 1811, um Mitternacht.
\end{center}
\paragraph{}
Lichtaschgraue, stark zusammenhängende Grundmasse, mit höchst feinen, schwer unterscheidbaren braunen Pünktchen. Auf polierten Fachen sind undeutliche, kugelige Ausscheidungen, vieles, teils fein, teils grob eingesprengtes metallisches Eisen und ziemlich viel, aber höchst fein eingesprengter Magnetkies wahrzunehmen; letzterer ist auch auf den Bruchflächen leicht zu unterscheiden. Den Stein durchziehen hie und da dünne, schwarze Adern; auch sind schwarze Ablösungsflächen vorhanden. Dicke, matte oder etwas schimmernde Rinde.

Zwei Bruchstücke von einem einzeln gefallenen Steine von 13 Pfund.

1. Bruchstück mit Rinde und einer anpolierten Fläche. — 8 27/32 Loth. — 1838. 28. 6. — Von der kaiserl. russischen Akademie der Wissenschaften zu Petersburg durch Herrn Professor Kupffer in Tausch erhalten.

2. Bruchstück mit Rinde. — 2 3/8 Lth. — 1841. 3. 18. — Von Hrn. Dr. Baader gekauft, welcher das Bruchstück von Hrn. Apotheker Krämmerer in Petersburg eintauschte.
\subsection{Slobodka.}
\begin{center}
\small
Gouv. Smolensk, Russland.

10. August 1818.
\end{center}
\paragraph{}
Lichtgraue, rostbraun gefleckte, mit feinen, schwarzen Adern durchzogene Grundmasse; mit vielen, aber undeutlichen kugeligen, meist jedoch eckigen, mit der Grundmasse fest verwachsenen Ausscheidungen, die dem Steine ein marmoriertes Aussehen geben; ziemlich viel, teils fein, teils mittelfein eingesprengtes metallisches Eisen; weniger, sehr fein eingesprengter Magnetkies; fast matte, oder nur schimmernde Rinde.

Drei Bruchstücke von einem einzeln gefallenen Steine von 7 Pfund.

1. Bruchstück von schwarzen Adern durchzogen, mit Rinde und einer anpolierten Fläche. — 4 3/32 Loth. — 1829. 41. 15. — Von Doktor Fiedler in Dresden gekauft, mit den imaginären Fundörtern: Ural und Ufa. Herr Fiedler erhielt dieses Fragment vom Herrn G. B. Sowerby in London.

2. Frisches Bruchstück mit etwas Rinde und einer Ablösungsfläche. — 3 3/32 Loth. — 1841. 14. 9. — Aus der Heath'schen (früher Heuland'schen) Meteoriten-Sammlung durch Herrn Pötschke gekauft, mit der Etiquette: Timochin. — Stammt aus der Sammlung des Sir Alexander Chrichton, welche in London durch Herrn Sowerby versteigert wurde. Dieses Fragment, sowie Nr. 1, sind von einem größeren Stücke abgeschlagen, welches nunmehr Baron Reichenbach in Wien aus der Heuland'schen Sammlung besitzt.

3. Fragment ohne Rinde, mit anpolierter Fläche. — 1 3/8 Loth. — 1839. 28. 2. — Aus der Mineralien-Sammlung der königl. Universität zu Berlin durch Professor Weiss mit dem Fundorte Slobodka‚ Gouv. Smolensk, Russland (gefallen 10. August 1818) in Tausch erhalten. Stammt aus der von der Berliner Universität angekauften Bergemann’schen Mineralien-Sammlung.

Ob die aufgestellten drei Fragmente wohl sicher von einem und demselben Fundorte, auch wirklich von Slobodka seien, bleibt noch etwas zweifelhaft.
\subsection{Milena.}
\begin{center}
\small
(Ungarisch: Milyan). Dorf Pusinsko Selo, eine Meile südlich von Milena, Warasdiner Komitat, Kroatien.

26. April 1842, 3 Uhr Nachmittags.
\end{center}
\paragraph{}
Lichtaschgraue Grundmasse mit braunen Rostflecken, undeutlichen, etwas dunkleren, kugeligen Ausscheidungen, ziemlich viel fein und mittelfein eingesprengtem metallischen Eisen und sehr fein eingesprengtem Magnetkies; matte oder schwach schimmernde Rinde. — Gehört zu der gewöhnlichen Abänderung der lichten, metallisches Eisen führenden Meteorsteine, und ist von den Meteoriten von Slobodka, Forsyth, Glasgow, Yorkshire, Kuleschofka, Politz, Zaborczika und Charkow kaum zu unterscheiden.

1. Fragment mit frischem Bruch, einer schwach anpolierten Schnittfläche und etwas Rinde, von einem der zwei oder drei allda gefallenen Steine oder Fragmente von mäßiger Größe, deren Gewicht wegen schneller Zertrümmerung der Steine durch die herbeigeeilten Landleute, nicht genau bekannt wurde, und etwa 10 bis 11 Pfund betragen haben mag. — 11 1/16 Loth. — 1842. 45. 1. — Von Sr. Excellenz dem Bischof von Agram, Georg von Haulik, als Geschenk erhalten.
\subsection[Forsyth.]{Forsyth,}
\begin{center}
\small
im Staate Georgien, Nord-Amerika.

8. Mai 1829, zwischen 3 und 4 Uhr Nachmittags.
\end{center}
\paragraph{}
Lichtgraue, etwas ins Dunkelgraue ziehende, rostbraun gefleckte Grundmasse, mit undeutlichen kugeligen Ausscheidungen; fein eingesprengtes gediegenes Eisen und meist sehr fein eingesprengter Magnetkies; dicke, matte Rinde.

Zwei Bruchstücke von einem einzeln gefallenen Stein von 36 Pfund.

1. Fragment mit Rinde und anpolierter Schnittfläche. — 2 7/8 Loth. — 1832. 43. 13. — Durch den k. k. General-Konsul in New-York, Baron Lederer in Tausch erhalten.

2. Fragment ohne Rinde, jedoch teilweise ganz frischen Bruchflächen. — 2 3/32 Loth. — 1834. 31. 22. — Wie Nr. 1 erhalten.
\subsection{Yorkshire.}
\begin{center}
\small
(Woldcottage) England.

13. Dezember 1795, 3 1/2 Uhr Nachmittags.
\end{center}
\paragraph{}
Lichtgraue, auf polierten Flächen ins Dunkelgraue geneigte, schwach rostbraun gefleckte Grundmasse, mit undeutlichen, ebenfalls grauen, mit der Grundmasse innig verbundenen, kugeligen, oder ovalen Ausscheidungen; ziemlich viel, teils fein, teils mittelfein eingesprengtes metallisches Eisen; viel, jedoch sehr fein eingesprengter Magnetkies; schwarze, glänzende Ablösungsflächen; matte, oder schwachschimmernde Rinde.

1. Bruchstück von einem einzeln gefallenen Steine von 56 englischen Pfund, welchen früher die Familie Sowerby in London besaß, der aber nunmehr im britischen Museum zu London aufbewahrt wird. — 2 3/32 Lth. — 1816. 76. 2. — Von Herrn Sowerby, Vater, aus London in Tausch erhalten.
\subsection{Glasgow.}
\begin{center}
\small
(High-Possil) Schottland.

5. April 1804, Vormittags.
\end{center}
\paragraph{}
Lichtgraue, rostbraun gefleckte, auch schwarz gesprenkelte Grundmasse, mit undeutlichen, grauen, kugeligen Ausscheidungen; mit teils fein, meist jedoch grob eingesprengtem gediegenen Eisen und sehr fein eingesprengtem Magnetkies; matte Rinde. — Steht den Meteorsteine aus Yorkshire sehr nahe.

1. Bruchstück von einem einzeln gefallenen Steine, wovon nur zwei Fragmente aufgefunden wurden, mit Rinde und einer anpolierten Fläche. — 7/8 Loth. — 1816. 76. 1. — Von Herrn Sowerby, Vater, aus London in Tausch erhalten.
\subsection{Berlanguillas.}
\begin{center}
\small
(Burgos, Aranda, Roa) Alt-Kastilien, Spanien.

8. Juli 1811, 8 Uhr Abends.
\end{center}
\paragraph{}
Fast lichtgraue, auf polierten Flächen dunkelgraue, rostbraun gelleckte Grundmasse, mit undeutlichen, mit der Grundmasse innig verbundenen kugeligen Ausscheidungen; viel, zum Teil fein, zum Teil grob eingesprengtes gediegenes Eisen; auch viel, sehr fein eingesprengter Magnetkies; matte Rinde. — Gleicht auffallend dem Meteorstein von Apt, nur ist die Masse etwas lichter und die Rinde ebener.

1. Ein Bruchstück von einem der drei oder vier daselbst gefallenen Steine unbekannten Gewichts, mit anpolierten Flächen und viel Rinde. — 11 9/32 Loth. — 1816. 31. 33. a. — Durch Vermittlung des Direktors von Schreibers von dem Museum der Naturgeschichte in Paris in Tausch erhalten. Wurde von einem daselbst aufbewahrten, 3 Pfund schweren, ganzen Steine abgeschnitten.
\subsection{Apt.}
\begin{center}
\small
(Saurette), Dép. de Vaucluse, Provence, Frankreich.

8. Oktober 1803. 10 Uhr Vormittags.
\end{center}
\paragraph{}
Fast lichtgraue, auf polierten Schnittflächen dunkelgraue, rostbraun gefleckte Grundmasse, mit einzelnen, meist lichteren, kugeligen Ausscheidungen; viel, meistens fein, zum Teil aber auch grob eingesprengtes, metallisches Eisen; viel, sehr fein eingesprengter Magnetkies; matte, raue Rinde.

Zwei Bruchstücke von einem einzeln gefallenen Steine von 7 Pfund 12 Loth.

1. Bruchstück mit viel Rinde und einer großen anpolierten Fläche. Ein darin befindliches großes Eisenkorn ist geätzt. — 16 7/16 Loth. — 1816. 31. 33. b. — Wurde auf Verwendung des Direktors von Schreibers im Jahre 1815 von dem im naturhistorischen Museum zu Paris aufbewahrten, 7 Pfund schweren, ganzen Steine abgeschnitten.

2. Kleines Bruchstück mit Rinde und einer nicht polierten Schnittfläche. — 2 1/8 Loth. — 1841. 19. 10. — Aus der Heuland'schen Meteoriten-Sammlung, durch Herrn Pötschke gekauft. Wurde mit der falschen Lokalität Casignano (Parma) erhalten, passt jedoch genau an das Stück Nr. 1 an; ist also bestimmt von Apt und wurde in Paris von Nr. 1 abgebrochen. Stammt aus der Mineralien-Sammlung des Herrn Marquis de Drée.
\subsection[Vouillé.]{Vouillé,}
\begin{center}
\small
bei Poitiers, Dépt. de la Vienne, Frankreich.

18. Juli (nach dem Kataloge des Pariser Museums), nach andern Angaben 13. Mai 1831.
\end{center}
\paragraph{}
Lichtaschgraue, doch schon stark ins Dunkelbläulichgraue geneigte Grundmasse, durch undeutliche, mit der Grundmasse fest verwachsene kugelige Ausscheidungen schwach gefleckt, zum Teil auch mit Rostflecken durchsäet; ziemlich viel, sehr fein, zum Teil aber auch grob eingesprengtes metallisches Eisen, sehr fein eingesprengter Magnetkies, schwach schimmernde, fast matte Rinde.

1. Fragment mit Rinde und einer unvollkommen (ohne Smirgel) polierten Fläche, von einem einzeln gefallenen Steine von 40 Pfund. — 5 1/16 Loth. — 1840. 29. 1. Vom königl. Museum der Naturgeschichte zu Paris auf Vermittlung des Herausgebers in Tausch erhalten. Das Pariser Museum bekam das Stück von Baron Cuvier zu Geschenk.
\subsection{Château-Renard.}
\begin{center}
\small
Gemeinde Triguères, Dépt. du Loiret, Frankreich.

12. Juni 1841.
\end{center}
\paragraph{}
Lichtgraue, doch etwas ins Dunkelbläulichgraue ziehende, durch undeutlich eingemengte kugelige Ausscheidungen gefleckt aussehende, zum Teil auch mit Rostflecken besäte und schwarz punktierte Grundmasse, von schwarzen, dickeren oder dünneren Adern durchzogen, die sich auf den Bruchflächen manchmal als schwarze Ablösungsflächen darstellen; viel fein und grob eingemengtes metallisches Eisen und sehr fein eingesprengter Magnetkies; matte schwarze Rinde.

Drei Bruchstücke von einem einzeln gefallenen, in zwei große und viele kleine Fragmente zersprungenen Steine von 70 bis 80 Pfund.

1. Fragment mit Rinde und anpolierter Fläche. — 18 9/16 Loth. — 1842. 28. 1. — Von Herrn Roussel in Paris in Tausch erhalten.

2. Scharfkantiges Bruchstück ohne Rinde, mit einer zum Teil dicken, schwarzen Ader. — 7 1/4 Loth. — 1842. 28. 2. — Mit und wie Nr. 1 erhalten.

3. Anpolierte Platte mit Rinde. — 2 1/16 Loth. — Von 1842. 28. 1. — Wurde von Nr. 1 abgeschnitten.
\subsection{Salés.}
\begin{center}
\small
Villefranche, Dépt. du Rhône, Frankreich.

8. oder 12. März 1798. 6 Uhr Abends.
\end{center}
\paragraph{}
Lichtgraue, doch schon etwas ins Dunkle und Braune ziehende, mit sehr feinen schwarzen Adern durchwebte, rostbraun gefleckte und fein schwarz punktierte Grundmasse, mit wenigen kugeligen Ausscheidungen, wovon einige schmutzig dunkelgrün, andere graulich-weiß sind; viel, teils fein eingesprengtes, teils in Körnern, (die manchmal an 2 Linien und darüber lang sind), eingewachsenes gediegenes Eisen und fein eingesprengter Magnetkies; undeutliche Ablösungsflächen; matte, dicke und raue Rinde.

Zwei Bruchstücke von einem einzeln gefallenen Steine von 20 Pfund.

1. Bruchstück mit Rinde und zwei anpolierten Flächen. — 16 11/16 Loth. — 1841. 14. 3. — Aus der Heuland'schen Meteoriten-Sammlung durch Herrn Pötschke gekauft. Stammt aus der Mineralien-Sammlung des Herrn Marquis de Drée.

2. Bruchstück mit einem kleinen Stück Rinde und einer anpolierten Fläche, in welcher zwei größere, mit Salpetersäure geätzte Körner von gediegenem Eisen eingewachsen sind. — Beschrieben und abgebildet in von Schreibers Beiträgen, S. 86, Taf. 7. — 2 13/32 Loth. — 1816. 35. 43. — Auf Verwendung des Direktors von Schreibers während dessen Anwesenheit in Paris im Jahre 1815, aus der Sammlung des Herrn Marquis de Drée in Tausch erhalten. (Ist damals von dem jetzt im k. k. Mineralien-Kabinette befindlichen Stücke Nr. 1 abgeschnitten worden.)
\subsection{Agen.}
\begin{center}
\small
Dépt. Lot et Garonne, Frankreich.

5. September 1814, Mittags.
\end{center}
\paragraph{}
Lichtgraue, auf polierten Flächen ins Dunkelgraue ziehende, rostbraun gefleckte und schwarz punktierte, auch mit vielen, meist sehr feinen schwarzen Adern durchzogene Grundmasse, mit dunkelgrauen, fest verwachsenen, kugeligen Ausscheidungen; viel, aber sehr fein eingesprengtem und gleichmäßig verteiltem gediegenen Eisen und mikroskopisch fein eingesprengtem Magnetkies; matte, stellenweise auch schlackige Rinde.

Zwei Bruchstücke von zwei der vielen allda gefallenen Steine, welche hinsichtlich der Helligkeit der Grundmasse und der Rostflecken voneinander etwas verschieden sind.

1. Bruchstück mit viel Rinde und einer anpolierten Fläche. — 4 9/32 Loth. — 1816. 31. 33. e. — Auf Vermittlung des k. k. Naturalien-Kabinetts-Direktors von Schreibers im Jahre 1815 aus dem königlichen Museum der Naturgeschichte in Paris durch Tausch erhalten.

2. Bruchstück (halber Stein ?) von lichterem Grau, und mit polierter Fläche. — 7 1/4 Loth. — 1841. 14. 7. — Aus der Heuland'schen Sammlung durch Herrn Pötschke gekauft. Ist von Nr. 1 etwas verschieden. Herr Heuland kaufte das Stück von Herrn Leman in Paris, als die Hälfte des angeblich größten der bei Agen gefallenen Steine, was jedoch unrichtig ist, da der größte 18 Pfund wog.

Die Meteorsteine von Agen sind vorzüglich merkwürdig durch den Umstand, dass sie, obwohl gediegenes Eisen führend, doch keinen Nickel enthalten, ein Fall, der unter allen anderen eisenführenden nur noch bei dem Meteorsteine von Wessely Statt findet.
\subsection{Nanjemoy.}
\begin{center}
\small
Maryland, in den vereinigten Staaten von Nord-Amerika.

10. Februar 1825, gegen Mittag.
\end{center}
\paragraph{}
Zwischen licht und dunkelaschgrau schwankende, feste und dichte, zum Teil mit Rostflecken durchsäete Grundmasse, mit teils lichteren, meist aber dunkleren, mit der Grundmasse fest verwachsenen, kugeligen Ausscheidungen; mit ziemlich viel fein eingesprengtem gediegenen Eisen, und höchst fein eingesprengtem Magnetkies; raue und matte Rinde, mit feinen Sprüngen durchwebt.

1. Fragment mit Rinde und einer anpolierten Fläche von einem einzeln gefallenen Steine von 16 Pfund. — 20 Loth. — 1835. 25. 1. — Aus der ehemals Heuland'schen, dann Heath'schen Meteoriten-Sammlung durch Herrn Pötschke gekauft. Herr Heuland in London erhielt das Stück von Professor Silliman aus Nord-Amerika.
\subsection[Asco.]{Asco,}
\begin{center}
\small
auf der Insel Korsika.

November 1805.
\end{center}
\paragraph{}
Fast lichtgraue Grundmasse, mit Rostflecken und kleinen undeutlichen kugeligen Ausscheidungen; mit vielem teils fein, teils mittelfein eingesprengtem metallischen Eisen und sehr fein eingesprengtem Magnetkies; undeutliche, sehr feine schwarze Adern; schwarze, metallisch glänzende Ablösungen. Rinde ist an dem Fragmente nicht vorhanden. (Dieser Stein ist von dem Meteorsteine von Nanjemoy kaum zu unterscheiden.)

1. Anpoliertes Fragment ohne Rinde. — 1 1/16 Loth. — 1838. 25. 4. — Aus der ehemals Heuland'schen, dann Heath'schen Meteoriten-Sammlung durch Herrn Pötschke gekauft. Das Stück stammt aus der Mineralien-Sammlung des Herrn Marquis de Drée in Paris, die Herr Heuland kaufte.

Über den Steinfall von Asco ist nichts öffentlich bekannt gemacht worden; dass er sich ereignet, wurde uns durch verlässliche Zeugnisse aus Korsika bekräftiget und dabei gemeldet, dass der gefallene Stein in einer Kirche aufbewahrt werde.
\subsection{Toulouse.}
\begin{center}
\small
(Permejean, Pechmeja), Dépt. der oberen Garonne, Frankreich.

10. April 1812, 8 1/4 Uhr Abends.
\end{center}
\paragraph{}
Schwach dunkelgraue, rostbraun gefleckte Grundmasse, mit kleinen und undeutlichen, mit der Grundmasse fest verwachsenen kugeligen Ausscheidungen; mit viel, ziemlich fein und gleichförmig eingesprengtem metallischen Eisen und höchst fein eingesprengtem Magnetkies; matte, mit kleinen runden Erhöhungen oder Narben besetzte Rinde.

1. Bruchstück von einem der mehreren allda gefallenen, aber meist nicht aufgefundenen Steine (das Aufgefundene soll nach Chladni höchstens 16 Loth wiegen), mit Rinde und einer polierten Fläche. — 15/16 Loth. — 1816. 31. 33. d. — Auf Verwendung des Direktors von Schreibers im Jahre 1815 durch Tausch aus dem königl. Museum der Naturgeschichte in Paris von dem daselbst aufbewahrten 6 Loth schweren Steine erhalten.

Das im kön. Mineralien-Kabinett zu Berlin befindliche, aus Chladnis Sammlung herrührende Stück von Toulouse ist dunkler als das in unserer Sammlung vorhandene, und ähnelt mehr den Steinen von Limerick und Tipperary. Chladni erhielt das Stück von Professor Laugier. — Man sehe auch das bei Nr. 3 im Anhange zur Lokalität Lucé Bemerkte.
\subsection{Blansko.}
\begin{center}
\small
Brünner Kreis, Mähren.

25. November 1833, 6 1/2 Uhr Abends.
\end{center}
\paragraph{}
Dunkelgraue, rostbraun gefleckte Grundmasse, mit ziemlich vielen dunkleren, kleinkugeligen Ausscheidungen, viel fein eingesprengtem metallischen Eisen, und sehr fein eingesprengtem Magnetkies; matte Rinde. — Ist von den Steinen von Toulouse und Wessely kaum zu unterscheiden.

1. a. Ein ganzer, überrundeter Stein, mit ein Paar kleinen Bruchflächen. Einer von den 8 Steinen, die durch Dr. Reichenbach mit vielen Kosten und großer Anstrengung aufgefunden wurden. Die von demselben abgeschnittene Ecke liegt unter:

1. b. Beide Schnittflächen sind poliert und lassen ein Paar feine Adern wahrnehmen; auch ist eine Ablösungsfläche sichtbar. — Beide Stücke wiegen zusammen 3 31/32 Loth. — 1834. 32. 1. — Von Herrn Dr. Reichenbach in Tausch erhalten.
\subsection{Wessely.}
\begin{center}
\small
(Znorow), Hradischer Kreis, Mähren.

9. September 1831, 3 Uhr Nachmittags.
\end{center}
\paragraph{}
Dunkel- fast bläulichgraue, schwach rostbraun gefleckte, mit sehr feinen schwarzen Adern durchzogene Grundmasse, mit undeutlichen, meist kleinen, kugeligen Ausscheidungen; viel fein eingesprengtem, gleichmäßig verteiltem metallischen Eisen und wenig höchst fein, fast mikroskopisch eingesprengtem Magnetkies; matte Rinde mit kleinen Erhöhungen oder Narben (wie am Steine von Toulouse). — Nahe verwandt mit den Meteorsteinen von Limerick und Tipperary; nur etwas heller.

1. Ein ganzer Stein, der einzige, welcher da fiel. An drei Stellen ist die Rinde beim Fallen in geringer Ausdehnung abgesprengt, an zwei anderen sind kleine Stückchen abgesägt worden; eine von den dadurch entstandenen Schnittflächen ist poliert.

Ausführlich ist dieser Meteorstein beschrieben in dem Berichte des Direktors von Schreibers über den Meteorstein-Fall von Wessely in Baumgartners Zeitschrift für Physik und verwandte Wissenschaften, Band 1. Seite 193. Auch ist am k. k. Hof-Mineralien-Kabinette eine sehr genaue Lithographie dieses Steines vorhanden. — 6 Pfund, 17 1/2 Loth. — 1832. 7. 1. — Vom herrschaftlichen Wirtschaftsamte zu Wessely an das k. k. Kreisamt zu Hradisch abgeliefert, gelangte dieser Meteorstein an das Landes-Präsidium zu Brünn und von da an die k. k. Hofkanzlei nach Wien, welche ihn Seiner Majestät dem Kaiser Franz vorlegte, auf dessen Befehl er in dem k. k. Hof-Mineralien-Kabinette hinterlegt wurde.

2. a. und b. Zwei kleine anpolierte Fragmente, wovon eines mit Rinde. — Zusammen 15/32 Loth. — Abfälle, erhalten beim Absägen eines kleinen Stückes für das Franzens-Museum zu Brünn, für welches auch ein Gips-Modell von diesem Steine angefertigt wurde.

Wir haben schon bei dem Meteorstein von Agen Nr. 48 bemerkt, dass außer diesem, unter allen Meteorsteinen die gediegenes Eisen einschließen, nur noch der von Wessely die merk würdige Erscheinung darbietet, keinen Nickel zu enthalten.
\subsection{Limerick.}
\begin{center}
\small
(Adare, Patrikswood, Scagh, Brasky, Faha), Grafschaft Limerick, Irland.

10. September 1813, 9 Uhr Morgens.
\end{center}
\paragraph{}
Dunkelasch- oder bläulichgraue, rostbraun gefleckte Grundmasse, mit einzelnen deutlichen, meist aber undeutlichen kugeligen Ausscheidungen; viel fein eingesprengtes metallisches Eisen (auf polierten Flächen zuweilen zu Linien vereinigt) und sehr fein eingesprengter Magnetkies; schwarze, mehr oder weniger deutliche, zum Teil metallisch glänzende Ablösungsflächen; matte, zuweilen aderige Rinde. — (Von dem Meteorsteine von Tipperary nicht unterscheidbar.)

Drei Bruchstücke von den mehreren daselbst gefallenen Steinen, deren Gesamtgewicht nicht bekannt ist.

1. Fragment mit anpolierter Fläche und stark aderiger, dicker Rinde. — 3 31/32 Loth. — 1818. 26. B. 194. — Geschenk von Professor Gieseke in Dublin.

2. Längliches Fragment mit brauner, glatter Rinde. — 3 1/8 Loth. — 1827. 27. 4054. — Aus der von der Nüll'schen Mineralien-Sammlung. Herr von der Nüll erhielt es von Professor Gieseke in Dublin.

3. Fragment mit metallisch glänzenden Ablösungen und ziemlich glatter Rinde. — 2 9/32 Loth. — 1821. 9. 12. — Von Herrn G. B. Sowerby in London gekauft.

Mit der Angabe des Fallortes Limerick erhielt das kaiserl, Mineralien-Kabinett später ein Fragment, das den hier aufgeführten drei Fragmenten nicht gleicht. (Siehe den Anhang zu Lucé.) Wir überzeugten uns, dass unsere drei Fragmente den im britischen Museum von der Lokalität Limerick aufbewahrten Stücken vollkommen ähnlich sind, durch ein Fragment, das Herr Heuland nach Wien schickte, und nun im Besitze von Baron Reichenbach ist. Sollten vielleicht bei Limerick, wie bei Weston, Steine von verschiedenem Aussehen gefallen sein?
\subsection{Grüneberg (Grünberg).}
\begin{center}
\small
(Heinrichau), Regierungsbezirk Liegnitz, Schlesien.

22. März 1841, Nachmittags 3 1/2 Uhr.
\end{center}
\paragraph{}
Dunkelasch- oder bläulichgraue Grundmasse, mit sehr undeutlichen , kleinkugeligen Einmengungen und schwarzen, glänzenden Ablösungsflächen. Da das kleine Fragment nicht anpoliert ist, so lässt sich über die Menge des eingemengten metallischen Eisens und des Magnetkieses kein sicheres Urteil fällen; ersteres scheint in ziemlicher Menge, aber fein eingesprengt vorhanden zu sein. Das Fragment zeigt auch nur unvollkommene, dünne Rinde. — (Dieser Meteorstein ist auf Bruchflächen von den Steinen von Tipperary und Limerick nicht zu unterscheiden.)

1. Kleines Fragment mit unvollkommener Rindenbildung, von einem einzeln gefallenen Steine, welcher in mehrere Stücke zerbrach, von denen zwei, in einem Gesamtgewichte von 2 Pfund 20 1/2 Loth aufgefunden wurden. — 1/2 Loth. — 1842. 35. 1. — Von Herrn Professor von Glocker in Breslau in Tausch erhalten.
\subsection{Tipperary.}
\begin{center}
\small
(Mooresfort), Grafschaft Tipperary, Irland.

August 1810, Mittags.
\end{center}
\paragraph{}
Dunkelasch-, fast bläulichgraue, mit einigen sehr feinen, schwarzen Adern durchzogene Grundmasse und wenigen schwachen Rostflecken; kleinkugelige, dunklere Ausscheidungen, die zuweilen auseinanderlaufend faserige Struktur zeigen; viel fein eingesprengtes gediegenes Eisen, und viel, sehr fein eingesprengter Magnetkies, von welch letzterem stellenweise auch einige größere Körner sichtbar sind; manchmal auch undeutliche, schwarze Ablösungsflächen; aderige, matte und dicke Rinde.

Zwei Bruchstücke von einem einzeln gefallenen Steine von 7 3/4 Pfund.

1. Bruchstück mit Rinde und einer anpolierten Fläche. — 14 17/32 Loth. — 1816. 75. 1. — Ein Geschenk des Professors Gieseke in Dublin.

2. Kleines Bruchstück mit ganz frischem Bruche, großen Rostflecken, einer Ablösungsfläche und etwas Rinde. — 1 3/8 Loth. — 1839. 4. 14. — Durch Herrn Dr. Baader von Herrn Heuland in London gekauft. Dieses Fragment rührt wahrscheinlich von dem jetzt im britischen Museum befindlichen Stücke von Tipperary her, ist aber jedenfalls, wie Herr Sowerby durch Herrn Heuland verbürgen ließ, damit vollkommen identisch.
\subsection{Gouvernement Kursk.}
\begin{center}
\small
Russland.

Ohne nähere Angabe des Fallortes, und ohne Angabe der Fallzeit erhalten.
\end{center}
\paragraph{}
Dunkelaschgraue Grundmasse, mit teils lichteren, teils dunkleren, zuweilen fast schwärzlichen, kugeligen Ausscheidungen; mit fein und mittelfein, wie es scheint, ungleichförmig eingesprengtem gediegenen Eisen und fein eingesprengtem Magnetkies; matte Rinde. (Wegen Kleinheit der vorhandenen Fragmente ist die Diagnose vielleicht nicht ganz vollständig.)

1. Drei sehr kleine Fragmente, sämtlich mit Rinde, eines davon anpoliert. — Zusammen 3/16 Loth. — 1838. 28. 5. — Von der kaiserl. russischen Akademie der Wissenschaften zu Petersburg durch Professor Kupffer in Tausch erhalten.

Über diesen Steinfall ist eben so wenig etwas öffentlich bekannt gemacht worden, wie über die aus den Gouvernements Simbirsk und Poltawa (nicht Kuleschofka); auch konnten wir darüber aus St. Petersburg, unserer Bemühungen ungeachtet, bisher keine nähere Notiz erhalten.
\subsection[Lixna.]{Lixna,}
\begin{center}
\small
bei Dunaburg, Gouv. Witepsk, Russland (ehemals polnisch Liefland oder Litauen).

12. Juli (oder 30. Juni alten Styls) 1820, zwischen 5 und 6 Uhr Abends.
\end{center}
\paragraph{}
Fast dunkelaschgraue, mit kleinen Rostflecken durchsäete und von schwarzen Linien durchzogene Grundmasse, mit zahlreichen, aber kleinen, dunkelgrauen, mit der Grundmasse fest verwachsenen und daher aus derselben auf Bruchflächen nicht hervortretenden kugeligen Ausscheidungen; viel fein und mittelfein eingesprengtes gediegenes Eisen und sehr fein eingesprengter Magnetkies; zahlreiche schwarze und glänzende Ablösungsflächen, welche diesen Stein besonders auszeichnen; matte, etwas raue Rinde.

1. Fragment mit Rinde, zwei Absonderungs- und einer anpolierten Schnittfläche, von einem der mehreren allda gefallenen Steine. — 14 11/32 Loth. — 1838. 9. 1. — Von Herrn Doktor Estreicher, Professor der Naturgeschichte an der Universität zu Krakau, in Tausch erhalten.
\subsection{Tabor.}
\begin{center}
\small
(Plan, Strkow u. s. w.) im Taborer (ehemals Bechiner) Kreise, Böhmen.

3. Juli 1753, 8 Uhr Abends.
\end{center}
\paragraph{}
Dunkel- fast bläulichgraue, rostbraungefleckte, dichte und stark zusammenhängende Grundmasse, mit meist kleinen und nicht sehr deutlichen kugeligen Ausscheidungen; viel fein und grob eingemengtes, zum Teil auch zu Adern und rundlichen Partien vereinigtes metallisches Eisen; sehr fein eingesprengter Magnetkies; matte Rinde. — Einer der eisenreichsten Meteorsteine.

Sieben Stücke, darunter ein großer ganzer Stein, ein kleinerer fast ganzer und ein kleiner ganzer, von den ziemlich vielen der allda gefallenen Steine.

1. Großer, ganzer Stein, fast 7 Zoll lang, 2 1/2 Zoll hoch, verschoben viereckig; die Rinde an zwei Stellen angebrochen; auch sonst noch kleine, vom Falle herrührende Beschädigungen an Ecken und Kanten. In einer Vertiefung steckt ein bohnengroßes Eisenkorn; Spuren von anderen sind an der Rinde vorhanden. — Beschrieben und abgebildet in von Schreibers Beiträgen, S. 10. Taf. 2. — 4 Pfund 31 Loth. — \mars 1. 4. — Ist von dem damaligen Kreishauptmann zu Tabor, Grafen von Wratislaw, gleich nach der Begebenheit im Jahre 1753 mit einem umständlichen Berichte an das böhmische Gubernium, und von diesem an die k. k. Hofkanzlei eingesendet worden.

2. Fast ganzer Stein, von vierseitig prismatischer Form, oben mit einer frischen Bruchfläche; an einer der 4 Seitenflächen eine Ablösungsfläche. — 1 Pfund 3 1/2 Loth. — 1840. 11. 1. — Von Herrn Ludwig von Scala aus der Mineralien-Sammlung des verstorbenen Grafen Gregor Razoumovsky gekauft, wo der Stein ohne Bezeichnung des Fundortes lag. Stammt wahrscheinlich aus der Mineralien-Sammlung des Fürsten Sinzendorf, die Graf Razoumovsky kaufte.

3. Ein ganzer, aber entzweigebrochener kleiner Stein, von interessanter prismatischer Form, an dem einen Ende dicker. — 1 7/8 Loth. — 1832. 6. 7. — Stammt aus der Mineralien-Sammlung des verstorbenen Baron Thavonat, und wurde später durch Doktor Baader an das k. k. Mineralien-Kabinett verkauft.

4. Vierseitig pyramidales, stark umrundetes Bruchstück eines großen Steines; die Bruchfläche, welche während des Falles entstanden und unvollkommen überrindet ist, zum Teil anpoliert. — 31 Loth. — 1841. 14. 2. — Aus der Heuland'schen Sammlung durch Herrn Pötschke gekauft. War irrtümlich als L'Aigle bezeichnet.

5. Dünner, plattenförmiger Abschnitt, mit polierter Fläche und viel Rinde. — 2 29/32 Loth. — 1838. 4. 1. — Von Herrn Grafen von Beroldingen, k. k. Kämmerer, eingetauscht. Lag früher in der Mineralien-Sammlung des Herrn Morgenbesser, ohne Angabe des Fallortes.

6. Ein kleines Bruchstück mit Rinde. — 2 19/32 Loth. — 1811. 16. 1. — Von Doktor Pohl in Prag zu Tausch erhalten.

7. Eine viereckige, von beiden Seiten polierte Platte, mit einer Eisenader, — 2 17/32 Loth. \mars 1. 5. — Vom verstorbenen Kabinetts-Kustos von Mühlfeld erhalten.

Die Meteorsteine von Tabor (1753) sind die ersten, die in wissenschaftliche Sammlungen kamen, Der große Stein von Tabor (Nr. 1) in der Sammlung des k. k. Mineralien-Kabinettes und die ebenfalls darin befindliche berühmte Eisenmasse von Agram (1751), von der später die Rede sein wird, waren die ersten in ihrer Integrität verbliebenen Meteoriten, die Chladni, der Meister in der Meteoritologie, Leopold v. Buch und andere Gelehrte zu sehen bekamen. Die Fragmente des älteren Meteorsteines von Ensisheim (1492) wurden, wie bei dieser Lokalität (Nr. 15) bemerkt worden ist, erst später in Zirkulation gesetzt und hatten als Fragmente, wie die Bruchstücke des Pallasischen Eisens weniger Interesse. — Frühere Gelehrte ließen sich durch den Anblick der erwähnten Massen von ihrer vorgefassten Meinung nicht abbringen. Born äußert in seinem Lithophylacium Bornianum, B. 1. S. 125, bei den Steinen von Tabor: quae 3 Julii anni 1753 inter tonitrua e Coelo pluisse creduliores quidam asserunt.
\subsection{Charsonville.}
\begin{center}
\small
(Orléans, Beaugency, Mortelle, Villerai, Moulin-brule) Dépt. du Loiret, Frankreich.

23. November 1810, 1 1/2 Uhr Nachmittags.
\end{center}
\paragraph{}
Ins Dunkelasch- oder Bläulichgraue ziehende, dichte und feste, von vielen Rostflecken wie marmoriert aussehende Grundmasse; die kugeligen Ausscheidungen undeutlich und mit der Grundmasse innig verwachsen; sehr viel fein und gleichförmig eingesprengtes gediegenes Eisen und höchst fein eingesprengter Magnetkies; dickere und dünnere, etwas verzweigte Adern; matte, etwas schimmernde Rinde.

Zwei Bruchstücke von einem der zwei aufgefundenen Steine, wovon einer 40, der andere 20 Pfund wog.

1. Großes Bruchstück mit Rinde und einer großen anpolierten Schnittfläche. — Beschrieben und abgebildet in von Schreibers Beiträgen, S. 65, Taf. 7. — 30 Loth. — 1816. 31. 33. f. — Von dem königl. Museum der Naturgeschichte zu Paris auf Vermittlung des Direktors von Schreibers von dem daselbst aufbewahrten 11 Pfd. schweren Bruchstücke in Tausch erhalten.

2. Kleines Bruchstück ohne Rinde und mit polierter Fläche. — 4 Loth. — 1841. 14. 8. — Aus der Heuland'schen Meteoriten-Sammlung durch Herrn Pötschke gekauft.

Dem Meteorsteine von Charsonville oder Orléans verleihen die darin befindlichen schwarzen Adern oder Gänge, die in keinem anderen Meteorsteine so schön und in solcher Deutlichkeit zu finden sind, ein besonderes Interesse. Große polierte Schnittflächen, wie die bei dem Stücke Nr. 1, stellen die Beschaffenheit und Verzweigung dieser Gänge am schönsten dar.
\subsection{Doroninsk.}
\begin{center}
\small
Gouv. Irkutsk, Sibirien.

25. März 1805, 5 Uhr Nachmittags.
\end{center}
\paragraph{}
Dunkelaschgraue, durch eine Menge von Rostflecken fast braun erscheinende, sehr dichte Grundmasse, mit undeutlichen und kleinen lichteren kugeligen Ausscheidungen, welche, mit der Grundmasse fest verwachsen, als kleine Flecken erscheinen; viel fein eingesprengtes gediegenes Eisen und höchst fein eingesprengter Magnetkies; schwarze Ablösungsflächen und undeutliche, sehr feine, die Masse durchziehende Adern; matte, schwarze Rinde. — (Ist von dem Meteorsteine von Seres in Makedonien kaum zu unterscheiden.)

1. Fragment mit Rinde und einer anpolierten Fläche, von einem der zwei allda aufgefundenen Steine, deren einer 7 und der andere 2 1/2 Pfund wog. — 1 3/4 Loth. — 1839. 22. 2. — Aus der Mineralien-Sammlung der königl. Universität zu Berlin durch Herrn Professor Weiss eingetauscht. Befand sich früher in der Klaproth'schen Mineralien-Sammlung.
\subsection{Seres.}
\begin{center}
\small
Makedonien, Turkey.

Juni 1818.
\end{center}
\paragraph{}
Dunkelasch- oder bläulichgraue, rostbraun gefleckte, sehr dichte Grundmasse, mit rundlichen lichteren Stellen, welche von kugeligen, aber mit der Grundmasse innig verbundenen Ausscheidungen herrühren; viel, meist fein eingesprengtes metallisches Eisen und höchst fein eingesprengter Magnetkies; gestreifte Ablösungsflächen, matte Rinde.

Zwei Bruchstücke von einem einzeln gefallenen Steine von 15 Pfund, welchen Yussuf-Pascha, Statthalter von Seres in Makedonien, seinem Leibarzte, Herrn Grohmann, schenkte. Letzterer brachte den Stein nach Wien und verehrte ihn seinem ehemaligen Lehrer, Hrn. Professor Andreas Ritter von Scherer, in dessen Besitz er sich noch befindet.

1. Bruchstück mit einer kleinen anpolierten Fläche. — 6 21/32 Loth. — 1832. 11. 1. — Geschenk von Herrn Ritter Pittoni von Dannenfeld in Grätz, der das Fragment von Doktor Grohmann erhielt.

2. Bruchstück mit frischem Bruche, ohne Rinde, mit einer verrosteten, schwach gefurchten Ablösungsfläche. — 3 3/32 Loth. — 1842. 26. 1. — In Tausch von Baron Lederer, k. k. österreichischen General-Konsul zu New-York, erhalten. Baron Lederer kaufte das Stück mit einer Mineralien-Sammlung in Wien mit der Etiquette: Syrmien, 1824, und schickte dasselbe zur näheren Bestimmung aus Nord-Amerika an das k. k. Mineralien-Kabinett, wo es sogleich für Seres erkannt wurde. Für die Richtigkeit der Bestimmung zeugte der später aufgefundene Umstand, dass eine Bruchfläche dieses Stückes an eine Bruchfläche des Stückes Nr. 1 anpasst. (Auch andere lange getrennt gewesene, aneinander passende Stücke anderer Lokalitäten haben sich in der Meteoriten-Sammlung des k. k. Mineralien-Kabinettes wieder zusammen gefunden.)
\subsection[]{Sigena.}
\begin{center}
\small
Dorf Sena, Bezirk Sigena, Aragonien, Spanien.

17. November 1773, Mittags.
\end{center}
\paragraph{}
Fast dunkelgraue, rostbraun gefleckte Grundmasse; mit wenigen kugeligen Ausscheidungen, viel fein eingesprengtem metallischen Eisen und wenig sehr fein eingesprengtem Magnetkies. — (Ist nahe verwandt mit den Steinen von Barbotan.)

1. Ein sehr kleines Bruchstück ohne Rinde, von einem einzeln gefallenen Steine von 9 Pfund 2 Loth. — 7/32 Loth. — 1816. 31. 33. g. — Wurde auf Verwendung des Direktors v. Schreibers im Jahre 1815 von dem kleinen Stücke von Sigena im königl. Museum der Naturgeschichte zu Paris abgeschnitten und in Tausch erhalten.
\subsection{Barbotan.}
\begin{center}
\small
(Roquefort, Créon, Juillác, Mezin, Eause, Armagnac, Losse, Agen, St. Sever, La Grange), im Dépt. des Landes, im Dépt. du Lot et Garonne und im Dépt. du Gers (Gascogne), Frankreich. (Werden zuweilen auch Meteorsteine von Bordeaux genannt.)

24. Juli 1790, 9 Uhr Abends.
\end{center}
\paragraph{}
Fast dunkelgraue, stark rostbraun gefleckte, feste Grundmasse, mit sehr wenig kugeligen Ausscheidungen; sehr viel, meist fein eingesprengtes metallisches Eisen, das hie und da in größeren, zuweilen linsen- und bohnengroßen Körnern, und auch in unvollkommenen Hexaedern hervortritt; sehr fein eingesprengter Magnetkies; schimmernde schwarze Ablösungsflächen; matte Rinde.

Zwei Bruchstücke von zwei der vielen allda gefallenen Steine, wovon einige über 20 Pfund wogen.

1. Großes Bruchstück mit Rinde und einer ansehnlichen, polierten Fläche; an der Bruchfläche ragen zwei unvollkommene Hexaeder von gediegenem Eisen hervor. — 19 3/4 Loth. — 1841. 14. 4. — Aus der Heuland'schen, später Heath'schen Meteoriten-Sammlung durch Herrn Pötschke gekauft.

2. Bruchstück ohne Rinde, mit lachen linsengroßen Eisenkörnern und zwei schwarzen Ablösungsflächen. — 15 21/32 Loth. — 1827. 27. 4052. Aus der von der Nüll'schen Mineralien-Sammlung.

Der Meteorsteinfall von Barbotan ist einer der beträchtlichsten und ausgedehntesten gewesen, da er sich über mehrere Ortschaften verschiedener, jedoch benachbarter Departements erstreckte. Er fiel in die Zeit völligen Unglaubens und größter Verstockung (1790). Die Äußerungen von Bertholon im Journal des sciences utiles vom Jahre 1790 über den Verbal-Prozess, den die Munizipalität von Juillac über das Phänomen und den Steinfall abgefasst hatte (d'un fait évidement faux, d'un phénomène physiquement impossible) und über die Erzählungen von Augenzeugen des Ereignisses (qui ne peuvent qu'exciter la pitié, nous ne dirons pas seulement des physiciens, mais de tous les gens raisonnables) verdienen, dass sie als Beiträge zur Geschichte der Meteoriten nicht in Vergessenheit geraten.
\subsection{Eichstädt.}
\begin{center}
\small
(Wittens), Franken, Königreich Bayern.

19. Februar 1785, Mittags.
\end{center}
\paragraph{}
Dunkelgraue Grundmasse mit vielen Rostflecken; zahlreiche, auf Bruchflächen aus der Grundmasse hervorragende kleinkugelige Ausscheidungen; viel mittelfein eingesprengtes metallisches Eisen; weniger und sehr fein eingesprengter Magnetkies; dicke, matte Rinde.

Zwei Bruchstücke von einem, so viel bekannt, einzeln gefallenen Steine von 5 Pfund 22 Loth.

1. Bruchstück mit Rinde und einer anpolierten Fläche. — Beschrieben und abgebildet in von Schreibers Beiträgen, Seite 13, Taf. 2. — 6 31/32 Loth. — \mars 1. 8. — Wurde durch den Domherrn von Hompesch in Eichstädt, um das Jahr 1789, dem damaligen Direktors-Adjunkten am k. k. Naturalien-Kabinette, Abbé Stütz, mitgeteilt.

2. Kleines Bruchstück mit frischem Bruche und etwas Rinde. — 11/32 Loth. — 1840. 4. 6. — Aus der Heuland'schen, später Heath'schen Meteoriten-Sammlung durch Herrn Pötschke gekauft. Stammt aus der de Drée'schen Mineralien-Sammlung (in welcher es beim Verkaufe durch Verwechslung die Etiquette mit dem Fallorte Mauerkirchen führte).
\subsection[Groß-Divina.]{Groß-Divina,}
\begin{center}
\small
nächste Budetin, unweit Sillein, im Trentschiner-Komitate, Ungarn.

24. Juli, 1837, 1/2 12 Uhr Mittags.
\end{center}
\paragraph{}
Zwischen dunkel- und lichtaschgrau schwankende, mit braunen Rostflecken erfüllte Grundmasse, mit einer großen Anzahl von kleinen, dunkelgrauen kugligen Ausscheidungen, die auf Bruchflächen aus der Grundmasse zum Teil hervorragen; mit ziemlich viel fein eingesprengtem metallischen Eisen und höchst fein eingesprengtem Magnetkies; matte, teils ziemlich glatte, teils höchst raue Rinde. — (Steht den Meteorsteinen von Timochin, Zebrak und Eichstädt sehr nahe.)

1. Anpoliertes Fragment, mit Rinde von zweierlei Beschaffenheit, von einem einzeln gefallenen Steine von 19 Pfund Wiener Gewicht, der sich jetzt im k. National-Museum zu Pesth befindet. — 3 11/16 Loth. — 1838. 1. 1. — Geschenk von Herrn Johann Lottner, Pfarrer zu Groß-Divina.

Über den Steinfall von Groß-Divina oder Budetin besitzen wir nur Zeitungsnachrichten (darunter eine in der allgemeinen Zeitung vom 27. August 1837) und eine kleine Notiz von H. Zipser in Leonhards und Bronns Jahrbuch für Mineralogie (Jahrgang 1840. S. 89.) Wir haben uns kurz nach dem Falle sowohl um den einzeln gefallenen Stein, der jedoch schon dem ungarischen National-Museum in Pesth zugesichert worden war, als um Nachrichten über das Ereignis beworben. Der Stein kam durch die Gefälligkeit der Gräfin Ludmilla von Csaky, Herrschaftsbesitzerin von Budetin, zur Ansicht in das k. Mineralien-Kabinett nach Wien, wo wir davon Zeichnungen und ein Gipsmodell anfertigen ließen. Er gehört durch seine Form und Überrundung und durch die Eindrücke au einem Teile seiner Oberfläche zu den merkwürdigsten Meteorsteinen. Herr Professor Sadler, Kustos am National-Museum zu Pesth, ist beschäftiget, über diesen und den Meteorstein von Milena in Kroatien (siehe Nr. 39) Notizen zu sammeln und diese, nebst dem Resultate der chemischen Untersuchung der zwei Steine, der wissenschaftlichen Welt bekannt zu geben.
\subsection{Zebrak.}
\begin{center}
\small
(Horzowitz, Praskoles), Berauner Kreis, Böhmen.
\end{center}
\paragraph{}
Mehr dunkel- als lichtgraue, aber ganz mit braunen Rostflecken erfüllte Grundmasse; viele kleine auf Bruchflächen aus der Grundmasse zum Teil heraustretende und daher mit ihr nicht fest verwachsene kuglige Ausscheidungen; viel ziemlich fein eingesprengtes gediegenes Eisen und viel sehr fein eingesprengter Magnetkies; dicke und matte Rinde. — (Ist den Steinen von Eichstädt und Smolensk nahe verwandt.)

1. Bruchstück mit viel Rinde und anpolierter Fläche, von einem in drei Stücke zerfallenen Steine, welche zusammen etwa 4 Pfund gewogen haben sollen. (Es fiel auch ein zweiter Stein, der aber nicht aufgefunden worden zu sein scheint.) — 20 3/16 Loth. — 1832. 31. 1. — Vom vaterländischen Museum in Prag durch Tausch erhalten.
\subsection{Timochin.}
\begin{center}
\small
Gouv. Smolensk, Russland.

13. März, 1807. Nachmittags.
\end{center}
\paragraph{}
Zwischen licht- und dunkelaschgrau schwankende Grundmasse, mit vielen Rostflecken und dunkleren, aus der Grundmasse heraustretenden kugligen Ausscheidungen; viel fein eingesprengtes gediegenes Eisen und sehr fein eingesprengter Magnetkies; dicke und matte Rinde.

Zwei Bruchstücke von einem einzeln gefallenen Steine von 140 Pfund.

1. Bruchstück mit Rinde und anpolierter Fläche. — Beschrieben und abgebildet in von Schreibers Beiträgen Seite 63. Taf. 7. — 4 3/4 Loth. — 1810. 2. 3. — Von dem verstorbenen Professor und Medizinalrat Klaproth in Berlin als Abschnitt von einem größeren Bruchstücke gekauft.

2. Fragment mit etwas Rinde und ganz frischem Bruche. — 3 1/4 Loth. — 1838. 28. 3. — Von der kaiserl. russischen Akademie der Wissenschaften in Petersburg durch Herrn Professor Kupffer in Tausch erhalten.
\subsection[Macao.]{Macao,}
\begin{center}
\small
Dorf am Flüsse Açu (Assu), oder Amargoro, Provinz Rio grande do Norte (nicht Ceara), Brasilien.

11. November (nicht 11. Dezember) 1836, 5 Uhr Morgens.
\end{center}
\paragraph{}
Fast dunkelaschgraue, mit einer großen Menge von Rostflecken durchsäete, sehr feste Grundmasse, mit undeutlichen, kugligen Ausscheidungen; mit einer großen Menge meist fein eingesprengten metallischen Eisens, das sich jedoch oft zu geraden, oder krummen dicken Linien zusammenhäuft, und viel sehr fein eingesprengtem Magnetkies; matte oder schwach schimmernde, meist stark verrostete, zuweilen verschlackte Rinde. — (Hat große Ähnlichkeit mit dem Meteorstein von Timochin.)

Drei kleine ganze Steine und vier Bruchstücke von der ungeheureren Menge der allda gefallenen Steine.

1. Rundlicher, vollkommen ganzer Stein, mit verrosteter Rinde. — 3 Loth. — 1839. 27. 5. — Auf Vermittlung des Herrn Johann Natterer, Kustos-Adjunkten am k. k. Hof-Naturalien-Kabinette, durch gütige Bemühung des Herrn Tegetmeyer, österreichischen Vize-Konsuls-Stellvertreter zu Pernambuco, und des H. Breisky, österreichischen Vize-Konsuls zu Bahia, von dem ersteren mit den folgenden sechs Stücken als Geschenk erhalten.

2. Ganzer, rundlicher Stein, an einer Seite gewölbt, an der andern flach; an letzterer etwas angebrochen, verrostet und mit verschlackter Rinde. — (Erhielt nach seiner Einsendung Sprünge, die durch das fortgesetzte Rosten des Eisens entstanden.) — 7 7/16 Loth. — 1839. 27. 3. — Mit Nr. 1 erhalten.

3. Ganzer, länglicher Stein, an der Oberfläche stark verrostet, mit Rinde von zweierlei Beschaffenheit, ohne Bruchfläche. — 11 5/16 Loth. — 1839. 27. 2. — Mit Nr. 1 erhalten. —

4. Bruchstück eines größeren Steines mit Rinde, alten verrosteten Bruchflächen und einer großen anpolierten Schnittfläche, auf welcher sich Linien von gediegenem Eisen hinziehen. — 9 1/2 Loth. — Von: 1839. 27. 1. Mit Nr. 1 erhalten.

5. Längliches Fragment, ungefähr ein halber Stein, mit verrosteter Rinde und rostiger Bruchfläche. — 3 5/8 Lth. — Von: 1839. 27. 4. — Mit Nr. 1 erhalten.

6. Fragment mit anpolierter Fläche, dann veralteten und frischen Bruchflächen und etwas Rinde. — 2 13/32 Loth. — Von: 1839. 27. 1. — Mit Nr. 1 erhalten.

7. Fragment mit polierter Fläche, mit alten und neuen Bruchflächen und etwas Rinde. Die eisenreiche Schnittfläche ist zur Hälfte schwach, zur Hälfte stark geätzt, wodurch auf den, eine unterbrochene Linie bildenden Eisenteilchen feine Linien, die fast Widmanstättensche Figuren bilden, zum Vorschein kamen. — 1 Loth. — Von: 1839. 27. 1. — Mit Nr. 1 erhalten.

Über den Steinregen von Macao oder Rio Assu sind nur ungenügende Nachrichten bekannt gemacht worden. Sie rühren von einem Franzosen, H. F. Berthou, her, der damals in Olinda, in der Provinz Ceara wohnte. Er hat die Gegend des Steinfalls nicht besucht und das Ereignis nur nach Hörensagen beschrieben. Nach Nachrichten, die sich Herr Breisky in Bahia auf Ansuchen von Herrn Tegetmayer in Pernambuco aus der Stadt Ceara verschaffte, liegt das kleine Dorf Macao, das erst im J. 1828 zu entstehen anfing, und im J. 1839 31 Häuser zählte, am Ufer des Flusses Assu oder Amargoro, nicht weit von seinem Ausflusse in das Meer und gehört zur Provinz Rio grande do Norte. Der Steinregen hat sich nach unsern Nachrichten den 11. November, nach dem Briefe des H. Berthou in dem Compte rendu der Pariser Akademie vom 14. August 1837, S. 211, den 11. Dezember 1836 ereignet. Die Anzahl der gefallenen Steine muss ungeheuer groß gewesen sein, da sich der Niederfall auf 3 Meilen, von dem Gute Cacimbas bis zur Mündung des Riv Assu erstreckte (nach dem Berichte von H. Berthou über einen Flächenraum von mehr als 10 Lieues); eigentlich scheint aber das Phänomen zwei große Explosionen, eine zu Cacimbas, die andere an der Flussmündung gemacht zu haben. Die gefallenen Steine sind nach den Nachrichten, die uns zukamen, klein gewesen, meist von der Größe eines Taubeneies; der französische Bericht In dem Compte rendu spricht dagegen von Steinen von 1 bis 80 Pfund. Sie sollen nach demselben Berichte einige Ochsen getötet oder verwundet haben. Ein Muster dieser Meteorsteine ist an die Akademie der Wissenschaften in Paris geschickt worden, welche die Analyse (die jedoch bisher noch nicht erschienen ist) Herrn Berthier übertrug. Außer diesem Muster und den Exemplaren, die das k. k. Mineralien-Kabinett durch die eifrige und gütige Bemühung der Herren Tegetmayer und Breisky erhielt, scheinen keine anderen von diesem Ereignisse nach Europa gekommen zu sein. Aus England und auf andern Wegen auch von hier aus hat man sich fruchtlos darum bemühet. Da die Steine von Macao zu den eisenreichsten gehören, werden sie bei längerem Liegen in der Erde durch Oxydierung des Eisens zerbersten und endlich zerfallen, so dass man nach einigen Jahren, wenigstens von den kleineren, kaum mehr etwas wird finden können. Selbst ein paar Stücke unserer Sammlung, die mehr als zwei Jahre in der Erde gelegen haben mögen, zerbarsten seit sie sich darin befinden und dürften kaum zu retten sein.
\begin{center}
{\LARGE Meteoreisen.}

Nr. 70 bis 94.
\end{center}
\subsection[--- Meteoreisen --- Nr. 70 bis 94. --- Atacama.]{Atacama.}
\begin{center}
\small
Bei dem Dorfe San Pedro, 20 Leagues von dem Hafen Cobija entfernt, Provinz Atacama, Republik Bolivia (ehemals Peru), an der Grenze von Chili, Süd-Amerika.
\end{center}
\paragraph{}
Der wissenschaftlichen Welt seit 1827 bekannt; die Fallzeit ist unbekannt. — An dem bezeichneten Fundorte soll eine Masse von 3 Zentnern (Quintals) und viele kleinere Stücke zerstreut herum liegen.

Ein Gemenge von gediegenem Eisen mit einem gleichen Verhältnis von lichtgrünem, fast grünlich weißen (vom Eisen jedoch später teilweise rostbraun gefärbten) Olivin, in meist feinkörnigem Gefüge, und mit Magnetkies, der aber nur in sehr geringer Menge vorhanden und nur auf polierten Schnittflächen unterscheidbar ist. Das metallische Eisen bildet ein ästiges oder schwammförmiges, von dem Olivin ausgefülltes Gerippe. Auf Durchschnitten zeigt sich das Eisen in Feldern mit aus- und einspringenden Winkeln, und die vom Olivin erfüllten Zellen daher ebenfalls eckig, selten rund. Durch Behandlung des Eisens mit Säuren entstehen in der Mitte der Eisenpartie eckige mit den Rändern derselben parallele, durch glänzende Leisten eingefasste und öfter von Linien durchzogene dunkle Felder, während der größere Teil des den Rändern näher liegenden metallischen Eisens weniger oder gar nicht angegriffen wird, und daher den Metallglanz behält. Der Olivin ist in größeren Körnern von dem Eisen nicht trennbar, sondern zerbröckelt, vermöge seiner kleinkörnigen Struktur.

1. Ein großes merkwürdiges Stück, rings von gestreiften, teilweise glänzenden natürlichen Ablösungsflächen umgeben; mit einer polierten Schnittfläche. — 5 Pfd. 5 1/2 Loth. — 1842. 1. 1. — Aus der ehemals Heuland'schen, später Heath'schen Meteoriten-Sammlung, durch H. Carl Pötschke in Wien gekauft. Herr Heuland kaufte das Stück durch Vermittlung des Herrn Brooke in London von dem Agenten einer südamerikanischen Bergwerksgesellschaft.

2. Bruchstück, durch das Entzweireißen der Masse an der Oberfläche kurzzahnig, mit anpolierter Fläche. — 1 Pfund, 1 Loth. — 1834. 5. 1. — Durch Doktor Bondi in Dresden von Herrn Heuland in London zu Kauf erhalten.

3. Ein dickes Plättchen von Nr. 2 abgeschnitten, mit polierter und dann geätzter Schnittfläche. — 6 Loth. — Von 1834. 5. 1.

Herr Hippolyte Jubin, königl. französischer Schiffslieutenant, brachte ein Meteoreisen aus Peru nach Frankreich (in das Museum von Angers), das bei Potosi in Bolivia gefunden worden sein soll. (Siehe Chronique scientifique, eine Beilage zum Institut, Nr. 8, von 24. Februar 1839, dann Poggendorffs Annalen, Band 47. S. 470.) Es gleicht nach der Beschreibung ganz dem Meteoreisen von Atacama, und der Fundort Potosi ist wohl nur eine irrige Angabe von Seite des Mittheilers des Stückes an Herrn Jubin.
\subsection{Krasnojarsk.}
\begin{center}
\small
Zwischen Krasnojarsk und Abakansk, Gouv. Jeniseisk, Sibirien.
\end{center}
\paragraph{}
Auch Pallasisches Eisen genannt, weil der Naturforscher Pallas, der diese Eisenmasse im Jahre 1772 kennen lernte, dieselbe später im Jahre 1776 im 3. Teile seiner Reise durch verschiedene Provinzen des russischen Reiches ausführlich bekannt machte. Die Fallzeit ist unbekannt. Die Masse wurde von einem Kosaken im J. 1749 aufgefunden. Sie wog ursprünglich 40 Pud oder 1600 russische Pfund.

Die Eisenmasse von Krasnojarsk, gegenwärtig in der Mineralien-Sammlung der k. Akademie der Wissenschaften zu St. Petersburg aufbewahrt, wiegt jetzt noch 1270 russische Pfund. (Siehe G. Rose, Reise nach dem Ural u. s. w. 1. Band. S. 43.) Außer dieser großen durch Pallas im J. 1776 nach St. Petersburg gesendeten Masse dürften Später noch mehrere kleinere Stücke hei Krasnojarsk aufgefunden, und nach Europa gebracht worden sein; eine Vermutung, die wir einer gefälligen Mittheilung von Herrn Heinrich Heuland in London entnehmen. Derselbe schrieb, dass sein ehemaliger Associe Sitrikoff im J. 1807 über zwei Zentner des Pallasischen Eisens zu Moskau auf dem dortigen Trödlermarkte (der Lausemarkt genannt)als altes Eisen nach dem Pfund in der Bude eines Eisenhändlers kaufte, wo es unter zerbrochenen Eisengefäßen lag. Herr Heuland sah die Pallas'sche Eisenmasse in St. Petersburg zu wiederholten Mahlen, und ist überzeugt, dass die erwähnten Stücke wenigstens nicht nach dem J. 1796, wo er die Masse das erste mahl sah, von derselben abgeschlagen worden sind. Die größten von den am Lausemarkt zu Moskau verkauften Stücken kamen in die Sammlung des ehemaligen russischen Reichskanzlers Romanzow und an die Oxforder Universität. — Sollten die Moskauer Stücke etwa von einer neuen, nicht bekannt gewordenen Lokalität herrühren? Es ist zu bedauern, dass man damals der Sache nicht mehr nachspürte.

Ein Gemenge von gediegenem Eisen mit Olivin (oder Chrysolith) ungefähr in gleichem Verhältnis, und mit etwas Magnetkies. Das gediegene Eisen bildet ein ästiges oder schwammförmiges Gerippe, das der Olivin ausfüllt. Die ästige Gestalt zeigt sich nur dann deutlich, wenn der Olivin aus den Höhlungen herausgefallen ist. Der in weit geringerer Menge vorhandene Magnetkies liegt zwischen den Verzweigungen des Eisens und hilft mit das Gerippe zu konstituieren; ist jedoch fast nur auf Schnittflächen wahrzunehmen. Auf diesen erscheint der Olivin zwischen dem metallischen Eisen und dem Magnetkies in runden, nicht in eckigen Zellen. Durch mäßiges Ätzen mit Säuren werden die Ränder oder die Außenwände des Eisens fast nicht angegriffen und bleiben glänzend oder fast glänzend, während die Mitte oder der Kern des Eisens in ein mit den Rändern paralleles graues und mattes Feld umgeändert wird, das von erhöhten leisten eingefasst und zuweilen mit einzelnen, oder auch mehreren untereinander parallelen Linien durchzogen ist. Der Olivin zeigt verschiedene Nuancen von Grün, vom schönsten und hellsten Pistaziengrün bis zu einem dunklen und schmutzigen Bräunlichgrün. Die einzelnen Olivinkörner besitzen oft unregelmäßige Flächen, die durch Berührung mit andern Olivinkörnern und mit den Wänden des Eisens entstanden sind (jedoch sind daran von Gustav Rose auch wahre Krystallflächen beobachtet worden), auch durchziehen den Olivin zuweilen dünne Linien oder Adern von metallischem Eisen.

Das Meteoreisen von Krasnojarsk unterscheidet sich von jenem von Atacama vorzüglich durch die Beschaffenheit des Olivins und die runde Form der Zellen.

1. Großes Bruchstück, zum Teil mit natürlicher Oberfläche und unvollkommener Rindenbildung; in einigen Höhlungen schwärzlich angelaufen; mit wenigen zahnigen Hervorragenden; der Olivin ist noch größtenteils vorhanden. — 4 Pfund 15 Loth. — \mars 1. 1. — Dieses Stück kann nicht aus der Mineralien-Sammlung des Freiherrn von Baillou, welche der römische Kaiser Franz 1. im J. 1748 kaufte, wie in dem Verzeichnisse der Meteormassen des k. k. Mineralien-Kabinettes in Chladnis Werke über Feuer-Meteore vermutet wird, herstammen, sondern muss später angekauft worden sein.

2. Ein zum Teil ausgezeichnet ästiges Stück, da der Olivin (oder eigentlich Chrysolith) aus einem Teile der Masse herausgefallen ist. Von diesem Mineral befindet sich an einer Stelle der Oberfläche ein schön pistaziengrünes, durchsichtiges, erbsengroßes Korn, mit glatten glänzenden Flächen, das man für einen Kristall halten könnte. Das Stück besitzt eine polierte, aber nur zur Hälfte geätzte Schnittfläche, wodurch sowohl das Gemenge von Eisen, Magnetkies und Olivin, als auch auf dem geätzten Teile die charakteristische Zeichnung des Eisens zum Vorschein kamen. — Ein schönes, höchst ausgezeichnetes und lehrreiches Stück. — 27 1/32 Loth. — 1827. 27. 4042. — Beschrieben und abgebildet in v. Schreibers Beiträgen, Seite 84. Taf. 8. — Aus der von der Nüll'schen Mineralien-Sammlung.

3. Ein vollkommen ästiges Stück, fast ohne Olivin, der beim Zerschlagen herausgefallen ist. — 12 7/16 Lth. — 1827. 27. 4043. — Aus der von der Nüll'schen Mineralien-Sammlung.

4. Kleines Stück mit Olivin und einer geätzten Schnittfläche. — 8 Loth. — 1842. 1. 4. — Aus der Heuland'schen Sammlung durch Herrn Pötschke gekauft.

5. Kleines Stück mit Olivin und einer anpolierten Fläche. — 6 7/32 Loth. — 1841. 14. 13. — Aus der Heuland‘schen Sammlung durch Herrn Pötschke gekauft.

6. Lose, ausgewählte Körner von Olivin, zusammen 7/8 Loth wiegend. — 1838. 25. 15. — Von Herrn Pötschke gekauft, als Auswahl aus den Abfällen bei dem Zerschlagen eines großen 6 Pfund 12 Loth schweren Stückes von diesem Meteoreisen, aus der ehemals Heulandischen Sammlung.
\subsection{Brahin.}
\begin{center}
\small
Am Zusammenflusse des Dnieper und Prypetz, Retschitzer Kreis, Gouv. Minsk, Russland (ehemals Litauen).
\end{center}
\paragraph{}
Es wurden da im Jahre 1810 (nicht 1809) zwei Stücke gefunden, die zusammen ungefähr 200 Pfund wogen, und wovon der größte Teil in den Jahren 1821 und 1822 an die Universität zu Wilna und von da später mit den Sammlungen der Universität nach Kiew kam.

Aus gediegenem Eisen und Olivin gemengte Masse (Magnetkies wird wohl auch vorhanden sein, ist aber in dem kleinen, uns zu Gebote stehenden Stücke nicht sichtbar). Das metallische, das schwammförmige Gerippe bildende Eisen scheint schmälere, weniger ausgedehnte Partien zu bilden, als dies bei den analogen Massen von Atacama und Krasnojarsk der Fall ist, und der Olivin (dem sibirischen im Aussehen gleich) fast der vorwaltende Gemengteil zu sein. Auszeichnend für die Brahiner Masse ist der Umstand, dass bei Ätzung des Eisens, die matt werdenden Mittelfelder verhältnismäßig sehr ausgedehnt und dagegen die sie einschließenden glänzenden Ränder sehr schmal sind. An dem kleinen Stücke der k. Sammlung ist eine mit ganz kleinen braunen Olivinkörnern besäete Stelle vorhanden; wahrscheinlich eine Stelle der natürlichen Oberfläche der Masse.

1. Fragment mit Olivin und 2 kleinen anpolierten und geätzten Flächen. — 1 1/16 Loth. — 1839. 28. 3. — Aus der Mineralien-Sammlung der königl. Universität zu Berlin durch Herrn Professor Weiss in Tausch erhalten. Das im Berliner Museum befindliche größere Exemplar ist ein Geschenk des Herrn Professors Eichwald in Wilna (jetzt in St. Petersburg).

Über die Eisenmassen von Brahin ist in deutschen wissenschaftlichen Zeitschriften wenig gesagt worden. Nähere Nachrichten darüber findet man in einer kleinen, polnisch geschriebenen Schrift von Felix Drzewinski, die in Jahre 1825 zu Wilna erschien, und den Titel führt: Über Meteorsteine und die mögliche Ursache ihrer Entstehung. Wir verdanken sie der gütigen Mittheilung des kaiserl. russischen Staatsrates und Professors von Eichwald in Petersburg.
\subsection{Sachsen.}
\begin{center}
\small
Steinbach, zwischen Johann Georgenstadt und Eibenstock; oder Naunhof bei Grimma?
\end{center}
\paragraph{}
Man sehe was Chladni in seinem Werke über Feuer-Meteore, Seite 326, über die Herstammung dieser Eisenmasse bemerkt. — Die Vermutung, dass die in den Sammlungen vorhandenen Stücke des sächsischen Meteoreisens Teile von der vor der Mitte des sechzehnten Jahrhunderts (zwischen den Jahren 1540 und 1550) im Walde bei Naunhof, nicht weit von Grimma (nicht Grimme, wie Chladni schreibt) gefallenen Masse seien, ist höchst unwahrscheinlich. Die wenigen Stücke, die man vom sächsischen Meteoreisen in Sammlungen findet (wir entdeckten zwei kleine Stücke erst im vorigen Sommer noch in den Mineralien-Sammlungen der königl. Akademie der Wissenschaften zu Stockholm und der Universität zu Uppsala, wo sie als sibirisches Eisen lagen), mögen teils von der ansehnlichen Stufe herrühren, die Marggraf bei den Steinbacher Seifenwerken zwischen Eibenstock und Johann Georgenstadt gefunden hat (siehe Lehmann Einleitung in einige Teile der Bergwerkswissenschaft. Berlin, 1751, S. 79), teils von dem Stücke oder den Stücken, die in der Sammlung des ehemaligen sächsischen Berghauptmannes von Schönberg mit der Etiquette lagen: "ein kurioses Stück gediegen Eisen, so auf dem Felde gefunden worden," und das wohl auch von Steinbach sein dürfte. Weitere Daten über die Ungewissheit, die noch über die Herstammung des vermeintlichen Meteoreisens von Sachsen herrscht, werden weiter unten folgen.

Aus gediegenem Eisen und einem braunen, stellenweise ins Grüne ziehenden olivinartigen Mineral gemengte Masse, der nur höchst sparsam etwas Magnetkies beigemengt ist. Das Eisen schwammförmig oder ästig, an der Oberfläche der Bruchflächen kurzzahnig, auf Schnittflächen in der olivinartigen Substanz, mehr oder weniger schmale, meist wurmförmig gekrümmte Ausscheidungen mit rundlichen Sinuositäten bildend; die Farbe mehr in das Graue als das Silberweiße geneigt (nicht so hell wie bei den fast silberweißen Eisenmassen von Atacama und Sibirien). Werden die Eisenpartien geätzt, so erscheinen dieselben bedeckt mit geraden, nach drei Richtungen gekehrten, glänzenden und hervorragenden Linien und zwischen ihnen matte graue Felder. (Widmanstättensche Figuren.) Das braune, stellenweise ins Grüne ziehende, mit dem Eisen gemengte Mineral ist körnig, jedoch stark zusammenhängend, mit mehr oder weniger deutlicher blätteriger Struktur (Teilbarkeit), und soll nach den Untersuchungen von Stromayer, obwohl mit dem Olivin im Äußeren verwandt, ein eigentümliches Mineral sein, da es nicht wie der Olivin der Eisenmassen aus Sibirien, Litauen und Bolivia, als ein einfaches Talkerde-Silicat, sondern als ein Talkerde-Trisilikat zu betrachten ist. (Siehe Göttinger gelehrte Anzeigen, Stück 208 und 209 vom Monat Dezember 1824.) — Diese interessante Eisenmasse unterscheidet sich von denen von Krasnojarsk, Brahin und Atacama auffallend durch die auf geätzten Flächen zum Vorschein kommenden Widmanstättenschen Figuren, da bei diesen auf den dunkleren Mittelfeldern entweder gar keine Figuren, oder nur einzelne Linien erscheinen.

1. a. Kleines Bruchstück, ästig, an der Oberfläche rostbraun; eine kleine Fläche ist poliert und geätzt, wodurch Widmanstättensche Figuren zum Vorschein kamen. — 7/32 Loth. — 1809. 9. 1. — Wurde von dem verstorbenen Präsidenten von Schlotheim in Gotha im Jahre 1809 als Geschenk erhalten, und ist ein Abschnitt von dem Stücke von 145/, Loth Wiener Gewicht, das ehemals in von Schlotheims Mineralien-Sammlung lag, (aus der es das kaiserl. Mineralien-Kabinett um den Preis von 100 Stück Louis d'or hätte kaufen können) und die nach seinem Tode nach St, Petersburg verkauft wurde. Es stammt, wie das 1 Pfund 29 Loth Wiener Gewicht schwere Stück im herzogl. Naturalien-Kabinett zu Gotha, aus der Sammlung des ehemaligen sächsischen Berghauptmannes von Schönberg.

1. b. Kleines Bruchstück, angeblich, aber fälschlich vom Senegal, mit stark geätzter Fläche. — 9/32 Loth. — 1838. 25. 14. — Mit der Lokalität Senegal aus der ehemals Heuland'schen, dann Heath'schen Meteoriten-Sammlung, durch Herrn Pötschke erkauft. Das Stück lag früher in der von Herrn Heuland erkauften Sammlung des Marquis de Drée, in welcher es Bigot de Morogues und Chladni sahen und desselben in ihren schon früher angeführten Werken, Seite 33 und 335, erwähnen.

1. c. Kleines Bruchstück, angeblich, aber fälschlich von Tabor in Böhmen (nach Borns irriger, durch eine Verwechslung entstandener Angabe im Lithophylacium Bornianum B. 1. S. 125), mit kleiner anpolierter Fläche. — 14/32 Loth, gut. — 1839. 22. 8. — Von der Mineralien-Sammlung der königl. Universität zu Berlin durch Professor Weiss in Tausch erhalten, Stammt aus der Klaproth'schen Mineralien-Sammlung.

1. d. Kleines Bruchstück, angeblich, aber fälschlich aus Sibirien, mit geätzter Fläche. — 23/32 Loth. — Aus den Doubletten. — Wurde durch Herrn Direktor von Schreibers vor mehreren Jahren von Hofrath von Gersdorf eingetauscht, welcher dieses Stück von Professor Chladni ebenfalls in Tausch als Meteoreisen von Sibirien erhielt. (War damals jedoch weder poliert noch geätzt.)

2. Großes Bruchstück, das gediegene Eisen an der Oberfläche (die aus Bruchflächen besteht), kurzzahnig, mit einer anpolierten Schnittfläche. — 1 Pfund 14 1/32 Lth. — \mars 1. 2.

Dieses und das folgende davon abgeschnittene Stück sollen aus Norwegen sein, und man sehe, was Chladni in Gilberts Annalen der Physik, Band50, S.259, und in seinem Werke über Feuer-Meteore, S. 325, darüber sagt. Chladni und alle, die das Stück sahen, hielten dasselbe für sibirisches oder Pallasisches Eisen, bis der Herausgeber dieser Bogen das Stück entzweischneiden, polieren und ätzen ließ, wo die große Verschiedenheit desselben von der sibirischen und anderen Eisenmassen und seine vollkommene Identität mit dem sächsischen Meteoreisen an Tag kam. Das fast gleich große und gleich schwere Stück im Museum zu Gotha, das wir im vorigen Sommer sahen, ist unserem Stücke (mit dem sub Nr. 3 folgenden Abschnitte verbunden) vollkommen ähnlich, besitzt jedoch keine, das wahre Wesen der Masse aufschließende polierte und geätzte Schnittfläche. Das Stück der kaiserl. Sammlung und das davon abgetrennte unter Nr. 3 stammen aus der, unter der Regierung des Kaisers Joseph 2. mit dem k. k. Mineralien-Kabinette vereinigten Mineralien-Sammlung der Theresianischen Ritter-Akademie. In diese kam das Stück aus der Stieglitzischen Sammlung, ehemals in Leipzig, Von dieser ließ der damalige Besitzer im Jahre 1769 eine Übersicht unter dem Titel drucken: Spicilegium quorundam rerum subterranearum Lipsiae collectarum. Darin ist unser Stück auf der Tafel 11. roh abgebildet; die Erklärung dazu sagt weiter gar nichts als: Zackig gewachsen Eisen, in einer gründlichen, glas- oder eisengranatartigen Stein-Gangart; aus Norwegen." Es wird dasselbe später auch in dem "Verzeichnisse der Fossilien in dem zur allgemeinen Ökonomie gewidmeten Gebäude der kaiserl. Theresianischen Akademie. Wien 1776." Seite 240 fehlerhaft beschrieben: "Gediegenes, zahnicht und zackicht gewachsenes Eisen mit körnichtem Quarz (!) und gelblichtem Flussspat (!) Aus Norwegen." Von einem Meteoreisen aus Norwegen ist übrigens weiter nie etwas bekannt geworden, und auch in der Versammlung der skandinavischen Naturforscher zu Stockholm, wo wir die Sache in Anregung brachten, war darüber keine Aufklärung zu erhalten.

3. Anpolierter und dann geätzter Abschnitt von dem Stücke Nr. 2. — 15 31/32 Loth. — Von 1. 2.
\subsection[Bitburg.]{Bitburg,}
\begin{center}
\small
in der Eifel, nördlich von Trier, preußische Provinz Niederrhein.
\end{center}
\paragraph{}
Die Masse, von ungefähr 33-34 Zentnern an Gewicht, wurde im Jahre 1805 bei Gelegenheit der Erweiterung eines Weges gefunden, und bald darauf in einem Frischofen eingeschmolzen; sie ist durch den amerikanischen Obersten Gibbs, obwohl dieser die unveränderte Masse kurz nach ihrer Auffindung schon im Jahre 1805 sah, erst im Jahre 1814 im ersten Bande von Bruces American Mineralogical Journal (New-York. 1814.) der gelehrten Welt, jedoch fälschlich als Meteoreisen aus den Ardennen, wohin er Bitburg versetzte, bekannt geworden; bekannter wurde sie erst durch Chladni im Jahre 1819 und durch einen Aufsatz von den Herren Nöggerath und Bischof in Schweiggers Journal für Chemie und Physik (43. Band v. J. 1825).

Gediegenes Eisen, gemengt mit einer gelblichen, ins Braune und Grünliche ziehenden olivinartigen Substanz, die jedoch nur einen geringen Teil der Masse eingenommen zu haben scheint, daher man das Eisen nicht ästig, sondern derb nennen, und das erdige Mineral nur als beigemengt ansehen muss. Auf polierten Flächen (und nur auf diesen ist das beigemengte Mineral deutlich zu sehen) erscheinen durch Ätzen deutliche Widmanstättensche Figuren.

Diese, wegen Kleinheit der zu Gebote gestandenen zwei Stückchen unvollkommene Beschreibung bezieht sich, wie es sich von selbst versteht, nur auf das unveränderte Bitburger Eisen, von dem, mit Ausnahme der zwei kleinen Abschnitte im Wiener kaiserl. Kabinette nur drei kleine Stückchen (in den Sammlungen von Trier, Berlin und New-Haven) bekannt sind. Das durch den Frischprozess veränderte, welches hier als Kunstprodukt nicht in Betracht gezogen werden kann, befindet sich fast in allen Meteoriten- und Mineralien-Sammlungen, verdient aber doch insofern Aufmerksamkeit, als die in ihm vorhandenen, oft mit Schlackenstückchen ausgefüllten Höhlungen die ehemalige Anwesenheit des olivinartigen Minerals andeuten, und auf polierten Durchschnittsflächen beim Ätzen einzelne kleine Stellen zum Vorschein kommen, die Widmannstättische Figuren zeigen, wo also das Eisen beim Frischprozesse nicht verändert worden ist.

1. Ganz kleines (unverändertes) Stückchen mit schön geätzter Fläche , worauf kleine, meist rundliche Partien eines erdigen braunen Minerals eingemengt erscheinen. — 1/16 Loth oder 15 1/2 Gran. — 1840. 1. 2. — Vom Mineralien-Kabinette der königl. Universität zu Berlin durch Herrn Professor Weiss in Tausch erhalten. Wurde von dem kleinen Stücke von nahe 200 Gran abgeschnitten, das früher in Chladnis Sammlung lag. Der verstorbene Chladni erhielt es durch Herrn Professor Steininger von der Gesellschaft nützlicher Forschungen in Trier. Das Stück war früher in Besitz von Doktor Schmitz, Kreisphysikus zu Hillesheim in der Eifel. (Siehe darüber Schweiggers Journal für Chemie und Physik. Band 46, S. 38 und 392).

Das in der Mineralien-Sammlung des Gymnasiums zu Trier noch vorhandene kleine Stück unveränderten Bitburger Eisens (das wir im Jahre 1840 auf einer bloß in dieser Absicht unternommenen Seitentour zu sehen Gelegenheit nahmen) wiegt 1 Loth 1 Quäntchen und 51 Gran Nürnberger Apotheker Gewicht. Es war damals nicht anpoliert und daher nicht aufgeschlossen. Wegen Kleinheit und der sehr unebenen Beschaffenheit des Stückes konnten wir davon keinen Abschnitt für unsere Sammlung erhalten. Die olivinartige Substanz ist daran demungeachtet gut erkennbar; sie soll nach Herrn Professor Steininger leicht schmelzbar sein, und aus einem Eisensilicat bestehen, Diess kostbare Eisenstückchen stammt von dem verstorbenen Appellationsgerichtsrate Seippel in Trier her, der es selbst von der Masse abschlug, als sie auf dem Durchwege von Bitburg nach dem Frischofen zu Pluwig in Trier gewogen wurde. Von ihm kam das Stück an den Domdechant Castello und später mit dem anderen, nun in Berlin befindlichen Stückchen an Doktor Schmitz, durch welchen Herr Professor Nöggerath in Bonn im Jahre 1814 die erste mündliche Nachricht, aber leider nur von dem Vorhandengewesensein dieser höchst merkwürdigen Masse erhielt.

2. Kleines, doch fast 1 Zoll langes (nicht im Feuer gewesenes) Stückchen, mit einer polierten, zur Hälfte schwach geätzten Schnittfläche. Das Eisen ist darauf in 3 Partien gesondert, die voneinander durch eine schmutziggelbe, ins Braune ziehende olivinartige Substanz getrennt sind, welche wieder höchst feine Pünktchen von gediegenem Eisen enthält; auf der Rückseite, wo das Stückchen vielleicht durch Auswitterung der olivinartigen Substanz, eine poröse Beschaffenheit zeigt, scheint etwas natürliche Oberfläche vorhanden zu sein. — 5/16 Loth, oder 75 1/2 Gran. — 1842. 34. 2. — Durch den Kurator am Yale-College zu New-Haven im Staate Connecticut, Herrn Professor Silliman in Tausch erhalten.

In das Museum des Yale-College in New-Haven kam mit der Mineralien-Sammlung des verstorbenen, schon früher erwähnten Obersten Gibbs das Stückchen unveränderten Bitburger Eisens, das derselbe im Jahre 1805 von der Masse noch in Bitburg selbst abschlug. Nun liegt aber im Yale-College kein gediegenes Eisen weder mit der Etiquette Bitburg, noch mit der Etiquette Ardennen, wohin der Oberste Bitburg früher vorlegte, sondern ein 500 Gran schweres Stück gediegenes Eisen, von dem das unter Nr. 2 beschriebene ein Abschnitt ist, mit der Bezeichnung: "aus der Auvergne." Durch Zerstreuung oder einen Gedächtnisfehler musste also Bitburg aus den Ardennen nunmehr nach der Auvergne wandern; denn aus dem letzteren Lande ist außer natürlichem Stahl, der ein pseudovulkanisches Produkt ist, nie eine Lokalität von Meteoreisen bekannt geworden und das aus New-Haven erhaltene Muster stimmt fast ganz mit den zwei anderen noch vorhandenen Stückchen unveränderten Bitburger Eisens in den Sammlungen von Trier und Berlin überein.
\subsection{Toluca.}
\begin{center}
\small
Xiquipilco, in der Gerichtsbarkeit von Ixtlahuaca, nördlich von Toluca, Mexiko.
\end{center}
\paragraph{}
Seit 1784 durch Nachrichten in der Gazeta di Mexico bekannt; mehr durch Chladni, der seine Nachrichten aus einem seltenen Buche von Sonneschmidt (Beschreibung der vorzüglichsten Bergwerks-Reviere in Mexiko oder Neu-Spanien 1804) schöpfte. Die Fallzeit ist unbekannt.

Gediegenes Eisen, derb, dicht, ohne sichtbare Beimengung (auch Magnetkies ist an dem kleinen Stücke der Sammlung, das außer einer stark geätzten, meist nur gehämmerte oder gequetschte Flächen besitzt, nicht zu sehen, obwohl er kaum fehlen wird). Durch Ätzung mit Säuren erscheinen (wenigstens an dem Stücke der Sammlung) Widmanstättische Figuren, welche verschobene Vierecke, also Streifung nur nach zwei Richtungen (nicht wie es gewöhnlich ist, Streifung nach drei Richtungen und daher Dreiecke) zeigen.

Da zu dieser Diagnose uns nur das kleine Stück unserer Sammlung zu Gebote stand, halten wir dieselbe für unvollständig. Im Berliner Museum liegen mehrere, aber zur wissenschaftlichen Untersuchung noch nicht zugerichtete Stücke von Toluca, aus verschiedenen Quellen.

1. Dreieckiges Stück von einer der größtenteils kleinen, nur einzelne Pfunde oder Unzen schweren Massen, welche da auf den Feldern herum lagen (noch liegen?), und von den Indianern und Grundeigentümern zu Ackergeräten verwendet wurden; mit einer stark geätzten Fläche mit Widmanstättischen Figuren, zwei, etwas unebenen Schnitt-, einer kleinen gehämmerten und einer noch kleineren Bruchfläche mit hackigem Bruche. Eine körnige Fläche scheint von der natürlichen Oberfläche der Masse zu sein. — 3 7/32 Loth. — 1810. 2. 1. — Beschrieben und (jedoch nicht gut) abgebildet in von Schreibers angeführten Beiträgen, Seite 78, Taf. 8, Mittelfigur links. — Vom verstorbenen Professor Klaproth in Berlin als Abschnitt von einem von Herrn Alexander v. Humboldt herrührenden, und nun in der Berliner königl. Mineralien-Sammlung befindlichen Stücke gekauft.
\subsection[Elbogen.]{Elbogen,}
\begin{center}
\small
bei Carlsbad in Böhmen.
\end{center}
\paragraph{}
Die Fallzeit ist unbekannt. Die ursprünglich 191 Wiener Pfund schwer gewesene Eisenmasse war seit Jahrhunderten unter dem Namen: der verwünschte Burggraf auf dem Rathause zu Elbogen aufbewahrt, ist aber erst im Jahre 1811 (von dem Herrn Gubernialrat Neumann in Prag) als Meteorit erkannt worden.

Derbes und dichtes gediegenes Eisen, mit hie und da in Körnern eingesprengtem, oder in Linien eingewachsenem Magnetkies. Durch Anlaufen in der Hitze, noch besser durch Ätzen mit Säuren, erscheinen jene merkwürdigen von ihrem Entdecker Herrn von Widmannstätten in Wien benannten Figuren in großer Vollkommenheit. Sie bilden meist gleichseitige Dreiecke, die von Streifen von mäßiger Breite umschlossen sind.

Das Nähere und bisher einzige Erschöpfende über diese Widmanstättenschen Figuren findet man in den schon oft angeführten Beiträgen zur Geschichte und Kenntnis meteorischer Stein- und Metall-Massen von Karl von Schreibers, S. 70 u. f. Es kann hier nur angeführt werden, dass bei diesen, mit der kristallinischen Struktur und der chemischen Beschaffenheit der Meteoreisen-Massen zusammenhängenden Figuren dreierlei zu unterscheiden ist: 1. Streifen, die meist nach drei Richtungen gehen, beim Ätzen des Meteoreisens mit Säuren, da sie das reinste oder am wenigsten mit Nickel legierte Eisen enthalten, am leichtesten aufgelöst werden, und daher die vertieftesten Stellen der geätzten Oberfläche bilden. 2. Zwischenfelder, von diesen Streifen eingeschlossen, Dreiecke, Vierecke und andere Figuren darstellend, und aus einer körnigen, von Säuren weniger als die Streifen angreifbaren Masse bestehend, die von feinen erhabenen Linien (Schraffierungslinien oder Leisten) nach einer oder mehreren Richtungen durchzogen ist. 3. Einfassungsleisten, oder erhabene und glänzende, die Streifen und Zwischenfelder einfassende und sie voneinander trennende Linien, die von der Säure nicht oder nur wenig angegriffen werden, daher am meisten hervorragen und noch den Glanz der polierten Fläche beibehalten haben. Sie enthalten nach den Untersuchungen von Berzelius mehr Nickel als die Streifen und Zwischenfelder, und sind daher am wenigsten von Säuren angreifbar.

1. Eine unregelmäßige Masse von fast dreieckiger Form (die man der eines Pferdekopfes ähnlich gefunden hat), 141 Wiener Pfund an Gewicht, von welcher vor der Ausfolgung in Elbogen das vordere oder dünnere Ende von mehr als 40 Pfunden abgesäget, und zum Andenken daselbst zurückbehalten wurde.

Die Masse wog, wie schon angeführt worden ist, ursprünglich 191 Pfund, wovon nach des Direktors von Schreibers Verzeichnis der Meteoriten des k. k. Mineralien-Kabinettes in Chladnis Feuer-Meteoren, S. 433, 150 Pfund nach Wien gekommen sind. Durch das Abschneiden einer Platte‚ die zerschnitten, zu Tauschen und Versuchen verwendet wurde, wie auch durch das Polieren und Ätzen der neuen großen Schnittfläche, hat das Stück 9 Pfund verloren. — Außer dem großen im k. k. Mineralien-Kabinette aufbewahrten Stücke, das mehr als zwei Drittel der ursprünglichen Elbogner Eisenmasse ausmacht, sind nur noch zweiandere Stücke dieser Eisenmasse von ansehnlicher Größe vorhanden. Das auf dem Bathhause zu Elbogen befindliche Stück wog im Jahre 1838 noch 27 Pfund. Das Universitäts-Museum zu Prag bewahrt ein von dem Elbogner Stück abgesägtes Exemplar von 11 Pfund 18 1/4 Loth. Wir haben von diesen zwei Stücken durch die gütige Bemühung der Herren Wilhelm Haidinger und Professor Presl Gipsabgüsse erhalten, um die Masse im Wiener Kabinette ergänzen zu können.

Das am kaiserl. Mineralien-Kabinette befindliche große Stück ist 11 1/2 Zoll lang und 12 1/2 Zoll hoch; die große geätzte Schnittfläche ist 8 Zoll hoch, 7 Zoll breit und zeigt auf das Vollkommenste die Widmanstättenschen Figuren. Ein unmittelbarer Abdruck oder Autograph von dieser Fläche befindet sich in des Direktors v. Schreibers öfter angeführtem Werke, Taf. 9., die Beschreibung S. 72. Die Oberfläche der Masse ist auf den hervorragenderen Teilen stark abgerieben, und zeigt auch da das kristallinische Gefüge ungemein deutlich. — 1812. 59. 1. — Wurde auf Veranlassung des Naturalien-Kabinetts-Direktors v. Schreibers im Jahre 1812 von dem Magistrate zu Elbogen als ein kostbares Geschenk erhalten. — Das Weitere über diese höchst merkwürdige Masse findet man in Chladnis angeführtem Werke über Feuer-Meteore, S. 327, und in einem kleinen zu Carlsbad im Jahre 1834 erschienenen Werkchen: "Der verwünschte Burggraf von Elbogen. Ein Andenken an Elbogen für die Herren Karlsbader Brunnen-Gäste."

2. Ein parallelepipedisches Stück, von welchem 5 Seiten, sämtlich Schnittflächen, geätzt und dann angelaufen wurden, wodurch das kristallinische Gefüge und die ungleiche Fähigkeit zum Anlaufen auf das Schönste zum Vorschein kamen; die sechste Fläche ist eine natürliche von der Oberfläche der Masse. — 2 15/16 Loth. — Wurde beim Schneiden der großen Fläche am Stücke Nr. 1 als Abschnitt erhalten.

3. a. Ein ringsum mäßig geätztes parallelepipedisches Stückchen mit Magnetkies. — 1 1/4 Loth.

3. b. Kleines parallelepipedisches Stückchen, ringsum schwach geätzt mit Magnetkies. — 9/16 Loth.

3. c. und d. Zwei, ringsum mäßig geätzte kleine Würfel; an einem davon ist eine Würfelecke abgeschnitten. Der Magnetkies ist beim Ätzen zum Teil herausgefallen. — Zusammen 1/2 Loth. — Wie a und b als Abschnitte von Nr. 1 erhalten. Sie dienen zum Studium der Widmanstättenschen Figuren, da die große Masse Nr. 1 nicht leicht beweglich ist, und die kleinen Stücke auch mehrere aufeinander senkrecht stehende Flächen darbieten.

4. a. Ein parallelepipedisches Stückchen, ringsum poliert, mit fein eingesprengtem Magnetkies. — 1 3/32 Loth.

4. b. Ein poliertes, und dann mittelst Hitze angelaufenes Plättchen. — 1/2 Loth.

4. c. Ein ebenso behandeltes, ringsum poliertes Plättchen. — 15/32 Loth. — Ebenfalls wie auch a und b als Abschnitte von Nr. 1. erhalten.
\subsection{Agram.}
\begin{center}
\small
Dorf Hraschina (Chladni schreibt fälschlich Hradschina) bei Agram, Kroatien.

Gefallen am 26. Mai 1751, Abends um 6 Uhr.
\end{center}
\paragraph{}
Wenn nicht durch gleichzeitige Zeitungsnachrichten ist dieser Meteoritenfall, (der aber damals und noch lange darnach nicht geglaubt, vielmehr verlacht wurde) wahrscheinlich erst durch Güssmanns Lithophylaceum Mitisianum im J. 1785 der wissenschaftlichen Welt bekannt geworden.\footnote{Der Verfasser sagt daselbst, nachdem das Ereignis von Hraschina kurz erzählt worden, S. 127 weiters von unserer Eisenmasse: Massa illarum major caesareo Viennensi illata Museo paucos adhuc ante annos in eodem ostendabatur, primum inter lapides e coelo lapsos relata, non plane falso, sed male aptato vocabulo; deinde cum phaenomeni causa ignota esset, hoc nomine irrisa ac neglecta. Ein Vorstand des k. k. Mineralien-Kabinetts Abbé Stütz fügte noch im J. 1790 bei Gelegenheit der Bekanntmachung der Urkunde über den Fall der Agramer Eisenmasse (im 2. Bande der Berghaukunde, Seite 407) folgende Bemerkung bei: dass das Eisen vom Himmel gefallen sein soll, mögen der Naturgeschichte Unkundige glauben, mögen wohl im J. 1751 selbst Deutschlands aufgeklärtere Köpfe bei der damals unter uns herrschenden Ungewissheit in der Naturgeschichte und praktischen Physik geglaubt haben; aber in unsern Zeiten wäre es unverzeihlich, solche Märchen auch nur wahrscheinlich zu finden.}

Allgemeiner bekannt wurde dieser Meteoritenniederfall und unsere Eisenmasse im J. 1794 durch die berühmte Schrift von Chladni: "Über den Ursprung der von Pallas gefundenen und anderer ihr ähnlichen Eisenmassen, und über einige damit in Verbindung stehende Naturerscheinungen." Riga 1794. 4.

Gediegenes Eisen, derb, dicht, hie und da mit Magnetkies gemengt, auf polierten Flächen durch Anlaufen oder Ätzen sich mit sehr vollkommenen Widmanstättenschen Figuren bedeckend, die meist gleichseitige Dreiecke darstellen. Diese Figuren sind am meisten mit den bei dem Elbogner Eisen zum Vorschein kommenden verwandt; die Streifen sind jedoch bei dem Agramer Eisen meist feiner. Andere Unterschiede, die in den Widmanstättenschen Figuren der verschiedenen Eisenmassen stattfinden und einem geübten Auge Merkmale darbieten, sie voneinander zu unterscheiden, lassen sich nicht gut mit Worten ausdrücken, aber leicht durch gute Abbildungen deutlich machen.

1. Ganze Masse von dreieckiger Form, flach; auf der einen Seite mit vielen kleinen, auf der anderen Seite mit weniger aber weit ausgedehnteren Eindrücken; hier ist die Rinde viel deutlicher und dicker, und auf dieser Seite ist auch ein, wie es scheint, ziemlich dünnes Stück von der Masse abgeschnitten worden; die 4 3/4 Zoll lange Schnittfläche ist geätzt, und mit Widmanstättenschen Figuren bedeckt. — (Von der anderen Seite ist nur eine ganz kleine Hervorragung abgeschnitten und die Schnittfläche schwach geätzt.) — 70 Pfund — (wog ursprünglich fast 71 Pfund.) — \mars 1. 3. — Genau beschrieben und von der einen Seite sehr gut abgebildet in v. Schreibers Beiträgen. S. 1. Taf. 1. — Sie ist die Eine und zwar größere der zwei vor vielen Augenzeugen bei dem Dorfe Wraschina unweit Agram herabgefallenen Eisenmassen, welche nach der, wenige Tage nach dem Ereignisse von Seite des bischöflichen Konsistoriums zu Agram aus freiem Antriebe an Ort und Stelle gepflogenen amtlichen Untersuchung , samt einer darüber in lateinischer Sprache ausgestellten Urkunde, die sich noch im Original am k. k. Mineralien-Kabinette befindet, durch den Bischof von Agram, Freiherrn von Klobuschitzky, auf dem eben zu jener Zeit zu Preßburg in Ungarn abgehaltenen Landtage dem römischen Kaiser Franz 1. und der Kaiserin Maria Theresia überreicht wurde. Auf Befehl der Kaiserin ward dieselbe nach Wien gesendet, wo sie anfänglich in der k. k. Schatzkammer aufbewahrt, in der Folge aber an das k. k. Mineralien-Kabinett abgegeben wurde. — Die zweite kleinere Masse von 16 Pfund ist in Kroatien in Verlust geraten.

Von allen bekannten meteorischen Eisenmassen ist die Agramer die einzige, deren Herabfallen samt allen Nebenumständen beobachtet worden ist; sie ist sowohl in dieser Hinsicht, als durch ihre vortreffliche Erhaltung, durch die Eigentümlichkeit ihrer Oberfläche, die Rinde von doppelter Beschaffenheit, die Vollkommenheit der Widmanstättenschen Figuren u. s. w. die merkwürdigste und kostbarste von allen in Sammlungen aufbewahrten Meteoreisenmassen, welcher nur zwei andere Massen, nämlich die ebenfalls noch vollkommen ganze, aber an der Oberfläche stark abgenützte, dagegen weit größere, 591 Kilogrammen wiegende Eisenmasse von Caille, Departement du Var, Frankreich, im k. Museum der Naturgeschichte zu Paris. dann die noch größere Eisenmasse von 3000 Pfund im Yale-Collegium zu New-Haven im nordamerikanischen Staate Connecticut, die jedoch ebenfalls undenkliche Zeit lang unter freiem Himmel lag, etwa den Rang streitig machen könnten, Andere meteorische Eisenmassen von ansehnlicher Größe, womit jedoch der wissenschaftliche Werth nicht zunimmt, befinden sich im britischen Museum zu London und in der Mineralien-Sammlung der Akademie der Wissenschaften zu St. Petersburg; im ersteren ein Block, an Gewicht 1400 engl. Pfund, von der auf 30.000 Pfund geschätzten Eisenmasse in der Provinz Tucuman, Republik Buenos-Ayres; in der zweiten die Pallasische Eisenmasse von Krasnojarsk in Sibirien, gegenwärtig noch 1270 russische Pfunde schwer, von deren Oberfläche aber so viele Stücke abgeschlagen worden sind, dass sie jetzt eine ganz abgerundete Masse darstellt. An wissenschaftlichem und historischem Interesse kann, vorzüglich durch die herrliche von Herrn v. Widmannstätten geätzte große Schnittfläche, die deren Inneres aufschloss, nur die große Elbogner Masse im k. k. Mineralien-Kabinette der Masse von Agram an die Seite gesetzt werden, Einen noch nicht aufgeschlossenen Schatz besitzt das vaterländische Museum zu Prag an der Bohumilitzer Eisenmasse.

2. Kleines Stück, ein Teil des vom vorigen abgesägten, mit natürlicher, schwach überrundeter, und mit drei aufeinander senkrecht stehenden, schwach geätzten Schnittflächen. Um einen Teil dieses kostbaren Stückchens, (da von der Masse nichts mehr abgesägt werden soll) zieht sich zwischen dem gediegenen Eisen der Masse und der schwarzen Rinde, und mit dieser zum Teil in Schichten abwechselnd, eine zweite Rinde von Magnetkies herum. — 2 1/8 Loth.

3. Kleine Platte, ebenfalls ein Teil des Stückes, das von der Masse 1 abgesägt wurde; 2 Zoll 2 Linien lang; die eine Fläche schön geätzt und mit Widmanstättenschen Figuren bedeckt, die andere poliert. — 19/32 Loth. — Beschrieben und nicht gut abgebildet in von Schreibers Beiträgen, Seite 76. Tafel 8, Mittelfigur rechts.

4. Etwas größere, aber sehr dünne Platte, ebenfalls von der Masse Nr. 1 abgeschnitten; 2 Zoll 10 Linien lang; beide Flächen poliert und durch sehr gelungenes Anlaufen mittelst Einwirkung von Hitze mit den herrlichsten Widmanstättenschen Figuren bedeckt. — 7/16 Loth.

5. a. Dünnes Blättchen, von einer Seite fein poliert; die Rückseite ist geätzt, und zeigt noch Spuren, dass sie angelaufen war. — 11/32 Loth.

5. b. Ein kleines dreieckiges, aber durch lange fortgesetzte Behandlung mit Säuren ungemein lehrreich gewordenes Plättchen. Die Streifen, aus dem reineren Eisen bestehend, sind nämlich wie Rinnen tief ausgehöhlt; über diesen erheben sich die Zwischenfelder, von Einfassungsleisten begrenzt. Die letzteren sind auch zuweilen ohne Mittelfeld als Begränzung zweier Streifen vorhanden, und erscheinen an einer Ecke des Stückes als dünne freie Blättchen hinausragend. — 3/64 Lth. — (genauer 10 1/8 Gran.)

5. c. und d. Zwei Stückchen von der schwarzen dicken Rinde, welche die eine Seite der Masse bedeckt. Die Außenseite der Rindenfragmente ist zum Teil glatt, zum Teil durch eine Menge von kleinen Erhabenheit körnig und rau; die innere Seite derselben glatt. Im Bruche ist die Rinde fasrig (aus parallelen Fasern). Die Fragmente wiegen zusammen 9/64 Loth (genauer 33 3/4 Gran.)
\subsection{Lenarto.}
\begin{center}
\small
An der galizischen Grenze bei Bartfeld, Saroscher Komitat, Ungarn.
\end{center}
\paragraph{}
Gefunden im Jahre 1814; die Fallzeit ist unbekannt. Die ganze Masse, wovon das größte Stück, 134 Pfund an Gewicht, im National-Museum zu Pesth aufbewahrt wird, wog 194 Pfund.

Derbes und dichtes gediegenes Eisen mit mehr oder weniger, in Körnern und Linien eingesprengtem, zuweilen auch in größeren Nieren, Zapfen u. s. w. eingewachsenem Magnetkies; durch Anlaufen und Ätzen ausgezeichnete Widmanstättensche Figuren darstellend, die meist gleichschenkliche Dreiecke sind; die Zwischenfelder oft schön schraffiert; die Einfassungsleisten meist breiter als bei den Eisenmassen von Elbogen und Agram.

1. Ein großes dreieckiges Stück, an zwei Seiten mit natürlicher Oberfläche, mit einer Schnittfläche von 4 1/2 Zoll Länge, die schön geätzt und mit Widmanstättenschen Figuren bedeckt ist, einer kleinen rohen Schnitt- und einer Bruchfläche mit deutlichem blättrigen Gefüge — 5 Pfund schwach. — 1818. 45. 1. — Von dem verstorbenen Prof. Sennovitz in Eperies gekauft.

2. Eine dreieckige Platte, 2 Zoll 10 Linien lang; die eine Fläche geätzt, die andere angelaufen, und beide mit Widmanstättenschen Figuren bedeckt. 4 7/32 Lth. — 1815. 39. 1. — Beschrieben und abgebildet in v. Schreibers Beiträgen. S. 77. Taf. 8. — Ein Geschenk des verstorbenen Freiherrn v. Brudern in Pesth.

3. Tiefgeätzte dreieckige Platte, aus der die Einfassungsleisten ungemein schön hervorragen; Mittelfelder sind weniger vorhanden. Merkwürdig ist dieses Stück auch durch den Umstand, dass die eben nicht dicke Platte an den zwei großen einander entgegengesetzten parallelen Flächen auffallend verschiedene Zeichnung zeigt, da auf der einen nur wenige Einfassungsleisten und fast gar keine Zwischenfelder vorhanden sind. Die feinkörnige Masse, welche die Streifen bildet, ist hier sehr ausgedehnt und vorherrschend. — 3 19/32 Loth. — 1840. 24. 1. — Durch Herrn Ritter von Pittoni in Grätz zu Kauf erhalten. — Stammt aus der Mineralien-Sammlung des H. v. Patschovski in Triest.

4. Unregelmäßige Platte, von der einen Seite fein poliert, von der anderen sehr schwach geätzt. Die Streifen der geätzten Fläche sind punktiert, ein Umstand, der etwas rätselhaft ist, und von ungleicher Auflöslichkeit im ersten Stadium der Säureeinwirkung herrühren mag, Die zweite, spiegelhell polierte Fläche zeigt außer fein eingesprengtem Magnetkies, nach gewissen Richtungen gehalten, ziemlich deutlich die feinen Einfassungsleisten und somit eine Skizze der Widmanstättenschen Figuren, was dadurch erklärlich wird, dass die Einfassungsleisten (eine Legierung von Eisen mit Nickel oder sogenannter Meteorstahl) härter sind, als die Streifen und Mittelfelder und daher eine höhere Politur annehmen. — 1 31/32 Loth. — Von Nr. 1 abgeschnitten.

5. Eine gleich große Platte, durch Teilung des vorigen Stückes erhalten; beiderseits fein poliert und ungemein schön angelaufen, mit den vollkommensten Widmanstättenschen Figuren geziert. — 2 5/32 Loth. — Von Nr. 1 und rücksichtlich Nr. 4.

6. Ein dickes viereckiges Plättchen oder niederes Parallelepipedum, von allen Seiten geätzt und dann angelaufen, wodurch die Widmanstättenschen Figuren ungemein schön in Farben dargestellt wurden. (Die Einfassungsleisten und die Schraffierungsleisten der Zwischenfelder sind messinggelb, die Streifen meist lasurblau, zuweilen ins Gelbe schillernd.) — 2 27/32 Loth. — Abschnitt vom Stücke Nr. 1.

7. a. und b. Zwei durch Auseinandersägen erhaltene kleine Stücke, jedes mit zwei anpolierten Flächen; durch den Umstand merkwürdig, dass die Durchschnittsfläche zufällig die Mitte einer mehr als bohnengroßen, 8 Linien langen Ausscheidung von Magnetkies durchschnitt, wovon man in dem einen Stücke noch die Hälfte eingewachsen sieht; in dem andern Stücke ist die Vertiefung sichtbar, welche die zweite Hälfte, die beim Zerschneiden zerbröckelte, einnahm. — Zusammen 4 5/16 Lth. — 1824. 43. 1. 2. — Durch Prof. Sadler in Pesth zu Kauf erhalten.
\subsection{Red-River.}
\begin{center}
\small
Am Red-River oder roten Flüsse im Staate Louisiana (New-Orleans) in den vereinigten Staaten von Nord-Amerika, über 100 englische Meilen oberhalb der Stadt Natchitoches.
\end{center}
\paragraph{}
Seit dem Jahre 1814 bekannt durch zwei Aufsätze von Gibbs in Bruces American Mineralogical-Journal, Vol. 1. Seite 124 und S. 218.

Am roten Flüsse, der aus Texas kommt und sich in den Mississippi ergießt, sollen mehrere Eisenmassen zerstreut herum liegen. Eine von da weggebrachte, von ungefähr 3000 Pfund, war früher im Besitze des amerikanischen Obersten Gibbs, kam dann als Depositum auf das Lyceum zu New-York und von da endlich in die Mineralien-Sammlung des Yale Collegiums zu New-Haven im Staate Connecticut, Das in den letzteren Jahren aus Texas bekannt gewordene Meteoreisen wird wohl mit dem von Louisiana identisch sein, da der Red-River beiden Staaten angehört. Wir erhielten durch Herrn Professor Silliman aus der Sammlung des Yale-Collegiums ein schönes Stück Meteoreisen, dass nun in der k. Mineralien-Sammlung (Schrank 55, unter Nr. 83. a. liegt, mit der Lokalität-Bezeichnung: Louisiana oder Texas.

Derbes und dichtes gediegenes Eisen mit Magnetkies gemengt; durch Anlaufen oder Ätzen sehr vollkommene, ziemlich feinstreifige Widmanstättensche Figuren bildend; auf dem Bruche, (wie dies übrigens auch bei dem Eisen von Lenarto und anderen Meteoreisenmassen von kristallinischem Gefüge der Fall ist) ausgezeichnet blättrige Struktur zeigend.

1. Fast viereckiges Stück mit mehreren vom Abmeißeln und Abhämmern herrührenden, etwas verrosteten Flächen, mit einer schwach geätzten Schnittfläche, einer Bruchfläche von ausgezeichnet blättrigem Gefüge und einer ziemlich großen natürlichen Fläche. — 1 Pfund 5 Loth. — 1822. 49. 1. — Vom Obersten Gibbs in New-York durch den k. k. General-Konsul Baron Lederer in Tausch erhalten.

2. Kleineres Stück mit einer polierten Fläche, mit zwei Bruchflächen von ausgezeichnet blättrigem Gefüge und etwas natürlicher Oberfläche. — 11 5/8 Loth. — 1822 25. 1. — Auf dieselbe Art in Tausch erhalten wie Nr. 1.

3. Unregelmäßiges dickes Plättchen mit zwei sehr stark geätzten Flächen. — 2 23/32 Loth. — Wurde von dem Stücke Nr. 1 abgeschnitten.

4. Ein kleines Plättchen, eine Fläche poliert, die andere schwach geätzt; beide blau angelaufen und schöne Figuren zeigend. — 9/16 Loth. — Abschnitt von Nr. 2.
\subsection{Durango.}
\begin{center}
\small
Mexico.
\end{center}
\paragraph{}
Seit 1811 durch Nachrichten von Herrn Alexander von Humboldt bekannt. (Siehe dessen Essai politique sur le royaume de la Nouvelle Espagne. Tome 1. S. 293.) — Herr von Humboldt erwähnet einer Eisenmasse von 300 bis 400 Zentnern; es mögen aber bei Durango noch andere kleinere Massen vorhanden sein. — Die Fallzeit ist unbekannt.

Man sehe was Chladni (Feuermeteore, S. 337) über diese Lokalität bemerkt. Er bezweifelt, ob die von Herrn von Humboldt nach Europa mitgehrachten Stücke wirklich von Durango sind und äußert die irrige Vermutung, dass sie von Zacatecas sein dürften, Sie dürften eher von Toluca sein. Über die Herstammung der in den Sammlungen vorhandenen Eisenmassen aus verschiedenen Teilen von Mexico herrscht übrigens noch mancher Zweifel.

Derbes und dichtes gediegenes Eisen, mit wenig beigemengtem Magnetkies; von ausgezeichnet blättriger Struktur; durch Anlaufen und Ätzen sehr vollkommene Widmannstätten'sche Figuren darstellend, die, wenigstens in unseren Stücken, das Eigentümliche besitzen, dass stellenweise die Mittelfelder verschwinden und dafür nur aneinander stoßende parallele Streifen mit ihren Einfassungsleisten vorhanden sind. Die größeren Zwischenfelder zeigen auch eine eigentümliche Art von Schraffierungsleisten, indem diese oft wellenförmig gekrümmt und unterbrochen sind.

1. Flaches, fast parallelepipedisches Stück, wovon eine der größeren Flächen eine natürliche zu sein scheint; eine der schmalen Flächen zeigt deutlichen blättrigen Bruch, eine geätzte Fläche Widmannstätten'sche Figuren und endlich eine größere polierte, außer der licht stahlgrauen Farbe (die allem Meteoreisen zukommt, nur bei einigen mehr ins Silberweiße und bei anderen mehr ins Bleigraue zieht), auch etwas eingemengten Magnetkies. — 1 Pfund, 1 1/16 Loth. — 1834. 26. 4. — Von dem Freiherrn von Karawinsky in München zu Kauf erhalten, welcher dieses Stück aus Mexico mitbrachte, wo es nach seiner Angabe von einem mehrere hundert Pfund wiegenden Klumpen, der in der Ebene nordöstlich von Durango liegt, abgetrennt wurde.

2. Ein kleineres, von Nr. 1 abgeschnittenes Stück. Die schön geätzte Schnittfläche dieses Stückes steht auf der geätzten Schnittfläche des Stückes Nr. 1 senkrecht und zeigt schmälere Streifen. — 11 1/32 Loth. — Von 1834. 26. 4.

3. Plättchen von dem Stücke Nr. 1 abgeschnitten; eine Fläche poliert, die andere stark geätzt mit scharfen und schönen Widmannstätten'schen Figuren. — 1 15/16 Loth. — Von 1834. 26. 4.

4. a. Unregelmäßiges von Nr. 1 abgeschnittenes Plättchen, blau angelaufen. — 3/4 Loth. — Von 1834. 26. 4.

4. b. Kleines viereckiges Plättchen, von dem Stücke Nr. 1 nach einer anderen Richtung abgeschnitten; ebenfalls blau angelaufen. — 11/32 Loth. — Von 1834. 26. 4.
\subsection{Guilford.}
\begin{center}
\small
Nord-Carolina, in den vereinigten Staaten von Nordamerika.
\end{center}
\paragraph{}
Ist im Jahre 1830 durch Professor Shepard in Sillimans American Journal bekannt gemacht worden, der es jedoch anfänglich (auch noch in seinem Treatise on Mineralogy, Vol. 2. 1835. S. 70.) als terrestrisches oder tellurisches Eisen beschrieb, und erst später (1841) dessen meteorische Natur erkannte.

Die bei Guilford befindlich gewesene Eisenmasse, deren ursprüngliches Gewicht unbekannt ist, wurde von den Schmieden der Umgegend lange zur Verfertigung von Nägeln, Hufeisen u. dgl. benützt. Den Rest der Masse, die nur noch 200 Grammen wog, brachte Professor Denison Olmstedt nach dem Yale-Collegium zu New-Haven in Connecticut.

Derbes und dichtes gediegenes Eisen (wie fast nicht zu bezweifeln mit etwas Magnetkies gemengt); auf polierten, und dann durch Hitze oder mit Säuren behandelten Flächen, sehr vollkommene Widmannstätten'sche Figuren zeigend. (Eine weitere Beschreibung erlaubt die Kleinheit des uns zu Gebote stehenden Stückes nicht.)

1. Kleines unregelmäßiges dreieckiges Plättchen, von einer Seite poliert und geätzt, von der anderen angefeilt. — 15/32 Loth. — 1842. 34. 3. — Vom Yale-College zu New-Haven in Nord-Amerika durch den Kurator Herrn Silliman in Tausch erhalten.
\subsection[Caille.]{Caille,}
\begin{center}
\small
bei Grasse, Dépt. du Var, im südlichen Frankreich.
\end{center}
\paragraph{}
Für die wissenschaftliche Welt entdeckt durch Herrn Brard im J. 1828; lag aber schon 200 Jahre lang vor der Kirche von Caille, wo die Masse als Bank diente.

Derbes und dichtes gediegenes Eisen, dem wenig Magnetkies beigemengt zu sein scheint. Die durch Ätzen zum Vorschein kommenden Widmannstätten'schen Figuren zeichnen sich durch geschlängelte hervorragende Linien aus, die mit den geraden Linien der Dreiecke nicht parallel gehen. (Es steht uns nur eine kleine Fläche zu Gebot, die keine sichere Diagnose erlaubt.) Dieses Eisen scheint eine leichtere Spaltbarkeit zu besitzen, als (mit Ausnahme von Durango) die meisten der bisher beschriebenen Lokalitäten.

1. Ein Fragment mit deutlich blättrigem Gefüge und etwas natürlicher Oberfläche, dann einer polierten Fläche, welche zum Teil, aber unvollkommen geätzt ist. — 8 3/8 Loth. — 1838. 32. 1. — Vom königl. Museum der Naturgeschichte zu Paris durch gütige Vermittlung der Herren Regierungsrat Baumgartner und Professor Alexander Brongniart in Tausch erhalten.

2. Fragment mit geätzter Fläche. — 4 7/8 Loth. — 1840. 29. 3. — Vom königl. Museum der Naturgeschichte in Paris durch Herrn Cordier auf Vermittlung des Herausgebers während seiner Anwesenheit zu Paris in Tausch erhalten.
\subsection{Ashville.}
\begin{center}
\small
Buncombe-County, Nord-Carolina, in den vereinigten Staaten von Nord-Amerika.
\end{center}
\paragraph{}
Bekannt seit 1839, durch einen Aufsatz von Herrn Charles Upham Shepard, Professor der Chemie am medizinischen Collegium des Staates von Süd-Carolina, in Sillimans American Journal. — Es wurde da nur eine runde Masse von der Größe eines Menschenkopfes lose auf dem Erdboden und, ungeachtet vielen Suchens, weiter nichts gefunden.

Derbes und dichtes, mit etwas Magnetkies gemengtes gediegenes Eisen, auf polierten Flächen durch Ätzen sehr ausgezeichnete, feinstreifige Widmanstättensche Figuren darstellend. Keines von allen bisher betrachteten Meteoreisen zeigt eine so ausgezeichnet blätterige Struktur und eine so große Tendenz, durch Oxydierung, parallel den oktaedrischen Teilungsflächen, in Oktaeder, Tetraeder und in rhomboederähnlichen Gestalten (wenn zwei parallele Teilungsflächen fehlen) zu zerklüften und endlich zu zerfallen, wie das Ashviller Eisen; eine Erscheinung, die leider das allmählige zu Grundegehen der Masse nach sich zieht.

1. Ein an der Oberfläche ganz zerklüftetes Stück, mit einer polierten Schnittfläche und vielen dabei liegenden abgebröckelten Abfällen. — 19 5/16 Loth — (samt den Abfällen). — 1840. 19. 1. — Wurde von Herrn Shepard an Herrn Heuland in London gesendet, und von diesem durch Herrn Doktor Bondi in Dresden zu Kauf erhalten.

2. An beiden Seiten schön geätztes Plättchen. — 15/16 Loth. — 1840. 19. 3. — Abschnitt von Nr. 1.
\subsection{Tennessee.}
\begin{center}
\small
Cosbys-Creek, Cocke-County, im östlichen Tennessee, in den vereinigten Staaten von Nord-Amerika.
\end{center}
\paragraph{}
Bekannt seit 1840 durch einen Aufsatz von Professor Troost zu Nashville in Sillimans American-Journal. — Es soll da eine Masse von 2000 Pfund liegen.

Derbes und dichtes gediegenes Eisen, mit wenig Magnetkies (und nach der Angabe von Professor Troost mit viel Graphit) gemengt. Uns stehen nur kleine durch Verwitterung von Brauneisenstein umgebene, kristallinische Bröckchen zu Gebote, in welche die Masse, gleich der Ashviller, sehr leicht zu zerfallen scheint. Die kristallinische Struktur lässt vermuten, dass sich auf geätzten Flächen deutliche Widmanstättensche Figuren darstellen würden.

1. Meist sehr kleine und einige größere braune, verrostete Bröckchen, durch das Zerfallen des Eisens parallel den Zusammensetzungs- oder Teilungsflächen entstanden. — 1 5/16 Loth. — 1843. 4. 68. — Von Herrn Doktor Bondi in Dresden zu Kauf erhalten. (Durch Professor Shepard kamen diese Fragmente aus Nord-Amerika an Herrn Heinrich Heuland in London.)
\subsection{Bohumilitz.}
\begin{center}
\small
Prachiner Kreis, Böhmen.
\end{center}
\paragraph{}
Gefunden im Monate September 1829. Die Fallzeit ist unbekannt; die Masse mag aber, da sie mit einer dicken Rinde von Eisenoxyd überzogen war, mehrere Jahrhunderte in der Erde gelegen sein. Sie wog bei ihrer nach heftigem Regenwetter, dass sie bloß legte, erfolgten Auffindung 103 Pfund und befindet sich jetzt, nach Abtrennung einiger Stücke, im vaterländischen Museum zu Prag.

Derbes und dichtes metallisches Eisen, stellenweise mit viel Magnetkies und einem schwarzen, nicht sehr harten, problematischen Minerale gemengt, das nicht Graphit sein kann, den chemische Analysen in diesem Eisen gefunden haben. In diese schwarze Substanz, die sowohl in der Mitte der Masse als an der Oberfläche derselben zuweilen in fast zolllangen Partien auftritt, ist wieder gediegenes Eisen und auch Magnetkies fein eingesprengt. Diese schwarzen Partien sind ringsum von einer Rinde von Magnetkies umschlossen, von dessen äußerem Rande höchst sonderbare kurze und feine Streifen auslaufen und sich zuweilen in die Eisenmasse noch weiter verzweigen, Der Magnetkies ist an manchen Stellen in Nieren von der Größe einer Mandel ausgeschieden und zeigt sich auch in sonderbaren eckigen Ausscheidungen mit aus- und einspringenden Winkeln. — Durch Anlaufen und Ätzen dieses Eisens erscheinen die Widmannstätten'schen Figuren in der Beziehung nicht ganz vollkommen, als die Mittel- oder Zwischenfelder mit den feinen Schraffierungsleisten, welche die eigentlichen Figuren (Dreiecke, Vierecke u. s. w.) bilden, nur in sehr geringer Anzahl vorhanden, dagegen die Streifen sehr ausgezeichnet und breit sind, und fast die ganze Masse erfüllen. Diese dickeren Streifen sind partienweise (d. h. mehrere nebeneinander liegende immer zusammengenommen) bald nach dieser, bald nach jener Richtung mit ganz feinen und parallelen Linien, meist auch mit Linien, die nach einer, oder nach zwei anderen Richtungen laufen und die ersteren durchschneiden, schraffiert, wodurch, wenn man die geätzte Fläche nach verschiedenen Richtungen wendet, jener abwechselnde Glanz erscheint, den man metallischen Schimmer (moire metallique) nennt. Die Einfassungsleisten treten zwischen den Streifen nicht deutlich hervor, und sind auf angelaufenen Platten deutlicher zu sehen, als auf geätzten. — Ein höchst merkwürdiges Meteoreisen, das noch andere interessante Eigentümlichkeiten zeigt, die hier nicht weiter beschrieben werden können, und noch genauere mineralogische und chemische Untersuchung verdient.

1. Großes Stück von dreieckiger Form; eine der zwei größeren Flächen ist ein Teil der natürlichen Oberflächen der Masse und mit Eindrücken versehen; die andere große, 5 Zoll 2 Linien lange Fläche ist eine schön geätzte Schnittfläche mit unvollkommenen Widmannstätten'schen Figuren bedeckt; an einer schmalen, 5 1/2 Zoll langen, fein polierten Seitenfläche zeigt sich, außer kleineren Partien von Magnetkies, ein sich auf die geätzte Fläche hinüberziehender Flecken von der erwähnten schwarzen problematischen Substanz, die auch noch an zwei anderen Stellen der geätzten Fläche auftritt, An einer anderen Stelle der polierten Schnittfläche ist, und zwar an der Oberfläche der Masse, eine sonderbare, breccienartige Bildung zu unterscheiden, von welcher noch Erwähnung gemacht werden wird, — 4 Pfund 19 5/8 Loth. — 1831. 34. 1. — Durch Vermittlung des k. k. geheimen Rates, Grafen Caspar von Sternberg, vom vaterländischen Museum zu Prag als wertvolles Geschenk erhalten.

2. Ein dünnes, längliches Plättchen, schwach, aber vortrefflich geätzt, auf welchem vorzüglich die erwähnten feinen Linien auf den breiten Streifen ungemein schön erscheinen; mit Magnetkies und Körnern des schwarzen Minerals. — 23/32 Loth. — Abschnitt von Nr. 1.

3. Dickes, poliertes Plättchen. An dem äußeren, der Rinde der Masse angehörigen Rande des Plättchens befindet sich der Durchschnitt einer 5/4 Zolllangen, schon bei dem Stücke Nr. 1. erwähnten breccienartigen Bildung, die mit dem Eisen innig zusammenhängt und kaum von einer späteren Entstehung (während des Liegens der Eisenmasse in der Erde) sein dürfte. Sie scheint aus einer Grundmasse von Brauneisenstein zu bestehen, in welche braune und grünliche Körner von harten Mineralien und auch zwei kleine Parzellen von gediegenem Eisen eingewachsen sind. — 2 21/32 Loth. — Abschnitt von Nr. 1.

4. a. Längliches dickes Plättchen, von beiden Seiten poliert; von hohem Interesse durch die schon bei der allgemeinen Beschreibung der Bohumilitzer Masse erwähnten Beimengungen, die auf diesem Plättchen, nebst anderen merkwürdigen Verhältnissen, ungemein schön zu beobachten sind. Nebst einem Flecken, aus dem schwarzen, problematischen Mineral bestehend, in welches gediegenes Eisen und Magnetkies fein eingesprengt sind, ist auch, und zwar an das schwarze Mineral anstoßend, eine Partie Magnetkies von 3/4 Zoll im Durchmesser eingewachsen, aber teilweise Behufs einer damit vorzunehmenden chemischen Analyse herausgebrochen. Das schwarze Mineral zieht sich an ein Paar Stellen auch in Zickzacklinien durch das Eisen. — 2 19/32 Loth. — Abschnitt von Nr. 1.

4. b. Ein kleines keilförmiges, poliertes Stück, eine Mandel von Magnetkies einschließend. — 1 1/4 Loth. — Abschnitt von Nr. 1.

5. Dickes, blau angelaufenes Blättchen. Das schwarze Mineral erhielt durch das Anlaufen eine dunkel bleigraue Farbe. — 2 13/16 Loth. — Abschnitt von Nr. 1.
\subsection{Bahia.}
\begin{center}
\small
Am Bache (Riacho) Bemdegò, der in den Rio San Francisco fällt, nördlich von Monte Santo, Capitanie Bahia, Brasilien.
\end{center}
\paragraph{}
Gefunden 1784; bekannt seit 1816, durch einen Bericht von A. F. Mornay in den Philosophical-Transactions of the Roy. Soc. of London for 1816. P. 2. — Die Fallzeit ist unbekannt. Die daselbst noch im Freien liegende, nicht transportable Masse wiegt, nach der Berechnung des Professors Martius in München, der in seiner Reise in Brasilien über dieselbe interessante Nachrichten mittheilt (2. Band Seite 736, u. f.) 17300 Pfund.

Derbes und dichtes gediegenes Eisen, stellenweise, obwohl wie es scheint nicht häufig, mit Magnetkies und wahrscheinlich auch, jedoch wohl selten, mit jenem schwarzen, problematischen Mineral gemengt, dessen bei dem Eisen von Bohumilitz Erwähnung geschah. Die durch Ätzen oder Anlaufen zum Vorschein kommenden unvollkommenen Widmannstätten'schen Figuren haben im Allgemeinen große Ähnlichkeit mit der Zeichnung des Meteoreisens von Bohumilitz; die Zwischenfelder sind aber bei Bahia noch seltener vorhanden, die feinen Linien auf den breiten Streifen weniger regelmäßig und auch die Einfassungsleisten weniger deutlich. Die Felder, welche abwechselnd den metallischen Schimmer (moire metallique), zeigen, sind bei dem Eisen von Bahia ausgedehnter als bei jenem von Bohumilitz.

1. Ein großes dreieckiges Stück mit viel natürlicher Oberfläche, über welche eine scharfe Kante hinwegläuft, einer großen geätzten und einer kleinen polierten Fläche. — 3 Pfund, 14 1/2 Loth. — 1842. 1. 2. — Aus der Heuland'schen, später Heath'schen Meteoriten-Sammlung durch Herrn Pötschke gekauft. Herr Heuland erhielt das 10englische Pfund schwer gewesene Stück, von welchem das unserige abgeschnitten wurde, als Geschenk von dem gewesenen englischen Gesandten in Brasilien, Sir Eduard Thorton.

2. Längliches, gekrümmtes Stück, mit schwarzer natürlicher Oberfläche, einer veralteten Bruchfläche und einer fein polierten Schnittfläche, worauf wie dieses auf gut polierten Flächen der Eisenmassen von Bahia und Bohumilitz nicht selten ist, die Begränzung der breiten Streifen durch vertiefte Linien angedeutet erscheint. — 15 3/16 Loth. — 1822. 54. 1. — Wurde durch die Herren Spix und Martius aus Brasilien nach München gebracht, und von der königl. bayerischen Akademie der Wissenschaften dem k. k. Mineralien-Kabinette als Geschenk mitgeteilt.

3. Abschnitt von dem Stücke Nr. 2, mit einer geätzten und einer alten Bruchfläche, die deutliches blättriges Gefüge wahrnehmen lässt. — 1/4 Loth. — Abschnitt von Nr. 2.

4. Auf beiden Seiten geätztes Plättchen. — 1 23/32 Loth. — 1841. 14. 12. — Wie das Stück Nr. 1 erhalten, von dem es abgeschnitten ist.

5. Kleines, angelaufenes Plättchen. — 13/32 Lth. — Abschnitt von Nr. 1.

6. Kleines Plättchen, geätzt und dann angelaufen. — 1/2 Loth. — Abschnitt von Nr. 1.
\subsection{Zacatecas.}
\begin{center}
\small
Mexico.
\end{center}
\paragraph{}
Den Eingebornen seit undenklichen Zeiten bekannt, einem größeren Publicum in Amerika durch die Gazeta de Mexico von 3. April 1792, den europäischen Gelehrten aber erst seit 1804 durch das schon früher bei Toluca erwähnte Werk von Sonneschmidt, der die Masse auf 20 Zentner schätzte.

Neuere Nachrichten über dieses Eisen verdanken wir Herrn Burkart (Aufenthalt und Reisen in Mexico in den Jahren 1825 bis 1834. Band 1. S. 389.) Man erzählte Herrn Burkart in Zacatecas, dass diese in einem Hause der Tacuba-Strasse liegende Eisenmasse (die er für schwerer hält, als sie Sonneschmidt schätzte), aus dem Norden nach der Stadt Zacatecas gebracht worden sei, Dass sie nicht immer da lag, scheint auch aus einer Erzählung in den angeführten Blatte der Gazeta de Mexico, das wir der gültigen Mittheilung des Herrn Alexander v. Humboldt verdanken, hervorzugehen.* Sollte sie nicht, wie die Meteoreisenmasse von Charcas, von der Meierei San Jose del Sitio dahin gebracht worden sein? da Herr Burkart die Eisenmasse, welche jetzt zu Charcas an der Ecke der Kirche als Radabweiser dient, jener in der Stadt Zacatecas liegenden in ihrem äußeren Ansehen, in Bruch, Farbe, u. s. w. ganz ähnlich fand, Eine Entscheidung darüber lässt sich nur durch Untersuchung polierter und geätzter Schnittflächen von beiden Eisenmassen erlangen. Leider ist es Herrn Burkart aller Anstrengung ungeachtet nicht gelungen, von der etwa 8 bis 9 Zentner schweren Eisenmasse von Charcas ein Stückchen abzutrennen. (Siehe das angeführte Werk B. 2. S. 127.)

*) Die Volkssage von der Eisenmasse von Zacatecas lautet nämlich: die Masse war ursprünglich Silber und wurde aus dem berühmten Bergwerk la Quebradilla nach der Haustür des Bergwerks-Eigentümers in Zacatecas gebracht, weil dieser die Absicht hatte, sie in der Gestalt eines Heiligen der Ehre Gottes zu widmen, Später änderte er jedoch seinen Entschluss und als er die Masse mit Keilen auseinanderschlagen wollte, war sie nicht mehr Silber, sondern Eisen.

Derbes und dichtes gediegenes Eisen, mit einer ganz ungewöhnlichen, durch die ganze Masse zerstreuten Menge von Magnetkies (und auch gemeinem Schwefelkies?) der darin in meist runden oder linsenförmigen Partien eingewachsen ist. Wenn man größere polierte Platten dieses merkwürdigen Eisens zu sehen Gelegenheit hat (wie die im Besitze des Herrn Baron Reichenbach in Wien befindliche, mehr als spannenlange, die von einem großen, durch Herrn Burkart nach Europa gebrachten Stücke abgeschnitten wurde), so zeigt sich der Kies in dem gediegenen Eisen so verteilt, dass er darin gleichsam ein unvollkommenes, netzförmiges Geflechte bildet. Die oben angedeutete Vermutung, dass der Kies von zweierlei Beschaffenheit sei, rechtfertigt sich durch den Umstand, dass man an größeren polierten Kiesflecken eine doppelte Farben-Nuance und Dichtigkeit zu unterscheiden vermag, und diese auch mit verschiedenen Farben anlaufen. Das Eisen durchziehen zickzackförmige Sprünge, und auf gut polierten Flächen werden in dem Eisen feine, etwas vertiefte Linien sichtbar, die nach verschiedenen Richtungen ziehend, sich oft berühren und schneiden. Durch mäßiges und vorsichtiges Ätzen dieses sehr schwer angreifbaren und wegen des vielen Kieses in Säuren überhauptschwer zu behandelnden Eisens bilden sich keine wahren Widmannstätten'schen Figuren, sondern die schon auf polierten Flächen zum Vorschein kommenden geraden und langen vertieften Linien, wovon gewöhnlich zwei nahe an einander liegende parallel laufen, werden schärfer und deutlicher, und die zwischen den Linienpaaren liegenden, meist viereckigen Felder sind mit Punkten und feinen Strichelchen erfüllt, die, unter sich selten parallel, meist nach allen Richtungen und oft fast strahlenförmig aus einander laufen. — Ein höchst charakteristisches und merkwürdiges, nach seiner Zeichnung schwer zu beschreibendes Meteoreisen.

1. Ein Stück mit natürlicher Oberfläche, zwei Bruch- und einer anpolierten Schnittfläche. — 24 1/2 Loth. — 1839. 2. 1. — Von dem königl. preußischen Oberbergamts-Sekretär Herrn Burkart in Bonn, der dieses Stück von der Masse in Zacatecas abhämmerte, gefälligst zu Kauf erhalten.

2. Schwach geätztes Plättchen. — 2 1/16 Loth. — Von 1839. 2. 1. — Abschnitt von Nr. 1.

3. Blau und violett angelaufenes Plättchen, wodurch jedoch keine Zeichnung zum Vorschein kam. — 1 1/2 Loth. — Von 1839. 2. 1. — Abschnitt von Nr. 1.

4. Ein länglicher Abschnitt mit einer Schnittfläche, die zum Teil poliert, zum Teil sehr schwach und zum Teil sehr stark geätzt ist (auf letzterem ist die charakteristische Zeichnung zum Teil verschwunden, und nur ein körniges Gefüge, mit linienförmigen Einschnitten zu sehen). — 4 29/32 Loth. — 1838. 25. 12. — Aus der ehemals Heuland'schen, später Heath'schen Meteoriten-Sammlung durch Herrn Pötschke gekauft. Herr Heuland überkam das Stück mit dem Mineralien-Vorrate seines im Jahre 1806 zu St. Petersburg gestorbenen Oheims Jakob Forster.
\subsection[Rasgatà.]{Rasgatà,}
\begin{center}
\small
nordöstlich von Santa Fe de Bogotá, in der Nähe der Salinen von Zipaquira, Republik Columbia (jetzt Republik Neu-Granada), Süd-Amerika.
\end{center}
\paragraph{}
Gefunden 1810; bekannt seit dem Jahre 1823 (?) durch einen zu Santa Fe de Bogotá spanisch gedruckten Bericht der Herren Mariano de Rivero und Bousingault, der im 25. Bande der Annales de Chimie et de Physique v. Jahre 1824 übersetzt wurde. — Es scheinen da mehrere Stücke gefunden worden zu sein. Die genannten Berichterstatter sprechen von zwei Massen von 41 Kilogrammen (73 Pfund, 1 Loth Wiener Gewicht) und von 22 Kilogrammen (39 Wiener Pfund und 6 Loth), die wohl die größten gewesen sein dürften.

Derbes und dichtes gediegenes Eisen, zuweilen mit Schwefelkies (Magnetkies ?) gemengt, der jedoch, wie es scheint, nie in dem Eisen eingesprengt ist, sondern nur Höhlungen in demselben teilweise ausfüllt. Das Eisen ist von gebogenen oder zickzackförmigen Sprüngen durchzogen und es sind darin auch größere und kleinere Höhlungen vorhanden, Durch Anlaufen oder Ätzen kommt wohl eine gewisse Zeichnung zum Vorschein, aber wahre Widmannstätten'sche Figuren zeigen sich nicht. Die Zeichnung besteht aus sehr feinen, meist geraden, seltener gekrümmten Linien, die nach mehreren Richtungen ziehen, sich aus der Masse etwas erheben und glänzen (weil sie durch Säuren nicht leicht angegriffen werden), sich zuweilen, aber selten berühren und folglich nur selten geschlossene Zwischenfelder oder Figuren darstellen; den übrigen Raum erfüllen feine kurze Strichelchen und Punkte, die sich ebenfalls schwach erheben und glänzen.

1. Ein größtenteils von natürlichen Flächen umschlossenes Stück, dass eine hervorragende Ecke gebildet hat, mit einer polierten Schnittfläche, über die sich zickzackförmige Sprünge hinziehen. Das Stück zeigt sehr merkwürdige Oberflächen-Verhältnisse (Vertiefungen und Höhlungen) und eine schlackenartige Rinde mit sonderbaren Poren oder feinen Löchern. — 1 Pfund 3 29/32 Loth. — 1840. 4. 7. — Aus der ehemals Heuland'schen, später Heath'schen Meteoriten-Sammlung durch Herrn Pötschke in Wien gekauft. Ist (wie die folgenden Stücke Nr. 2 bis Nr. 5) von einem großen Stücke abgeschnitten, das bei seiner Ankunft in Wien 13 Pfund 2 Loth Wiener-Gewicht wog, und von Herrn Mariano de Rivero aus Columbien an Herrn Heuland in London gesendet wurde.

2. Eine 6 Linien dicke und 3 1/2 Zoll lange Platte, an den Rändern von natürlicher Oberfläche umgeben; die zwei großen polierten Schnittflächen sehr schwach geätzt; vorzüglich merkwürdig durch den Umstand, dass die eine Schnittfläche eine ovale Höhlung durchschnitt, die teilweise mit porösem Schwefelkies (die Farbe ist nicht die des Magnetkieses) ausgekleidet ist. Sprünge durchziehen diese Platte ebenso, wie die Schnittfläche des Stückes Nr. 1. — 30 5/8 Loth. — 1842. 1. 3. — Wie Nr. 1 akquiriert.

3. Dicker Abschnitt, von zwei Seiten mit Rinde umgeben, mit zwei schmalen und einer breiten, polierten, dann einer größeren schön geätzten Schnittfläche. Auf einer der polierten Flächen befindet sich eine sonderbare kleine längliche Höhlung. — 5 13/16 Loth. — 1838. 25. 8. Wie Nr. 1 akquiriert.

4. Kleines, an beiden Seiten schwach geätztes Plättchen. — 7/8 Loth. — 1838. 25. 9. Wie Nr. 1 akquiriert.

5. Kleines, blau und violett angelaufenes Plättchen mit Zeichnungen. — 5/8 Loth. — 1838. 25. 10. Wie Nr. 1 akquiriert.

Die Herren Rivero und Bousingault haben sowohl in dem Meteoreisen von Santa Rosa in Columbien, das wir in der k. Berliner Mineralien-Sammlung zu sehen und hinsichtlich der heim Ätzen sich zeigenden Figuren zu untersuchen Gelegenheit hatten, wobei es sich wie Rasgatà verhält, als auch in diesen letzteren einen nicht unbeträchtlichen Antheil von Nickel gefunden. Wiederholte in Wien mit unserem Eisen von Rasgatà, das doch von den Herrn Rivero und Bousingault herstammt, angestellte Versuche konnten darin keinen Nickel entdecken. Dieses merkwürdige Eisen verdiente wohl eine wiederholte genaue chemische Untersuchung.
\subsection[Tucuman.]{Tucuman,}
\begin{center}
\small
15 Meilen von Otumpa (das nach einigen Angaben im Bezirke, jetzt Staate, St. Jago del Estero liegen soll) in einer wüsten Gegend des Staates Tucuman, einer der vereinigten Provinzen von Rio de la Plata (argentinische Republik), Süd-Amerika.
\end{center}
\paragraph{}
Wurde von Don Miguel Rubin de Celis im Auftrage der spanischen Regierung im Jahre 1783 aufgesucht, und ist durch eine Übersetzung seines Berichtes in den Londoner Philos. Transact. vom Jahre 1788. T. 1. der wissenschaftlichen Welt bekannt geworden. — Die Fallzeit ist unbekannt. — Rubin de Celis schätzte das Gewicht der Masse auf 300 Zentner.

Derbes und dichtes gediegenes Eisen, oft mit größeren oder kleineren Höhlungen, die zuweilen ganz oder teilweise mit Schwefelkies ausgefüllt sind, der auch sonst noch in kleineren Partien in der Masse zerstreut ist. Auf Bruchflächen kommt eine kristallinische Struktur, parallel den Flächen des Oktaeders zum Vorschein, auf polierten Flächen kurze, nach verschiedenen Richtungen gekehrte, linienförmige Einschnitte. Durch mäßiges Ätzen erscheinen auf diesem, durch Salpetersäure schwer angreifbaren Meteoreisen keine Widmanstätten‘schen Figuren, sondern kurze, etwas erhöhte Linien, die nach mehreren Richtungen gekehrt sind, sich auch berühren und gegenseitig schneiden, und dem Ganzen ein gestricktes oder federartiges Ansehen verleihen, je nachdem die Striche sich unter rechten oder schiefen Winkeln berühren oder schneiden. Die geätzten Flächen gleichen in dieser Beziehung, d. h. hinsichtlich ihrer Zeichnung, der langsam erkalteten kristallinischen Oberfläche mancher, Metallkuchen, z. B. von Antimon, Tellur, Wismuth, oder auch der Zeichnung, welche oft auf gefrorenen Fensterscheiben zum Vorschein kommt. Durch sehr starkes Ätzen bietet dieses Eisen eine körnige Oberfläche dar, von tiefen Einschnitten nach verschiedenen Richtungen durchkreuzt. — Ein Meteoreisen von merkwürdiger, nur mit dem Eisen von Senegal verwandter Beschaffenheit.

1. Ein Stück mit viel natürlicher Oberfläche, mit Bruchflächen, worauf sich oktaedrische Teilungsgestalten befinden, und einer polierten Schnittfläche, wodurch zwei Höhlungen durchschnitten wurden, wovon die größere teilweise mit Schwefelkies ausgefüllt ist. — 19 23/32 Loth. — 1840. 4. 8. — Aus der Heuland'schen, später Heath'schen Meteoriten-Sammlung durch Herrn Pötschke gekauft. Kam durch einen in Chili ansässigen Engländer nach London.

2. Platte, von der einen Seite schwach und von der anderen teilweise stark geätzt. — 3 1/16 Loth. — Von 1840. 4. 8. — Abschnitt von Nr. 1.

3. a. Kleines Stück, mit in Folge der Abmeißelung gekrümmten Blättern, mit einer kleinen geätzten und einer noch kleineren blau angelaufenen Fläche, die jedoch wegen Zerquetschung des Eisens nichts Lehrreiches darbieten. — 1 3/32 Loth. — 1807. 22. 16. — Durch den verstorbenen v. Fichtel aus Madrid zu Kauf erhalten.

3. b. Kleines Stück mit natürlicher Oberfläche und einer kleinen polierten Fläche. — 1 1/8 Loth. — 1827. 27. 4044. — Aus der von der Nüll'schen Mineralien-Sammlung, in die es ebenfalls durch den verstorbenen v. Fichtel gekommen ist.

3. c. Ganz kleines, schön angelaufenes Plättchen, mit kurzen feinen Strichelchen und kleinen Pünktchen; von der einen Seite geätzt. — 3/32 Loth. — Ein Abschnitt von Nr. 1.
\subsection{Senegal.}
\begin{center}
\small
Mehrere Gegenden am oberen Senegal in Afrika, besonders im Lande Siratik und im Lande Bambuk, von wo einige Stücke über Galam nach Fort Louis und nach den englischen Pflanzörtern kamen.
\end{center}
\paragraph{}
Ist durch den Reisenden Compagnon und durch die mineralogischen Lehrbücher von Baumer, Wallerius und anderen zwischen den Jahren 1760 und 1770 in Europa bekannt geworden. — Die Fallzeit ist unbekannt. — Es müssen da auf einer sehr großen Strecke sehr viele große und kleine Eisenmassen zerstreut herum liegen, die von den Eingebornen schon lange zur Verfertigung von Töpfen u. s. w. benützt werden.

Derbes und dichtes gediegenes Eisen, an welchem nur sehr selten eine geringe Einmengung von Schwefel- oder Magnetkies und (wenigstens an den uns zu Gebote stehenden Stücken) auch keine Höhlungen wahrzunehmen sind. Durch Ätzen mit Säuren kommen keine Widmannstätten‘schen Figuren, sondern nur kurze feine, nach mehreren Richtungen gekehrte Striche zum Vorschein, die sich zuweilen berühren und schneiden und gestrickte oder federartige Zeichnungen bilden. Die Masse erhält bei stärkerer Ätzung ein gekörntes Ansehen und nach verschiedenen Richtungen gekehrte Einschnitte, und ist daher dem Eisen von Tucuman hinsichtlich seines Verhaltens in Säuren nahe verwandt.

1. Dicke Platte mit natürlicher Oberfläche an den Rändern, dann mit einer rohen Schnitt- und mit einer fein polierten Fläche. — 12 23/32 Loth. — 1840. 13. 6. — Von dem französischen Naturalien-Händler, Herrn F. Marguier, gekauft, der dieses und die folgenden durch Zerschneiden erhaltenen Stücke Nr. 2, 3 und 4, von einem vom Senegal zurückgekehrten Naturforscher in Bordeaux erhielt.

2. Plättchen , von einer Seite sehr schwach geätzt, von der anderen poliert; mit etwas Schwefelkies. — 2 17/32 Loth. — 1840. 13. 7. — Wie und mit Nr. 1 akquiriert.

3. Dickes Plättchen, von allen Seiten sehr stark geätzt. — 3 15/16 Loth. — 1840. 21. 8. — Wie Nr. 1 akquiriert.

4. Dünnes Plättchen, poliert und dann blau angelaufen. — 31/32 Loth. — 1840. 13. 8. — Mit und wie Nr. 1 akquiriert.

5. a. und b. Ein Stück mit natürlicher Oberfläche und mit Bruchflächen, wovon ein dabei liegendes Plättchen abgeschnitten ist. Von den zwei Schnittflächen ist die Eine poliert, die andere schwach geätzt. — Zusammen 9 1/4 Loth. — 1843. 4. 67. — Durch Doktor Bondi in Dresden zu Kauf erhalten.
\subsection{Vorgebirge der guten Hoffnung.}
\begin{center}
\small
Kapland (oder Kapkolonie). Zwischen dem Sonntags- und Boschmanns-Flüsse, Afrika.
\end{center}
\paragraph{}
Gefunden 1793. Bekannt seit 1801 durch Barrows Reise in Süd-Afrika; besser seit 1804 durch eine Abhandlung von Van Marum in den Verhandlungen der Gesellschaft der Wissenschaften zu Harlem. — Die Fallzeit ist unbekannt. — Man fand da eine 300 Pfund schwere Masse, von der Mehreres verschmiedet wurde, und die später bei ihrer Überführung nach Europa (in das Naturalien -Kabinett zu Harlem) nur noch 171 Pfund wog.

Neuerlich ist wieder Meteoreisen am großen Fischflüsse in der Kap-Kolonie gefunden worden, (Siehe: Alexander Expedition of Discovery into the Interior of Afrika. London 1838. Vol. 2. Appendix S. 272.) Capitan Alexander hat Eisenmassen in großer Menge über eine große Strecke Landes zerstreut angetroffen. Da der Sonntags- und Boschmannsfluss, zwischen welchen die Barrow'sche Eisenmasse gefunden wurde, von dem großen Fischflüsse, namentlich in seiner oberen Strecke, nicht sehr entfernt sind, und die durch Capitan Alexander bekannt gewordenen Eisenmassen über eine große Landstrecke zerstreut liegen, so dürften wohl alle diese Eisenmassen von einem gemeinschaftlichen Ereignisse herrühren.

Derbes und dichtes gediegenes Eisen mit wenig und meist fein eingesprengtem Magnetkies. Durch Ätzen von polierten Flächen mit Säuren entstehen keine Widmannstätten'schen Figuren; man sieht nur über die graue sehr fein gekörnte Fläche schmälere oder breitere, gerade und gekrümmte undeutliche Bänder hinziehen, die sich jedoch nur zeigen, wenn man die Fläche nach gewissen Richtungen hält. Auf polierten Flächen bemerkt man, dass sich breite Streifen und verzweigte, fast dendritische Zeichnungen von der Oberfläche der Masse in das Innere ziehen, durch den Umstand, weil diese Stellen durch das Polieren weniger Glanz erlangen. Durch starkes Ätzen kommen manchmal vertiefte, etwas gekrümmte Streifen, an anderen Stellen auch kleine sternförmige Erhöhungen zum Vorschein. — Ein höchst sonderbares und eigentümliches, durch seine Eigenschaften ganz isoliert stehendes Meteoreisen, etwa nur dem folgenden von Clairborne in Alabama verwandt.

1. Unregelmäßiges Stück, zum Teil mit schwarzer, natürlicher Oberfläche und einer polierten Schnittfläche. Unterhalb der, wie es scheint, leicht absprengbaren Rinde sind kleine, runde, fast netzartige Vertiefungen wahrnehmbar. — 1 Pfund 2 5/16 Loth. — 1815. 33. 1. — Durch Professor Van Marum aus dem Naturalien-Kabinette der Gesellschaft der Wissenschaften zu Harlem in Tausch erhalten.

2. Platte, von einer Seite poliert, von der anderen geätzt, mit natürlicher Oberfläche an drei Rändern. — 8 1/2 Loth. — 1840. 23. 1. — Durch Vermittlung des Herrn Bondi in Dresden von Herrn Heuland in London zu Kauf erhalten. — Stammt aus der Mineralien-Sammlung des verstorbenen Sowerby Vater, (der von dem Kap'schen Eisen ein größeres von Barrow nach England gebrachtes Stück besaß, und davon einen Säbel für den Kaiser Alexander von Russland schmieden ließ).

3. a. Ein kleines, dreieckiges Stück, ringsum stark geätzt, wodurch auf einer Seite die bei der Beschreibung erwähnten, vertieften Streifen und auf der anderen die erwähnten Sternchen zum Vorschein kamen. — 17/32 Lth. — Abschnitt von Nr. 1.

3. b. Ein kleines angelaufenes Blättchen, ohne Zeichnung oder Figuren. — 3/8 Loth. — Ebenfalls Abschnitt von Nr. 1.
\subsection{Clairborne.}
\begin{center}
\small
Lime-Creek, Clarke-County, im Staate Alabama, Nord-Amerika.
\end{center}
\paragraph{}
Bekannt seit 1838 durch eine Abhandlung von Herrn Charles Jackson in Sillimans American-Journal, Vol. 34. — Es wurde da eine 10 Zoll lange und 1 bis 6 Zoll breite Masse gefunden, und man vermutet, dass noch mehrere Massen in der Gegend vorhanden sein mögen.

Derbes und dichtes gediegenes Eisen, worin Magnet- oder Schwefelkies teils in Körnern und Linien, teils in unendlich feinen Pünktchen, die letzteren gleichförmig durch die ganze Masse verteilt, eingemengt ist, Eine so feine und gleichmäßige Verteilung des Kieses findet bei keinem anderen Meteoreisen Statt. Das kleine uns zu Gebote stehende Blättchen dieses Meteoreisens zeigte durch starkes Ätzen mit Säuren (verdünnte Salpetersäure griff es gar nicht an) weder Widmannstätten‘sche Figuren, noch Linien, Striche oder Bänder; die Säure verwandelte die übrigens auch schwer polirbare und nur wenig Glanz erlangende Fläche nur in eine schwarzgraue Substanz, aus der die Kiespünktchen hervorglänzen, — Ein merkwürdiges, weitere Untersuchung verdienendes Eisen, das nach den Untersuchungen von Dr. Jackson Chlorine enthält.

1. Kleines, von einer Seite geätztes Plättchen (wegen schnellen Rostens mit Schellack-Firniss überzogen). — 13/32 Loth. — 1842. 46. 1. — Ein interessantes Geschenk von Herrn Charles Jackson zu Boston in Nord-Amerika.
\subsection{Oaxaca.}
\begin{center}
\small
Aus einem indischen Dorfe in der Misteca, im Staate Oaxaca in Mexico.
\end{center}
\paragraph{}
Die Existenz dieses Meteoreisens ist noch nicht durch öffentliche Nachrichten bekannt geworden. Herr Baron Karawinsky in München, der zu wiederholten Mahlen Mexico bereiste, brachte das Stück von da mit. Es wurde von einem Klumpen, wie er sich in einem Briefe ausdrückt, der an dem angegebenen Orte liegt, abgemeißelt.

Da das kleine Stückchen unserer Sammlung beim Abmeißeln gehämmert worden ist, so lässt sich von dieser Lokalität keine Diagnose geben. Das Eisen ist derb und dicht und zeigt durch Ätzen feine gekrümmte Streifen.

1. Ein kleines längliches und gekrümmtes Stückchen, mit einer ebenfalls gekrümmten, anpolierten, und dann geätzten Fläche. Das Stückchen scheint gehämmert und dadurch gestreckt worden zu sein. — 15/32 Loth. — 1834. 26. 8 — Von Baron Karawinsky in München als Geschenk erhalten.
\subsection{Grönland.}
\begin{center}
\small
In einer Sowallik genannten Gegend, an der nördlichen Küste der Baffings-Bay, unter 76° 22' der Breite und 58° westlicher Länge von Greenwich.
\end{center}
\paragraph{}
Bekannt seit 1819 durch die Reise des Capitans Ross. — Nach der Aussage der Eskimos sollen allda, 30 englische Meilen von der Küste entfernt, zwei große Eisenmassen vorhanden sein, von welchen sie Stücke abbrechen, um Geräte daraus zu verfertigen.\footnote{Möge es dem jetzt regierenden König von Dänemark, einem großen Gönner der Wissenschaften und der Mineralogie insbesondere, gefallen, die Sache ermitteln, und Stücke des unveränderten Eisens nach Europa bringen zu lassen.}

Die Verarbeitung des Eisens gestattet keine Diagnose; es lassen sich aber darin noch Schwefelkies und schwarze Einmengungen unterscheiden. Die geätzte Fläche bekam ein körniges Ansehen, und man hat sich bei Gelegenheit des Ätzens von der Anwesenheit des Nickels in diesem Eisen überzeugt.

1. Als kurze, dreieckige Messerklinge zum Abhäuten der Seehunde von den Eskimos verarbeitet. Die eine Fläche verrostet,, die andere zum Teil anpoliert, und zum Teil geätzt. — 7/32 Loth. — 1838. 25. 13. — Aus der ehemals Heuland'schen, dann Heath'schen Meteoriten-Sammlung durch Herrn Pötschke gekauft. Herr Heuland kaufte diese Messerklinge, die in ein Heft, aus dem Zahne eines Wallrosses angefertigt, befestigt war, in einer öffentlichen Versteigerung zu London. Das Stück stammt von der ersten englischen Expedition unter Capitan Ross, der dieses zur Fischerei benützte Messer von den Eskimos eintauschte.
\clearpage
\section{Zusätze.}
\paragraph{}
1. Die im Wiener k. k. Mineralien-Kabinette aufgestellten 258 Nummern von Meteoriten (von kleineren Stücken machen in mehreren Fällen zwei oder mehrere zusammen Eine Nummer) repräsentieren nach dem im Kabinetts-Kataloge jedem einzelnen Stücke beigefügten Werte oder Ankaufspreise eine Gesamtsumme von 33,196 1/2 Gulden in Conventions-Münze.\footnote{Wir liefern im einem Anhange, um einem mehrseitigen Wunsche zu entsprechen, die Schätzung der einzelnen Stücke.} Sie wiegen zusammen 330 Pfund und 14 29/32 Loth Wiener Gewicht. (Ein Wiener Pfund von 32 Wiener Lothen ist gleich 38,314 preußischen Lothen, oder 19,754 englischen Avoir du poids Unzen, oder 560,012 französischen Grammen. Ein Kilogramm ist gleich 1 Pfund 25 9/64 Loth Wiener Gewicht.)

2. Außer der unter Glas zur Schau gestellten Meteoriten-Sammlung des k. k. Mineralien-Kabinettes befinden sich daselbst in einem Schranke des vierten Saales in Schiebfächern (Schubladen) noch zwei kleinere Sammlungen von Meteoriten zur näheren naturhistorischen oder chemischen Untersuchung derselben und beinahe von allen Lokalitäten der großen Sammlung. Die erste, in 12 Kartons, besteht aus kleinen Stücken, von welchen die spezifischen Gewichte der angefügten Tabelle genommen wurden, aus Fragmenten, die zu anderen wissenschaftlichen Untersuchungen, z. B. mechanischer Sonderung ihrer Gemengteile u. s. w. dienten, und endlich auch aus manchem kleinen Stückchen, an dem sich ein interessantes Verhältnis darstellt, dass die Stücke der größeren Sammlung nicht darbieten. Die zweite der angeführten kleineren Sammlungen ist eine zum Behufe von mikroskopischen Untersuchungen hergerichtete, in welcher die Meteorsteine in grob gepulvertem Zustande in Schiebern unter zwei runden Glasplättchen; die Meteoreisen in eben solchen Schiebern in ganz dünn geschnittenen und dann mit Säuren geätzten kleinen Plättchen aufbewahrt werden.

3. Als Anhänge zu der Meteoriten-Sammlung werden ferneres allda noch aufbewahrt:

a. Alle Körper, die man fälschlich für Meteoriten angesehen hat, und wovon ein großer Teil die meisten der öffentlichen Meteoriten-Sammlungen (namentlich jene in Berlin und London) verunziert; so erstens künstliche Eisenmassen, die man für Meteoreisen hielt, wie die von Groß-Kamsdorf‚ Magdeburg, Aachen, Cilly, Collina di Brianza, Florac, Oswego u. s. w. Zweitens verschiedene Naturkörper, die aus der Atmosphäre gefallen sind oder gefallen sein sollen, z. B. die in Brauneisenstein umgewandelten Schwefelkieslinsen von Sterlitamak im Gouvernement Orenburg, die den Kern von Hagelkörnern gebildet haben sollen; Stückchen von porösem Braun- oder Raseneisenstein, die man für das Residuum einer bei Löbau in der Oberlausitz niedergefallenen Sternschnuppe hält\footnote{Über diese jedenfalls merkwürdige Substanz siehe: Ficinus Beobachtung des Falles eines Meteorsteines bei Löbau in der k. sächs. Oberlausitz, am 18. Januar 1835, in Erdmanns und Schweiggers Journal für praktische Chemie. B. 5. S. 41.}; Fragmente von Kalkspat, die auf ein Schiff in den amerikanischen Gewässern niedergefallen sein sollen; das sogenannte Meteorpapier von Rauden in Kurland; die Rückstände des roten Schnees aus den Schweizer Alpen; Staub von dem Schlammregen von Udine; die vielbesprochenen Raseneisensteinkörner von Iwan in Ungarn u. s. w.

b. Mehrere rohe, durchschnittene, polierte und zum Teil auch geätzte Stücke des, durch leidige Unkenntnis, im Frischfeuer veränderten höchst merkwürdigen Meteoreisens von Bitburg bei Trier.

c. Das vorgebliche tellurische Eisen von Canaan im Staate Connecticut, und ein damit scheinbar identisches schwarzes grafithaltiges Eisen, angeblich aus Kamtschatka.

d. Stücke von Roheisen, Eisenschlacken und andere Schmelzprodukte; auch Blitzröhren und vom Blitz getroffene Felssteine zum Vergleiche.

e. Einige Mineralien, die Ähnlichkeit mit Meteorsteinen besitzen (Dolerit, Basalt, Lava, Obsidianporphyr, Trachyt), oder an Widmanstättensche Figuren erinnernde Streifen nach drei Richtungen wahrnehmen lassen. (Magneteisenstein, Eisenglanz, Korund, Kalkspat.)

f. Einige Meteorsteine, namentlich von Stannern, mit welchen durch Herrn Hofrath und Direktor von Schreibers verschiedenartige Schmelzversuche im Porzellanfeuer, mit Brennspiegeln und mit elektrischen Strömungen angestellt worden sind, dann andere Meteorsteine, die längere Zeit in der Erde vergraben waren, um die Verwitterbarkeit derselben zu untersuchen (da es sonderbar ist, dass man noch nie Meteorsteine fand, deren Niederfallen nicht beobachtet worden wäre.)

g. Verarbeitetes (geschmiedetes, geschweißtes, in Bleche ausgezogenes....) Meteoreisen, namentlich von Agram, Elbogen und vom Kap der guten Hoffnung; technische Versuche, von Herrn v. Widmannstätten ausgeführt.

h. Die Original-Urkunde in lateinischer Sprache, ddo. 6. Juli 1751 mit Siegel und einer gleichzeitigen deutschen Übersetzung über das Niederfallen der zwei Meteoreisenmassen von Agram.

i. Eine Anzahl Gipsabgüsse von antiken Münzen, worauf Meteoriten oder geheiligte Steine (Bätylien, Cerauniten) abgebildet sind.

k. Einzelne Gipsabdrücke von Meteoriten, namentlich von dem durch seine Gestalt und die Eindrücke an der Oberfläche sehr merkwürdigen Meteorstein, gefallen im Jahre 1837 zu Groß-Divina bei Budetin in Ungarn, nun im National-Museum zu Pesth; eine Ergänzung in Gips von unserer Elbogner Eisenmasse, und Modelle von den Meteorsteinen von Tipperary und Wessely.

l. Eine große Anzahl von ausgezeichnet schönen in Farben ausgeführten Zeichnungen, dann viele höchst getreue Bleistift-Zeichnungen von Meteoriten, vorzüglich von Widmanstättenschen Figuren der Meteoreisenmassen; ferneres Autographen von geätztem Meteoreisen; Lithographien und Kupferstiche, Meteorsteine darstellend, endlich zwei Situationspläne der Gegenden von Stannern und Lissa, mit Bezeichnung der Punkte, auf welchen die daselbst aus der Luft gefallenen Steine aufgefunden worden sind.

(Eine fast vollständige kleine Sammlung von allen über Meteoriten erschienenen Werken und Abhandlungen befindet: sich unter den Büchern des k. k. Mineralien-Kabinetts.)

4. Von Meteoriten, deren Vorhandensein im Besitz von öffentlichen Sammlungen und Anstalten oder von Privaten bekannt ist, fehlen dem k. k. Mineralien-Kabinette (zum Teil ungeachtet vieler Bemühungen) noch folgende Lokalitäten:
\subsection{Meteorsteine.}
\paragraph{}
a. (Fallzeit) 1668. Verona (Vago). Obwohl das Gewicht der allda gefallenen 2 oder 3 großen Steine 500 Pfund betrug, ist jetzt von denselben doch nur noch ein ganz kleines Stückchen bekannt, dass sich in der Sammlung des verstorbenen Chemikers Laugier in Paris befand. Chladni, der es im Jahre 1818 sah, fand es den Steinen von Tabor und Barbotan ähnlich. (Es ist uns Hoffnung gemacht worden, dass in Verona doch noch ein Stück ausgemittelt werden dürfte.)

b. 1715. Garz (Schellin) bei Stargard in Pommern. Von diesem erst im Jahre 1822 (in Gilberts Annalen B. 71. S. 213) bekannt gemachten Steinfall sind gegenwärtig nur noch 2 oder 3 Stücke vorhanden; eines bei einen Gutsbesitzer in Pommern, ein anderes sehr kleines von 91 Gran in der k. Mineralien-Sammlung zu Berlin. (Wir sahen es da und fanden es den Steinen von Apt, Berlanguillas u. s. w. ähnlich.) Das kleine Fragment, das der verstorbene Prof. Gilbert in Leipzig besaß, scheint verloren gegangen zu sein.

c. 1815. Duralla bei Lodiana in Ostindien. Der allda gefallene Stein von 25 Pfunden, dem die Braminen große Verehrung bezeugten und einen eigenen Tempel bauen lassen wollten, wird im Hause der ostindischen Compagnie in London aufbewahrt, und davon auch nicht die Abtrennung eines winzigen Fragmentchens gestattet. Wir haben deshalb fruchtlose Schritte gemacht. Er soll zu der Abteilung der eisenhältigen Meteorsteine gehören.

d. 1822. Angers, Dépt. Maine et Loire, Frankreich. Wir sahen von diesem, hinsichtlich seiner Masse unbeträchtlichen Steinfall ein Stück im k. Museum der Naturgeschichte zu Paris. Der Stein gleicht dem von Vouillé.

e. 1822. Rourpoor bei Futichpore in Doab, Ostindien. Von dem unergiebigen Steinfall (die Steine wogen zusammen nur einige Pfunde) scheint nichts nach Europa gekommen zu sein.

f. 1824. Tounkin, Gouv. Irkutzk, Sibirien. Der 5 Pfund schwere Stein befand sich vor mehreren Jahren bei einem Gouverneur in Sibirien. Ein ganz kleiner Splitter, den Doktor Fiedler in Dresden auf seiner Reise durch Sibirien davon erhielt, befindet sich jetzt in der Sammlung des Freiherrn von Reichenbach in Wien.

g. 1825. Oriang in Malwate, Ostindien. Der Stein, der durch seinen Fall einen Mann tötete und eine Frau stark beschädigte, ist wohl nicht nach Europa gebracht worden.

h. 1827. Mhow im Distrikt Azim-Gesh; Ostindien. Es fielen einige Steine, wovon ein abgesprungenes Bruchstück ebenfalls einen Menschen tötete; sie sind wohl ebenfalls nicht nach Europa gekommen.

i. 1829. Deal im Staate New-Jersey, Nordamerika. Es fielen mehrere Steine, von denen wir durch die Gefälligkeit der Herren Silliman und Shepard etwas zu erhalten hoffen.

k. 1830. Launton bei Bicester, Oxfordshire, England. Es fiel nur Ein Stein von 2 Pfund 5 Loth, jetzt im Besitz eines Geistlichen in England.

l. 1834. Charwallas bei Hissar, Ostindien. Es fiel ebenfalls nur Ein, wenige Pfunde schwerer Stein, von dem wohl gleichfalls keine Fragmente nach Europa gekommen sein mögen.

m. 1837. Esnaude, Dépt. Charente inferieure. Von dem einzeln gefallenen Steine von 3 Pfunden sind Stücke an das naturhistorische Museum zu Bordeaux geschickt worden.\footnote{Ein drei Loth schweres Fragment des Meteorsteines von Esnaude hat das k. k. Mineralien-Kabinett während des Druckes des vorliegenden Kataloges durch H. Marguier erhalten.}

n. 1839. Little-Piney, im Staate Missouri, Nordamerika. Es wurde nur Ein faustgroßer Stein gefunden, den man zertrümmerte, Ein Fragment davon ist im britischen Museum zu London, andere in ein paar amerikanischen Sammlungen. Auch unsere Sammlung hat Hoffnung, davon ein Fragment zu erhalten.

o. 1840. Kirgisen-Steppe, am Flüsse Karokol, Asien. Der einzeln gefallene Stein von fast 6 Zoll Länge befindet sich im Museum der naturforschenden Gesellschaft zu Moskau.
\subsection{Meteoreisen.}
\begin{center}
(Die Zeit des Niederfallens unbekannt.)
\end{center}
\paragraph{}
r. Auf dem Alasej'schen Bergrücken in Sibirien, der das Flusssystem des Alasej von dem des Indigirka trennt, findet man in Menge gediegenes Eisen von vorzüglicher Güte, dass nur Meteoreisen sein kann, und von den Jakuten zu Messern, Beilen u.dgl. verarbeitet wird, (Siehe Wrangels Reise längs der Nordküste von Sibirien und auf dem Eismeere. 1. Band. Seite 175. Berlin. 1839.) Es scheint davon noch nichts nach Europa gekommen zu Sein.

s. In der Petropawlowsker Goldseife, Gouv. Omsk, Sibirien, hat man früher kleinere Stücke gediegenen nickelhaltigen Eisens, die man nicht beachtete, und erst kürzlich ein größeres 17 1/2 Pfund schweres Stück gefunden. (Siehe Erdmann Archiv für wissenschaftliche Kunde von Russland. 1841. 1. S. 314-320.) Wir hoffen, durch gütige Vermittlung des Herrn Generals von Tscheffkin, Chef des Stabes des k. russischen Bergingenieur-Corps, an den wir uns gewendet haben, davon etwas zu erhalten.

t. Wie wir bereits bei dem Meteoreisen vom Kap der guten Hoffnung (Nr. 91 des gegenwärtigen Verzeichnisses) bemerkten, ist es noch ungewiss, ob das früher zwischen dem Sonntags- und Boschmanns-Flüsse gefundene Meteoreisen mit den am großen Fischflüsse weit umher gestreuten Eisenmassen, die durch Capitan Alexander bekannt wurden, von ein und demselben Ereignis herrühren. Es wäre wünschenswert, das von C. Alexander entdeckte Eisen untersuchen zu können. Ob es schon nach Europa gebracht worden, ist uns nicht bekannt.

u. Das Meteoreisen von Santa Rosa oder Tocavita in Neu-Granada sahen wir in der k. Mineralien-Sammlung zu Berlin, wohin es durch Herrn Alexander von Humboldt kam, und fanden es dem bei Rasgatà aufgefundenen ganz ähnlich. Es ist nicht unwahrscheinlich, dass die Eisenmassen dieser zwei Lokalitäten, die voneinander nicht sehr entfernt sind, von Einem Meteore herrühren, das verschiedene Entladungen machte.

v. Ob das Meteoreisen aus der Sierra blanca unweit Villa nueva de Huaxuquilla in Mexiko (siehe Chladni S. 339.) und

w. dass von Charcas in Mexico (Chladni S. 337) mit anderen bekannten mexikanischen Lokalitäten (das erstere etwa mit Toluca, das andere mit Zacatecas oder eigentlich San Jose del Sitio) zusammenfallen dürften, ist noch ungewiss, und wird erst durch mineralogische und chemische Untersuchungen ermittelt werden können. Ein Stück Eisen mit dem angeblichen Fundorte Sierra blanca, in der, für die k. Mineralien-Sammlung in Berlin angekauften Bergemann'schen Mineralien-Sammlung fanden wir rücksichtlich der Widmannstätti'schen Figuren dem Eisen von Durango im Wiener kais. Mineralien-Kabinette ähnlich.
\clearpage
\section{Die spezifischen Gewichte der im k. k. Mineralien-Kabinette vorhandenen Meteoriten.}
\begin{center}
    \begin{longtable}{|p{7mm}|p{32mm}|p{30mm}|p{30mm}|}
    \hline
        Nr. & ~  & Wiegung im k. k. Mineralien-Kabinette durch Herrn C. Rumler. & Andere Wiegungen. \\ \hline
         ~ & \textbf{1. Meteorsteine.} & ~  & ~  \\ \hline
        1 & Alais & 1,70. & 1,94. Biot. \\ \hline
        2 & Simonod & 1,35. &   \\ \hline
        3 & Kapland & 2,69. & 2,94. Faraday. \\ \hline
        4 & Chassigny & 3,55. & 3,55. Schreibers. \\ \hline
        5 & Juvenas & 3,11. & 3,09. Flangergues. \\ \hline
        6 & Stannern & 3,01 bis 3,17. & 2,95 bis 3,16. Schreibers. - 3,19. Vauquelin. \\ \hline
        7 & Konstantinopel & 3,17. &   \\ \hline
        8 & Jonzac & 3,07 bis 3,08. & 3,12. Fleuriau de Bellevue. \\ \hline
        9 & Bialistock & 3,17. &   \\ \hline
        10 & Lontalax & 3,07. &   \\ \hline
        11 & Nobleborough & 3,09. &   \\ \hline
        12 & Massing & 3,21. & 3,36. Imhof. \\ \hline
          & eine Kugel daraus & 3,26. &   \\ \hline
        13 & Parma & 3,39 bis 3,40. & 3,3. bis 3,4. Guidotti. \\ \hline
        14 & Siena & 3,39 bis 3,40. & 3,41. Bournon. - 3,34. bis 3,40. Klaproth. - 3,35. Schreibers. \\ \hline
        15 & Ensisheim & 3,48. & 3,23. Barthold. - 3,48. bis 3,50. Schreibers. \\ \hline
        16 & L’Aigle & 3,39 bis 3,47. & 3,49. Schreibers. \\ \hline
          & Eisenkorn daraus & 7,08. &   \\ \hline
          & Eisenplättchen daraus & …. & 6,04. Schreibers \\ \hline
        17 & Liponas & 3,66. &   \\ \hline
        18 & Chantonnay &   &   \\ \hline
          & der lichte Teil & 3,46. & 3,44. bis 3,49. Schreibers. \\ \hline
          & der schwarze Teil & 3,48. &   \\ \hline
        19 & Renazzo & 3,24 bis 3,28. &   \\ \hline
        20 & Richmond & 3,37. & 3,29. bis 3,31. Shepard. \\ \hline
        21 & Weston & 3,47 bis 3,58. & 3,3. Warden. - 3,6. Silliman. \\ \hline
        22 & La Baffe & 3,66. &   \\ \hline
        23 & Benares & 3,36. & 3,35. Bournon. \\ \hline
          & Kugel aus demselb. & 3,04. &   \\ \hline
        24 & Gouv. Poltawa & 3,33. &   \\ \hline
        25 & Krasno-Ugol & 3,49. &   \\ \hline
        26 & Erxleben & 3,64. & 3,60. Klaproth. - 3,61. Stromayer. \\ \hline
        27 & Gouv. Simbirsk & 3,51 bis 3,55. &   \\ \hline
        28 & Mauerkirchen & 3,45. & 3,45. Imhof. - 3,5. Schreibers. \\ \hline
        29 & Nashville & 3,58. & 3,4. Seybert. \\ \hline
        30 & Lucé & 3,47. & 3,53. Lavoisier u. Cadet. \\ \hline
        31 & Lissa & 3,50. & 3,56. Reuß. \\ \hline
        32 & Owahu & 3,39. &   \\ \hline
        33 & Charkow & 3,49. &   \\ \hline
        34 & Zaborzika & 3,40. &   \\ \hline
        35 & Bachmut & 3,42. &   \\ \hline
        36 & Politz & 3,37. & 3,49. Stromayer. \\ \hline
        37 & Kuleschofka & 3,49. &   \\ \hline
        38 & Slobodka & 3,47. &   \\ \hline
        39 & Milena & 3,54. &   \\ \hline
        40 & Forsyth & 3,45 bis 3,48. & 3,37. Shepard. \\ \hline
        41 & Yorkshire &   & 3,58. Bournon. \\ \hline
          &   & 3,55. &   \\ \hline
          &   & 3,88. &   \\ \hline
          &   & 3,95. &   \\ \hline
          &   & 4,02. &   \\ \hline
        42 & Glasgow & 3,53. &   \\ \hline
        43 & Berlanguillas & 3,49. &   \\ \hline
        44 & Apt & 3,48. &   \\ \hline
        45 & Vouillé & 3,55. &   \\ \hline
        46 & Château-Renard & 3,54. & 3,56. Dufrenoy. \\ \hline
          & Eisenkorner daraus & …. & 6,48. Dufrenoy. \\ \hline
        47 & Salés & 3,47. &   \\ \hline
        48 & Agen & 3,59 bis 3,62. &   \\ \hline
        49 & Nanjemoy & 3,66. & 3,66. Chilton. \\ \hline
        50 & Asco & 3,66. &   \\ \hline
        51 & Toulouse & 3,73. & 3,66. bis 3,70. Bigot de Morogues. \\ \hline
        52 & Blansko & 3,70. &   \\ \hline
        53 & Wessely & 3,70. & 3,66. bis 3,68. Schreibers. \\ \hline
        54 & Limerick & 3,65. & 3,62. bis 4,23. Apjohn. \\ \hline
        55 & Grüneberg & 3,72. & 3,69. bis 3,73. Weinmann. \\ \hline
        56 & Tipperary & 3,64. & 3,67. Higgins. \\ \hline
        57 & Gouv. Kursk & 3,55. &   \\ \hline
        58 & Lixna & 3,66. & 3,76. Grotthuss. \\ \hline
        59 & Tabor & 3,65. & 3,66. Schreibers. - 4,28. Bournon. \\ \hline
        60 & Charsonville &   & 3,36. bis 3,67. Bigot de Morogues. - 3,71. Hauy. - 3,57. bis 3,65. Schreibers. \\ \hline
          &   & 3,64. &   \\ \hline
          &   & 3,71. &   \\ \hline
          &   & 3,75. &   \\ \hline
        61 & Doroninsk & 3,63. &   \\ \hline
        62 & Seres & 3,71. & 3,60. John. \\ \hline
        63 & Sigena & 3,63. &   \\ \hline
        64 & Barbotan & 3,62. &   \\ \hline
        65 & Eichstädt & 3,60. & 3,70. Schreibers. \\ \hline
        66 & Groß-Divina & 3,55 bis 3,56. &   \\ \hline
        67 & Zebrak & 3,60. & 3,6. Zippe. \\ \hline
        68 & Timochin & 3,60. & 3,7. Klaproth. \\ \hline
        69 & Macao & 3,72 bis 3,74. &   \\ \hline
          & 2. Meteoreisen. &   &   \\ \hline
        70 & Atacama & 7,44 bis 7,66. & 6,68. Turner. \\ \hline
          & der Olivin daraus Eisen angeblich aus Potosi & 3,33. & 7,73. Morreu. \\ \hline
        71 & Krasnojarsk &   & 6,48. Howard und Bournon. - 7,54. bis 7,70. Schreibers. \\ \hline
         ~ &   & 7,16. bis 7,42. &   \\ \hline
         ~ &   & 7,66. &   \\ \hline
         ~ &   & 7,78. bis 7,84. &   \\ \hline
        ~ & der Olivin daraus & 3,43. & 3,26. bis 3,30. Howard und Bournon. - 3,34. Stromayer. \\ \hline
        72 & Brachin & 7,58. & 6,2. Drzewinski. \\ \hline
        73 & Sachsen &   & 6,14. Howard und Bournon. \\ \hline
         ~ & angeblich Tabor aus Klaproths Samml. Geatzt & 7,50. &   \\ \hline
         ~ & angeblich Norwegen; nicht ganz rein, obwohl geatzt & 6,86. &   \\ \hline
        ~  & das olivinartige Mineral daraus & 3,23. & 3,27. Stromayer. \\ \hline
        74 & Bitburg & 6,52. & 6,14. Steininger. \\ \hline
        75 & Toluca & 7,72. & 7,60. bis 7,67. Schreibers. \\ \hline
        76 & Elbogen & 7,74. & 7,2. bis 7,3. Neumann. - 7,76. Mohs. - 7,78. Wehrle. - 7,80. bis 7,83. Schreibers. \\ \hline
         ~ & geglüht, wodurch der Magnetkies zerstört wurde & 7,87. & ~  \\ \hline
        77 & Agram & 7,82. & 7,73. bis 7,80. Schreibers. - 7,78. Wehrle. \\ \hline
        78 & Lenarto & 7,73. & 7,72. bis 7,80. Schreibers. - 7,79. Wehrle. \\ \hline
        79 & Red-River & 7,82. & 7,4. Gibbs. \\ \hline
        80 & Durango & 7,88. & ~ \\ \hline
        81 & Guilford & 7,67. &  ~ \\ \hline
        82 & Caille & 7,64. & ~  \\ \hline
        83 & Ashville & 7,90. & 6,5. bis 8,0. Shepard. \\ \hline
        84 & Tennessee & 7,26. & ~  \\ \hline
        85 & Bohumilitz & 7,61 bis 7,71. & 7,14. Steinmann. \\ \hline
          & Magnetkies daraus & 4,62. & ~  \\ \hline
        86 & Bahia & 7,48. & 7,73. Spix und Martius. \\ \hline
        87 & Zacatecas & 7,55. & 7,5. Burkart. - 7,2. bis 7,6. Sonneschmidt bei Chladni. \\ \hline
        88 & Rasgatà & 7,33 bis 7,77. & 7,6. Rivero und Bousingault. \\ \hline
        89 & Tucuman &   & 7,60. bis 7,65. Schreibers. - 7,64. Widmannstatten. \\ \hline
         ~ &  ~ & 7,54. & ~  \\ \hline
         ~ & ~  & 7,56. & ~  \\ \hline
         ~ & ~  & 7,60. & ~  \\ \hline
        90 & Senegal & 7,72. &   \\ \hline
        91 & Kap der guten Hoffnung & ~ & 6,65. bis 6,92. die dunkleren Teile. Widmannstatten. - 7,63. bis 7,87. die lichteren Teile. Widmannstatten. - 7,60. Van Marum. - 7,66. Wehrle. - 7,70. Dankelmann. \\ \hline
         ~ &  ~ & 6,63. &  ~ \\ \hline
         ~~ & ~  & 6,92. & ~  \\ \hline
         ~ & ~  & 6,99. & ~  \\ \hline
         ~ & ~  & 7,40. & ~  \\ \hline
         ~ & ~  & 7,59. & ~  \\ \hline
        ~  & ~  & 7,74. & ~  \\ \hline
         ~ & ~  & 7,76. & ~  \\ \hline
       ~   & ~  & 7,94. & ~  \\ \hline
        92 & Clairborne & 6,82. & 5,7., 6,0., 6,5. Verschiedene Teile dieses Eisens. Jackson. \\ \hline
         ~ & Anhang. &  ~ &  ~ \\ \hline
        93 & Oaxaca (etwas gehämmert) & 7,38. &   \\ \hline
        94 & Grönland (stark gehämmert) & 7,23. & \\ \hline
    \end{longtable}
\end{center}
\clearpage
\section{Schätzung der Meteoriten im k. k. Mineralien-Kabinette.}
\begin{center}
    \begin{longtable}{|l|l|l|}
    \hline
        Nr. &   & Werth in Conv. Mze. fl. \\ \hline
          & 1. Meteorsteine. &   \\ \hline
        1 & Alais Nr. 1 & 6 \\ \hline
        2 & Simonod Nr. 1 & 6 \\ \hline
        3 & Kapland Nr. 1 & 5 \\ \hline
        4 & Chassigny Nr. 1 & 20 \\ \hline
        5 & Chassigny Nr. 2 & 14 \\ \hline
          & Juvenas Nr. 1 & 90 \\ \hline
          & Juvenas Nr. 2 & 25 \\ \hline
          & Juvenas Nr. 3 & 15 \\ \hline
          & Juvenas Nr. 4 & 10 \\ \hline
        6 & Stannern Nr. 1 & 500 \\ \hline
          & Stannern Nr. 2 & 214 \\ \hline
          & Stannern Nr. 3 & 153 \\ \hline
          & Stannern Nr. 4 & 87 \\ \hline
          & Stannern Nr. 5 & 78 \\ \hline
          & Stannern Nr. 6 & 66 \\ \hline
          & Stannern Nr. 7 & 64 \\ \hline
          & Stannern Nr. 8 & 46 \\ \hline
          & Stannern Nr. 9 & 40 \\ \hline
          & Stannern Nr. 10 & 30 \\ \hline
          & Stannern Nr. 11 & 22 \\ \hline
          & Stannern Nr. 12 & 21 \\ \hline
          & Stannern Nr. 13 & 13 \\ \hline
          & Stannern Nr. 14 & 13 \\ \hline
          & Stannern Nr. 15 & 11 \\ \hline
          & Stannern Nr. 16a & 10 \\ \hline
          & Stannern Nr. 16b & 9 \\ \hline
          & Stannern Nr. 17 & 9 \\ \hline
          & Stannern Nr. 18 & 8 \\ \hline
          & Stannern Nr. 19 & 3 \\ \hline
          & Stannern Nr. 20 & 3 \\ \hline
          & Stannern Nr. 21 & 27 \\ \hline
          & Stannern Nr. 22 & 22 \\ \hline
          & Stannern Nr. 23 & 15 \\ \hline
          & Stannern Nr. 24 & 13 \\ \hline
          & Stannern Nr. 25 & 12 \\ \hline
          & Stannern Nr. 26 & 9 \\ \hline
          & Stannern Nr. 27 & 7 \\ \hline
          & Stannern Nr. 28 & 7 \\ \hline
          & Stannern Nr. 29 & 6 \\ \hline
          & Stannern Nr. 30 & 5 \\ \hline
          & Stannern Nr. 31 & 2 \\ \hline
          & Stannern Nr. 32 & 2 \\ \hline
          & Stannern Nr. 33 & 3 \\ \hline
        7 & Konstantinopel Nr. 1 & 10 \\ \hline
        8 & Jonzac Nr. 1 & 120 \\ \hline
          & Jonzac Nr. 2 & 21 \\ \hline
        9 & Bialistock Nr. 1 & 20 \\ \hline
        10 & Lontalax Nr. 1 & 10 \\ \hline
        11 & Nobleborough Nr. 1 & 10 \\ \hline
        12 & Massing Nr. 1a & 6 \\ \hline
          & Massing Nr. 1b & 6 \\ \hline
        13 & Parma Nr. 1 & 21 \\ \hline
          & Parma Nr. 2 & 6 \\ \hline
        14 & Siena Nr. 1 & 5 \\ \hline
          & Siena Nr. 2 & 5 \\ \hline
          & Siena Nr. 3 & 5 \\ \hline
          & Siena Nr. 4 & 5 \\ \hline
          & Siena Nr. 5 & 10 \\ \hline
          & Siena Nr. 6 & 10 \\ \hline
          & Siena Nr. 7 & 30 \\ \hline
        15 & Ensisheim Nr. 1 & 100 \\ \hline
          & Ensisheim Nr. 2 & 25 \\ \hline
          & Ensisheim Nr. 3 & 20 \\ \hline
          & Ensisheim Nr. 4 & 14 \\ \hline
          & Ensisheim Nr. 5 & 8 \\ \hline
        16 & L’Aigle Nr. 1 & 171 \\ \hline
          & L’Aigle Nr. 2 & 125 \\ \hline
          & L’Aigle Nr. 3 & 40 \\ \hline
          & L’Aigle Nr. 4 & 4 \\ \hline
          & L’Aigle Nr. 5 & 17 \\ \hline
          & L’Aigle Nr. 6 & 14 \\ \hline
          & L’Aigle Nr. 7 & 7 \\ \hline
          & L’Aigle Nr. 8a-b & 4 ½ \\ \hline
          & L’Aigle Nr. 9a-b & 4 \\ \hline
          & L’Aigle Nr. 10 & 2 \\ \hline
          & L’Aigle Nr. 11 & 28 \\ \hline
        17 & Liponas en Bresse Nr. 1 & 26 \\ \hline
          & Liponas en Bresse Nr. 2 & 6 \\ \hline
        18 & Chantonnay Nr. 1 & 300 \\ \hline
          & Chantonnay Nr. 2 & 21 \\ \hline
          & Chantonnay Nr. 3 & 36 \\ \hline
          & Chantonnay Nr. 4 & 8 \\ \hline
        19 & Renazzo Nr. 1 & 30 \\ \hline
          & Renazzo Nr. 2 & 6 \\ \hline
        20 & Richmond Nr. 1 & 36 \\ \hline
          & Richmond Nr. 2 & 30 \\ \hline
          & Richmond Nr. 3 & 7 \\ \hline
        21 & Weston Nr. 1 & 15 \\ \hline
          & Weston Nr. 2 & 13 \\ \hline
          & Weston Nr. 3 & 13 \\ \hline
          & Weston Nr. 4 & 8 \\ \hline
          & Weston Nr. 5 & 6 \\ \hline
        22 & La Baffe Nr. 1 & 12 \\ \hline
        23 & Benares Nr. 1 & 160 \\ \hline
          & Benares Nr. 2 & 25 \\ \hline
          & Benares Nr. 3 & 6 \\ \hline
        24 & Gouv. Poltawa Nr. 1 & 32 \\ \hline
        25 & Krasno-Ugol Nr. 1 & 10 \\ \hline
        26 & Erxleben Nr. 1 & 18 \\ \hline
        27 & Gouv. Simbirsk Nr. 1 & 15 \\ \hline
        28 & Mauerkirchen Nr. 1 & 120 \\ \hline
          & Mauerkirchen Nr. 2 & 50 \\ \hline
        29 & Nashville Nr. 1 & 20 \\ \hline
        30 & Lucé Nr. 1 & 8 \\ \hline
          & Lucé Nr. 2 & 48 \\ \hline
          & Lucé Nr. 3 & 4 \\ \hline
        31 & Lissa Nr. 1 & 500 \\ \hline
          & Lissa Nr. 2 & 15 \\ \hline
          & Lissa Nr. 3 & 13 \\ \hline
        32 & Owahu Nr. 1 & 15 \\ \hline
          & Owahu Nr. 2 & 20 \\ \hline
        33 & Charkow Nr. 1 & 6 \\ \hline
        34 & Zaborczika Nr. 1 & 6 \\ \hline
        35 & Bachmut Nr. 1 & 10 \\ \hline
        36 & Politz Nr. 1 & 13 \\ \hline
          & Politz Nr. 2 & 7 \\ \hline
          & Politz Nr. 3 & 7 \\ \hline
        37 & Kuleschofka Nr. 1 & 54 \\ \hline
          & Kuleschofka Nr. 2 & 18 \\ \hline
        38 & Slobodka Nr. 1 & 40 \\ \hline
          & Slobodka Nr. 2 & 15 \\ \hline
          & Slobodka Nr. 3 & 10 \\ \hline
        39 & Milena Nr. 1 & 66 \\ \hline
        40 & Forsyth Nr. 1 & 20 \\ \hline
          & Forsyth Nr. 2 & 15 \\ \hline
        41 & Yorkshire Nr. 1 & 20 \\ \hline
        42 & Glasgow Nr. 1 & 15 \\ \hline
        43 & Berlanguillas Nr. 1 & 66 \\ \hline
        44 & Apt Nr. 1 & 96 \\ \hline
          & Apt Nr. 2 & 12 \\ \hline
        45 & Vouillé Nr. 1 & 35 \\ \hline
        46 & Château-Renard Nr. 1 & 74 \\ \hline
          & Château-Renard Nr. 2 & 21 \\ \hline
          & Château-Renard Nr. 3 & 10 \\ \hline
        47 & Salés Nr. 1 & 84 \\ \hline
          & Salés Nr. 2 & 15 \\ \hline
        48 & Agen Nr. 1 & 22 \\ \hline
          & Agen Nr. 2 & 35 \\ \hline
        49 & Nanjemoy Nr. 1 & 126 \\ \hline
        50 & Asco Nr. 1 & 15 \\ \hline
        51 & Toulouse Nr. 1 & 15 \\ \hline
        52 & Blansko Nr. 1a & 40 \\ \hline
          & Blansko Nr. 1b & gepaart mit über \\ \hline
        53 & Wessely Nr. 1 & 600 \\ \hline
          & Wessely Nr. 2a & 10 \\ \hline
          & Wessely Nr. 2b & gepaart mit über \\ \hline
        54 & Limerick Nr. 1 & 20 \\ \hline
          & Limerick Nr. 2 & 16 \\ \hline
          & Limerick Nr. 3 & 12 \\ \hline
        55 & Grüneberg Nr. 1 & 10 \\ \hline
        56 & Tipperary Nr. 1 & 87 \\ \hline
          & Tipperary Nr. 2 & 10 \\ \hline
        57 & Gouv. Kursk Nr. 1 & 6 \\ \hline
        58 & Lixna Nr. 1 & 84 \\ \hline
        59 & Tabor Nr. 1 & 500 \\ \hline
          & Tabor Nr. 2 & 105 \\ \hline
          & Tabor Nr. 3 & 15 \\ \hline
          & Tabor Nr. 4 & 93 \\ \hline
          & Tabor Nr. 5 & 12 \\ \hline
          & Tabor Nr. 6 & 12 \\ \hline
          & Tabor Nr. 7 & 12 \\ \hline
        60 & Charsonville Nr. 1 & 120 \\ \hline
          & Charsonville Nr. 2 & 16 \\ \hline
        61 & Doroninsk Nr. 1 & 20 \\ \hline
        62 & Seres Nr. 1 & 40 \\ \hline
          & Seres Nr. 2 & 20 \\ \hline
        63 & Sigena Nr. 1 & 8 \\ \hline
        64 & Barbotan Nr. 1 & 59 \\ \hline
          & Barbotan Nr. 2 & 47 \\ \hline
        65 & Eichstädt Nr. 1 & 42 \\ \hline
          & Eichstädt Nr. 2 & 6 \\ \hline
        66 & Groß-Divina Nr. 1 & 30 \\ \hline
        67 & Zebrak Nr. 1 & 120 \\ \hline
        68 & Timochin Nr. 1 & 28 \\ \hline
          & Timochin Nr. 2 & 20 \\ \hline
        69 & Macao Nr. 1 & 18 \\ \hline
          & Macao Nr. 2 & 30 \\ \hline
          & Macao Nr. 3 & 40 \\ \hline
          & Macao Nr. 4 & 45 \\ \hline
          & Macao Nr. 5 & 12 \\ \hline
          & Macao Nr. 6 & 10 \\ \hline
          & Macao Nr. 7 & 6 \\ \hline
        178 & Nummern Meteorsteine & 7585 1/2 \\ \hline
          & 2. Meteoreisen. &   \\ \hline
        70 & Atacama Nr. 1 & 506 \\ \hline
          & Atacama Nr. 2 & 132 \\ \hline
          & Atacama Nr. 3 & 24 \\ \hline
        71 & Krasnojarsk Nr. 1 & 400 \\ \hline
          & Krasnojarsk Nr. 2 & 150 \\ \hline
          & Krasnojarsk Nr. 3 & 48 \\ \hline
          & Krasnojarsk Nr. 4 & 34 \\ \hline
          & Krasnojarsk Nr. 5 & 27 \\ \hline
          & Krasnojarsk Nr. 6 & 7 \\ \hline
        72 & Brahin Nr. 1 & 15 \\ \hline
        73 & Sachsen Nr. 1a & 10 \\ \hline
          & Sachsen Nr. 1b & 10 \\ \hline
          & Sachsen Nr. 1c & 14 \\ \hline
          & Sachsen Nr. 1d & 20 \\ \hline
          & Sachsen Nr. 2 & 300 \\ \hline
          & Sachsen Nr. 3 & 150 \\ \hline
        74 & Bitburg Nr. 1 & 20 \\ \hline
          & Bitburg Nr. 2 & 50 \\ \hline
        75 & Toluca Nr. 1 & 45 \\ \hline
        76 & Elbogen Nr. 1 & 10000 \\ \hline
          & Elbogen Nr. 2 & 24 \\ \hline
          & Elbogen Nr. 3a & 10 \\ \hline
          & Elbogen Nr. 3b & 5 \\ \hline
          & Elbogen Nr. 3c & 8 \\ \hline
          & Elbogen Nr. 3d & gepaart mit über \\ \hline
          & Elbogen Nr. 4a & 10 \\ \hline
          & Elbogen Nr. 4b & 5 \\ \hline
          & Elbogen Nr. 4c & 5 \\ \hline
        77 & Agram Nr. 1 & 10000 \\ \hline
          & Agram Nr. 2 & 50 \\ \hline
          & Agram Nr. 3 & 25 \\ \hline
          & Agram Nr. 4 & 25 \\ \hline
          & Agram Nr. 5a & 15 \\ \hline
          & Agram Nr. 5b & 10 \\ \hline
          & Agram Nr. 5c & 9 \\ \hline
          & Agram Nr. 5d & gepaart mit über \\ \hline
        78 & Lenarto Nr. 1 & 500 \\ \hline
          & Lenarto Nr. 2 & 25 \\ \hline
          & Lenarto Nr. 3 & 20 \\ \hline
          & Lenarto Nr. 4 & 10 \\ \hline
          & Lenarto Nr. 5 & 12 \\ \hline
          & Lenarto Nr. 6 & 15 \\ \hline
          & Lenarto Nr. 7a & 20 \\ \hline
          & Lenarto Nr. 7b & gepaart mit über \\ \hline
        79 & Red-River Nr. 1 & 150 \\ \hline
          & Red-River Nr. 2 & 50 \\ \hline
          & Red-River Nr. 3 & 14 \\ \hline
          & Red-River Nr. 4 & 3 \\ \hline
        80 & Durango Nr. 1 & 128 \\ \hline
          & Durango Nr. 2 & 44 \\ \hline
          & Durango Nr. 3 & 10 \\ \hline
          & Durango Nr. 4a & 4 \\ \hline
          & Durango Nr. 4b & 2 \\ \hline
        81 & Guilford Nr. 1 & 20 \\ \hline
        82 & Caille Nr. 1 & 50 \\ \hline
          & Caille Nr. 2 & 30 \\ \hline
        83 & Ashville Nr. 1 & 90 \\ \hline
          & Ashville Nr. 2 & 8 \\ \hline
        84 & Tennessee Nr. 1 & 15 \\ \hline
        85 & Bohumilitz Nr. 1 & 500 \\ \hline
          & Bohumilitz Nr. 2 & 6 \\ \hline
          & Bohumilitz Nr. 3 & 18 \\ \hline
          & Bohumilitz Nr. 4a & 25 \\ \hline
          & Bohumilitz Nr. 4b & 8 \\ \hline
          & Bohumilitz Nr. 5 & 17 \\ \hline
        86 & Bahia Nr. 1 & 455 \\ \hline
          & Bahia Nr. 2 & 68 \\ \hline
          & Bahia Nr. 3 & 20 \\ \hline
          & Bahia Nr. 4 & 7 \\ \hline
          & Bahia Nr. 5 & 4 \\ \hline
          & Bahia Nr. 6 & 4 \\ \hline
        87 & Zacatecas Nr. 1 & 74 \\ \hline
          & Zacatecas Nr. 2 & 8 \\ \hline
          & Zacatecas Nr. 3 & 6 \\ \hline
          & Zacatecas Nr. 4 & 20 \\ \hline
        88 & Rasgatà Nr. 1 & 216 \\ \hline
          & Rasgatà Nr. 2 & 186 \\ \hline
          & Rasgatà Nr. 3 & 47 \\ \hline
          & Rasgatà Nr. 4 & 7 \\ \hline
          & Rasgatà Nr. 5 & 5 \\ \hline
        89 & Tucuman Nr. 1 & 83 \\ \hline
          & Tucuman Nr. 2 & 15 \\ \hline
          & Tucuman Nr. 3a & 6 \\ \hline
          & Tucuman Nr. 3b & 6 \\ \hline
          & Tucuman Nr. 3c & 1 \\ \hline
        90 & Senegal Nr. 1 & 71 \\ \hline
          & Senegal Nr. 2 & 16 \\ \hline
          & Senegal Nr. 3 & 22 \\ \hline
          & Senegal Nr. 4 & 5 \\ \hline
          & Senegal Nr. 5a & 20 \\ \hline
          & Senegal Nr. 5b & gepaart mit über \\ \hline
        91 & Vorgebirge der guten Hoffnung Nr. 1 & 200 \\ \hline
          & Vorgebirge der guten Hoffnung Nr. 2 & 55 \\ \hline
          & Vorgebirge der guten Hoffnung Nr. 3a & 4 \\ \hline
          & Vorgebirge der guten Hoffnung Nr. 3b & 3 \\ \hline
        92 & Clairborne Nr. 1 & 10 \\ \hline
        93 & Oaxaca Nr. 1 & 5 \\ \hline
        94 & Grönland Nr. 1 & 5 \\ \hline
        80 & Nummern Meteoreisen. & 25611 \\ \hline
          & Hiezu die 178 Meteorsteine. & 7585 1/2 \\ \hline
          & Also die gesamten 258 Meteoriten. & 33196 1/2 \\ \hline
    \end{longtable}
\end{center}
\clearpage
\section{Erklärung der Abbildung.}
\paragraph{}
Die auf der Abbildung dargestellten Widmanstättenschen Figuren (siehe die Anmerkung auf Seite 100 des vorliegenden Werkes) sind durch Ätzen der polierten Schnittfläche eines Stückes Meteoreisen von Lenarto in Ungarn erhalten worden. Es ist dies das größte Stück von dieser Lokalität in der Meteoriten-Sammlung des k. k. Mineralien-Kabinettes und in dem vorliegenden beschreibenden Verzeichnisse Seite 108 unter Nummer 1 angezeigt, Um die Zeichnung des Originals zu vervielfältigen, ist von der mit Salpetersäure geätzten Fläche desselben ein Gipsabguss gemacht und in diesen die Metall-Legierung (Blei, Zinn und Antimon) ausgegossen worden. Dadurch wurde eine dem Originale vollkommen ähnliche Platte gewonnen, und von dieser sodann die Abdrücke auf Papier abgezogen.
\clearpage
\section{Verwandtschafts-Tabelle der Meteoriten.}
\subsection{Meteorsteine.}
\paragraph{}
Erdige Meteoriten entweder ohne metallischen Eisen, oder, wenn dieses eingemengt ist, bestehen wenigstens 3/4 der Masse nicht aus metallischem Eisen.
\subsubsection{Anomale Meteorsteine.}
\paragraph{}
Gediegenes Eisen und Schwefeleisen sind darin entweder gar nicht vorhanden, oder in so geringer Menge, dass man sie in der gepulverten Substanz nur mittelst des Mikroskope zu entdecken vermag.

1. Alais. Bröcklige, leicht zu Pulver zerreibliche schwarze Masse, effloreszierend, im Wasser zu einem Brei zerfallend. Gediegenes Eisen und Schwefeleisen nur durch das Mikroskop erkennbar.

2. Simonod. Bröcklige und scharfkantige nicht leicht zerreibliche schwarze Masse, ohne gediegenen Eisen und ohne Magnetkies.

3. Kapland. Zusammenhängende aber weiche, schwarze Masse, durch den Strich Glanz erlangend, ohne sichtbares Schwefeleisen (obwohl der Stein nach der chemischen Untersuchung 4 p. c. Schwefel enthält) und ohne Metall. Eisen. (Die chem. Analyse hat darin weder Metall. Eisen noch Nickel gefunden.) Matte schwarze Rinde.

4. Chassigny. Zusammenhängende, körnige, gelblichgrüne Grundmasse ohne metallischen Eisen und ohne Schwefeleisen, aber mit kleinen schwarzen Körnern von Chromeisen gemengt. (Die Grundmasse besteht nur aus Einem, und zwar einem olivinartigen, in Säuren löslichen Mineral.) Matte schwarze Rinde.
\subsubsection{Normale Meteorsteine.}
\paragraph{}
Stets ist darin Schwefeleisen, in den meisten Fällen, nebst dem Schwefeleisen auch gediegenes Eisen als Gemengteil leicht zu unterscheiden (sicherer auf polierten Flächen und mittelst der Lupe)

\vspace{2ex}

1. Mit Schwefeleisen, aber ohne metallischen Eisen. Die schwarze Rinde pech- oder firnissartig glänzend. (In der chemischen Zusammensetzung herrscht die Talkerde nicht vor.)

\vspace{2ex}

\hspace*{10mm}a. Von erdigen Gemengteilen sind 2 Mineralien, ein augit- und ein labradorartiges, in körnigem Gemenge zu unterscheiden.

\hspace*{15mm}5. Juvenas. Die Masse enthält kleine Höhlungen; die beiden erdigen Mineralien stets in frischem Zustande.

\hspace*{15mm}6. Stannern. Die Masse ist ohne Höhlungen; von den beiden erdigen Gemengteilen ist der weiße meist nicht ganz frisch.

\hspace*{15mm}7. Konstantinopel. Die Masse ist ohne Höhlungen; von den beiden erdigen Gemengteilen ist der weiße meist nicht ganz frisch.

\hspace*{15mm}8. Jonzac. Die Masse ist ohne Höhlungen; von den beiden erdigen Gemengteilen ist der weiße meist nicht ganz frisch.

\vspace{2ex}

\hspace*{10mm}b. Von erdigen Gemengteilen ist außer den zwei bei a. erwähnten noch wenigstens ein dritter olivengrüner vorhanden; breccienartiges Aussehen.

\hspace*{15mm}9. Bialistock. leicht zerreiblich.

\hspace*{15mm}10. Lontalax. leicht zerreiblich.

\hspace*{15mm}11. Nobleborough. leicht zerreiblich.

\hspace*{15mm}12. Mässing. En treten in der Masse auch graulichgrüne Kügelchen auf.

\vspace{5ex}

2. Mit Schwefeleisen (Magnetkies?) und mit gediegenem Eisen. Die schwarze Rinde matt oder schwach schimmernd.

\vspace{2ex}

(Von erdigen Mineralien sind in Folge chemischer Untersuchung darin vorhanden: a. Ein olivinartiges in Säuren lösliches, aber unschmelzbares Mineral. b. Ein oder vielleicht zwei, ein augit- und ein leucit- oder feldspatartiges Mineral, in Säuren unlöslich, aber schmelzbar Silicate von Talkerde, Kalkerde, Eisenoxydul, Manganoxydul, Tonerde, Kali und Natron). In der chemischen Zusammensetzung dieser Mineralien herrscht die Talkerde vor.)

\vspace{2ex}

13. Parma. Mehr Kies als Eisen. Lichtgraue Grundmasse. Breccienartig, oder doch breccienartige Zeichng.

14. Siena. Mehr Kies als Eisen. Lichtgraue Grundmasse. Breccienartig, oder doch breccienartige Zeichng.

15. Ensisheim. Mehr Kies als Eisen. Dunkelgraue Grundmasse. Breccienartig, oder doch breccienartige Zeichng.

16. L'Aigle. Mehr Eisen als Kies. Dunkelgraue Grundmasse. Breccienartig, oder doch breccienartige Zeichng.

17. Liponas. Mehr Eisen als Kies. Dunkelgraue Grundmasse. Breccienartig, oder doch breccienartige Zeichng.

18. Chantonnay. Vorherrschend schwarze, basaltartige Grundmasse, ohne porphyrartige Einmengungen; stellenweise graue Grundmasse.

19. Renazzo. Durchaus schwarze dichte Grundmasse, mit einem porphyrartig eingewachsenen weißen Mineral.

20. Richmond. Schwarzgraue, poröse Grundmasse, mit schwarzgrauen kugligen Ausscheidungen.

21. Weston. Die Grundmasse fleckig zum Teil lichtgrau, zum Teil dunkelgrau.

22. La Baffe. Lichtgraue Grundmasse. Aus der Grundmasse hervorragende, deutliche kuglige Ausscheidung.

23. Benares. Lichtgraue Grundmasse. Aus der Grundmasse hervorragende, deutliche kuglige Ausscheidung.

24. Gouv. Poltawa. Dunkelgraue Grundmasse. Aus der Grundmasse hervorragende, deutliche kuglige Ausscheidung. 

25. Krasno-Ugol. Dunkelgraue Grundmasse. Aus der Grundmasse hervorragende, deutliche kuglige Ausscheidung.

26. Erxleben. Dichte, fast homogen aussehende, dunkelgraue Grundmasse, ohne, oder nur mit einzelnen und undeutlichen kugligen Ausscheidungen.

27. Gouv. Simbirsk. Dichte, fast homogen aussehende, dunkelgraue Grundmasse, ohne, oder nur mit einzelnen und undeutlichen kugligen Ausscheidungen.

\vspace{2ex}

(28-69.): Die kugligen Ausscheidungen meist wenig deutlich, zuweilen auch, wenn sie mit der Grundmasse fest verwachsen sind, nur durch ein fleckiges Aussehen der Masse wahrzunehmen.
\begin{center}
(28., 29.)
\end{center}
\hspace*{6mm}28. Mauerkirchen. Lichtgraue Grundmasse. Die kugligen Ausscheidungen meist wenig deutlich, zuweilen auch, wenn sie mit der Grundmasse fest verwachsen sind, nur durch ein fleckiges Aussehen der Masse wahrzunehmen.

29. Nashville. Lichtgraue Grundmasse.

\vspace{2ex}

30. Lucé. Lichtgraue Grundmasse.

\vspace{2ex}

\begin{center}
(31., 32.)
\end{center}

31. Lissa. Lichtgraue Grundmasse.

32. Owahu. Lichtgraue Grundmasse.

\begin{center}
(33., 34., 35., 36., 37., 38., 39., 40., 41., 42.)
\end{center}

33. Charkow. Lichtgraue Grundmasse.

34. Zaborzika. Lichtgraue Grundmasse.

35. Bachmut. Lichtgraue Grundmasse.

36. Politz. Lichtgraue Grundmasse.

37. Kuleschofka. Lichtgraue Grundmasse.

38. Slobodka. Lichtgraue Grundmasse.

39. Milena. Lichtgraue Grundmasse.

40. Forsyth. Lichtgraue Grundmasse.

41. Yorkshire. Lichtgraue Grundmasse.

42. Glasgow. Lichtgraue Grundmasse.

\begin{center}
(43., 44.)
\end{center}

43. Berlanguillas. Die Grundmasse aus dem Lichtgrauen in das Dunkelgraue übergehend.

44. Apt. Die Grundmasse aus dem Lichtgrauen in das Dunkelgraue übergehend.

\vspace{2ex}

45. Vouillé. Die Grundmasse aus dem Lichtgrauen in das Dunkelgraue übergehend.

46. Château-Renard. Die Grundmasse aus dem Lichtgrauen in das Dunkelgraue übergehend.

47. Salés. Die Grundmasse aus dem Lichtgrauen in das Dunkelgraue übergehend.

48. Agen. Die Grundmasse aus dem Lichtgrauen in das Dunkelgraue übergehend.

\begin{center}
(49., 50.)
\end{center}

49. Nanjemoy. Die Grundmasse aus dem Lichtgrauen in das Dunkelgraue übergehend.

50. Asco. Die Grundmasse aus dem Lichtgrauen in das Dunkelgraue übergehend.

\vspace{2ex}

51. Toulouse. Die Grundmasse aus dem Lichtgrauen in das Dunkelgraue übergehend.

\begin{center}
(52., 53.)
\end{center}

52. Blansko. Dunkelgraue (bläulichgraue) Grundmasse.

53. Wessely. Dunkelgraue (bläulichgraue) Grundmasse.

\begin{center}
(54., 55., 56.)
\end{center}

54. Limerick. Dunkelgraue (bläulichgraue) Grundmasse.

55. Grüneberg. Dunkelgraue (bläulichgraue) Grundmasse.

56. Tipperary. Dunkelgraue (bläulichgraue) Grundmasse.

\vspace{2ex}

57. Gouv. Kursk. Dunkelgraue (bläulichgraue) Grundmasse.

58. Lixna. Dunkelgraue, oder zwischen licht- und dunkelgrau schwankende, durch eine Menge von eingesäten Rostflecken, mehr oder weniger ins Braune ziehende Grundmasse (Zugleich die eisenreichsten Meteorsteine.)

59. Tabor. Dunkelgraue, oder zwischen licht- und dunkelgrau schwankende, durch eine Menge von eingesäten Rostflecken, mehr oder weniger ins Braune ziehende Grundmasse (Zugleich die eisenreichsten Meteorsteine.)

60. Charsonville. Dunkelgraue, oder zwischen licht- und dunkelgrau schwankende, durch eine Menge von eingesäten Rostflecken, mehr oder weniger ins Braune ziehende Grundmasse (Zugleich die eisenreichsten Meteorsteine.)

\begin{center}
(61., 62.)
\end{center}

61. Doroninsk. Dunkelgraue, oder zwischen licht- und dunkelgrau schwankende, durch eine Menge von eingesäten Rostflecken, mehr oder weniger ins Braune ziehende Grundmasse (Zugleich die eisenreichsten Meteorsteine.)

62. Seres. Dunkelgraue, oder zwischen licht- und dunkelgrau schwankende, durch eine Menge von eingesäten Rostflecken, mehr oder weniger ins Braune ziehende Grundmasse (Zugleich die eisenreichsten Meteorsteine.)

\begin{center}
(63., 64.)
\end{center}

63. Sigena. Dunkelgraue, oder zwischen licht- und dunkelgrau schwankende, durch eine Menge von eingesäten Rostflecken, mehr oder weniger ins Braune ziehende Grundmasse (Zugleich die eisenreichsten Meteorsteine.)

64. Barbotan. Dunkelgraue, oder zwischen licht- und dunkelgrau schwankende, durch eine Menge von eingesäten Rostflecken, mehr oder weniger ins Braune ziehende Grundmasse (Zugleich die eisenreichsten Meteorsteine.)

\begin{center}
(65., 66., 67., 68.)
\end{center}

65. Eichstädt. Dunkelgraue, oder zwischen licht- und dunkelgrau schwankende, durch eine Menge von eingesäten Rostflecken, mehr oder weniger ins Braune ziehende Grundmasse (Zugleich die eisenreichsten Meteorsteine.)

66. Groß-Divina. Dunkelgraue, oder zwischen licht- und dunkelgrau schwankende, durch eine Menge von eingesäten Rostflecken, mehr oder weniger ins Braune ziehende Grundmasse (Zugleich die eisenreichsten Meteorsteine.)

67. Zebrak. Dunkelgraue, oder zwischen licht- und dunkelgrau schwankende, durch eine Menge von eingesäten Rostflecken, mehr oder weniger ins Braune ziehende Grundmasse (Zugleich die eisenreichsten Meteorsteine.)

68. Timochin. Dunkelgraue, oder zwischen licht- und dunkelgrau schwankende, durch eine Menge von eingesäten Rostflecken, mehr oder weniger ins Braune ziehende Grundmasse (Zugleich die eisenreichsten Meteorsteine.)

\vspace{2ex}

69. Macao. Dunkelgraue, oder zwischen licht- und dunkelgrau schwankende, durch eine Menge von eingesäten Rostflecken, mehr oder weniger ins Braune ziehende Grundmasse (Zugleich die eisenreichsten Meteorsteine.)
\subsection{Meteoreisen.}
\paragraph{}
Metallische Meteoriten, die wenigstens zur Hälfte aus metallischem Eisen, meist aber vorwaltend aus derbem Eisen bestehen, dem einige andere, gleichfalls meist metallische Mineralien nur in geringer Menge beigemengt sind.
\subsubsection[Ästiges Meteoreisen.]{Ästiges Meteoreisen,}
mit einer ungefähr gleichen Menge von Olivin oder einem olivinartigen Mineral in den Höhlungen oder Zwischenräumen, dessen Vorhandensein die ästige oder schwammförmige Gestalt des Eisens, das gleichsam das Skelett des Ganzen bildet, bestimmt.

\vspace{2ex}

1. Mit Olivin (Talkerde-Silicat).

\vspace{2ex}

70. Atacama. Der Olivin feinkörnig, zerreiblich. Durch Ätzen von polierten Flachen mit Sauren entstehen dem breiten nicht angreifbaren und daher Glanz behaltenden Rande der durchschnittenen Eisenpartien parallel gehende graue matte Felder, zuweilen von einzelnen Linien durchzogen.

71. Krasnojarsk. Der Olivin großkörnig, nicht zerreiblich. Durch Ätzen von polierten Flachen mit Sauren entstehen dem breiten nicht angreifbaren und daher Glanz behaltenden Rande der durchschnittenen Eisenpartien parallel gehende graue matte Felder, zuweilen von einzelnen Linien durchzogen.

72. Brahin. Der Olivin großkörnig, nicht zerreiblich. Durch Ätzen von polierten Flachen mit Sauren entstehen dem breiten nicht angreifbaren und daher Glanz behaltenden Rande der durchschnittenen Eisenpartien parallel gehende graue matte Felder, zuweilen von einzelnen Linien durchzogen.

\vspace{2ex}

2. Mit einem olivinähnlichen Mineral (einem Talkerde-Trisilikat).

\vspace{2ex}

73. Sachsen. Durch Ätzen entstehen auf den durchschnittenen Eisenpartien Widmanstättensche Figuren.

\subsubsection[Derbes Meteoreisen.]{Derbes Meteoreisen,}
von unbestimmter Form, mit Einmengungen, die auf die Gestalt des Eisens keinen Einfluss ausüben, und demselben nur in geringer Menge beigemengt sind. (Sie machen, etwa mit Ausnahme von Nr. 74, wohl nie mehr als den zwölften Teil des Ganzen aus.)

\vspace{2ex}

1. Die Einmengung besteht aus einem erdigen, grünlichen oder braunen olivinartigen Mineral (und wohl auch aus Magnetkies).

\vspace{2ex}

74. Bitburg.

\vspace{2ex}

2. Die Einmengung besteht aus Magnetkies (zuweilen wohl auch aus einer zweiten, noch nicht hinreichend untersuchten Schwefeleisen-Verbindung) und in selteneren Fällen aus einigen anderen Mineralien (Magneteisen, Chromeisen, Graphit, Verbindungen von Phosphoreisen mit Phosphornickel und Phosphormagnesium u. s. w.).

\vspace{2ex}

75. Toluca. Durch Ätzen mit Sauren oder Anlaufen durch Hitze entstehen vollkommene Widmanstättensche Figuren, d. h. mit der kristallinischen Struktur und der chemischen Beschaffenheit des Eisens, das teils rein, teils mit Nickel, Kobalt, Phosphor u. s. w. legiert ist, zusammenhangende Zeichnungen, die aus Streifen, Zwischenfeldern und Einfassungsleisten bestehen. Die Zwischenfelder sind schraffiert und wiederhohlen in Kleinen die Beschaffenheit der Masse in Großen.

76. Elbogen. Durch Ätzen mit Sauren oder Anlaufen durch Hitze entstehen vollkommene Widmanstättensche Figuren, d. h. mit der kristallinischen Struktur und der chemischen Beschaffenheit des Eisens, das teils rein, teils mit Nickel, Kobalt, Phosphor u. s. w. legiert ist, zusammenhangende Zeichnungen, die aus Streifen, Zwischenfeldern und Einfassungsleisten bestehen. Die Zwischenfelder sind schraffiert und wiederhohlen in Kleinen die Beschaffenheit der Masse in Großen.

77. Agram. Durch Ätzen mit Sauren oder Anlaufen durch Hitze entstehen vollkommene Widmanstättensche Figuren, d. h. mit der kristallinischen Struktur und der chemischen Beschaffenheit des Eisens, das teils rein, teils mit Nickel, Kobalt, Phosphor u. s. w. legiert ist, zusammenhangende Zeichnungen, die aus Streifen, Zwischenfeldern und Einfassungsleisten bestehen. Die Zwischenfelder sind schraffiert und wiederhohlen in Kleinen die Beschaffenheit der Masse in Großen.

78. Lenarto. Durch Ätzen mit Sauren oder Anlaufen durch Hitze entstehen vollkommene Widmanstättensche Figuren, d. h. mit der kristallinischen Struktur und der chemischen Beschaffenheit des Eisens, das teils rein, teils mit Nickel, Kobalt, Phosphor u. s. w. legiert ist, zusammenhangende Zeichnungen, die aus Streifen, Zwischenfeldern und Einfassungsleisten bestehen. Die Zwischenfelder sind schraffiert und wiederhohlen in Kleinen die Beschaffenheit der Masse in Großen.

79. Red-River. Durch Ätzen mit Sauren oder Anlaufen durch Hitze entstehen vollkommene Widmanstättensche Figuren, d. h. mit der kristallinischen Struktur und der chemischen Beschaffenheit des Eisens, das teils rein, teils mit Nickel, Kobalt, Phosphor u. s. w. legiert ist, zusammenhangende Zeichnungen, die aus Streifen, Zwischenfeldern und Einfassungsleisten bestehen. Die Zwischenfelder sind schraffiert und wiederhohlen in Kleinen die Beschaffenheit der Masse in Großen.

80. Durango. Durch Ätzen mit Sauren oder Anlaufen durch Hitze entstehen vollkommene Widmanstättensche Figuren, d. h. mit der kristallinischen Struktur und der chemischen Beschaffenheit des Eisens, das teils rein, teils mit Nickel, Kobalt, Phosphor u. s. w. legiert ist, zusammenhangende Zeichnungen, die aus Streifen, Zwischenfeldern und Einfassungsleisten bestehen. Die Zwischenfelder sind schraffiert und wiederhohlen in Kleinen die Beschaffenheit der Masse in Großen.

81. Guilford. Durch Ätzen mit Sauren oder Anlaufen durch Hitze entstehen vollkommene Widmanstättensche Figuren, d. h. mit der kristallinischen Struktur und der chemischen Beschaffenheit des Eisens, das teils rein, teils mit Nickel, Kobalt, Phosphor u. s. w. legiert ist, zusammenhangende Zeichnungen, die aus Streifen, Zwischenfeldern und Einfassungsleisten bestehen. Die Zwischenfelder sind schraffiert und wiederhohlen in Kleinen die Beschaffenheit der Masse in Großen.

82. Caille. Durch Ätzen mit Sauren oder Anlaufen durch Hitze entstehen vollkommene Widmanstättensche Figuren, d. h. mit der kristallinischen Struktur und der chemischen Beschaffenheit des Eisens, das teils rein, teils mit Nickel, Kobalt, Phosphor u. s. w. legiert ist, zusammenhangende Zeichnungen, die aus Streifen, Zwischenfeldern und Einfassungsleisten bestehen. Die Zwischenfelder sind schraffiert und wiederhohlen in Kleinen die Beschaffenheit der Masse in Großen.

83. Ashville. Durch Ätzen mit Sauren oder Anlaufen durch Hitze entstehen vollkommene Widmanstättensche Figuren, d. h. mit der kristallinischen Struktur und der chemischen Beschaffenheit des Eisens, das teils rein, teils mit Nickel, Kobalt, Phosphor u. s. w. legiert ist, zusammenhangende Zeichnungen, die aus Streifen, Zwischenfeldern und Einfassungsleisten bestehen. Die Zwischenfelder sind schraffiert und wiederhohlen in Kleinen die Beschaffenheit der Masse in Großen.

84. Tennessee. Durch Ätzen mit Sauren oder Anlaufen durch Hitze entstehen vollkommene Widmanstättensche Figuren, d. h. mit der kristallinischen Struktur und der chemischen Beschaffenheit des Eisens, das teils rein, teils mit Nickel, Kobalt, Phosphor u. s. w. legiert ist, zusammenhangende Zeichnungen, die aus Streifen, Zwischenfeldern und Einfassungsleisten bestehen. Die Zwischenfelder sind schraffiert und wiederhohlen in Kleinen die Beschaffenheit der Masse in Großen.

85. Bohumilitz. Durch Ätzen oder Anlaufen entstehen unvollkommene Widmanstättensche Figuren, d. h. die Einfassungsleisten sind wenig deutlich, die Zwischenfelder verschwinden fast ganz, die Streifen sind dagegen sehr breit und schimmern fleckenweise und abwechselnd wie moire metallique.

86. Bahia. Durch Ätzen oder Anlaufen entstehen unvollkommene Widmanstättensche Figuren, d. h. die Einfassungsleisten sind wenig deutlich, die Zwischenfelder verschwinden fast ganz, die Streifen sind dagegen sehr breit und schimmern fleckenweise und abwechselnd wie moire metallique.

87. Zacatecas. Durch Ätzen entstehen keine Widmanstättenschen Figuren, sondern längere feine Linien, die sich zwar öfter berühren und schneiden, und dadurch unvollkommene Zwischenfelder bilden; diese werden aber nur von diesen Linien (nicht von Streifen mit Begrenzungsleisten) umgeben, und sind zum Teil mit unterbrochenen kurzen Linien der Strichelchen ausgefüllt.

88. Rasgatà. Durch Ätzen entstehen keine Widmanstättenschen Figuren, sondern längere feine Linien, die sich zwar öfter berühren und schneiden, und dadurch unvollkommene Zwischenfelder bilden; diese werden aber nur von diesen Linien (nicht von Streifen mit Begrenzungsleisten) umgeben, und sind zum Teil mit unterbrochenen kurzen Linien der Strichelchen ausgefüllt.

89. Tucuman. Durch Ätzen entstehen keine Widmanstättenschen Figuren, sondern kurze feine Linien, die sich oft berühren und schneiden, ohne Mittelfelder zu bilden, und dem Ganzen eine gestrickte oder federartige Zeichnung verleihen.

90. Senegal. Durch Ätzen entstehen keine Widmanstättenschen Figuren, sondern kurze feine Linien, die sich oft berühren und schneiden, ohne Mittelfelder zu bilden, und dem Ganzen eine gestrickte oder federartige Zeichnung verleihen.

91. Kap der guten Hoffnung. Durch Ätzen entstehen entweder gar keine Figuren, oder es ziehen sich über die geätzte graue und feinkörnige Fläche einzelne, zuweilen mehrere parallele Bänder hin, die jedoch nur sichtbar sind, wenn die Fläche nach gewissen Richtungen gehalten wird.

92. Clairborne. Durch Ätzen entstehen entweder gar keine Figuren, oder es ziehen sich über die geätzte graue und feinkörnige Fläche einzelne, zuweilen mehrere parallele Bänder hin, die jedoch nur sichtbar sind, wenn die Fläche nach gewissen Richtungen gehalten wird.

\subsubsection{Anhang. Wegen Hammerung nicht untersuchbar.}
\hspace*{6mm}93. Oaxaca.

94. Grönland.
\clearpage
\end{document}
